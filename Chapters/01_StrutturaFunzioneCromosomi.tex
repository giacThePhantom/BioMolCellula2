\chapter{Struttura e funzione dei cromosomi}
\section{Organizzazione dei cromosomi}
L'informazione genetica \`e impacchettata in almeno una molecola di DNA molto lunga, un cromosoma. Ogni cromosoma contiene una molecola di DNA a doppio filamento con molti geni e regioni
di DNA non codificante. Si dicono intergeniche le regioni tra i geni. Se batteri ed archea possiedono crsomosomi circolari gli eucarioti ne possiedono di lineari. La distribuzione dei
geni varia tra gli organismi: quelli meno complessi tendono ad avere geni ordinati pi\`u densamente. La densit\`a genica pu\`o variare anche sui diversi cromosomi degli organismi. Il
numero dei cromosomi \`e caratteristico per una specie. Si possono fare incroci tra specie con numero di cromosomi diverso: in questo caso sono incapaci di accoppiarsi durante la prima 
parte della meiosi e l'incrocio risulta sterile.
\subsection{Ploidia}
Con ploidia si intende quanti cromosomi identici possiede un organismo:
\begin{itemize}
	\item Aploidia: $1$ cromosoma come nel lievito.
	\item Diploidia: $2$ cromosomi come negli umani.
	\item Poliploidia: pi\`u di $2$ cromosomi come nelle piante.
	\item Aneuploidia: un numero anormale di cromosomi, pu\`o avvenire in caso di sindromi genetiche o cancri.
\end{itemize}
La poliploidia viene sfruttata nei prodotti ortofrutticoli per aumentarne le dimensioni.
\subsubsection{Aneuploidia e aborti}
L'aneuploidia pu\`o essere sopportata in un certo numero dagli organismi: si nota per la trisomia del cromosoma $21$ (sindrome di Down) e le poliploidie, ma pu\`o essere mortale e
causare un aborto spontaneo.
\subsubsection{Gametogenesi femminile}
Si nota come con l'aumentare dell'et\`a della donna aumenta il rischio di aneuploidia per i figli. Questo avviene in quanto ogni donna nasce con tutte le uova diploidi gi\`a presenti 
anche se immature. Queste maturano una alla volta dopo la pubert\`a una volta al mese. La continuazione della meiosi bloccata comincia il giorno prima dell'ovulazione a causa dalla
gonadotropina. Un oocita primario pu\`o causare pi\`u errori durante la segregazione cromosomica nelle due fasi della meiosi rispetto a un uovo pi\`u giovane risultando in un uovo 
aploide con pi\`u o meno cromosomi. 
\subsection{Ulteriore DNA presente nelle cellule}
\subsubsection{Cellule eucariote}
Le cellule eucariote possono avere DNA addizionale oltre il DNA cromosomale, in particolare in:
\begin{itemize}
	\item Mitocondri: forniscono le cellule con ATP e sono organelli racchiusi da membrana con il proprio, solitamente circolare, cromosoma singolo.
	\item Cloroplasti: derivano l'energia dalla luce solare nelle piante, possiedono un proprio cromosoma.
\end{itemize}
Si pensa che questi organelli derivino da un batterio ancestrale assorbito e mantenuto da un altro organismo unicellulare.
\subsubsection{Cellule batteriche}
Le cellule batteriche possiedono DNA addizionale nelle proprie cellule: piccolo DNA circolare detto plasmide. Questi tipicamente codificano poche proteine che conferiscono un vantaggio
selettivo come una resistenza ad un antibiotico.
\subsubsection{Virus}
I virus sono agenti infettivi che trasportano informazioni genetiche come piccoli cromosomi a DNA o RNA. I cromosoma virale pu\`o essere lineare o circolare, a doppio o singolo 
filamento.
\section{Impacchettamento del DNA cromosomale}
Si nota come per potersi adattare alle dimensioni del nucleo, delle cellule o di organelli intracellulari il DNA deve essere compattato. Compattare il genoma svolge anche una funzione
di protezione, rendendolo meno accessibile da agenti esterni. 
\subsection{Il nucleoide}
Nei procarioti, in assenza di nucleo il DNA si organizza in un nucleoide. \`E composto per l'$80\%$ di DNA e per il restante $20$ di proteine di compattamento e RNA. Il cromosoma 
\`e pertanto composto da un grande complesso DNA proteine detto cromatina. Il nucleoide appare come una regione che esclude cromosomi, occupa $\frac{1}{3}$ del volume della cellula ed
\`e ancorato all'origine di replicazione nella membrana cellulare. Forma ``loops" o domini di circa $40kb$ grazie alla proteina \emph{HIF} (integration host factor), una piccola
proteina carica positivamente per bilanciare le cariche negative sul backbone. Il DNA \`e successivamente superavvolto da altre proteine che piegano il DNA. L'IHF \`e costituito da
un dimero su cui si forma il loop. Il superavvolgimento \`e controllato da altri fattori come fattori di trascrizione e l'attivit\`a di DNA ed RNA polimerasi che creano due
superavvolgimenti con polarit\`a opposta ai lati della bolla.
\subsection{DNA eucariotico}
Nel nucleo degli eucarioti i cromosomi subiscono cambi visibili durante il ciclo di divisione cellulare. Nelle cellule umane diploidi il DNA deve essere compattato 
\num{300000}-\num{400000} volte. 
\subsection{Il ciclo cellulare e la dinamicit\`a dei cromosomi}
Durante la fase $G_2$ del ciclo cellulare i cromosomi replicati si trovano in uno stato poco avvolto e si forma il centromero. Nella profase compaiono le fibre del fuso e i cromosomi
si condensano. Nella prometafase le fibre si attaccano ai cromosomi che continuano a condensarsi. Nella metafase i cromosomi si allineano. Nell'anafase i centromeri si dividono e
i cromatidi fratelli si muovono ai poli opposti. Durante la telofase si riforma la membrana nucleare, i cromosomi si decondensano e scompaiono le fibre del fuso. 
\section{Impacchettamento del DNA cromosomale degli eucarioti}
\subsection{Istoni}
Negli eucarioti gli istoni sono proteine leganti il DNA. Si trovano quattro istoni del nucleo: nascono molto presto nell'evoluzione ed essendo cruciali per la sopravvivenza sono 
altamente conservati. Gli istoni sono basici in quanto ricchi in lisina e arginina cariche positivamente grazie all'gruppo ammino \ce{+NH3} che stabilizzano le interazioni tra il DNA
e gli istoni. $146bp$ si arrotolano $1.75$ volte intorno a un complesso istonico in maniera sinistrorsa per formare un nucleosoma. Il complesso istonico prende il nome di ottamero 
istonico. Il superavvolgimento negativo facilita la separazione pi\`u facile, necessaria per la replicazione e la trascrizione. L'ottamero istonico ha due di ognuno dei quattro 
istoni del nucleo: \emph{H2A}, \emph{H2B}, \emph{H3}, \emph{H4}. Inizialmente due dimeri \emph{H3-H4} si associano con il DNA e reclutano poi due dimeri \emph{H2A-H2B} per la formazione
dell'ottamero. Nonostante tutto il DNA eucariote sia impacchettato dagli istoni i nucleosomi si formano preferenzialmente a sequenze di DNA. Il DNA \`e generalmente piegato dolcemente
intorno agli istoni ma presenta curve pi\`u acute ????????? La scanalatura minore deve diventare pi\`u stretta durante il piegamento, cosa pi\`u favorevole in regioni ricche di 
\emph{AT}. Gli istoni fanno $13$ interazioni con gli istoni del DNA nucleosomale: i due dimeri \emph{H3-H4} legano il centro e le terminazioni del DNA mentre \emph{2(H2A-H2B)} legano
$30bp$ su un lato del nucleosoma. Il core istonico \`e composto dai domini di histone-fold composti da tre $\alpha$-eliche. 
\subsubsection{Code istoniche}
Il nucleo di una proteina istonica \`e legato a una lunga coda N-terminale che si estende verso l'esterno. Sono lunghe tra i $20$ e i $39$ amminoacidi e non hanno strutture. 
Interagiscono con altri nucleosomi per aiutare un ulteriore compattamento del DNA e strutture cromatiniche di livello superiore. La coda pu\`o essere modificata chimicamente in modo
da modificare la struttura della cromatina e la sua funzione promuovendo o prevenendo il reclutamento di proteine che regolano la trascrizione. \emph{H2A} e \emph{H2B} presentano
anche code C-terminali che regolano la trascrizione.
\subsubsection{Varianti istoniche}
Le varianti istoniche sono alrte proteine con staibilit\`a diverse, domini specialisti che cambiano la funzione del cromosoma, sequenze diverse alle terminazioni. Hanno amminoacidi
diversi che possono essere diversamente modificati. Le varianti istoniche sono depositate da complessi di rimodellamento della cromatina dipendenti da ATP e possono essere 
dipendenti o indipendenti dalla replicazione. 
\subsubsection{Interazioni con il DNA}
Gli istoni interagiscono con il DNA attraverso interazioni elettrostatiche tra i gruppi fosfato del legame fosfodiestere e gli amminoacidi basici negli istoni e attraverso legami a
idrogeno tra l'atomo di ossigeno nei gruppi fosfato e gli atomi di idrogeno nei gruppi ammino degli istoni. Modifiche chimiche delle basi o code istoniche attraverso enzimi 
modificano le cariche locali e le interazioni. 
\subsection{Livelli di compattazione}
L'impacchettamento cromatinico ha diversi livelli di compattamento. 
\subsubsection{Primo livello}
Il primo livello di compattazione \`e la fibra di $10\si{\nano\metre}$, che nasce dall'associazione con il DNA dei
nucleosomi con apparenza di perline su un filo. 
\subsubsection{Secondo livello}
La fibra \`e ulteriomente compattata da una quinta proteina istonica \emph{H1} in una fibra di \num{30}\si{nm} nel secondo livello di 
compattamento, un ordinamento regolare che avvicina i nucleosomi. \emph{H1} si lega al DNA linker tra due nucleosomi successivi diminuendo la lunghezza di \num{7} volte. Anche le
code istoniche sono coinvolte nella formazione di questo secondo livello. 
\subsubsection{Terzo livello}
Il terzo livello di compattamento, con diametro di \num{300}\si{nm} si forma grazie a domini di loop radiali e al legame con la matrice nucleare nelle cellule in interfase. La
matrice nucleare \`e composta dalla lamina nucleare composta da fibre proteiche della matrice interna e da proteine che legano ad essa i cromosomi. Le proteine attaccano la
base di un loop di DNA alla fibra proteica grazie a sequenze specifiche \emph{MAR} (matrix-attachment region) e \emph{SAR} (scaffold attachment region). Si nota come ogni cromosoma
occupa nel nucleo un territorio determinato
\subsubsection{Quarto livello}
I loop radiali diventano altamente compattati e rimangono ancorati alla matrice nucleare. Mentre la cellula entra la profase la membrana nucleare si dissolve e non si trova pi\`u una
matrice nucleare: la compattazione aumenta drammaticamente nel quarto livello di compattazione o condensazione. Alla fine della profase i cromosomi sono interamente eterocromatici
con un diametro di \num{700}\si{nm}. Pertanto i cromosomi in metafase subiscono poca trascrizione e unicamente nel centromero. In questo momento i cromosomi hanno accesso al fuso
mitotico. 
\subsubsection{Quinto livello}
Il quinto livello di compattamento avviene con la formazione dei cromosomi visibili e grazie alla condensina. La condensina \`e una proteina che si sposta nel nucleo durante l'inizio 
della fase $M$, si lega ai cromosomi e compatta i loop radiali riducendo il loro diametro. Un'altra proteina coinvolta \`e la coesina caricata durante la fase $S$ per tenere uniti i 
cromatidi fratelli. 
\section{Modifiche covalenti degli istoni}
\subsection{Epigenetica}
Si intende per epigenetica l'ereditariet\`a di fenotipi non causati da cambi nella sequenza del DNA. \`E un fenomeno principalmente eucariote ed \`e causata da cambi strutturali
nella composizione dei nucleosomi (varianti istoniche), modifiche chimiche della coda o nucleo istonico che altera lo stato di compattazione della cromatina e l'attivit\`a del nucleo
del nucleosoma, metilazione del DNA alla citosina e dal legame di DNA o RNA con RNA non codificanti. Queste opzioni alterano l'espressione genetica. Cambi epigenetici sono 
trasferiti da cellula madre e figlia durante la replicazione del DNA e un numero di sindromi e cancri sono dovuti alla mal-regolazione di attivit\`a epigenetiche. Nei batteri la
trascrizione dipende principalmente dall'RNA polimerasi e la sua regolazione allo stadio di iniziazione. Metilazione di adenosina e citosina intervengono nell'espressione genica, 
nella replicazione e riparazione del DNA e come difesa contro attacchi virali. Le modifiche chimiche pi\`u comuni sono alle code istoniche ma anche gli amminoacidi del nucleo globulare 
degli istoni possono essere modificati. Le modifiche sono principalmente acetilazione, metilazione, fosforilazione, ubiquitinazione e sumoilazione: ``PUMAS". Tali modifiche vanno a 
colpire la struttura cromatinica e il recltuamento di proteine specifiche su di essa. Le modifiche epigenetiche sono molto veloci e reversibili attraverso enzimi e sono alla base di
una veloce e precisa regolazione dell'attivit\`a genica. 
\subsection{Acetilazione}
La maggior prate della cromatina possiede istoni acetilati, specialmente nelle code \emph{H3} e \emph{H4}. \`E associata con una trascrizione attiva: l'eucromatina \`e pi\`u acetilata.
L'acetilazione di code e nuclei ha effetto sulla struttura cromatina:
\begin{itemize}
	\item Direttamente: neutralizza le cariche positive sulla lisina sulla coda istonia criducendo le interazioni tra le code e il DNA rendendo la cromatina pi\`u accessibile da
		proteine leganti il DNA.
	\item Indirettamente: la lisina acetilata agisce come un sito di riconoscimento e legame per proteine contenenti bromodomini o lettori che possono reclutare altre proteine, 
		componenti di grandi complessi che regolano la trascrizione come \emph{HAT}, complessi di rimodellamento della cromatina e fattori di trascrizione che agiscono come
		\emph{HAT}.
\end{itemize}
L'enzima responsabile per l'aggiunta di un gruppo acetile (mono-acetilazione) al gruppo ammino \ce{+NH3} della lisina \`e l'istone acetiltrasferasi \emph{HAT}, mentre l'istone 
deacetilasi \emph{HDAC} lo rimuove. L'acetilazione della lisina pertanto neutralizza direttamente la carica positiva di essa riducendo l'attrazione tra DNA \ce{PO4-} e lisina \ce{NH3+}.
Inoltre diventa un sito di legame per proteine con bromodominio e rimodellatrici della cromatina aprendola e attivando la trascrizione. La deacetilasi agisce come repressione della
trascrizione. La (de)acetilazione in regioni promotrici ha un ruolo nell'iniziazione della trascrizione. Altre acetilazioni sono presenti lungo sequenze codificanti, con ruolo 
nell'allungamento della trascrizione. Se ne trovano ancora in enhancers o in varianti istoniche che presentano trascrizione attiva. Le proteine contenenti un bromodominio possono 
legarsi a una o pi\`u lisine acetilate attraverso il dominio e contengono altri domini come un dominio \emph{PHD} che si lega a lisine metilate. 
\subsection{Metilazione}
La metilazione sugli istoni avviene grazie a un istone metiltrasferasi \emph{HMT} che pu\`o aggiungere $1$, $2$ o $3$ gruppi metile sul gruppo ammino della lisina \emph{K} o $1$ o $2$
gruppi metile sul gruppo ammino dell'arginina \emph{A}. La metilazione della coda e del core istonico ha due effetti sulla struttura cromatinica:
\begin{itemize}
	\item Diretto: mantiene la carica locale della lisina positiva compattando il legame tra istoni e DNA.
	\item Indiretto: proteine contenenti cromodomini (\emph{HP1}, \emph{Polycomb}) riconoscono e legano a specifiche lisine metilate e reclutano proteine che causano il silenziamento
		trascrizionale (togliendo spazio al legame con fattori di trascrizione) o la sua attivazione.
\end{itemize}
La metilazione \`e associata sia con attivazione che con repressione della trascrizione in base al residuo che \`e metilato:
\begin{itemize}
	\item Mono-metilazione di \emph{K9} nella coda \emph{H3} causa una cromatina attiva trascrizionalmente.
	\item Mono- o tri-metilazione di \emph{K4} nella coda \emph{H3} causa una cromatina attiva trascrizionalmente.
	\item Di- o tri-metilazione di \emph{K9} nella coda \emph{H3} causa una cromatina silente trascrizionalmente.
\end{itemize}
\subsection{Fosforilazione}
I fosfati sono aggiunti da chinasi e rimossi da fosfatasi. La fosforilazione aggiunge una carica negativa alla coda istonica. Fosforilazione di \emph{S10} nella coda \emph{H3} permette
la crescita cellulare e trascrizione promuovendo l'acetilazione di \emph{K14} sulla coda \emph{H3}.La fosforolaizone di \emph{S10} e \emph{S27} nella coda \emph{H3} \`e correlata con
la condensazione dei cromosomi durante la mitose. \`E importante per la replicazione e riparazione del DNA e per l'apoptosi. 
\subsection{Ubiquitinazione e sumoilazione}
L'ubiquitinazione delle lisine consiste dell'aggiunta di una proteina di $76$ amminoacidi catalizzata dall'ubiquitina ligasi e rimossa dalla de-ubiquitinasi. Il suo ruolo non \`e 
compreso a fondo e avviene specialmente nelle code C-terminali di \emph{H2A} e \emph{H2B}. Regola la trascrizione reclutando rimodellatori e risposte al danno del DNA. Una 
mono-ubiquitinazione di \emph{H2A} causa repressione trascrizionale mentre se avviene a \emph{H2B} causa un'attivazione indiretta in quanto richiesta per la mono-metilazione di 
\emph{H3K4} e di \emph{H3K79}. La sumoilazione della lisina \`e una modifica simile all'ubiquitinazione e gioca un ruolo nella regolazione di trascrizione e riparazione di DNA.
\subsection{Codice istonico}
Le grandi possibili modifiche in aggiunta con le loro interazioni porta alla definizione di un codice istonico in cui modifiche uniche definiscono certi stati di cromatina e di 
espressione genica. 
\section{Complessi rimodellatori dei nucleosomi}
La cromatina compattata rappresenta una barriera per le proteine che devono accedere al DNA e pertanto inibisce processi come trascrizione. La composizione del nucleosoma, la compattezza
del suo legame con il DNA e la sua locazione possono essere fisicamente cambiati da complessi di rimodellamento dei nucleosomi dipendenti da ATP. Questi complessi possono introdurre
loop nel DNA avvolto intorno a un nucleo istonico, far scivolare il DNA lungo l'ottamero istonico o rimuovere l'intero ottamero o $1$-$2$ proteine istoniche e trasferirle da qualche 
altra parte. Possono attivare o reprimere la trascrizione ma non sono usati per la replicazione.
\subsection{Ruoli dei rimodellatori della cromatina}
I complessi di rimodellamento della cromatina hanno diversi ruoli nello stato cromatinico. Possono intervenire dopo la deposizione degli istoni durante la maturazione dei nucleosomi
portando a una loro spaziazione regolare. Possono inoltre alterare lo stato cromatinico riposizionando i nucleosomi, espellendoli completamente o solo alcune loro subunit\`a. Possono
inoltr compiere installazioni o rimozioni di varianti istoniche. 
\subsection{Sottofamiglie}
Esistono diverse classi di rimodellatori dei nucleosomi, ma tutte contengono dei domini chiave:
\begin{itemize}
	\item Dominio motore ATPasi come \emph{Dexx} e \emph{HELICc}.
	\item Bromodominio o cromodominio.
	\item Dominio legante actina \emph{HSA}.
	\item Dominio per il legame alla coda istonica \emph{SANT} e \emph{SLIDE}.
\end{itemize}
\subsubsection{Switch/sucrose non-fermentable}
Il complesso \emph{SWI/SNF} facilita l'accesso alla cromatina: fa scivolare ed espelle i nucleosomi per l'attivazione o repressione genica.
\subsubsection{Imitation switch}
Il complesso \emph{ISWI} assembla e spazia i nucleosomi princimpalmente per la repressione della trascrizione.
\subsubsection{Cromodomino elicasi legante il DNA}
Il complesso \emph{CDH} \`e usato per l'assemblaggio dei nucleosomi e la loro spaziazione, per l'accesso ai geni esponendo i promotori e l'editing attraverso l'incorporazione di 
\emph{H3.3}. Aiuta i repressori a legarsi alla cromatina e reprimere i geni attraverso \emph{HDAC} associate.
\subsubsection{Richiedenti inositolo}
Il complesso \emph{INO80} interagisce con \emph{HAT} per attivare la trascrizione. Interviene anche nell'assemblaggio e spaziazione dei nucleosomi oltre a sostituire \emph{H2A} con
\emph{H2A.Z} per la riparazione del DNA.
\section{Variazione nella struttura cromatinica}
I cromosomi subiscono varie fasi di compattazione diversa durante il ciclo cellulare. Durante l'interfase, quando i cromosomi sono relativamente poco condensati, i geni sono 
trascritti e il genoma \`e replicato si trova un gran numero di compattazione lungo il cromosoma. Della trascrizione pu\`o avvenire nelle regioni eterocromatiche, ma la traslocazione
di un gene da una regione eucromatica a una eterocromatica pu\`o prevenire attivamente la sua trascrizione. Il livello di compattamento della cromatina non \`e uniforme e l'epigenetica
rappresenta il suo ultimo livello di regolazione.
\subsection{Eucromatina}
Si dicono eucromatiniche le regioni dove le fibre di \num{30}\si{\nano\metre} formano domini radical loop formando cromatina a \num{300}\si{\nano\metre}. Questa zona \`e 
trascrizionalmente attiva.
\subsection{Eterocromatina}
Nell'eterocromatina i domini radical loop sono ulteriormente compattati attraverso metilazione della coda istonica a formare una cromatina a \num{700}\si{\nano\metre}. L'eterocromatina
si divide in costitutiva, o regioni sempre eterocromatiche permanentemente disattivate rispetto alla trascrizione o silenti e facoltativa, o regioni di cromatina che cambiano stato tra
eucromatina ed eterocromatina. Alcune zone dei cromosomi sono altamente eterocromatiche:
\begin{itemize}
	\item Telomeri: regioni di DNA alla terminazione dei cromosomi.
	\item Peri-centromeri.
	\item Regioni con sequenze di DNA altamente ripetute come l'rDNA nei nucleoli.
\end{itemize}
\subsection{Effetti della cromatina}
La cromatina ha effetto su trascrizione, replicazione, ricombinazione e trasmissione dei cromosomi. Riarrangiamenti che spostano un'origine di replicazione nell'eterocromatina 
causano una replicazione tardiva, arrivando fino a ritardare la divisione cellulare. La ricombinazione coinvolge rotture e riunioni di DNA di diverse molecole. Le regioni eterocromatiche
ne subiscono di meno, proteggendo la regione contro tale modifica, cosa che avviene come nei geni di ripetizione di DNA ribosomiale. I cromosomi devono essere completamente compattati
affinch\`e avvenga la trasmissione e segregazione dei cromosomi. 
\subsection{Nucleolo}
Il nucleolo \`e la parte del nucleo che contiene i geni di rDNA. Gli esseri umani possiedono cinque cluster di rDNA vicino la fine di cinque cromosomi. Si dice regione organizzatrice
dei nucleoli i trascritti di rRNA prodotti dalle ripetizioni dall'rDNA. rDNA codifica per l'RNA ribosomiale e molte cellule possiedono migliaia di ripetizioni di rDNA per riuscire
a soddisfare la richiesta di rRNA e produzione di ribosomi. Il nucleolo non \`e separato da una membrana: sono le proteine e le RNA ad esso specifiche che gli conferiscono diversi
pattern di colorazione. Un sottoinsieme di ripetizioni di rDNA sono silenti trascrizionalmente ed eterocromatiche in modo da aumentare la stabilit\`a delle regioni ripetute. 
\section{Metilazione del DNA}
Il DNA pu\`o essere modificato chimicamente attraverso la metilazione, che avviene in batteri ed eucarioti. I gruppi metile possono essere aggiunti a residui di citosina per 
creare la \emph{5-metil citosina} attraverso DNA metiltransferasi o DNA metilasi. La modifica \`e reversibile grazie alla DNA demetilasi. La metilazione \`e rischiosa in quanto pu\`o 
alterare il DNA permanentemente. Le citosine metilate infatti possono subire una spontanea deamminazione idrolitica che cambia la citosina in timina con cambio mutagenico. 
\subsection{DNA metilasi}
Le DNA metilasi utilizzano un base flipping per accedere alla citosina: una citosina \`e fatta uscire dalla doppia elica: un amminoacido dell'enzima \`e inserito temporaneamente al 
suo posto. La citosina viene poi metilata e reinserita nel DNA. 
\subsection{Effetti della metilazione}
\subsubsection{Nei procarioti}
Nei procarioti la metilazione del DNA distingue il DNA appena sintetizzato nel processo di riparazione: appena dopo la replicazione solo il filamento genitore \`e metilato: questa
regione si dice emi-metilata. Quando gli enzimi di riparazione del mismatch ne trovano uno leggono lo stato metilato per identificare correttamente il filamento parentale e riparare
quello appena sintetizzato. La metilazione permette anche ai batteri di distinguere il DNA genomico da quello virale invadente: enzimi di restrizione tagliano il DNA del fago a 
siti di riconoscimento specifici e durante il taglio il batterio protegge il proprio DNA metilando i siti di restrizione. 
\subsubsection{Negli eucarioti}
La metilazione del DNA negli eucarioti silenzia la trascrizione. \`E pertanto un altra forma di silenziamento epigenetico. Non cambia la carica della base e l'effetto repressivo \`e
indiretto in quanto comporta il reclutamento di proteine lettrici che riconoscono e legano la base metilata. La metilazione avviene tipicamente a siti \emph{CpG} o \emph{CpXpG}, dove
\emph{p} \`e il legame fosfodiestere e \emph{X} una base qualsiasi. Circa il $60\%$ delle \emph{CpG} umane sono metilate. La metilazione pu\`o anche essere ereditata. Alcuni complessi
si legano specificatamente a DNA metilato come enzimi di modifica istonica e complessi di rimodellamento della cromatina. Alcune proteine leganti istoni possono reclutare DNA
metil trasferasi. 
\paragraph{Isole \emph{CpG}}
Le sequenze \emph{CpG} non sono distribuite uniformemente nel genoma ma si trovano in lunghezze di $1$-$2kb$ dove il $60\%$ del contenuto di DNA forma queste isole \emph{CpG}. Sono 
studiate principalmente per la disattivazione del cromosoma $X$, si trovano in tutti i geni housekeeping, principalmente nella zona $5'$ nel promotore. Sono principalmente hypo-metilate,
protette dalla metilazione e si correla con un'alta attivit\`a di trascrizione. \emph{CpG} sono riconosciute da proteine \emph{MBD} (metil-CpG-binding domain) con un dominio di legame
di DNA e di un dominio repressore della trascrizione che possono reclutare complessi di rimodellazione della cromatina che disattivano la trascrizione. La metilazione pu\`o anche
proibire il legame con fattori di trascrizione alle proprie sequenze di riconoscimento del DNA in un processo di mascheramento di $C$. La demetilazione avviene quando un gene deve
essere trascritto. La metilazione di \emph{CpG} \`e ereditata grazie all'enzima DNA metiltrasferasi \emph{DNMT1} che riconosce il sito emimetilato e lo rende completamente metilato.
\subsection{La disattivazione del cromosoma \emph{X} \`e un esempio di silenziamento epigenetico della metilazione di cromatina nei mammiferi}
Un cromosoma $X$ in ogni cellula \`e disattivato nelle femmine in modo che abbiano la stessa quantit\`a di prodotto di gene $X$ come nei maschi che ne possiedono uno solo. Il DNA 
\`e altamente metilato, \emph{H2A} \`e sostituito con \emph{MacroH2A-Z}, gli istoni sono modificati come in eterocromatina ed avviene una regolazione basata su long non-coding RNA. La
disattivazione del cromosoma $X$ \`e casuale ed avviene alla gastrulazione nell'embrione, ognuna delle cellule possono scegliere individualmente quale dei due cromosomi $X$ disattivare
e la scelta viene ereditata. Uno dei due cromosomi si presenter\`a pertanto pi\`u denso, compatto e su un lato del nucleo. Circa il $15\%$ dei geni legati a $X$ non vengono disattivati
completamente e la loro attivit\`a genica varia tra i cromosomi disattivati. La maggior parte di questi si trova nelle regioni pseudoatosomiali \emph{PAR}, dove $X$ e $Y$ si 
accoppiano durante la meiosi. 
\subsubsection{Non corretta disattivazione di $X$ durante la gastrulazione}
\paragraph{Gatti calico}
La colorazione rossa del pelo dei gatti \`e dovuta a un gene nel cromosoma $X$, in cui l'allele rosso sintetizza un enzima che crea il pigmento arancio, mentre un altro non lo esprime
e causa una colorazione nera. Nel caso in cui un gene $X$ non sia disattivato e i maschi presentano $XXY$ presentano una colorazione arancio e nera, oltre ad essere sterili. 
\subsection{Imprinting genetico}
Una parte dell'attivit\`a genetica \`e controllata dall'imprinting genetico, che regola l'espressione di geni materni e paterni nell'embrione, casualmente in alcune cellule \`e 
silenziata la copia materna, in altre quella paterna. Se una delle copie di un gene \`e silenziata e l'altra \`e stata deleta non si trova espressione genica. 
\subsubsection{Disordini fisici e neurologici dovuti alla misregolazione dei geni soggetti a imprinting attraverso metilazione di citosina}
\paragraph{Sindrome di Rett}
Questa sindrome \`e dovuta a una mutazione disattivante in un allele del \emph{MECP2}. Avviene quando \emph{MECP2} in un allele non \`e espresso a causa di metilazione. Uno di questi
geni deve essere sempre espresso per la vitalit\`a.
\paragraph{Sindrome di Prader-Willy e di Angelman}
In queste due sindromi sono colpiti gli stessi alleli del cromosoma $15$. \emph{PWS} avviene quando una regione paterna di $7$ geni \`e eliminata. \emph{AS} avviene quando \`e eliminata
la regione materna. Le sindromi si manifestano quando l'altro allele parentale \`e espresso sub-ottimamente a causa dell'imprinting. Un insieme allelico parentale deve essere intatto 
per la sopravvivenza dell'embrione. 
\section{La separazione dei domini cromatinici da elementi barriera}
\subsection{Variegazione da effetto di posizione}
Un effetto epigenetico \`e la variegazione da effetto di posizione. Un suo esempio \`e il colore dell'occhio di Drosophila in cui si presentano rossi grazie all'espressione del gene 
$white^+$. In alcuni casi gli occhi possono presentare sfaccettature bianche se il gene viene convertito in una regione eterocromatica in qualche cellula in cui risulta silenziato.
\subsection{Elementi di barriera}
Le cellule possiedono elementi di barriera che separano eu ed eterocromatina. Questi elementi possono prevenire la diffusione dell'eterocromatina. In S. pombe due elementi di barriera
affiancano una regione di eterocromatina silente intorno al centromero. Gli \emph{H3} negli elementi di barriera sono altamente metilati a \emph{K9}silenziando la regione, mentre
quelli fuori la barriera sono altamente metilati a \emph{K4} attivando la regione. La rimozione di questi elementi permette la diffusione di metilazione \emph{K9} e della zona 
silenziata. L'eterocromatina pu\`o infatti diffondersi attraverso modifiche di istoni successive come deacetilazione di \emph{H3} la sua metilazione a \emph{K9} e il legame 
della proteina di silenziamento \emph{Swi6}. GLi elementi di barriera agiscono come barriere fisiche e possono essere sequenze specifiche a cui si legano proteine regolatrici delle
modifiche istoniche o grandi loop di cromaina. Elementi di sequenze di barriera possono ancorare gli anelli nella lamina nucleare: l'eterocromatina si trova nelle regioni periferiche
del nucleo in quanto \emph{SAR/MAR} affiancano spesso elementi di sequenza di barriera. 
\section{Elementi richiesti per la funzione dei cromosomi}
\subsection{Origine di replicazione}
Le origini di replicazione sono regioni del DNA con sequenze specifiche richieste per la replicazione in batteri ed eucarioti. Le \emph{Ori} sono dove il dsDNA \`e svolto e separato per 
prepararsi all'attacco delle proteine di replicazione. La replicazione \`e bidirezionale:
\begin{itemize}
	\item Nei batteri si trova un \emph{Ori} per cromosoma e si indica con \emph{ter} gli elementi di terminazione della replicazione.
	\item Negli eucarioti si trovano diverse \emph{Ori} lungo il cromosoma in quanto si ha pi\`u cromosoma da replicare
\end{itemize}
\subsection{Centromeri}
I centromeri si trovano in tutti i cromosomi eucarioti. Sono sequenze che si trovano tipicamente al centro del cromosoma e sono necessarie per la segregazione dei cromosomi durante la
divisione cellulare. La maggior parte delle specie ne possiedono $1$ per cromosoma. Si trova in una regione eterocromatica e dopo la replicazione del DNA $2$ cromatidi fratelli si
formano e sono uniti dai complessi di anelli di coesina. Il centromero appare come una costrizione dovuta all'arricchimento locale di coesina che si trova un quantit\`a minore lungo
l'intero paio di cromatidi. Il centromero ha dimensioni variabili a seconda della specie e si distingue in:
\begin{itemize}
	\item Centromero puntiforme con sequenze definite di poche centinaia di basi.
	\item Centromero regionale con centinaia di kilobasi.
\end{itemize}
Alcuni organismi possiedono molti centromeri lungo il cromosoma e sono detti olocentrici. I microtubuli si attaccano su tutta la lungheza del cromosoma. Durante la mitosi si possono
segregare frammenti di cromosomi.
\subsubsection{Il cinetocore}
Il centromero recluta piu\`u di $100$ proteine che formano il cinetocore che attacca i cromatidi fratelli ai microtubuli che si estendono da poli opposti del fuso, permettendo ad esso
di separare i cromatidi attraverso la depolimerizzazione dei microtubuli. Questo meccanismo di segregazione \`e altamente conservato. 
\subsubsection{Esempi di centromeri}
\paragraph{S. cerevisiae} Il centromero \`e lungo $125bp$ e possiede tre regioni \emph{CDEI}, \emph{CDEII} e \emph{CDEIII}, la prima e la terza possiedono sequenze altamente conservate
e singole mutazioni possono rompere la funzione del centromero. \emph{CDEII} invece \`e una regione ricca di \emph{AT} e la sequenza esatta non \`e fondamentale.
\paragraph{S. pombe} Ogni cromosoma di S. pombe presenta un cromosoma con una sequenza di centromero leggermente diversa con un nucleo unico di $5$-$6kb$ con lunghe sequenze di 
ripetizioni inverse che lo affiancano.
\paragraph{Esseri umani} I centromeri sono lunghi $1Mb$ e sono fatte di sequenze ripetute dette ripetizioni $\alpha$-satellite, lunghe $171bp$ ordinate in ripetizioni di ordine pi\`u 
alto. I nucleosomi centromerici possiedono varianti istoniche di \emph{H3} \emph{CENP-A} particolarmente nelle regioni ricche di \emph{AT} che potrebbe riconoscere gli \emph{i-motivi}
e diadi. Il centromero marcato da \emph{CENP-A} \`e dove il cinetocore si assembla. Una sovraespressione di \emph{CENP-A} causa un legame del cinetocore con tutto il cromosoma e una
sua rottura durante la segregazione. 
\subsection{Telomeri}
I telomeri sono regioni alle terminazioni dei cromosomi lineari e funzionano come cappucci protettivi. Negli esseri umani sono formati da sequenze ripetute centinaia di migliaia di 
volte di \emph{TTAGGGG}, marcano la terminazione del cromosoma definendolo e impedendo la fusione di cromosomi alle loro terminazioni. Infatti pi\`u un cromosoma \`e lungo pi\`u 
\`e propenso a subire rotture. IL DNA dei telomeri consiste di un filamento ricco di $G$ e uno ricco di $C$. La lunghezza totale delle ripetiizoni varia tra i \num{50000} e i 
\num{30000}\si{bp} in base alla specie. La sequenza ricca di $G$ si estende $5'$-$3'$ verso la terminazione del cromosoma dove termina in una regione corta a filamento singolo. 
Negli organismi con telomeri lunghi questa regione pu\`o essere processata in un rolled back T-loop formato dall'invasione e accoppiamento di basi del filamento singolo con la
sequenza a doppio filamento a monte. Le ripetizioni sono un sito di legame per proteine che le marcano come terminazioni naturali distinguendoli dalle rotture del DNA. La DNA 
polimerasi non pu\`o copiare la terminazione di una molecola di DNA e pertanto interviene la telomerasi per mantenere le terminazioni dei cromosomi. 
\subsubsection{Telomerasi}
Le proteine \emph{TRF1} e \emph{TRF2} (\emph{TTAGGGG} repeat binding factor), \emph{TIN2} e \emph{RAP1} si legano ai telomeri e proteggono le loro terminazioni. La terminazione di ogni 
telomero forma un T-loop composto da una ripetizione \emph{TTAGGGG} $3'$ a filamento singolo che lega una sequenza complementare in una sequenza a monte denaturata detta D-loop 
(displacement loop) e viene stabilizzata da copie multiple di \emph{POT1} che vi si lega. La telomerasi \`e una speciale DNA polimerasi che possiede una proteine e una componente 
a RNA: forma un \emph{RNP}. L'RNA della telomerasi fornisce un corto stampo che specifica la sequenza della ripetizione telomerica che deve essere aggiunta. La telomerasi pertanto
sintetizza il DNA telomerico usando l'RNA come stampo. La lunghezza dei telomeri \`e mantenuta nelle cellule staminali e germinali, mentre nei tessuti maturi si trova una telomerasi
insufficiente e avviene un accorciamento dei telomeri. Che limita il numero di divisioni cellulari che la cellula pu\`o avere. Una sovraattivazione della telomaerasi \`e implicata 
in molti cancri e permette alle cellule di continuare a crescere e a dividersi. 

