\chapter{Struttura e funzione dei cromosomi}
\section{Organizzazione dei cromosomi}
L'informazione genetica \`e impacchettata in almeno una molecola di DNA molto lunga, un cromosoma. Ogni cromosoma contiene una molecola di DNA a doppio filamento con molti geni e regioni
di DNA non codificante. Si dicono intergeniche le regioni tra i geni. Se batteri ed archea possiedono crsomosomi circolari gli eucarioti ne possiedono di lineari. La distribuzione dei
geni varia tra gli organismi: quelli meno complessi tendono ad avere geni ordinati pi\`u densamente. La densit\`a genica pu\`o variare anche sui diversi cromosomi degli organismi. Il
numero dei cromosomi \`e caratteristico per una specie. Si possono fare incroci tra specie con numero di cromosomi diverso: in questo caso sono incapaci di accoppiarsi durante la prima 
parte della meiosi e l'incrocio risulta sterile.
\subsection{Ploidia}
Con ploidia si intende quanti cromosomi identici possiede un organismo:
\begin{itemize}
	\item Aploidia: $1$ cromosoma come nel lievito.
	\item Diploidia: $2$ cromosomi come negli umani.
	\item Poliploidia: pi\`u di $2$ cromosomi come nelle piante.
	\item Aneuploidia: un numero anormale di cromosomi, pu\`o avvenire in caso di sindromi genetiche o cancri.
\end{itemize}
La poliploidia viene sfruttata nei prodotti ortofrutticoli per aumentarne le dimensioni.
\subsubsection{Aneuploidia e aborti}
L'aneuploidia pu\`o essere sopportata in un certo numero dagli organismi: si nota per la trisomia del cromosoma $21$ (sindrome di Down) e le poliploidie, ma pu\`o essere mortale e
causare un aborto spontaneo.
\subsubsection{Gametogenesi femminile}
Si nota come con l'aumentare dell'et\`a della donna aumenta il rischio di aneuploidia per i figli. Questo avviene in quanto ogni donna nasce con tutte le uova diploidi gi\`a presenti 
anche se immature. Queste maturano una alla volta dopo la pubert\`a una volta al mese. La continuazione della meiosi bloccata comincia il giorno prima dell'ovulazione a causa dalla
gonadotropina. Un oocita primario pu\`o causare pi\`u errori durante la segregazione cromosomica nelle due fasi della meiosi rispetto a un uovo pi\`u giovane risultando in un uovo 
aploide con pi\`u o meno cromosomi. 
\subsection{Ulteriore DNA presente nelle cellule}
\subsubsection{Cellule eucariote}
Le cellule eucariote possono avere DNA addizionale oltre il DNA cromosomale, in particolare in:
\begin{itemize}
	\item Mitocondri: forniscono le cellule con ATP e sono organelli racchiusi da membrana con il proprio, solitamente circolare, cromosoma singolo.
	\item Cloroplasti: derivano l'energia dalla luce solare nelle piante, possiedono un proprio cromosoma.
\end{itemize}
Si pensa che questi organelli derivino da un batterio ancestrale assorbito e mantenuto da un altro organismo unicellulare.
\subsubsection{Cellule batteriche}
Le cellule batteriche possiedono DNA addizionale nelle proprie cellule: piccolo DNA circolare detto plasmide. Questi tipicamente codificano poche proteine che conferiscono un vantaggio
selettivo come una resistenza ad un antibiotico.
\subsubsection{Virus}
I virus sono agenti infettivi che trasportano informazioni genetiche come piccoli cromosomi a DNA o RNA. I cromosoma virale pu\`o essere lineare o circolare, a doppio o singolo 
filamento.
\section{Impacchettamento del DNA cromosomale}
Si nota come per potersi adattare alle dimensioni del nucleo, delle cellule o di organelli intracellulari il DNA deve essere compattato. Compattare il genoma svolge anche una funzione
di protezione, rendendolo meno accessibile da agenti esterni. 
\subsection{Il nucleoide}
Nei procarioti, in assenza di nucleo il DNA si organizza in un nucleoide. \`E composto per l'$80\%$ di DNA e per il restante $20$ di proteine di compattamento e RNA. Il cromosoma 
\`e pertanto composto da un grande complesso DNA proteine detto cromatina. Il nucleoide appare come una regione che esclude cromosomi, occupa $\frac{1}{3}$ del volume della cellula ed
\`e ancorato all'origine di replicazione nella membrana cellulare. Forma ``loops" o domini di circa $40kb$ grazie alla proteina \emph{HIF} (integration host factor), una piccola
proteina carica positivamente per bilanciare le cariche negative sul backbone. Il DNA \`e successivamente superavvolto da altre proteine che piegano il DNA. L'IHF \`e costituito da
un dimero su cui si forma il loop. Il superavvolgimento \`e controllato da altri fattori come fattori di trascrizione e l'attivit\`a di DNA ed RNA polimerasi che creano due
superavvolgimenti con polarit\`a opposta ai lati della bolla.
\subsection{DNA eucariotico}
Nel nucleo degli eucarioti i cromosomi subiscono cambi visibili durante il ciclo di divisione cellulare. Nelle cellule umane diploidi il DNA deve essere compattato 
\numprint{300000}-\numprint{400000} volte. 
\subsection{Il ciclo cellulare e la dinamicit\`a dei cromosomi}
Durante la fase $G_2$ del ciclo cellulare i cromosomi replicati si trovano in uno stato poco avvolto e si forma il centromero. Nella profase compaiono le fibre del fuso e i cromosomi
si condensano. Nella prometafase le fibre si attaccano ai cromosomi che continuano a condensarsi. Nella metafase i cromosomi si allineano. Nell'anafase i centromeri si dividono e
i cromatidi fratelli si muovono ai poli opposti. Durante la telofase si riforma la membrana nucleare, i cromosomi si decondensano e scompaiono le fibre del fuso. 
\section{Impacchettamento del DNA cromosomale degli eucarioti}
\subsection{Istoni}
Negli eucarioti gli istoni sono proteine leganti il DNA. Si trovano quattro istoni del nucleo: nascono molto presto nell'evoluzione ed essendo cruciali per la sopravvivenza sono 
altamente conservati. Gli istoni sono basici in quanto ricchi in lisina e arginina cariche positivamente grazie all'gruppo ammino \ce{+NH3} che stabilizzano le interazioni tra il DNA
e gli istoni. $146bp$ si arrotolano $1.75$ volte intorno a un complesso istonico in maniera sinistrorsa per formare un nucleosoma. Il complesso istonico prende il nome di ottamero 
istonico. Il superavvolgimento negativo facilita la separazione pi\`u facile, necessaria per la replicazione e la trascrizione. L'ottamero istonico ha due di ognuno dei quattro 
istoni del nucleo: \emph{H2A}, \emph{H2B}, \emph{H3}, \emph{H4}. Inizialmente due dimeri \emph{H3-H4} si associano con il DNA e reclutano poi due dimeri \emph{H2A-H2B} per la formazione
dell'ottamero. Nonostante tutto il DNA eucariote sia impacchettato dagli istoni i nucleosomi si formano preferenzialmente a sequenze di DNA. Il DNA \`e generalmente piegato dolcemente
intorno agli istoni ma presenta curve pi\`u acute ????????? La scanalatura minore deve diventare pi\`u stretta durante il piegamento, cosa pi\`u favorevole in regioni ricche di 
\emph{AT}. Gli istoni fanno $13$ interazioni con gli istoni del DNA nucleosomale: i due dimeri \emph{H3-H4} legano il centro e le terminazioni del DNA mentre \emph{2(H2A-H2B)} legano
$30bp$ su un lato del nucleosoma. Il core istonico \`e composto dai domini di histone-fold composti da tre $\alpha$-eliche. 
\subsubsection{Code istoniche}
Il nucleo di una proteina istonica \`e legato a una lunga coda N-terminale che si estende verso l'esterno. Sono lunghe tra i $20$ e i $39$ amminoacidi e non hanno strutture. 
Interagiscono con altri nucleosomi per aiutare un ulteriore compattamento del DNA e strutture cromatiniche di livello superiore. La coda pu\`o essere modificata chimicamente in modo
da modificare la struttura della cromatina e la sua funzione promuovendo o prevenendo il reclutamento di proteine che regolano la trascrizione. \emph{H2A} e \emph{H2B} presentano
anche code C-terminali che regolano la trascrizione.
\subsubsection{Varianti istoniche}
Le varianti istoniche sono alrte proteine con staibilit\`a diverse, domini specialisti che cambiano la funzione del cromosoma, sequenze diverse alle terminazioni. Hanno amminoacidi
diversi che possono essere diversamente modificati. Le varianti istoniche sono depositate da complessi di rimodellamento della cromatina dipendenti da ATP e possono essere 
dipendenti o indipendenti dalla replicazione. 
\subsubsection{Interazioni con il DNA}
Gli istoni interagiscono con il DNA attraverso interazioni elettrostatiche tra i gruppi fosfato del legame fosfodiestere e gli amminoacidi basici negli istoni e attraverso legami a
idrogeno tra l'atomo di ossigeno nei gruppi fosfato e gli atomi di idrogeno nei gruppi ammino degli istoni. Modifiche chimiche delle basi o code istoniche attraverso enzimi 
modificano le cariche locali e le interazioni. 
\subsection{Livelli di compattazione}
L'impacchettamento cromatinico ha diversi livelli di compattamento. 
\subsubsection{Primo livello}
Il primo livello di compattazione \`e la fibra di $10\si{\nano\metre}$, che nasce dall'associazione con il DNA dei
nucleosomi con apparenza di perline su un filo. 
\subsubsection{Secondo livello}
La fibra \`e ulteriomente compattata da una quinta proteina istonica \emph{H1} in una fibra di \num{30}\si{nm} nel secondo livello di 
compattamento, un ordinamento regolare che avvicina i nucleosomi. \emph{H1} si lega al DNA linker tra due nucleosomi successivi diminuendo la lunghezza di \num{7} volte. Anche le
code istoniche sono coinvolte nella formazione di questo secondo livello. 
\subsubsection{Terzo livello}
Il terzo livello di compattamento, con diametro di \num{300}\si{nm} si forma grazie a domini di loop radiali e al legame con la matrice nucleare nelle cellule in interfase. La
matrice nucleare \`e composta dalla lamina nucleare composta da fibre proteiche della matrice interna e da proteine che legano ad essa i cromosomi. Le proteine attaccano la
base di un loop di DNA alla fibra proteica grazie a sequenze specifiche \emph{MAR} (matrix-attachment region) e \emph{SAR} (scaffold attachment region). Si nota come ogni cromosoma
occupa nel nucleo un territorio determinato
\subsubsection{Quarto livello}
I loop radiali diventano altamente compattati e rimangono ancorati alla matrice nucleare. Mentre la cellula entra la profase la membrana nucleare si dissolve e non si trova pi\`u una
matrice nucleare: la compattazione aumenta drammaticamente nel quarto livello di compattazione o condensazione. Alla fine della profase i cromosomi sono interamente eterocromatici
con un diametro di \num{700}\si{nm}. Pertanto i cromosomi in metafase subiscono poca trascrizione e unicamente nel centromero. In questo momento i cromosomi hanno accesso al fuso
mitotico. 
\subsubsection{Quinto livello}
Il quinto livello di compattamento avviene con la formazione dei cromosomi visibili e grazie alla condensina. La condensina \`e una proteina che si sposta nel nucleo durante l'inizio 
della fase $M$, si lega ai cromosomi e compatta i loop radiali riducendo il loro diametro. Un'altra proteina coinvolta \`e la coesina caricata durante la fase $S$ per tenere uniti i 
cromatidi fratelli. 
\section{Modifiche covalenti degli istoni}
\subsection{Epigenetica}
Si intende per epigenetica l'ereditariet\`a di fenotipi non causati da cambi nella sequenza del DNA. \`E un fenomeno principalmente eucariote ed \`e causata da cambi strutturali
nella composizione dei nucleosomi (varianti istoniche), modifiche chimiche della coda o nucleo istonico che altera lo stato di compattazione della cromatina e l'attivit\`a del nucleo
del nucleosoma, metilazione del DNA alla citosina e dal legame di DNA o RNA con RNA non codificanti. Queste opzioni alterano l'espressione genetica. Cambi epigenetici sono 
trasferiti da cellula madre e figlia durante la replicazione del DNA e un numero di sindromi e cancri sono dovuti alla mal-regolazione di attivit\`a epigenetiche. Nei batteri la
trascrizione dipende principalmente dall'RNA polimerasi e la sua regolazione allo stadio di iniziazione. Metilazione di adenosina e citosina intervengono nell'espressione genica, 
nella replicazione e riparazione del DNA e come difesa contro attacchi virali. 
\section{Complessi rimodellatori dei nucleosomi}

\section{Variazione nella struttura cromatinica}

\section{Metilazione del DNA}

\section{La separazione dei domini cromatinici da elementi barriera}

\section{Elementi richiesti per la funzione dei cromosomi}

\section{Centromeri}

\section{Telomeri}
