\chapter{Strumenti e tecniche della biologia molecolare}

\section{Separazione di molecole biologiche}

	\subsection{Separazione per elettroforesi su gel}
	La separazione di molecole biologiche avviene attraverso elettroforesi su Gel.
	Si possono usare due matrici sia per DNA, RNA che proteine, pur essendo queste interscambiabili.

		\subsubsection{Tipologie di gel}
		\begin{multicols}{2}
			\begin{itemize}
				\item Agarosio: formato da \emph{D-galattosio} e \emph{3,6-anidro-L-galattosio}, forma una struttura a maglie in gel.
					Un aumento di concentrazione causa una riduzione della dimensione dei pori.
				\item \emph{PAGE}: polyacrylamide gel electrophoresis.
					Viene tipicamente utilizzato per le proteine.
			\end{itemize}
		\end{multicols}

	\subsection{Processo}

		\subsubsection{Preparazione del gel}
		L'agarosio, polisaccaride di agarobioso forma una trama a maglie.
		Durante la preparazione:
		\begin{multicols}{2}
			\begin{itemize}
				\item Si pesa l'agarosio.
				\item Lo si sospende in buffer salini.
				\item Si porta a bollore per solubilizzare.
				\item Si pone il gel in una cassetta di plastica.
				\item Si aggiungono dei pettinini per formare i pozzetti dove verranno messi i campioni.
			\end{itemize}
		\end{multicols}

		\subsubsection{Separazione delle molecole}
		Dopo l'aggiunta dei campioni nei pozzetti si applica al gel una carica elettrica che porta alla formazione di due poli, uno positivo e uno negativo.
		RNA e DNA sono carichi negativamente a causa del fosfato e tenderanno a spostarsi verso l'elettrodo positivo.
		Il gel rallenta il movimento e permette una dimensione secondo grandezza:
		\begin{multicols}{2}
			\begin{itemize}
				\item Molecole pi\`u grandi si troveranno in posizione pi\`u alta in quanto si muovono pi\`u lentamente.
				\item Molecole pi\`u piccole si troveranno in posizione pi\`u bassa in quanto si muovono pi\`u lentamente.
			\end{itemize}
		\end{multicols}
		
	\subsection{Colorazione delle molecole}
	Gli acidi nucleici vengono colorati con un fluorescente come l'etidio bromuro o \emph{SYBR green}.
	Questa molecola \`e idrofobica e si intercala nel DNA inducendo mutazioni frameshift.
	Quando intercalata aumenta la fluorescenza di $20$ volte quando illuminata da raggi \emph{UV} grazie ai propri gruppi aromatici.
	Quando si intercala diminuisce il grido della molecola allungandola e inducendo un superavvolgimento negativo.

	\subsection{Stima della taglia del DNA}
	I frammenti di DNA sono comparati con degli standard di DNA detti markers, ovvero sequenze di DNA di una lunghezza prestabilita che determinano una scala.

	\subsection{\emph{PFGE}}
	\emph{PFGE} o elettroforesi a campo pulsato pu\`o risolvere grandi frammenti da $200kb$ fino a $6000kb$.
	Viene utilizzata per grandi DNA come cromosomi.
	Il campo elettrico \`e pulsato e proviene da tre direzioni diverse.
	Queste si dispongono a forma esagonale con $6$ elettrodi che lavorano a coppie opposte.
	La corsa del DNA \`e pertanto a zig-zag, ma il risultato \`e comunque una separazione verticale.
	Questo avviene in quanto il tempo di pulsazione di ogni coppia di elettrodi \`e equivalente.

	\subsection{Sodium dodcyl polyacrylamide gel electrophoresis}
	Sodium dodcyl polyacrylamide gel electrophoresis o \emph{SDS-PAGE} utilizza \emph{SDS}, un detergente che si lega alle zone idrofobiche delle proteine denaturandole.
	La proteina in questo modo torna ad avere una struttura primaria con cariche negative uniformi.
	Questo permette uno spostamento delle proteine dovuto unicamente alla loro massa.
	La denaturazione avviene grazie anche a un agente riducente, il \emph{$\beta$-marcaptoetanolo}, che rompe i legami disolfuro quando riscaldato a $95\si{\celsius}$ per $5min$.
	Vengono utilizzati marker come scala e coloranti come nitrato d'argento e coomassie blue.

	\subsection{Concentrazione di gel}
	La concentrazione del gel viene determinata in base alla grandezza dei frammenti che si vogliono separare: maggiore la concentrazione pi\`u grandi i frammenti.

\section{Amplificazione di sequenze di RNA e DNA}

	\subsection{Panoramica}
	L'amplificazione delle sequenze di DNA o RNA specifici si rende necessaria per l'esaminazione o per una manipolazione ulteriore di date molecole.

	\subsection{Polymerase chain reaction}
	La polymerase chain reaction o \emph{PCR} viene usata per amplificare DNA di interesse attraverso DNA polimerasi stabili e attive ad alte temperature ricavate da \emph{Thermus acquaticus}.
	Coinvolge $3$ passaggi principali ripetuti tra le $20$ e le $40$ volte.
	Il numero di molecole raddoppia ad ogni passaggio.
	\begin{multicols}{2}
		\begin{enumerate}
			\item Denaturazione: del DNA a $95\si{\celsius}$.
			\item Annealing dei primers: a $50$-$60\si{\celsius}$.
			\item Sintesi: del nuovo DNA $72\si{\celsius}$ per $20$ secondi per ogni $kb$.
		\end{enumerate}
	\end{multicols}
	I termociclatori automatizzano il processo.

	\subsection{Temperatura di annealing}
	Per calcolare la temperatura di annealing $T_{ann}$ si deve considerare la temperatura di melting $T_m$:
	\[T_{ann} = T_{m} - 2\si{\celsius}\]
	La temperatura di melting si calcola come:
	\[T_m = [4\si{\celsius} \cdot (n_G + n_C)] + [2\si{\celsius} \cdot (n_T + n_A)]\]

	\subsection{\emph{PCR} mutagenica}
	La \emph{PCR} nonostante sia principalmente utilizzata per amplificare una sequenza di DNA fedelmente pu\`o anche essere usata per introdurre mutazioni o aggiungere sequenze alla fine delle molecole.
	Questo pu\`o per esempio essere fatto alle terminazioni dei primer.
	Queste presenteranno uno stampo per i cicli successivi al primo.

	\subsection{Amplificazione basata su \emph{PCR} di RNA attraverso \emph{cDNA}}
	Il RNA non pu\`o essere clonato direttamente, pertanto un DNA complementare deve essere sintetizzato attraverso una trascrittasi inversa.
	In quanto le terminazioni di una molecola di RNA potrebbero essere sconosciute a causa di splicing alternativo si utilizza \emph{RACE} o rapid amplification of cDNA ends.

		\subsubsection{Rapid amplification of cDNA ends}
		Rapid amplification of cDNA ends o \emph{RACE} \`e una procedura di \emph{PCR} modificata che permette l'identificazione di terminazione di molecole di RNA
		Nella \emph{RACE $3'$} un primer con una regione di poli-T e una sequenza a scelta viene annealed alla coda di mRNA poli-A.
		La trascrittasi inversa crea cDNA e il secondo filamento a DNA viene sintetizzato estendendo da un primer interno.
		Il primer interno e il primo possono essere usati successivamente in una \emph{PCR} standard.

\section{Clonaggio di DNA}
	
	\subsection{Panoramica}
	\`E oggi possibile clonare geni.
	La maggior parte delle tecniche richiede che il DNA sia isolato e modificato.

		\subsubsection{Vettori di DNA}

			\paragraph{Plasmidi}
			Introducendo un frammento di DNA in un plasmide molte copie del frammento possono essere generate.
			I plasmidi contengono tipicamente un marcatore selezionabile che permette la crescita delle sole cellule che lo hanno integrato.
			Possono integrare frammenti di DNA fino a $15kb$.

			\paragraph{Bacterial artificial chromosome}
			I bacterial artificial chromosome \emph{BAC} possono possedere inserzioni fino a centinaia di chilobasi.

			\paragraph{Shuttle vector}
			I shuttle vectors sono vettori capaci di crescere in diversi organismi.
			Necessitano di un origine di replicazione per ogni organismo in cui crescono e possiedono un marcatore diverso per ogni organismo

			\paragraph{Marcatori per la selezione}
			I marcatori per la selezione sono tipicamente geni di resistenza batterica.
	
		\subsubsection{Utilizzi}
		I cloni di DNA possono essere utilizzato per complementare una mutazione o per aggiungere tag a proteine in modo che possano essere purificate o visualizzate all'interno di cellule.

	\subsection{Isolamento dei plasmidi}
	\begin{multicols}{2}
		\begin{enumerate}
			\item Sospensione delle cellule.
			\item Lisi delle cellule.
			\item Neutralizzazione del lisato.
			\item Legame del DNA dal surnatante a una matrice.
			\item Lavaggio della matrice.
			\item Diluizione del DNA.
		\end{enumerate}
	\end{multicols}

	\subsection{Enzimi di restrizione}
	Fondamentali per un clonaggio ottimale sono gli enzimi di restrizione.
	Questi agiscono su sequenze palindromiche e sono derivati dai batteri che li usano come meccanismo di difesa contro i batteriofagi.
	Sono distinti in base al taglio che operano:
	\begin{multicols}{2}
		\begin{itemize}
			\item Blunt \emph{Hpal}.
			\item $3'$ overhangs \emph{Kpnl}.
			\item $5'$ overhangs \emph{EcoR1}, \emph{HINDIII}
		\end{itemize}
	\end{multicols}
	Gli ultimi due formano delle estremit\`a sticky coesive.

	\subsection{Strategie per il clonaggio genico}

		\subsubsection{Ligazione}
		Nella ligazione il frammento e il vettore sono tagliati attraverso enzimi di restrizione in modo che le terminazioni delle molecole abbiano terminazioni complementari.
		Quando i due frammenti sono incubati insieme le terminazioni complementari formano accoppiamenti da basi e le due molecole si uniscono.
		Queste possono essere unite covalentemente da DNA ligasi.
		Questo viene poi introdotto nei plasmidi per l'amplificazione attraverso trasformazione.

			\paragraph{Trasformazione}
			L'organismo pi\`u utilizzato per il clonaggio \`e E. coli.

		\subsubsection{Gibson cloning}
		Nel processo di Gibson cloning diversi frammenti possono essere clonati simultaneamente.
		I frammenti sono progettati in modo che possiedano terminazioni che corrispondono le terminazioni degli altri frammenti e del vettore.
		Sono mischiati con il vettore per permettere l'anneal e i gap sono riempiti da DNA polimerasi e i frammenti sono ligati insieme.

		\subsubsection{Ricombinazione omologa}
		I geni possono essere inseriti nei vettori attraverso ricombinazione omologa.
		I siti di ricombinazione sono inseriti al prodotti di PCR nello stesso modo aggiungendo siti di restrizione.
		La ricombinazione omologa avviene tra due regioni di DNA omologhe.
		Diverse sequenze alle due terminazioni dell'inserzione controllano la direzione dell'inserimento del frammento.
		Gli enzimi di ricombinasi sono necessari per la reazione in vitro.

	\subsection{Mutagenesi sito-diretta}
	La mutagenesi sito-diretta viene usata per alterare la sequenza di DNA in posizioni specifiche per creare proteine mutanti e studiare la loro funzione.
	Si effettua \emph{PCR} dove vengono creati i primers e vengono usate polimerasi ad alta fedelt\`a.
	Le metilazioni del DNA vengono rimosse dall'enzima \emph{DpnI}.
	La molecola viene amplificata e usata per trasformazione.

	\subsection{Libreria genica}
	Si intende per libreria genica una collezione di geni clonati in un plasmide.
	Vengono usate per identificare una mutazione.
	Si trasforma un mutante con la libreria in modo che ogni individuo possieda un vettore e gli organismi con la mutazione soppressa conterranno il vettore desiderato.
	Questo clone pu\`o essere isolato dalle cellule trasformate.
	Le librerie possono inoltre essere costruite da mRNA espressi.

\section{Manipolazione genomica}


\section{Identificazione di specifiche molecole di DNA}

\section{Identificazione di specifiche molecole di RNA}

\section{Identificazione di specifiche proteine}

\section{Identificazione di interazioni tra molecole}

\section{Sequenziamento di molecole di DNA e RNA attraverso nanopori}
