\chapter{Strumenti e tecniche della biologia molecolare}

\section{Separazione di molecole biologiche}

	\subsection{Separazione per elettroforesi su gel}
	La separazione di molecole biologiche avviene attraverso elettroforesi su Gel.
	Si possono usare due matrici sia per DNA, RNA che proteine, pur essendo queste interscambiabili.

		\subsubsection{Tipologie di gel}
		\begin{multicols}{2}
			\begin{itemize}
				\item Agarosio: formato da \emph{D-galattosio} e \emph{3,6-anidro-L-galattosio}, forma una struttura a maglie in gel.
					Un aumento di concentrazione causa una riduzione della dimensione dei pori.
				\item \emph{PAGE}: polyacrylamide gel electrophoresis.
					Viene tipicamente utilizzato per le proteine.
			\end{itemize}
		\end{multicols}

	\subsection{Processo}

		\subsubsection{Preparazione del gel}
		L'agarosio, polisaccaride di agarobioso forma una trama a maglie.
		Durante la preparazione:
		\begin{multicols}{2}
			\begin{itemize}
				\item Si pesa l'agarosio.
				\item Lo si sospende in buffer salini.
				\item Si porta a bollore per solubilizzare.
				\item Si pone il gel in una cassetta di plastica.
				\item Si aggiungono dei pettinini per formare i pozzetti dove verranno messi i campioni.
			\end{itemize}
		\end{multicols}

		\subsubsection{Separazione delle molecole}
		Dopo l'aggiunta dei campioni nei pozzetti si applica al gel una carica elettrica che porta alla formazione di due poli, uno positivo e uno negativo.
		RNA e DNA sono carichi negativamente a causa del fosfato e tenderanno a spostarsi verso l'elettrodo positivo.
		Il gel rallenta il movimento e permette una dimensione secondo grandezza:
		\begin{multicols}{2}
			\begin{itemize}
				\item Molecole pi\`u grandi si troveranno in posizione pi\`u alta in quanto si muovono pi\`u lentamente.
				\item Molecole pi\`u piccole si troveranno in posizione pi\`u bassa in quanto si muovono pi\`u lentamente.
			\end{itemize}
		\end{multicols}
		
	\subsection{Colorazione delle molecole}
	Gli acidi nucleici vengono colorati con un fluorescente come l'etidio bromuro o \emph{SYBR green}.
	Questa molecola \`e idrofobica e si intercala nel DNA inducendo mutazioni frameshift.
	Quando intercalata aumenta la fluorescenza di $20$ volte quando illuminata da raggi \emph{UV} grazie ai propri gruppi aromatici.
	Quando si intercala diminuisce il grido della molecola allungandola e inducendo un superavvolgimento negativo.

	\subsection{Stima della taglia del DNA}
	I frammenti di DNA sono comparati con degli standard di DNA detti markers, ovvero sequenze di DNA di una lunghezza prestabilita che determinano una scala.

	\subsection{\emph{PFGE}}
	\emph{PFGE} o elettroforesi a campo pulsato pu\`o risolvere grandi frammenti da $200kb$ fino a $6000kb$.
	Viene utilizzata per grandi DNA come cromosomi.
	Il campo elettrico \`e pulsato e proviene da tre direzioni diverse.
	Queste si dispongono a forma esagonale con $6$ elettrodi che lavorano a coppie opposte.
	La corsa del DNA \`e pertanto a zig-zag, ma il risultato \`e comunque una separazione verticale.
	Questo avviene in quanto il tempo di pulsazione di ogni coppia di elettrodi \`e equivalente.

	\subsection{Sodium dodcyl polyacrylamide gel electrophoresis}
	Sodium dodcyl polyacrylamide gel electrophoresis o \emph{SDS-PAGE} utilizza \emph{SDS}, un detergente che si lega alle zone idrofobiche delle proteine denaturandole.
	La proteina in questo modo torna ad avere una struttura primaria con cariche negative uniformi.
	Questo permette uno spostamento delle proteine dovuto unicamente alla loro massa.
	La denaturazione avviene grazie anche a un agente riducente, il \emph{$\beta$-marcaptoetanolo}, che rompe i legami disolfuro quando riscaldato a $95\si{\celsius}$ per $5min$.
	Vengono utilizzati marker come scala e coloranti come nitrato d'argento e coomassie blue.

	\subsection{Concentrazione di gel}
	La concentrazione del gel viene determinata in base alla grandezza dei frammenti che si vogliono separare: maggiore la concentrazione pi\`u grandi i frammenti.

\section{Amplificazione di sequenze di RNA e DNA}

	\subsection{Panoramica}
	L'amplificazione delle sequenze di DNA o RNA specifici si rende necessaria per l'esaminazione o per una manipolazione ulteriore di date molecole.

	\subsection{Polymerase chain reaction}
	La polymerase chain reaction o \emph{PCR} viene usata per amplificare DNA di interesse attraverso DNA polimerasi stabili e attive ad alte temperature ricavate da \emph{Thermus acquaticus}.
	Coinvolge $3$ passaggi principali ripetuti tra le $20$ e le $40$ volte.
	Il numero di molecole raddoppia ad ogni passaggio.
	\begin{multicols}{2}
		\begin{enumerate}
			\item Denaturazione: del DNA a $95\si{\celsius}$.
			\item Annealing dei primers: a $50$-$60\si{\celsius}$.
			\item Sintesi: del nuovo DNA $72\si{\celsius}$ per $20$ secondi per ogni $kb$.
		\end{enumerate}
	\end{multicols}
	I termociclatori automatizzano il processo.

	\subsection{Temperatura di annealing}
	Per calcolare la temperatura di annealing $T_{ann}$ si deve considerare la temperatura di melting $T_m$:
	\[T_{ann} = T_{m} - 2\si{\celsius}\]
	La temperatura di melting si calcola come:
	\[T_m = [4\si{\celsius} \cdot (n_G + n_C)] + [2\si{\celsius} \cdot (n_T + n_A)]\]

	\subsection{\emph{PCR} mutagenica}
	La \emph{PCR} nonostante sia principalmente utilizzata per amplificare una sequenza di DNA fedelmente pu\`o anche essere usata per introdurre mutazioni o aggiungere sequenze alla fine delle molecole.
	Questo pu\`o per esempio essere fatto alle terminazioni dei primer.
	Queste presenteranno uno stampo per i cicli successivi al primo.

	\subsection{Amplificazione basata su \emph{PCR} di RNA attraverso \emph{cDNA}}
	Il RNA non pu\`o essere clonato direttamente, pertanto un DNA complementare deve essere sintetizzato attraverso una trascrittasi inversa.
	In quanto le terminazioni di una molecola di RNA potrebbero essere sconosciute a causa di splicing alternativo si utilizza \emph{RACE} o rapid amplification of cDNA ends.

		\subsubsection{Rapid amplification of cDNA ends}
		Rapid amplification of cDNA ends o \emph{RACE} \`e una procedura di \emph{PCR} modificata che permette l'identificazione di terminazione di molecole di RNA
		Nella \emph{RACE $3'$} un primer con una regione di poli-T e una sequenza a scelta viene annealed alla coda di mRNA poli-A.
		La trascrittasi inversa crea cDNA e il secondo filamento a DNA viene sintetizzato estendendo da un primer interno.
		Il primer interno e il primo possono essere usati successivamente in una \emph{PCR} standard.

\section{Clonaggio di DNA}
	
	\subsection{Panoramica}
	\`E oggi possibile clonare geni.
	La maggior parte delle tecniche richiede che il DNA sia isolato e modificato.

		\subsubsection{Vettori di DNA}

			\paragraph{Plasmidi}
			Introducendo un frammento di DNA in un plasmide molte copie del frammento possono essere generate.
			I plasmidi contengono tipicamente un marcatore selezionabile che permette la crescita delle sole cellule che lo hanno integrato.
			Possono integrare frammenti di DNA fino a $15kb$.

			\paragraph{Bacterial artificial chromosome}
			I bacterial artificial chromosome \emph{BAC} possono possedere inserzioni fino a centinaia di chilobasi.

			\paragraph{Shuttle vector}
			I shuttle vectors sono vettori capaci di crescere in diversi organismi.
			Necessitano di un origine di replicazione per ogni organismo in cui crescono e possiedono un marcatore diverso per ogni organismo

			\paragraph{Marcatori per la selezione}
			I marcatori per la selezione sono tipicamente geni di resistenza batterica.
	
		\subsubsection{Utilizzi}
		I cloni di DNA possono essere utilizzato per complementare una mutazione o per aggiungere tag a proteine in modo che possano essere purificate o visualizzate all'interno di cellule.

	\subsection{Isolamento dei plasmidi}
	\begin{multicols}{2}
		\begin{enumerate}
			\item Sospensione delle cellule.
			\item Lisi delle cellule.
			\item Neutralizzazione del lisato.
			\item Legame del DNA dal surnatante a una matrice.
			\item Lavaggio della matrice.
			\item Diluizione del DNA.
		\end{enumerate}
	\end{multicols}

	\subsection{Enzimi di restrizione}
	Fondamentali per un clonaggio ottimale sono gli enzimi di restrizione.
	Questi agiscono su sequenze palindromiche e sono derivati dai batteri che li usano come meccanismo di difesa contro i batteriofagi.
	Sono distinti in base al taglio che operano:
	\begin{multicols}{2}
		\begin{itemize}
			\item Blunt \emph{Hpal}.
			\item $3'$ overhangs \emph{Kpnl}.
			\item $5'$ overhangs \emph{EcoR1}, \emph{HINDIII}
		\end{itemize}
	\end{multicols}
	Gli ultimi due formano delle estremit\`a sticky coesive.

	\subsection{Strategie per il clonaggio genico}

		\subsubsection{Ligazione}
		Nella ligazione il frammento e il vettore sono tagliati attraverso enzimi di restrizione in modo che le terminazioni delle molecole abbiano terminazioni complementari.
		Quando i due frammenti sono incubati insieme le terminazioni complementari formano accoppiamenti da basi e le due molecole si uniscono.
		Queste possono essere unite covalentemente da DNA ligasi.
		Questo viene poi introdotto nei plasmidi per l'amplificazione attraverso trasformazione.

			\paragraph{Trasformazione}
			L'organismo pi\`u utilizzato per il clonaggio \`e E. coli.

		\subsubsection{Gibson cloning}
		Nel processo di Gibson cloning diversi frammenti possono essere clonati simultaneamente.
		I frammenti sono progettati in modo che possiedano terminazioni che corrispondono le terminazioni degli altri frammenti e del vettore.
		Sono mischiati con il vettore per permettere l'anneal e i gap sono riempiti da DNA polimerasi e i frammenti sono ligati insieme.

		\subsubsection{Ricombinazione omologa}
		I geni possono essere inseriti nei vettori attraverso ricombinazione omologa.
		I siti di ricombinazione sono inseriti al prodotti di PCR nello stesso modo aggiungendo siti di restrizione.
		La ricombinazione omologa avviene tra due regioni di DNA omologhe.
		Diverse sequenze alle due terminazioni dell'inserzione controllano la direzione dell'inserimento del frammento.
		Gli enzimi di ricombinasi sono necessari per la reazione in vitro.

	\subsection{Mutagenesi sito-diretta}
	La mutagenesi sito-diretta viene usata per alterare la sequenza di DNA in posizioni specifiche per creare proteine mutanti e studiare la loro funzione.
	Si effettua \emph{PCR} dove vengono creati i primers e vengono usate polimerasi ad alta fedelt\`a.
	Le metilazioni del DNA vengono rimosse dall'enzima \emph{DpnI}.
	La molecola viene amplificata e usata per trasformazione.

	\subsection{Libreria genica}
	Si intende per libreria genica una collezione di geni clonati in un plasmide.
	Vengono usate per identificare una mutazione.
	Si trasforma un mutante con la libreria in modo che ogni individuo possieda un vettore e gli organismi con la mutazione soppressa conterranno il vettore desiderato.
	Questo clone pu\`o essere isolato dalle cellule trasformate.
	Le librerie possono inoltre essere costruite da mRNA espressi.

\section{Manipolazione genomica}

	\subsection{Panoramica}

		\subsubsection{Forward genetics}
		Nella forward genetics un mutante viene identificato grazie a un difetto in un certo processo.
		Il gene mutante identificato permette di comprendere meglio il processo e pathway.

		\subsubsection{Reverse genetics}
		Nella reverse genetics un gene viene mutato e si esamina la sua funzione.
		Mutazioni casuali creano una via molto comune molto usata per esaminare il genoma.
		Le mutazioni possono essere indotte con metodi mutageni.
		La ricombinazione omologa permette di creare mutazioni su un gene specifico.

	\subsection{Inserimento di un trasposone}
	L'inserimento di un trasposone avviene casualmente nei genomi e i geni possono essere rotti se un trasposone interrompe la loro sequenza.
	Un componente contiene il gene della trasosasi controllato da un promotore, mentre l'altro elemento un marcatore selezionabile.
	Dopo la trasposizione la sequenza che circonda il trasposone pu\`o essere isolata e usata per indentificare la sua locazione nel genoma.

	\subsection{Genomi delle piante}
	I genomi delle piante sono stati mutati estensivamente utilizzando inserti a $T$-DNA.
	Le linee di \emph{Arabidospis} sono disponibili con inserti in quasi tutti i geni.
	I plasmidi \emph{Ti} induttori del tumore da \emph{Agrobacterium tumefaciens} contiene un origine di replicazione con il T-DNA affiancato da sequenze di inserimento e i fattori di virulenza codificano un enzima di trasfoemrinto.
	In natura la sezione $T$ codifica ormoni delle piante che inducono una divisione cellulare incontrollata che causa tumori crown gall.
	Il T-DNA pu\`o essere sostituito da DNA di interesse.

		\subsubsection{Mutazioni}
		Le mutazioni avvengono quando \emph{Agrobacterium} inserisce il plasmide \emph{Ti} attraverso ricombinazione non omologa illegittima nel genoma della pianta.

	\subsection{Ricombinazione omologa}
	La ricombianzione omologa \`e utilizzata per creare cambi specifici.
	Un costrutto lineare viene inserito in un organismo e le regioni omologhe si allineano  ricombinano in modo che il DNA dell'ospite viene sostituito dal DNA costruito.
	La sequenza originale viene degradata.
	Addizioni o delezioni possono essere ottenute se le sequenze di DNA omologhe differiscono.
	Viene anche usata per aggiungere tag a geni.

		\subsubsection{Aumentare l'efficienza}
		La bassa efficienza di ricombinazione omologa pu\`o essere aumentata inducendo una rottura a doppio filamento nel locus di interesse utilizzando sistemi basati su \emph{CRISPR} batterica.

			\paragraph{Clustered regulatory interspaced short palindromic sequences}
			Clustered regulatory interspaced short palindromic sequences o \emph{Crispr} utilizza piccole lunghezze di DNA obiettivo e RNA non codificanti per dirigere l'attivit\`a della nucleasi \emph{Cas9} verso il luogo desiderato.

\section{Identificare la composizione di molecole biologiche}

	\subsection{Sequenziamento del DNA}
		
		\subsubsection{Sequenziamento chain-terminated dideoxy}
		Nel sequenziamento chain-terminated dideoxy un singolo primer viene annealed alla sequenza di interesse e il primer viene esteso con la DNA polimerasi $I$
		La soluzione di reazione contiene tutti e $4$ i deossi nucleotidi e una piccola quantit\`a di dideossi nucleotidi ognuno etichettato con una diversa fluorescenza.
		L'estensione del polinucleotide si ferma se un nucleotide dideossi viene aggiunto alla catena in quanto non possiede un gruppo \emph{$3'$-OH} disponibile per la reazione.
		La reazione genera un insieme di molecole a diverse grandezze ognuna terminante con un dideossinucleotide.

			\paragraph{Lettura dei risultati}
			Originariamente i frammenti etichettati radioattivamente venivano fatti correre su un gel \emph{PAGE} molto lungo e letti ad occhio.
			Metodi odierni utilizzano una macchina sequenziatrice che separa i polinucleotidi attraverso elettroforesi su \emph{PAGE} e misura il colore di ogni dimensione del frammento con un laser.
			I risultati accurati fino a $500bp$ a causa di esaustione di nucleotidi sono mostrati sul cromatogramma.

		\subsubsection{Ottenere la sequenza genomica completa}
		La sequenza pu\`o essere ottenuta assemblando frammenti sequenziati.
		\begin{multicols}{2}
			\begin{enumerate}
				\item Il DNA genomico viene rotto in piccoli frammenti di restrizione a cui sono attaccati linkers.
				\item I frammenti sono amplificati secondo la sequenza linker.
				\item I frammenti sono sequenziati.
				\item Computazionalmente le sequenze linker sono rimosse dai dati di sequenza e le regioni sovrapposte sono usate per determinare l'ordine e l'assemblaggio delle sequenze di frammenti.
			\end{enumerate}
		\end{multicols}
		Viene richiesto sovra-sequenziamento a causa della sovrapposizione dei frammenti e per assicurare copertura completa ed accuratezza.

			\paragraph{Problematiche}
			\begin{multicols}{2}
				\begin{itemize}
					\item Un sequenziamento $8\times$ \`e richiesto per ottenere un'accuratezza del $99.9\%$.
					\item Le regioni ripetitive come centromeri e telomeri sono difficilmente allineabili ridotte crando \emph{YACs}.
					\item La velocit\`a viene aumentata da macchine di sequenziamento moderne.
				\end{itemize}
			\end{multicols}

		\subsubsection{Next-generation sequencing}

			\paragraph{Illumina sequencing}
			Durante l'illumina sequencing il DNA genomico viene frammentato, vengono legati adattatori e il DNA \`e denaturato.
			I frammenti sono aggiunti a una superficie con primer complementari agli adattatori legati.
			Il DNA viene amplificato da un bridge di dsDNA, viene denaturato per creare cluster di molecole amplificate identiche.
			Le molecole in ogni cluster sono sequenziate simultaneamente.
			Un primer specifico e nucleotidi etichettati fluorescentemente vengono utilizzati per la sintesi.
			Dopo ogni addizione il nucleotide aggiunto viene determinato dalla sua fluorescenza.
	
	\subsection{Sequenziamento delle proteine}

		\subsubsection{Degradazione di Edman}
		La degradazione di Edman \`e una degradazione a passi dalla terminazione $N$.
		IL fenilisotiocianato \`e utilizzato per etichettare i residui $N$ terminali in condizioni debolmente alcaline.
		Il risultante feniltioidantoin \emph{PHT}-derivato induce instabilit\`a nel legame peptidico $N$ terminale tra i primi due amminoacidi che possono essere idrolizzati causando la rottura del legame.
		Il complesso \emph{PTH-aa} viene identificato attraverso cromatografia \emph{HPLC}.

		\subsubsection{Spettrometria di massa}
		La spettrometria di massa viene usata per identificare proteine in miscele complesse.
		Le molecole ionizzate attraverso matrix assisted lase desorption/ionization \emph{MALDI} o electron-spray ionization \emph{ESI} permettono un calcolo del rapporto tra massa e carica.
		Il rapporto permette di determinare la massa di ogni frammento e quindi l'amminoacido in esso contenuto.

			\paragraph{Altre ionizzazioni}
			Le ionizzazioni \emph{MALDI} e \emph{ESI} sono soft e non rompono i peptidi.
			Ionizzazioni hard come \emph{EI} sono usate per piccole molecole formando lo ione molecolare radical cationico.

	\subsection{\emph{BLAST}}
	\emph{BLAST} \`e un tool informatico utilizzato per comparare sequenze di acidi nucleici e di sequenze proteiche.
	Si confronta una query con tutte le sequenze depositate nel database.
	Quelle simili alla nostra query si dicono subjects e sono ordinate con scores di similarity.

\section{Identificazione di specifiche molecole di DNA}

	\subsection{Panoramica}
	Enzimi di restrizione tagliano il DNA in frammenti di certa lunghezza che possono essere comparati con frammenti di lunghezza conosciuta utilizzando elettroforesi su gel d'agarosio.

	\subsection{Ibridazione southern blot}
	Il southern blot viene utilizzato per identificare una specifica sequenza di DNA all'interno di un campione.
	
		\subsubsection{Processo}
		\begin{multicols}{2}
			\begin{enumerate}
				\item Il DNA viene tagliato con enzimi di restrizione.
				\item I frammenti vengono separati con elettroforesi per dimensione.
				\item I frammenti denaturati in filamenti singoli in ambiente alcalino presentano basi esposte e trasferite su una membrana.
				\item Sonde di ssDNA marcato radioattivamente con sequenza complementare ibridano i frammenti con la sequenza target.
				\item I frammenti ibridati vengono identificati con un autoradiogramma.
			\end{enumerate}
		\end{multicols}

	\subsection{DNA fingerprint}
	Il genoma di persone non imparentate differisce per lo $0.1\%$ a causa di mutazioni o polimorfismi.
	Le seconde sono una variazione di DNA senza fenotipo clinico che avviene in regioni del genoma che non codificano proteine.
	
		\subsubsection{Restriction fragment length polymorphism}
		Restriction fragment length polymorphism \emph{RFLP} \`e identificato rompendo il DNA in frammenti con enzimi di restrizione.
		La lunghezza del frammento di restrizione \`e alterata se la variante genetica crea o distrugge un sito di restrizione.

			\paragraph{Cause di varianza}

				\subparagraph{Cambi a singola base}
				Cambi a singola base nella sequenza nucleotidica \emph{SNP} rappresentano il $90\%$ del genoma umano e possono causare mutazioni.

				\subparagraph{Cambi nel numero di ripetizioni di certe sequenze}
				Il numero variabile di ripetizioni tandem \emph{VNTR} sono sequenze corte di DNA ripetute in tandem a specifiche locazioni.
				Il numero \`e variabile per individuo e causa una identificazione univoca per ogni individuo.

			\paragraph{Utilizzo}
			La rottura da parte di enzimi di restrizione del genoma seguita da ibridazione southern blot rivela frammenti di diversa lunghezza in base a quante ripetizioni di ogni frammento sono contenute.

		\subsubsection{Analisi \emph{RFLP}}
		L'analisi \emph{RFLP} viene usata in ambito forense per identificare un individuo grazie a materiali che contengono DNA come:
		\begin{multicols}{2}
			\begin{itemize}
				\item Regioni mini satellite subtelomerici ricche in $GC$.
				\item Regioni micro satellite regioni distribuite nel genoma.
			\end{itemize}
		\end{multicols}
		Questi pattern sono ereditati attraverso regole mendeliane.

		\subsubsection{DNA fingerprinting}
		Il DNA fingerprinting avviene attraverso ibridazione southern blot in cui il DNA viene tagliato ai lati di \emph{VNTR}.
		Si fa correre la digestione su gel e si fa un blot.
		La probe mostra pattern complessi con i microsatelliti marcati.
		Le analisi forensi sono effettuate con amplificazione \emph{PCR}.
		I primer vengono generati in modo che siano ai lati dei loci.

			\paragraph{Utilizzo dell'analisi \emph{RFLP} per diagnosticare l'anemia falciforme}
			Una mutazione del codone $6$ dell'esone $1$ \`e un sito di restrizione per l'enzima \emph{MstII}.
			La \emph{PCR} amplifica l'esone $1$ che presenta due frammenti per sano e uno per malato.
			Le bande vengono marcate con sonde specifiche per l'esone $1$.

	\subsection{Ibridazione di colonie}
	L'ibridazione di colonie permette di identificare un clone da una libreria batterica cresciuta su plates.
	\`E utile per lo screening di una libreria genetica per un singolo gene.
	Avviene una lisi con \emph{SDS} e denaturazione del DNA che viene ibridato con sonde radioattive.
	Viene utilizzata la \emph{PCR} con primer specifici.

	\subsection{Cariotipo}
	Il cariogramma viene ottenuto e reso visibile con colorazione \emph{Giemsa} utilizzata nei test di diagnosi prenatale.
	Viene ottenuto attraverso tecniche invasive come amniocentesi o come amplificazione del materiale genetico nel sangue materno.
	Questo consente di visualizzare i cromosomi in metafase con pattern di bandeggio caratteristici.
	Le \emph{G bands} sono bande costituite da eterocromatina, scure, mentre le chiare da eucromatina.

		\subsubsection{Processo}
		Il DNA viene colorato da un mix di blu di metilene che si lega al legame fosfodiestere, eosina che si lega a siti cationici.
		Questo permette l'identificazione di anomalie cromosomiche.

		\subsubsection{Aberrazioni}
		Poliploidia, aneuploidia bilanciata e sbilanciata con perdita di materiale genetico.
		Inserzioni, delezioni.
		Mosaicismo.
		Disomia uniparentale in cui due cromosomi omologhi provengono dallo stesso genitore.

	\subsection{Fluorescent in situ hybridization}
	Fluorescent in situ hybridization o \emph{FISH} fissa le cellule a un vetrino e il DNA viene sondato con un analisi southern.
	Le sonde condengono tag fluorescenti.

	\subsection{Spectral karyotyping}
	Lo spectral karyotyping o \emph{SKY} \`e una modifica di \emph{FISH} in cui i cromosomi metafasici sono isolati e ognuno di essi \`e colorato diversamente in modo da semplificare l'analisi \emph{Giemsa}.

	\subsection{Array comparative genomic hybridization}
	Array comparative genomic hybridization \emph{aCGH} viene usata per identificare piccoli cambi nella sequenza di DNA.
	Pu\`o misruare la variazione di numero di copie.
	Un microarray \`e un insieme di frammenti di DNA fissato a uno slide che rappresentano il genoma.
	Due campioni di DNA cono confrontati uno etichettato verde e uno rosso.
	Sono mischiati e ibridizzati a un microarray.
	Quando l'array \`e scvansionato ogni punto avr\`a fluorescenza.
	Il colore \`e dipendente dal numero di copie relativo nel campione.
	Giallo in entrambe o rosso o verde se \`e maggiore in una delle due.
	Il rapporto tra i numeri di copie pu\`o essere confrontato lungo cromosomi o nel genoma.

\section{Identificazione di specifiche molecole di RNA}
	
	\subsection{Panoramica}
	L'espressione genica \`e spesso controllata dai livelli di trascrizione, pertanto \`e utile esaminare i livelli di RNA.

	\subsection{Ibridazione northern blot}
	Specifiche sequenze di RNA possono essere identificate attraverso l'ibridazione northern blot.
	Sono svolti nella stessa maniera dei southern blot ma il RNA non deve essere frammentato.
	La quantit\`a di RNA nel campione pu\`o essere determinata in molti modi.

		\subsubsection{Reverse transcription quantitative polymerase chain reaction}
		Reverse transcription quantitative polymerase chain reaction \emph{RT-qPCR} crea una copia di cDNA del RNA generato.
		Questo viene amplificato con una reazione a PCR con primer specifici.
		La quantit\`a di prodotto generata \`e proporzionale alla quantit\`a di RNA nel campione originale.
		La quantit\`a di prodotto pu\`o essere determinata dopo ogni ciclo.

		\subsubsection{Metodo delta-delta Ct di quantificazione}
		La quantificazione del metodo delta-delta Ct, dove Ct \`e il primo ciclo della reazione in cui la fluorescenza \`e emessa la disopra della soglia.
		\begin{multicols}{2}
			\begin{enumerate}
				\item Si calcola il \emph{$\Delta CT$} per normalizzare il campione escludendo variazioni di quantit\`a di campione.
					\[\Delta CT = CT\ medio\ GENE\ TARGET - CT\ MEDIO\ GENE\ HOUSEKEEPING\]
				\item Calcolo del $\Delta\Delta CT$, confrontando i campioni incogniti rispetto a uno scelto come controllo.
					\[\Delta\Delta CT = \Delta CT\ campione\ INCOGNITO - \Delta CT\ campione\ CONTROLLO\]
				\item Calcolo del fold change, se il valore \`e maggiore di $2$ il gene target \`e pi\`u espresso, meno espresso se minore di $0.5$, se si trova tra questi due valori la differenza non \`e significativa.
					\[2^{-\Delta\Delta CT}\]
			\end{enumerate}
		\end{multicols}

	\subsection{Microarray per profilo trascrittosomico}
	La tecnica dei microarray per profilo trascrittosomico permette una quantificazione del mRNA.
	I microarray comprendono tutti i geni dell'organismo, pertanto \`e possibile fare una esamina dell'espressione genica del genoma.
	Se un trascritto viene espresso differenzialmente il microarray appare verde o rosso, altrimenti giallo.
	Viene sostituita da \emph{RNA-seq}.

	\subsection{Trascrizione di un gene reporter}
	Per determinare l'espressione di un gene si possono fondere le sue regioni regolatorie a un gene reporter come $\beta$-galattosidasi o \emph{GFP}.

\section{Identificazione di specifiche proteine}

\section{Identificazione di interazioni tra molecole}

\section{Sequenziamento di molecole di DNA e RNA attraverso nanopori}
