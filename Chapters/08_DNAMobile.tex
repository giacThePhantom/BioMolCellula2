\chapter{DNA mobile}

\section{Panoramica degli elementi trasponibili}
Il DNA pu\`o muoversi nel o fuori dal genoma attraverso trasposizione (non specifica alla sequenza) o ricombinazione conservativa sito specifica \emph{CSSR}.
Gli elementi trasponibili o trasposoni sono sequenze di DNA che possono muovere intorno al genoma e inserirsi a siti obiettivo.
Si dividono in:
\begin{itemize}
	\item Autonomi: i trasposoni codificano le proteine necessarie al loro movimento nel DNA.
	\item Non autonomi: i trasposoni si basano su proteine sintetizzate da altri trasposoni autonomi.
\end{itemize}
Il primo trasposone umano identificato fu \emph{LINE L1} era un inserzione nell'esone $14$ del fattore $VIII$ della coagulazione e causava emofilia: la persona non \`e capace di bloccare il sanguinamento.
Molte malattie umane sono dovute a inserimenti di trasposoni e circa $\frac{1}{200}$ delle nascite umane hanno una trasposizione diversa rispetto ai genitori. 
La quantit\`a di trasposoni attiva nel genoma \`e molto bassa e non si spostano molto: tra lo $0.5$-$1\%$.
Possono anche aiutare a livello di espressione di geni essenziali.

	\subsection{Trasposoni e malattie umane}

		\subsubsection{\emph{LINE L1}}
		Sui geni che causano emofilia.
		\emph{RP2} per retinite pigmentosa X-linked, causa problemi con la vista e mancanza della visione periferica.
		\emph{APC} che causa cancro del colon.
		\emph{HBB} che causa beta-talassemia che causa stanchezza e un trasporto ridotto di ossigeno.

		\subsubsection{Delezione da ricombinazione omologa di due elementi trasponibili}
		La zona tra due elementi dopo la ricombinazione tra i due elementi pu\`o essere eliminata dal cromosoma.
		Un esempio \`e la sindrome di Alport's che agisce sui reni con abbondanza di sangue nell'urina.
		La delezione del Von Wiliebrand factor nella cascata del pathway della coagulazione questa si lega al fattore proteico $VIII$ e se mancante la sua attivit\`a impedisce la coagulazione.

	\subsection{Eventi di trasposizione}
	Gli elementi trasponibili, \emph{TE} o jumping genes sono elementi di DNA mobili che si muovono da una posizione all'altra sullo stesso o su un altro cromosoma.
	Influenzano l'espressione genica, causando mutazioni con proteine non-funzionali e possono riorganizzare la struttura cromosomica, con un ruolo attivo e passivo nell'evoluzione.
	I trasposoni possono:
	\begin{multicols}{2}
		\begin{itemize}
			\item Inserirsi in un esone:
				\begin{itemize}
					\item Troncamento.
					\item Esonizzazione.
					\item Splicing alternativo.
				\end{itemize}
			\item Inserirsi in un introne:
				\begin{itemize}
					\item Esonizzazione.
					\item Esonizzazione e troncamento.
					\item Splicing alternativo.
				\end{itemize}
			\item Distruggere un promotore o un enhancer:
				\begin{itemize}
					\item Up-regolazione.
					\item Cambio della specificit\`a del promotore.
					\item Sotto-regolazione.
					\item Repressione.
				\end{itemize}
			\item Creare un enhancer:
				\begin{itemize}
					\item Up-regolazione.
					\item Cambio della specificit\`a del promotore.
				\end{itemize}
			\item Silenziamento di un promotore o un enhancer:
				\begin{itemize}
					\item Sotto-regolazione.
					\item Repressione.
				\end{itemize}
		\end{itemize}
	\end{multicols}

		\subsubsection{Risvolti positivi}
		Possono essere positivi nella risposta allo stress: nel caso dei batteri un \emph{TE} si replica prima di fare un salto e lasciare o dare pi\`u capacit\`a alla cellula di rispondere a condizioni di stress.
		\subsubsection{Piante}
		Nelle piante, per esempio i \emph{TE} sono molto abbondanti ($60\%$) e possono rispondere a condizioni di stress per eliminare o alzare l'espressione genica per rispondere allo stress.
		Le piante utilizzano questo sistema per regolare l'espressione genica e le proteine in quanto per le piante \`e l'unico modo di creare un ambiente meno sfavorevole.
			
			\paragraph{Mais}
			Il mais in natura originalmente \`e nero e non giallo.
			I colori diversi dei chicchi di mais sono conseguenze di \emph{jumping genes}.
			Il chicco di mais contiene una base che si lega alla struttura di base.
			Al di sopra si trova l'embrione da cui si sviluppa la pianta, endosperma interno con la risorsa di carboidrati per l'embrio.
			La parte superiore o aleurone \`e la parte che contiene i pigmenti che determina il colore del mais che viene coperto dal pericarpo che tipicamente \`e trasparente.
			Studiando il mais McClintock nota mais di vari colori.

				\subparagraph{Colorazione del mais}
				La colorazione del mais \`e una conseguenza del jumping gene che entra nella sequenza del gene che codifica per il colore. 
				Si trova sul cromosoma $9$, gene $C$ che quando espresso crea un pigmento viola.
				Il colore viola \`e recessivo.
				Nel cromosoma $9$ si notano gli elementi $Ac$ e $Ds$ due jumping genes.
				Questi si possono spostare all'interno del gene $C$.
				$Ac$ codifica per una trasposasi che attiva la trasposizione di $Ds$ e questo pu\`o trasporsi nel gene $C$.
				La trasposizione in un gene $C$ causa un mutante formando il chicco giallo: $C^{Ds}$ \`e dominante.
				Pu\`o ancora succedere un ulteriore evento di trasposizione, dove $Ac$ attiva $Ds$ che si sposta dal gene $C$ che ritorna normale, portando a geni maculati.
				Pertanto se il trasposone \emph{Ds} si inserisce nel gene $C$ si nota un tessuto giallo.
				Trasposizione durante lo sviluppo precoce che ripristina il gene $C$ causa una sezione colorata di viola grossa.
				Trasposizione durante lo sviluppo tardivo crea sezioni viola piccole.
	
	\subsection{Scoperta dei trasposoni}
	Gli elementi trasponibili vengono scoperti nel $1940$ da Barbara McClintock e si nota come sono visibili nella pigmentazione dei chicchi di mais.
	Negli anni $60$ invece si scoprono questi elementi nei batteri e negli anni $80$ negli umani.
	I trasposoni non sono specifici alla sequenza e il $44\%$ del genoma umano \`e occupato da trasposoni ed elementi ripetitivi simili ai trasposoni, ma solo una piccola proporzione di loro $< 0.05\%$ \`e attiva.

	\subsection{Caratteristiche della trasposizione}
	\begin{itemize}
		\item La trasposizione di questi elementi non \`e specifica alla sequenza: tipicamente sono le condizioni di crescita che determinano dove questi elementi si spostano.
		\item La trasposizione dipende dall'enzima trasposasi che fa uscire l'elemento dal posto dove \`e e lo fa inserire nell'elemento obiettivo.
		\item L'inserzione di trasposoni risulta in una duplicazione del sito dove si inserisce il trasposone: la corta sequenza al sito di inserimento viene duplicata a causa della riparazione del danno al DNA.
	\end{itemize}

		\subsubsection{Inserzione dei trasposoni}
		L'inserzione del trasposone pu\`o essere non-replicativa o replicativa:
		\begin{itemize}
			\item Non replicativa (classe $II$) taglia e incolla o trasposizione diretta: avviene solo l'escissione del DNA e si muove a un nuovo sito.
			\item Replicativa (classe $I$): 
				\begin{itemize}
					\item Tipo $I$: nick and paste la replicazione a DNA di un trasposone che non si muove su un sito diverso.
					\item Tipo $II$ o retro-trasposione: il trasposone si replica attraverso un intermedio a cDNA.
				\end{itemize}
		\end{itemize}

	\subsection{Tipi di trasposoni}
	I trasposoni si trovano in tutti gli organismi e possono comprendere grandi parti dei loro genomi.
	La frequenza di nuove mutazioni attribuita ai trasposoni varia tra organismi.
	I tipi di trasposoni sono classificati in base a:
	\begin{itemize}
		\item Struttura e composizione.
		\item Proteine codificate.
		\item Metodo di trasposizione.
	\end{itemize}

		\subsubsection{Solo a DNA cut and paste}
		I trasposoni pi\`u semplici a DNA non replicativi: codificano per una trasposasi e ripetizioni invertite a livello terminali o \emph{TIR} che reclutano la trasposasi, si legano ad essa per
		liberare il trasposone.
		La trasposasi in caso di trasposone autonomo viene codificata all'interno.
		In assenza di trasposasi vengono detti non autonomi.

		\subsubsection{Elementi con lunghe terminazioni terminali \emph{LTR}}
		Sono replicativi di classe $I$ e sono presenti nei retrovirus.
		Sono autonomi e codificano diverse proteine come una trascrittasi inversa.
		Si muovono attraverso un intermedio a RNA.
		Alle terminazioni presentano dei long terminal repeat elements \emph{LTR}.

		\subsubsection{Elementi non \emph{LTR}}
		Sono replicativi di classe $I$ e sono retrotrasposoni.
		Possono essere autonomi o non autonomi, codificano proteine con diverse propriet\`a.
		Si muovono attraverso un intermedio a RNA;
		\begin{itemize}
			\item Long interspersed elements \emph{LINE}: codificano proteine che mediano la propria trasposizione.
			\item Short interspersed elements \emph{SINE}: non codificano proteine per il proprio movimento e si basano su quelle codificate da \emph{LINE}.
		\end{itemize}

	\subsection{Elementi trasponibili nei procarioti}
	Sono a DNA e tre tipi:
	\begin{itemize}
		\item Elementi a insertion sequence \emph{IS}.
		\item Trasposoni \emph{Tn}.
		\item Fagi trasponibili \emph{Mu} fago.
			Quando si integra nel genoma di E. coli pu\`o farlo in qualsiasi sequenza di coli e replicarsi ed amplificarsi nel suo genoma.
	\end{itemize}
	I trasposoni solo a DNA contengono sempre uno o pi\`u elementi di sequenza che sono affiancati a destra e sinistra da corte ripetizioni invertite riconosciute dalla trasposasi auto-codificata.
	Questi trasportano geni codificanti enzimi della trasposasi, fattori patogenici, resistenza agli antibiotici.
	Esistono virtualmente in tutti gli organismi e negli eucarioti contengono solo il gene della trasposasi.

		\subsubsection{Elementi a insertion sequence \emph{IS}}
		Questi sono stati i primi trasposoni a essere identificati: \emph{IS1} nell'operone del galattosio e in ceppi particolari fino a $19$ copie.
		Solo in E. coli ne esistono fino a centinaia.
		Sono gli elementi trasponibili pi\`u semplici e lunghi tra le $700$ e le $1500bp$.
		Codificano per la trasposasi il cui gene \`e affiancato da sequenze di ripetizioni terminali dette \emph{TIR}, \emph{IR} o sequenze a mosaico \emph{ME} a cui si lega la trasposasi.
		Le ripetizioni \emph{IR} formano un complesso sinaptico con la trasposasi legata ad essi. 
		Hanno principalmente un meccanismo replicativo ma possono averlo anche non-replicativo.

		\subsubsection{Trasposoni \emph{Tn}}
		I trasposoni sono lunghi tra le $2100$ e le $2300bp$ e hanno una composizione pi\`u complessa dagli elementi \emph{IS}.
		Un elemento \emph{Tn} \`e un trasposone composto: due elementi \emph{IS} affiancano uno o pi\`u geni: quelli interni sono funzionali e quelli che codificano per le trasposasi si trovano all'interno degli elementi \emph{IS}.
		Ogni elemento \emph{IS} pu\`o trasporsi indipendentemente o l'intero elemento \emph{Tn} pu\`o agire come un trasposone singolo.
		Si trovano composti o non composti.

			\paragraph{Composti}
			Come \emph{Tn5} e \emph{Tn10} contengono segmenti centrali con geni per la resistenza agli antibiotici.
			Sono affiancati a entrambe le terminazioni da elementi \emph{IS}, dello stesso tipo di un traspososne particolare.
			Gli elementi possono essere orientati in orientamenti inversi o lo stesso.
			La sequenza esterna \emph{IS-50R} codifica per le trasposasi.
			La lunghezza totale pu\`o essere diverse migliaia di nucleotidi e si traspongono attraverso un meccanismo cut-and-paste non replicativo.
			Si traspongono attraverso il meccanismo non-replicativo cut-and-paste.

			\paragraph{Non-composti}
			Non contengono elementi \emph{IS} ma semplici \emph{IR} che affiancano il trasposone da entrambi le parti lunghe tra i $2$ e i $20nt$.
			Codificano per un gene di resistenza a un farmaco, una trasposasi e una resolvasi, rispettivamente per inserzione e ricombinazione.
			Si traspongono attraverso il meccanismo replicativo nick-and-paste.

			\paragraph{Trasposone composto \emph{Tn5}}
			\emph{Tn5} consiste di due abbastanza identici elementi \emph{IS}:
			\begin{itemize}
				\item \emph{IS-50L} guida il gene che codifica per la resistenza a canamicina, neomicina, bleomicina e streptomicina.
				\item \emph{IS-50R} codifica la trasposasi.
			\end{itemize}
			Ogni \emph{IS-50} \`e circondato da due sequenze invertite di $19bp$ dette outer and inner end \emph{OE, IE} che definiscono la fine degli elementi \emph{IS} a cui la trasposasi agisce.
			La metilazione di \emph{IE} riduce l'abilit\`a della trasposasi di legarsi e agire.

\section{Panoramica dei trasposoni a DNA}
Nei trasposoni solo a DNA non viene coinvolto nessun intermedio a DNA.
La maggior parte di questi si muovono attraverso il meccanismo cut-and-paste non replicativo: il trasposone si escisse completamente e si inserisce nel target utilizzando una piccola quantit\`a di replicazione per riparare i siti di inserimento.
Alcuni si muovono attraverso il meccanismo replicativo nick-and-paste, ma solo nei batteri.
Il trasposone rimane attaccato al DNA donatore ed \`e unito al target formando un coniugato o co-integrato attraverso ricombinazione omologa.
Eventualmente il co-integrato si separa in due molecole, ognuna contenente un trasposone.

	\subsection{Trasposizione DNA-only non replicativa cut-and-paste}
	In questo caso si notano inverted repeats a sinistra e destra e la codifica della trasposasi al centro che viene tradotta.
	Nel target si trova un sito, tipicamente determinato da un numero di nucleotidi. 
	Viene tagliato il target DNA, il trasposone si inserisce e si chiudono i gaps da DNA polimerasi e il sito di taglio viene replicato e a destra del trasposone.
		
		\subsubsection{Processo}
		Questo processo avviene per i \emph{Tn} compositi come \emph{Tn5}, \emph{Tn7} o \emph{Tn10}.
		Avviene attraverso strutture di trasposoni:
		\begin{enumerate}
			\item Il DNA \`e riconosciuto da $2$ trasposasi che si legano alle sequenze \emph{IR}.
			\item Le trasposasi dimerizzano e portano le terminazioni del trasposone vicine formando il transpososoma o complesso sinaptico.
				Una trasposasi che si lega a un \emph{IR} va a tagliare quello opposto.
				Avviene pertanto un taglio incrociato, in modo da assicurare che le terminazioni siano vicine prima della rottura.
			\item Questo attiva l'attivit\`a del trasposone e causa la sua uscita dal DNA donatore.
			\item IL complesso transpososoma si lega al DNA target e le sue terminazioni \emph{$3'$-OH} fanno un attacco nucleofilo al sito target in presenza di \emph{$Mg^{2+}$} causando una doppia rottura \emph{ssDNA}.
			\item I gaps a singolo filamento sono riempiti dall'attivit\`a di riparazione della DNA polimerasi dell'ospite.
		\end{enumerate}

		\subsubsection{Liberazione del trasposone}
		Ci sono $3$ possibili meccanismi di auto-liberazione dei trasposoni.

			\paragraph{\emph{Tn7}}
			Avviene una rottura a doppio filamento mediata da \emph{TnsA}, che porta alla liberazione di due single strands.

			\paragraph{Con hairpin \emph{Tn10, Tn5}}
			Avviene un taglio del filamento singolo e per liberarsi completamente per attaccare il target DNA si deve liberare l'altro filamento.
			I single strand il \emph{$3'$-OH} come donatore di elettroni si presenta come donatore di elettroni anche per l'altra rottura: fa nel DNA donatore un attacco nucleofilo per tagliare il filamento rimanente formando un hairpin.
			L'hairpin viene aperto da acqua che agisce come donatore di elettroni.
			In questo modo si pu\`o formare il legame fosfoesterico con il DNA accettore.

			\paragraph{\emph{Hermes}}
			Avviene una reazione di transesterificazione con formazione di hairpin sul DNA donatore.

		\subsubsection{Taglio al sito donatore}
		Quando il DNA trasposone escinde dal sito donatore viene lasciato un gap che deve essere riparato.
		Il pathway potrebbe cambiare o non cambiare il sito donatore.
	
			\paragraph{Non-homologous end joining \emph{NHEJ}}
			\emph{NHEJ} riunisce le terminazioni.
			Il sito \`e raramente viene ripristinato allo stato originale, ma pi\`u spesso il sito \`e ripristinato in maniera approssimativa.

			\paragraph{Homology-directed repair}
			\emph{HDR} utilizza il cromatide fratello come un template di riparazione.
			Il trasposone \`e perfettamente ricopiato nel sito originale.

	\subsection{Trasposizione DNA-only replicativa nick-and-paste}
	Due esempi batterici sono \emph{Mu phage} e \emph{Tn3}.

		\subsubsection{\emph{Tn3}}
		A livello della trascrizione la trasposasi codificata da \emph{Tn3} catalizza la formazione di un cointegrato tra plasmidi donatori e recipienti.
		Durante il processo \emph{Tn3} \`e replicato cos\`i che si trovi una copia dell'elemento ad ogni giunzione del cointagrato.
		La resolvasi prodotta dal gene \emph{tnpR} risolve il cointegrato mediando la ricombinazione tra due elementi \emph{Tn3}.
		Il plasmide donatore e recipiente si separano, ognuno con una copia di \emph{Tn3}.

		\subsubsection{\emph{Mu fage}}
		Il \emph{Mu fage} \`e il trasposone pi\`u lungo e codifica per $55$ proteine, con numerosi geni per la produzione delle proteine di testa e coda del fago.
		Il DNA si trova lineare nel fago. 
		Dopo l'infezione si circolarizza nel DNA e si integra attraverso ricombinazione sito-specifica nel genome di E. coli.
		Dopo l'induzione dello stato litico si amplifica nel genoma di E. coli attraverso trasposizione per produrre $100$ cromosomi virali e produce le proteine codificate.
		Vengono assemblate le particelle del fago che si escindono dal DNA e causano lisi.
		\`E l'unico elemento trasponibile che esiste in uno stato extracromosomico.

		\subsubsection{Processo}
		\begin{enumerate}
			\item La trasposasi crea un taglio ssDNA ad entrambe le terminazioni $3'$ del DNA del trasposone.
			\item Ogni terminazione \emph{$3'$-OH} si attacca al DNA obiettivo per creare una rottura a filamento singolo.
			\item Il DNA donatore si trasferisce al DNA target formando un intermedio Shapiro con $2$ interruzioni.
			\item Una forcella di replicazione si forma all'interruzione di sinistra e il DNA viene sintetizzato per chiudere i gaps.
			\item Si forma un co-integrato che si divide a met\`a a causa dell'attivit\`a ricombinante delle resolvasi auto-codificate.
		\end{enumerate}


\section{Retrotrasposoni}
I trasposoni coinvolgono intermedi a RNA e tascrittasi inverse.
Si dividono in \emph{LTR} (long terminal repeat) o elementi non \emph{LTR}.
I retrovirus \emph{LTR} hanno fasi extracellulari e si trovano solo nei vertebrati.
Elementi retroviral-like \emph{LTR} si muovono solo intracellularmente e si trovano in funghi, piante ed animale.
Elementi non-\emph{LTR} si trovano in tutti i domini della vita.
Possono inoltre costituire parti significative dei genomi eucarioti.
	
	\subsection{Trasposoni umani}
	\begin{center}
		\begin{tabular}{|c|c|c|c|c|c|}
			\hline
			\multicolumn{2}{|c|}{Elemento} & \makecell{Numero \\di copie \\$\times\num{1000}$} & \makecell{Lunghezza \\totale \\ $Mb$} & \makecell{Genoma \\ $\%$} & Attivit\`a \\
			\hline
						     & \makecell{Retrotrasposoni \\\emph{LTR}} & $443$ & $227$ & $8.3$ & \\
			\hline
			\multicolumn{2}{|c|}{Line (non-\emph{LTR} autonomous)}  & $868$ & $462$ & $20.4$ & \\
			\hline
						& \emph{LINE-1} & $516$ & $462$ & $16.9$ & Attivo \\
			\hline
						& \emph{LINE-2} & $315$ & $88$ & $3.2$ & \\
			\hline
						& \emph{LINE-3} & $37$ & $8$ & $0.3$ & \\
			\hline
			\multicolumn{2}{|c|}{Sine (non-\emph{LTR} non-autonomous)}  & $1558$ & $360$ & $13.3$ & \\
			\hline
						& \emph{Alu} & $1090$ & $290$ & $10.6$ & Attivo con \emph{LINE-1}\\
			\hline
						& \emph{MIR/MIR3} & $468$ & $69$ & $2.5$ & \\
			\hline
						& \emph{SVA} & $2.76$ & $4$ & $0.15$ & Attivo con \emph{LINE-1}\\
			\hline
			\multicolumn{2}{|c|}{Trasposoni a DNA} & $294$ & $78$ & $2.8$ & \\
			\hline
		\end{tabular}
	\end{center}


	\subsection{Retrotrasposoni \emph{LTR}}
	Gli \emph{LTR} retrotrasposoni e i retrovirus sono simili nella struttura.
	Possiedono entrambe lunghe ripetizioni terminali.
	I retrovirus possiedono un elemento codificante il capside \emph{ENV} necessario per la sopravvivenza al di fuori della cellula.

		\subsubsection{Attivit\`a}
		L'integrazione nel genoma ospite del retrotrasposone \emph{LTR} \`e mediata da un integrasi codificata dall'elemento interno attraverso lo stesso meccanismo per gli elementi solo a DNA.
		La terminazione \emph{$3'$-OH} dell'elemento attacca il DNA target e la sequenza target \`e duplicata.
		Gli elementi \emph{LTR} possono causa mutazioni di inserimento alterando l'espressione del gene avendo effetto sui promotori, enhancer e siti di splicing.

			\paragraph{Processo}
			\begin{enumerate}
				\item Gli elementi \emph{LTR} sono trascritti in un mRNA provirus dalla cellula ospite.
				\item Il provirus viene esportato nel nucleo e proteine sono prodotte nel mRNA>
				\item Il mRNA e le proteine si assemblano in una particella simile a un virus.
				\item Nella particella il RNA \`e convertito in cDNA da trascrittasi inverse.
				\item Il DNA \`e integrato nel genoma dell'ospite da integrasi.
				\item Nei retrovirus esiste uno stato extracellulare che richiede \emph{ENV}>
			\end{enumerate}

			\paragraph{Sintesi da trascrittasi inversa}
			Le \emph{LTR} che contengono corte sequenze ripetute $R$ $U5$ e $U3$ sono i siti di inizio per la trascrittasi inversa.
			Sono sintetizzate da trascrizione inversa e passaggi di degradazione del RNA.

			\paragraph{Escissione}
			I trasposoni \emph{LTR} si muovono attraverso trasposizione replicativa e pertanto non sono escissi dal sito donatore e non si deve riparare.
			Ricombinazione omologa pu\`o avvenire tra i due \emph{LTR} di un trasposone causando la delezione del DNA tra gli \emph{LTR}.
			Questo non \`e mobile.

	\subsection{Retrotrasposoni non-\emph{LTR}}
	I trasposoni non \emph{LTR} come \emph{LINES} e \emph{SINES} sono i pi\`u comuni nei genomi eucariot.
	Quando si muovono una copia di RNA si associa con il sito obiettivo e agisce come stampo per la trascrizione inversa.
	La maggior parte degli elementi \emph{LINE} non hanno un promotore e non sono attivi con l'eccezione di \emph{LINE-1}.
	\emph{LINE ORF1} codifica per una proteine legante RNA, \emph{ORF2} una proteina ibrida con attivit\`a endonucleasica e di trascrittasi inversa.
	Elementi \emph{LINE} possiedono una sequenza poli-A nel loro DNA come i \emph{SINE}.
	Gli elementi \emph{SINE} non codificano le proteine richieste per la trasposizione e dipendono da \emph{LINE}
	Contengono $A$ e $B$ boxes per la trascrizione da RNA polimerasi $III$.

		\subsubsection{Attivit\`a}
		Gli elementi non-\emph{LTR} si inseriscono attraverso trascrizione inversa al sito di inserimento.
		\emph{ORF1} \`e un chaperon legante il DNA.
		\emph{ORF2} \`e una singola proteina con attivit\`a endocleasica e di trascrittasi inversa.
		\begin{enumerate}
			\item L'elemento \emph{LINE} viene trascritto in RNA che viene tradotto.
			\item Il sito target, regione ricca in $AT$ \`e nicked dall'attivit\`a endonucleasica.
			\item La coda poli-A del \emph{LINE RNA} si inserisce nel dsDNA e si accoppia con la sequenza poli-T.
			\item La trascrizione inversa si estende dal DNA utilizzando lo stampo a RNA.
			\item Il RNA \`e degradato da \emph{RNAasi H}/
			\item IL secondo filamento di DNA del sito target viene rotto e il filamento prodotto agisce come stampo per la DNA polimerasi.
			\item I processi di riparazione del DNA riempiono i gaps.
		\end{enumerate}
		La trascrittasi inversa fallisce nel completare la terminazione $5'$ causando inserzioni troncate senza \emph{ORF1} che non possono trasporsi.
		La trasposizione ha meccanismo replicativo.
		Gli elementi \emph{SINE} utilizzano le proteine codificate da \emph{LINE-1} e la trasposizione avviene con lo stesso meccanismo.

		\subsubsection{\emph{ALU} - elemento \emph{SINE}}
		\emph{ALU} (arthrobacter luteus restriction endonuclease element) \`e lungo $280bp$ e si traspone nel genoma umano causando malattie e cancro.
		Viene trascritto dalla RNA polimerasi III.
		
			\paragraph{Integrazione}
			L'integrazione di \emph{ALU} pu\`o essere deleteria se distrugge l'espressione genica.
			Contribuisce inoltre a una diversit\`a nell'espressione genica fornendo siti di legame per fattori di trascrizione.
			Possono introdurre eventi di ricombinazione come duplicazione o delezione di segmenti cromosomali.
			Accelera l'evoluzione e si trova spesso in sequenze geniche codificanti pre-mRNA come pare di introni o \emph{UTR}.

		\subsubsection{Introni mobili di gruppo $\mathbf{II}$}
		Un secondo gruppo di retrotrasposoni non-\emph{LTR} sono gli introni mobili di gruppo $II$.
		Possono avere due tipi di mobilit\`a: retrohoming e retrotrasposizione.
		Il retrohoming avviene con alta frequenza in un sito specifico con omologia considerevole alla sequenza intronica.
		La retrotrasposizione \`e pi\`u rara e avviene con frequenza pi\`u bassa a siti non specifici.
		Il primo passo per entrambi \`e lo splicing dell'elemento dal mRNA per formare un lariat intronico da maturasi. 
		Nel retrohoming il RNA escisso fa splicing inverso in uno dei filamenti del DNA.
		L'endonucleasi codificata dall'introne \emph{RME} fa un nick al DNA fuori dal sito di inserzione, una trascrittasi inversa fa una copia del DNA dell'introne attraverso la terminazione $3'$ del taglio del DNA come primer.
		Proteine batteriche rimuovono il RNA e una copia di DNA viene usata come filamento stampo dal lariat.


\section{Controllo della trasposizione}
La trasposizione pu\`o portare a diversit\`a genica, ma troppa trasposizione \`e principalmente svantaggiosa.
Ci sono pertanto meccanismi da parte della cellula capaci di prevenire eventi di trasposizione nel proprio genoma.
La trasposizione pu\`o essere controllata positivamente o negativamente.
Se le proteine dei trasposoni hanno un ruolo centrale nei passi catalitici quelle dell'ospite hanno un ruolo centrale nella trasposizione.

	\subsection{Meccanismi di controllo}

		\subsubsection{Frequenza di trasposizione}
		La frequenza di trasposizione \`e collegata alla concentrazione di trasposasi.
		La trasposasi \`e sotto il controllo a livelli traduzionali e trascrizionali.

			\paragraph{\emph{Tn10}}
			Caratteristiche specifiche del \emph{IS10} mantengono la frequenza di trasposizione bassa:
			\begin{itemize}
				\item L'espressione della trasposasi \`e guidata da un promotore debole $P_{in}$.
				\item Un trascritto anti-senso viene prodotto da $P_{out}$.
				\item L'accoppiamento delle basi con il mRNA da $P_{in}$ produce dsRNA che inibisce la traduzione.
			\end{itemize}

		\subsubsection{Metilazione in $\mathbf{P_{in}}$}
		La metilazione completa blocca la trasposizione impedendo l'espressione della trasposasi.
		Durante la replicazione del DNA, il DNA alla forcella di replicazione \`e emi-metilato causando un aumento dell'espressione dell'espressione della trasposasi da $P_{in}$.
		Siccome solo un filamento subisce la trasposizione ed \`e rotto, il filamento donatore pu\`o essere riparato da riparazione direzionata da omologia usando il braccio intatto come stampo.

			\paragraph{Drosophila - trasposone $\mathbf{P}$}
			La trasposizione del trasposone $P$ cut-and-paste della Drosophila \`e controllata da splicing alternativo.
			Nelle cellule germinali il mRNA \`e spliced correttamente in modo che la trasposizione possa avvenire permettendo la trasmissione dell'elemento $P$ da una generazione alla prossima.
			Nelle cellule somatiche il terzo introne non viene rimosso producendo una trasposasi troncata non funzionale.

		\subsubsection{Drosophila - controllo nella linea germinale}
		Specifici loci in Drosophila producono grandi numeri di piccole molecole di RNA \emph{piRNA} che bloccano la trasposizione utilizzando interferenza a RNA.
		Questi si accoppiano con le basi degli mRNA dei trasposoni.

			\paragraph{RNA regolatori}
			Regioni di DNA detti cluster di piRNA contengono trasposoni inattivi o frammentati che sono trascritti dalla RNA polimerasi II.
			Il lungo trascritto di piRNA viene processato dalla proteina Argonauta Piwi e vengono prodotti piRNA lunghi $23nt$.
			QUesti si legano al mRNA dei trasposoni attivi, causando la loro rottura da parte della proteina argonauta \emph{Ago}.
			Il trasposone mRNA rotto si lega a un altro lungo piRNA trascritto parentale.
			La rottura e processamento causa una produzione di pi\`u piRNA, che sono importati nel nucleo con la proteina argonauta per silenziare la trascrizione dei trasposoni attivi a livello di DNA o trasposoni reclutando enzimi epigenetici.

\section{Panoramica di \emph{CSSR}}
La ricombinazione sito-specifica conservativa avviene tra siti specifici alla sequenza con corte regioni di omologia.
Questi siti comprendono:
\begin{itemize}
	\item Gli spacer omologi con direzionalit\`a.
	\item I siti di legame delle ricombinasi che affiancano ogni spacer.
\end{itemize}
Quattro recombinasi si legano a due duplex.
Il DNA viene tagliato in una maniera divisa, creando overhangss in modo che le molecole di DNA ricombinate contengono spacer eteroduplex.
Il reclutamento per formare eteroduplexes impne l'accuratezza dell'evento di ricombinazione.
\emph{CSSR} pu\`o causare diversi tipi di riarrangiamento genetico in base alla posizione relativa e orientamento dei siti di ricombinazione.
Gli enzimi dei fagi che causano integrazione del genoma del fago nel DNA ospite sono detti integrasi ma non sono imparentati con le integrasi dei trasposoni.
Due siti di ricombinazione nella stessa direzione su un cromosoma singolo circolare causa escissione.
Due siti di ricombinazione che si trovano in direzioni opposte su un singoli cromosomi circolari causa inversione.
	
	\subsection{Ricombinasi sito-specifiche}
	Le ricombinasi specifiche alla sequenza sono topoisomerasi specifiche alla sequenza.
	Ce ne sono due famiglie di \emph{CSSR} ricombinasi e in entrambe un amminoacido nucleofilo attacca il DNA.
	Non richiedono \emph{$Mg^{2+}$} come co-fattore.
	
		\subsubsection{Ricombinasi tirosina}
		Le ricombinasi tirosina funzionano rompendo il DNA e formando un legame covalente \emph{DNA-$3'$-P-tirosina} come intermedio di reazione.

			\paragraph{Meccanismo}
			La tirosina ricombinasi forma un dimero $R1$-$R2$ e uno $R3$-$R4$.
			Avviene una rottura del filamento top da parte di $R1$ e $R3$, in cui si forma il legame \emph{$3'$-P-tirosina}.
			Le terminazioni libere $5'$ non legate alla ricombinasi possono invadere formando una giunzione di Holliday.
			Avviene una risoluzione verticale con una rottura del filamento bottom da $R2$ e $R4$/
			Infine avviene uno scambio del filamento bottom per terminare la ricombinazione.

		\subsubsection{Ricombinasi serina}
		Le ricombinasi serina funzionano rompendo il DNA e formando un legame covalente \emph{DNA-$5'$-P-serina} come intermedio di reazione.

			\paragraph{Meccanismo}
			La serina ricombinasi forma un dimero $R1$-$R2$ e $R3$-$R4$.
			La rottura forma terminazioni \emph{$3'$-OH}.
			Avviene uno scambio di partner e un legamento \emph{$3'$-OH-$5'$-P}

		\subsubsection{Confronto}
		Le due famiglie non sono imparentate, sono strutturalmente diverse e con diversi meccanismi di reazione.
		La reazione da parte di entrambi i tipi di ricombinasi produce lo stesso risultato.
		Non usano \emph{ATP}, un co-fattore bivalente e sintesi del DNA.
		In assenza di \emph{$Mg^{2+}$} le catene laterali di istidine lisine o arginine clustered istigano l'attacco nucleofilo da parte di tirosina o serina.
	
	\subsection{Conversione \emph{CSSR} di dimeri di DNA in monomeri}
	La conversione di un DNA circolare in due cerchi di DNA \`e una funzione importante di \emph{CSSR}.
	La ricombinazione omologa pu\`o causare multimerizzazione.
	\emph{CSSR} contribuisce alla segregazione dei cromosomi risolvendo cromosomi dimerici.
	Durante la replicazione del DNA di un plasmide su un genoma circolare la ricombinazione omologa tra il vecchio e il nuovo DNA sintetizzato pu\`o risultare in un DNA dimerico.
	Per riformare le due forme monomeriche deve avvenire una ricombinazione revertente sito-specifica come mediata dalla tirosina ricombinasi \emph{XerCD}.

\section{Integrazione ed escissione del batteriofago $\mathbf{\lambda}$}
Quando $\lambda$ infetta E. coli e entra nello stato lisogenico avviene \emph{CSSR} tra i siti \emph{attP} e i siti batterici \emph{attB}.
Entrambi contengono una regione spacer $O$ che \`e affiancata da diversi siti di legame per le tirosine ricombinasi codificate da $\lambda$ dette integrasi.
L'integrazione del DNA del fago ai siti \emph{attP} e \emph{attB} produce i siti \emph{attR} e \emph{attL}.
L'escissione del profago, mediata dall'integrasi di $\lambda$ ricrea i siti \emph{attP} e \emph{attB}.
La struttura unica di ogni sito \emph{att} causa in diverse richieste di reclutamento di proteine per la ricombinazione dei diversi substrati di DNA.

	\subsection{Integrasi $\mathbf{\lambda}$ \emph{Int}}
	L'integrasi $\lambda$ \emph{Int} contiene due domini leganti il DNA: uno N terminale che si lega ai siti del braccio di Int e quello C terminale che si lega ai siti nucleari.
	Il piegamento del DNA \`e richiesto per il corretto legame di Int, la proteina \emph{IHF} \`e richiesto.
	Il super-avvolgimento gioca un ruolo importante nella promozione del piegamento del DNA e l'avvolgimento del DNA \emph{attP} intorno all'intrasoma \emph{attP}.

	\subsection{Integrazione del fago $\mathbf{\lambda}$}
	Il \emph{attP}$+$\emph{attB} con i suoi quattro tetrameri $R1$, $R2$, $R3$ e $R4$ forma l'intasoma integratico.
	I domini catalitici dei due dimeri di \emph{Int} causa uno scambio di filamento tra il filamento del fago e del batterio.

	\subsection{Escissione del fago $\mathbf{\lambda}$}
	L'escissione richiede l'assemblaggio di \emph{attL} e \emph{attR} intasoma escissivo.
	Nono sono sufficienti \emph{IHF} e \emph{Int} in quanto i bracci $P$ e $P'$ usati per piegare il DNA si trovano su diversi filamenti, pertanto si richiedono \emph{Xis} e \emph{FIS} per piegare il DNA.
	Dopo l'assemblaggio corretto i siti di \emph{Int} nucleari sono piazzati affianco e pu\`o avvenire un cambio di filamento.
