\chapter{Replicazione}
\section{Replicazione del DNA semi-conservativa}
Durante la divisione cellulare l'informazione genetica deve essere copiata e distribuita equamente tra le cellule figlie. Dopo la scoperta della struttura a doppia elica del DNA
si ragion\`o come i due filamenti complementari sono copiati e replicati.
\subsection{Modelli di replicazione}
\begin{itemize}
	\item Replicazione conservativa: il DNA rimane intatto come un doppio filamento e agisce come stampo.
	\item Replicazione semi-conservativa: un filamento agisce come stampo per sintetizzare un nuovo filamento complementare.
	\item Replicazione dispersiva: il doppio filamento si rompe nella sua lunghezza e frammenti sovrapposti servono come stampi per la sintesi.
\end{itemize}
\subsubsection{Determinazione del modello semi-conservativo}
Per determinare il modello di replicazione Meselson e Stahl idearono un esperimento che utilizzava un gradiente di densit\`a e ultra-centrifugazione. 
\paragraph{Tecnica del gradiente di densit\`a}
Gli scienziati presero un tubo di plastica in cui era presente un gradiente del sale cloruro di cesio. In questo modo aggiungendo componenti di diversa densit\`a in cima al gradiente e 
centrifugandoli ad alta velocit\`a la forza gravitazionale trasporta le componenti nel gradiente in modo che si fermino quando la loro densit\`a \`e uguale alla densit\`a locale della 
soluzione di \emph{CsCl}. Le componenti a bassa densit\`a sono posizionate pi\`u in alto nel gradiente, mentre quelle a densit\`a pi\`u alta in basso. Quando le componenti si sono mosse 
attraverso il gradiente durante la centrifugazione si prendono campioni dal basso all'alto attraverso frazionamento. 
\paragraph{Rendere il DNA pi\`u pesante}
Per rendere il DNA di nuova sintesi pi\`u pesante rispetto a quello originale viene utilizzato un isotopo dell'azoto \emph{\ce{15N}} in quanto \`e la massa dell'elemento pi\`u frequente
e pu\`o essere sintetizzato in forma radioattiva.
\paragraph{L'esperimento}
Gli scienziati fecero crescere una coltura di E. coli per $4$ divisioni cellulari in un medio minimale contenente glucosio e con \emph{\ce{15NH4Cl}} come l'unica fonte di azoto. Il 
DNA alla fine pertanto conterr\`a \emph{\ce{15N}} nelle basi nucleotidiche. Prendendo un campione della coltura della quarta divisione cellulare e isolandolo. Successivamente si isola
il resto dei batteri attraverso centrifugazione, li si lava e risospende nel medio minimale con \emph{\ce{14NH4Cl}} come unica fonte di azoto. Si lascia crescere e dividere la coltura
cos\`i ottenuta prendendo campioni ogni divisione. Si isola il DNA dai vari campioni e lo si carica su un tubo a gradiente di \emph{\ce{CsCl}} separato. Successivamente si 
ultracentrifugano tutti i tubi in parallelo, si fraziona i campioni e si fa correre il DNA su un gel di agarosio e li si trasferisce su membrana di nitrocellulosa. Infine si espone la 
membrana a un foto film. 
\paragraph{Conclusioni}
Si nota come in base al modello si osserverebbero comportamenti diversi:
\begin{itemize}
	\item Modello conservativo: il numero di batteri con \emph{\ce{15N}} rimarrebbe costante e aumenterebbe quella con \emph{\ce{14N}}, presentando pertanto due bande, una per 
		l'isotopo e una per \emph{\ce{14N}}.
	\item Modello dispersivo: i batteri presenterebbero tutti del DNA ibrido contenente sia \emph{\ce{15N}} che \emph{\ce{14N}}, presentando pertanto una banda unica all'ibrido.
	\item Modello semi-conservativo: si troverebbe nella popolazione un numero di molecole contenenti uno strand con \emph{\ce{15N}} e l'altro \emph{\ce{14N}}, mentre il resto tutto 
		a \emph{\ce{14N}}, pertanto si noterebbero due bande, una per \emph{\ce{14N}} e una per l'ibrido.
\end{itemize}
Si osserva che avviene il terzo caso, determinando che la replicazione \`e semi-conservativa.
\section{Il modello dei repliconi}

\section{Identificazione delle origini di replicazione}

\section{Panoramica della replicazione del DNA}

\section{Iniziazione}

\section{Allungamento}

\section{Terminazione}

\section{Replicazione dei telomeri}

\section{Correzione degli errori post-replicativa}

\section{Mantenimento delle modifiche istoniche}

\section{DNA polimerasi specializzate}
