\chapter{Replicazione}
\section{Replicazione del DNA semi-conservativa}
Durante la divisione cellulare l'informazione genetica deve essere copiata e distribuita equamente tra le cellule figlie. Dopo la scoperta della struttura a doppia elica del DNA
si ragion\`o come i due filamenti complementari sono copiati e replicati.
\subsection{Modelli di replicazione}
\begin{itemize}
	\item Replicazione conservativa: il DNA rimane intatto come un doppio filamento e agisce come stampo.
	\item Replicazione semi-conservativa: un filamento agisce come stampo per sintetizzare un nuovo filamento complementare.
	\item Replicazione dispersiva: il doppio filamento si rompe nella sua lunghezza e frammenti sovrapposti servono come stampi per la sintesi.
\end{itemize}
\subsubsection{Determinazione del modello semi-conservativo}
Per determinare il modello di replicazione Meselson e Stahl idearono un esperimento che utilizzava un gradiente di densit\`a e ultra-centrifugazione. 
\paragraph{Tecnica del gradiente di densit\`a}
Gli scienziati presero un tubo di plastica in cui era presente un gradiente del sale cloruro di cesio. In questo modo aggiungendo componenti di diversa densit\`a in cima al gradiente e 
centrifugandoli ad alta velocit\`a la forza gravitazionale trasporta le componenti nel gradiente in modo che si fermino quando la loro densit\`a \`e uguale alla densit\`a locale della 
soluzione di \emph{CsCl}. Le componenti a bassa densit\`a sono posizionate pi\`u in alto nel gradiente, mentre quelle a densit\`a pi\`u alta in basso. Quando le componenti si sono mosse 
attraverso il gradiente durante la centrifugazione si prendono campioni dal basso all'alto attraverso frazionamento. 
\paragraph{Rendere il DNA pi\`u pesante}
Per rendere il DNA di nuova sintesi pi\`u pesante rispetto a quello originale viene utilizzato un isotopo dell'azoto \emph{\ce{15N}} in quanto \`e la massa dell'elemento pi\`u frequente
e pu\`o essere sintetizzato in forma radioattiva.
\paragraph{L'esperimento}
Gli scienziati fecero crescere una coltura di E. coli per $4$ divisioni cellulari in un medio minimale contenente glucosio e con \emph{\ce{15NH4Cl}} come l'unica fonte di azoto. Il 
DNA alla fine pertanto conterr\`a \emph{\ce{15N}} nelle basi nucleotidiche. Prendendo un campione della coltura della quarta divisione cellulare e isolandolo. Successivamente si isola
il resto dei batteri attraverso centrifugazione, li si lava e risospende nel medio minimale con \emph{\ce{14NH4Cl}} come unica fonte di azoto. Si lascia crescere e dividere la coltura
cos\`i ottenuta prendendo campioni ogni divisione. Si isola il DNA dai vari campioni e lo si carica su un tubo a gradiente di \emph{\ce{CsCl}} separato. Successivamente si 
ultracentrifugano tutti i tubi in parallelo, si fraziona i campioni e si fa correre il DNA su un gel di agarosio e li si trasferisce su membrana di nitrocellulosa. Infine si espone la 
membrana a un foto film. 
\paragraph{Conclusioni}
Si nota come in base al modello si osserverebbero comportamenti diversi:
\begin{itemize}
	\item Modello conservativo: il numero di batteri con \emph{\ce{15N}} rimarrebbe costante e aumenterebbe quella con \emph{\ce{14N}}, presentando pertanto due bande, una per 
		l'isotopo e una per \emph{\ce{14N}}.
	\item Modello dispersivo: i batteri presenterebbero tutti del DNA ibrido contenente sia \emph{\ce{15N}} che \emph{\ce{14N}}, presentando pertanto una banda unica all'ibrido.
	\item Modello semi-conservativo: si troverebbe nella popolazione un numero di molecole contenenti uno strand con \emph{\ce{15N}} e l'altro \emph{\ce{14N}}, mentre il resto tutto 
		a \emph{\ce{14N}}, pertanto si noterebbero due bande, una per \emph{\ce{14N}} e una per l'ibrido.
\end{itemize}
Si osserva che avviene il terzo caso, determinando che la replicazione \`e semi-conservativa.
\section{Il modello dei repliconi}
Il modello dei repliconi \`e stato proposto nel $1963$. Si indica con replicone la parte del DNA che sta venendo replicata. La replicazione inizia a una particolare sequenza di origine
o replicatore. Una proteine iniziatrice si lega al replicatore per iniziare il processo di replicazione. 
\subsection{Scoperta del modello}
La scoperta del modello dei repliconi avviene grazie a Cairns nel $1963$ attraverso un'analisi autoradiografica del genoma in replicazione di E. coli. Si cresce la cellula in un medium
contenente glucosio e azoto. Si aggiunge ad essa \emph{\ce{[3H]}-timidina} e si fanno avvenire due replicazioni del DNA in modo che questa si incorpori due volte nel filamento di 
nuova sintesi. Si lisa la cellula e la si espone a un foto film per due mesi. 
\subsubsection{Osservazioni}
Si nota come dopo una replicazione un filamento non \`e radioattivo mentre l'altro lo \`e. All'inizio della seconda replicazione si forma una sezione con entrambi i filamenti 
radioattivi, permettendo di visualizzare come il DNA si replica in maniera semi-conservativa nella cellula. Nel replicone o \emph{Ori} il DNA si apre formando bolle tra due 
forcelle di replicazione. La bolla si estende in maniera bidirezionale.	Lo si nota osservando la radioattivit\`a ai due estremi della bolla di replicazione. 
\subsection{Origini di replicazione}
\subsubsection{Origine di replicazione singola}
In caso di una singola \emph{Ori} in DNA circolare questo si comincia a svolgere in tale sequenza producendo una bolla di replicazione con una forcella ad ogni terminazione. Le forcelle
procedono lungo il cerchio producendo il modello $\Theta$. Successivamente interviene la topoisomerasi II \emph{girasi} che separa le due molecole di nuova formazione.
\subsubsection{Origini di replicazioni multiple}
In caso di multiple \emph{Ori} in DNA lineare si formano varie bolle di replicazione con forcelle ad ogni estremit\`a. Le bolle mano a mano che ne incontrano altre si fondono tra di 
loro.
\section{Identificazione delle origini di replicazione}
Nei genomi di batteri, batteriofagi, virus e plasmidi si trova un \emph{Ori} per molecola di DNA, nei primi in quanto il loro DNA si replica in maniera indipendente da quella 
dell'ospite. L'origine di replicazione in E. coli o \emph{OriC} \`e stata trovata attraverso un esperimento.
\subsection{Esperimento}
Si prende una coltura di E. coli e la si trasforma con un plasmide contenente del DNA di E. coli ottenuto attraverso enzimi di restrizione e un gene che codifica la resistenza 
all'ampicillina. In questo modo ogni colonia che cresce in presenza dell'antibiotico contiene un \emph{Ori} nel plasmide. Si continua a ridurre la lunghezza del frammento fino a che
non si verifica pi\`u la resistenza. In questo modo si riesce a determinare la sequenza minima e specifica dell'\emph{Ori}. In E. coli \`e lunga $245bp$ e la parte che si apre 
formata da $3\times 13bp$ \`e ricca in $A$ e $T$ in quanto le basi formando solo due legami a idrogeno sono pi\`u facili da aprire. La sequenza contiene inoltre $5\times 9bp$ siti di 
legame per $5$ proteine iniziatrici \emph{DnaA}.
\subsection{Origini di replicazione negli eucarioti}
Si nota come i lunghi cromosomi lineari degli eucarioti possiedono multiple origine di replicazione in modo da replicare il DNA in un tempo ragionevole. Attraverso autoradiografia si
identificano diverse origine di replicazione attraverso le bolle con diversa dimensione in base al tempo di formazione: se precoce o tardiva. Le forcelle di replicazione si muovono
comunque in maniera bidirezionale e si uniscono tra di loro quando si incontrano. La lunghezza di repliconi individuali \`e di $100bp$ in lievito e mosche e tra i \num{75000} e 
\num{175000} in cellule animali e umane. Il tasso di replicazione negli eucarioti \`e di \num{2000}$\frac{bp}{min}$, molto pi\`u lento rispetto ai batteri. Si nota come dalla velocit\`a
di replicazione il genoma di un mammifero potrebbe essere replicato in un'ora. Nonostante questo la fase $S$ dura pi\`u di $6$ ore in una cellula somatica. Questo avviene in quanto non
pi\`u del $15\%$ dei repliconi sono attivi in un dato momento. Ci sono eccezioni come le divisioni degli embrioni di Drosophila, con fase $S$ molto pi\`u breve. 
\subsubsection{Identificazione degli \emph{ARS} nel lievito S. cerevisiae}
Si intende con \emph{ARS} la sequenza replicante autonomamente o \emph{Ori}. L'identificazione avviene in maniera simile a quella dell'\emph{Ori} di E. coli: frammenti di DNA ottenuti
attraverso enzimi di restrizione vengono introdotti un un plasmide e in cellule del lievito incapaci di crescere in coltura priva di istidina. Le colture in grado di crescervi 
contenevano un'origine di replicazione. Si nota come si trova un \emph{ARS} ogni \num{135000}\si{bs}. 
\subsubsection{Ruolo della struttura cromosomica nella replicazione}
Negli eucarioti non tutte le \emph{Ori} sono utilizzate durante la replicazione: la loro attivazione \`e regolata nella fase $S$ da proteine che regolano il ciclo cellulare, 
l'ambiente locale di cromatina (effetto di posizionamento). \emph{ARS1} si trova vicino al centromero e \emph{ARS501} vicino a un telomero. Il cambio di posizione cambia il momento
di firing dell'\emph{ARS}. Nel lievito le \emph{ARS} vicine al centromero si attivano precocemente. Fattori di replicazione, modificatori della cromatina e rimozioni degli istoni
leggono il codice istonico aprendo e chiudendo la cromatina determinando domini di replicazione precoce e altri di replicazione tardiva. 
\subsubsection{Mappatura fisica di \emph{Ori} attraverso elettroforesi su gel d'agarosio}
Questa tecnica, detta anche ibridizzazione del Southern blot traccia le aperture e i movimenti di un \emph{Ori} in un pezzo di DNA durante la replicazione. Per farlo si isola il 
DNA genomico da una cultura asincrona e lo si taglia con un enzima di restrizione specifico. Si traccia il DNA con una sonda con un Southern blot su gel 2D. Ogni cellula 
rappresenta uno stato intermedio di attivit\`a di replicazione locale. Sul medium a 2D con agarosio con presente \emph{EtBr} pi\`u denso in modo che la forma del frammento influisca 
sulla sua posizione. Il DNA osservato pu\`o assumere diverse forme in base allo stato replicativo: 
\begin{itemize}
	\item Y semplice con un grafico ad arco Y che rappresenta un replicatore passivo.
	\item Y doppia con un grafico ad arco a doppia Y che rappresenta una terminazione.
	\item Bolla simmetrica con un arco a bolla che rappresenta un'iniziazione. 
	\item bolla asimmetrica con un arco a bolla e transizione ad arco ad Y che rappresenta un'iniziazione con un'origine non centrata.
\end{itemize}
\section{Panoramica della replicazione del DNA}
Affinch\`e la cellula si divida deve avvenire la replicazione del DNA. Si intende per replicazione la completa e fedele copia del DNA nei cromosomi della cellula. La replicazione \`e
semi-conservativa: ogni filamento della doppia elica parentale agisce come stampo per la sintesi di un nuovo filamento per la cellula figlia. In modo da copiare lo stampo la base dello
stampo deve essere identificata e deve essere aggiunta la base complementare. Questo garantisce a meno di mutazione che ogni doppia elica figlia sia identica a quella parentale e che
ogni cellula figlia riceva molecole di DNA identiche. 
\subsection{Le fasi della replicazione del DNA}
\subsubsection{Iniziazione}
Durante l'iniziazione viene riconosciuta l'origine di replicazione da una proteina iniziatrice che apre la doppia elica localmente e recluta elicasi. Le DNA elicasi continuano a 
svolgere l'elica per esporre DNA a singolo filamento che \`e circondato da proteine leganti \emph{ssDNA}. L'iniziazione \`e controllata in modo che avvenga una sola volta per ogni 
ciclo cellulare. La sintesi del DNA necessita di un primer in quanto pu\`o aggiungere nucleotidi a una terminazione $3'-OH$ esistente. Il primer \`e un piccolo filamento di RNA 
sintetizzato da una DNA primasi. 
\subsubsection{Allungamento}
Dopo la sintesi del DNA primer la pinza scorrevole simile ad un anello \`e reclutata all'ibrido a doppio filamento ssDNA RNA primer. La DNA polimerasi si lega al DNA attraverso la 
pinza e il macchinario di replicazione o repliosoma si muove lungo il DNA copiando i filamenti. Ogni base nel DNA parentale \`e letta dalla DNA polimerasi che aggiunge basi complementari
al filamento in crescita in una direzione $5'$-$3'$. 
\subsubsection{Terminazione}
La terminazione avviene quando due forcelle diverse si incontrano o quando questa raggiunge la terminazione del cromosoma lineare. Il complesso di replicazione \`e disassemblato, i 
primer a RNA sono rimossi e sostituiti con DNA e la DNA ligasi connette le sequenze di DNA di nuova sintesi. 
\section{Iniziazione}
Le origini di replicazione sono i siti in cui il DNA \`e inizialmente svolto. Le proteine iniziatrici si legano alle origini a siti di legame degli iniziatori permettendo il legame co n
l'elicasi e continuando a svolgere il DNA. Alcuni organismi hanno specifiche sequenze come origine ma \`e l'abilit\`a di legare la proteina iniziatrice che definisce un'origine. Le
proteine inizatrici sono proteine leganti \emph{ATP} \emph{$AAA^+$}. In E. coli si chiama \emph{DnaA}. Negli eucarioti l'inizatore \`e il complesso di riconoscimento dell'origine 
\emph{ORC}, in S. cerevisiae ha $6$ subunit\`a e si chiama \emph{Orc1-6}. L'\emph{ATP} regola il legame dell'iniziatore. Il legame di \emph{ATP} con \emph{Orc1} \`e richiesto per 
il legame di \emph{ORC} con il DNA. Spesso le origini di replicazione possiedono una sequenza di DNA definita con DNA unwinding element, regioni ricche di $AT$ che facilitano lo 
svolgimento in quanto possiedono solo $2$ legami a idrogeno. In queste regioni l'iniziatore separa i due filamenti quando si lega al DNA e recluta altre proteine. 	
\subsection{Svolgimento dell'\emph{Ori} nei procarioti - E. coli}
In E. coli l'origine di replicazione \emph{OriC} \`e una sequenz a $245bp$ con $5$ \emph{DnaA} box di $9bp$, $3$ con alta affinit\`a e $2$ con bassa che legano $15$ molecole di 
\emph{DnaA}. Nelle $3$ ad alta affinit\`a il \emph{DnaA} \`e sempre legato ad essi, mentre in quelle a bassa affinit\`a si lega \emph{DnaA-ATP} solo quando la replicazione deve iniziare. 
Tutte le proteine \emph{DnaA} legano \emph{ATP} e si multimerizzano in un filamento a spirale. Il filamento distorce il DNA producendo un superavvolgimento positivo locale svolgendo 
a valle la regione ricca di $AT$ di $3\times 13bp$ che subisce invece un superavvolgimento negativo. Successivamente \emph{DnaA-ATP} e $6$ molecole di \emph{DnaC-ATP} caricano 
l'anello omoesamerico fomrato dall'elicasi \emph{DnaB} sui singoli filamenti dell'origine. \emph{DnaC} successivamente lascia l'\emph{OriC} a seguito dell'idrolisi dell'\emph{ATP}. 
\emph{DnaB} recluta la DNA primasi \emph{DnaG} che sintetizzer\`a il RNA primer. La pinza scorrevole si lega alla sequenza ibrida ssDNA-RNA primer. 
\subsubsection{Regolazione dello svolgimento di \emph{OriC}}
\paragraph{Inattivazione regolata di \emph{DnaA} (\emph{RIDA})}
La replicazione del DNA \`e un punto di non ritorno: si devono prevenire rireplicazioni alla stessa origine e deve essere integrata con le altre attivit\`a di divisione cellulare. In
E. coli il discriminate \`e la presenza di \emph{DnaA-ATP} contro \emph{DnaA-ADP}: solo il primo pu\`o multimerizzarsi e svolgere il DNA. Dopo l'iniziazione l'\emph{ATPasi $AAA^+$} 
\emph{HDA} lega la pinza e stimola l'idrolisi in \emph{DnaA-ADP} che si dissocia dall'\emph{OriC} e non pu\`o riattivare la replicazoine. \emph{RIDA} \`e il principale meccanismo 
regolatorio che previene la ri-replicazione nei procarioti. 
\paragraph{Metilazione di \emph{OriC}}
L'iniziazione pu\`o essere prevenuta attraverso metilazione del DNA: la DNA adenine metilasi \emph{DAM metilasi} metila i residui $A$ nella sequenza $GATC$ lungo il genoma di E. coli. 
$11$ siti $GATC$ in \emph{OriC} sovrappongono i siti di legame di \emph{DnaA}. Dopo la replicazione solo un filamento di DNA \`e metilato (emimetilazione). La proteina \emph{SeqA} si 
lega ai siti $GATC$ emimetilati e blocca il legame della \emph{Dam  metilasi}. In questo modo previene la metilazione di entrambi i filamenti e il legame di \emph{DnaA} con le sue 
box. Il blocco \`e temporaneo: l'origine \`e completamente metilata sul nuovo filamento dopo $10$ minuti causando la dissociazione di \emph{SeqA} dalla doppio filamento. I cromosomi
completamente metilati si segregano nelle cellule figlie e sono in grado di legare \emph{DnaA}. 
\paragraph{Sequestro di \emph{DnaA} a \emph{datA}}
Il \emph{DnaA} pu\`o legarsi alla sequenza di DNA \emph{datA} che si trova vicino all'\emph{OriC}. La regione pu\`o legare $370$ molecole di \emph{DnaA}. Dopo la replicazione \emph{DnaA}
viene sequestrato alla regione \emph{datA} e non \`e disponibile per il legame in \emph{OriC}. All'inizio di un nuovo ciclo di replicazione cambi locali nella sequenza \emph{datA} 
causano una dissociazione di \emph{DnaA} che pu\`o legare \emph{OriC}. Il \emph{DnaA} \`e attivato in \emph{DnaA-ATP}. 
\subsection{Svolgimento dell'\emph{Ori} negli eucarioti}
Le origini di S. cerevisiae sono simili a quelle di E. coli. Sono lunghe tra $100$ e i $200bp$, siti $A$ e $B1$ a cui si lega \emph{ORC}. I siti $B2$ e $B3$ sono adiacenti ad esse e 
ricchi di $AT$. \emph{ORC} si lega intorno ai siti $A$ e $B1$. La struttura cromatinica \`e importante per le origini negli organismi multicellulari: la replicazione inizia in 
regioni cromosomiche sepcifiche, ma le sequenze di DNA non sono conservate. In Drosophila \emph{ORC} si lega a sequenze \emph{Ori} come le code istoniche sonoiuper acetilate, pertanto
\emph{HAT} potrebbero essere coinvolti nella regolazione delle origini di replicazione. 
\subsubsection{Regolazione dell'attivazione di \emph{Ori} negli eucarioti}
Gli eucarioti possiedono multiple origini di replicazione. Ognuna di esse deve attivarsi una volta per ciclo cellulare. Sono selezionate in $G_1$ e attivate in $S$. Le origini non possono
essere riutilizzate fino alla riselezione nella fase $G_1$ successiva.
\paragraph{Selezione}
Durante la fase $G_1$ \emph{ORC} si lega a un'origine. \emph{ORC} \`e costituito da $6$ subunit\`a: \emph{Orc1-6}. $5$ di esse sono \emph{$AAA^+$ ATPasi}. \emph{ORC} sottost\`a alla
formazione del complesso \emph{pre-RC}, formato da \emph{ORC}, \emph{Cdt1}, \emph{Cdc6} e l'elicasi \emph{MCM2-7}. 
\paragraph{Attivazione}
Durante la fase esse nel \emph{pre-RC} le elicasi \emph{Mcm2-7} vengono fosforilate dalla chinasi \emph{Dbf4} \emph{DDK}. Sono reclutate \emph{Sld2} e \emph{Sld3} per formare il complesso
\emph{SDS} al \emph{pre-RC} e vengono fsforilate dalla chinasi dipendente dalla fase $S$ o \emph{S-Cdk}. Entrambe le fosforilazoini risultano in una completa attivazione del 
\emph{pre-RC}. \emph{Cdt1} e \emph{Cdc6} lasciano il complesso. Avviene il reclutamento dei fattori di iniziazione \emph{Cdc45} e dei complessi \emph{GINS} e \emph{SDS} che permette 
l'apertura e svolgimento dell'origine. La replicazione deve essere completa prima che avvenga la segregazione. Se avviene uno stallo delle forcelle viene attivata la risposta al 
danno del DNA e l'entrata in mitosi \`e bloccata fino alla correzione degli errori. 
\subparagraph{\emph{Rif1} (Rap1 interacting factor 1)}
Questa proteina regola positivamente l'attivamento delle origini precoci e negativamente quello delle tardive. Previene il reclutamento del fattore di iniziazione \emph{Cdc45} al 
\emph{pre-RC}. \`E conservata nelle cellule umane ed \`e il regolatore chiave del programma di replicazione del DNA, ovvero dell'ordine temporale di attivazione delle \emph{Ori}. 
\subsubsection{Regolazione dell'iniziazione della replicazione negli eucarioti multicellulari}
Essendoci molte origini lungo un cromosoma lineare l'ordine di attivazione dipende da:
\begin{itemize}
	\item \emph{Rif1}.
	\item Lo stato di acetilazione della cromatina: \emph{HAT} sono richieste per il reclutamento delle proteine del \emph{pre-RC}.
	\item Lo stato di trascrizione: le \emph{Ori} si trovano spesso vicino a siti di inizio della trascrizione e la forcella di trascrizione produce uno stato di superavvolgimento 
		posivito a monte e negativo a valle, semplice da aprire per la replicazione.
\end{itemize}
In $G_1$ le origini selezionate che verranno attivate nella fase $S$ sono determinate dal punto di decisione delle origini \emph{ODP}. Specifici domini di replicazione sono regioni di 
DNA cromosomiale che contengono origini attivate nello stesso momento. Le proteine richieste per la replicazione sono concentrate in strutture nucleari dette replication factories, dove 
i domini di replicazione co-localizzano con $14$ forcelle di replicazione per factory. La pinza scorrevole resa fosforescente \emph{PCNA-GFP} rende visibile i foci di replicazione al 
microscopio. Le origini dormenti presentano un \emph{pre-RC} assemblato ma non sono attive, ma possono essere attivate velocemente quando una forcella di replicazione vicina entra in 
stallo. Nei mammiferi l'eucromatina nell'interno del nucleo \`e replicata precocemtente, mentre l'eterocromatina nella periferia \`e replicata tardivamente. 
\subsection{DNA elicasi} 
Dopo l'apertura e attivazione all'\emph{Ori} il dsDNA deve essere ancora svolto. Questa operazione viene catalizzata dalla DNA elicasi, un esamero che si lega a un DNA a singolo filamento
aperto alla sequenza \emph{Ori} che si muove in basso verso il doppio filamento per svolgerlo alla forcella di replicazione in svolgimento. L'energia necessaria \`e fornita 
dall'\emph{ATP}. L'elicasi \`e inoltre responsabile del reclutamento di altre proteine richieste per la replicazione: il complesso repliosomico. Il filamento rimanente viene legato 
da una proteina. In E. coli l'elicasi \`e la \emph{DnaB} e possiede $6$ subunit\`a identiche (omoesamero), negli eucarioti ed archea l'elicasi \`e complesso \emph{MCM2-7} e comprende
$6$ subunit\`a diverse (eteroesamero). Ognuna delle $6$ subunut\`a lega \emph{ATP} a coppie causando un cambio di conformazione, mentre l'idrolisi e rilascio di \emph{ADP} causa un
ritorno alla conformazione iniziale. L'elicasi assume un comportamento pulsante: svolge il DNA e si spinge in avanti. L'elicasi batterica si muove lungo il filamento principale, mentre 
quella eucariotica in quello lagging. 
\subsubsection{Risoluzione delle strutture secondarie}
Essendo che ssDNA pu\`o formare strutture secondarie che rende la sua copia difficoltosa e pi\`u sensibile a danno proteine \emph{SSB} nei batteri e \emph{RPA} (replication protein A) 
negli eucarioti si devon legarsi come omotetrameri al ssDNA in modo da tenerlo aperto. Sono fisicamente rimosse durante la polimerizazione del DNA. 
\subsubsection{Risoluzione del superavvolgimento}
Mentre il DNA viene svolto viene introdotto uno stress torsionale in quanto la separazione dei filamenti risulta in un superavvolgimento a valle di essa. Questo rende pi\`u difficile 
per l'elicasi proseguire nella separazione. Per questo devono intervenire topoisomerasi che risolvono il problema rompendo transientemente il DNA e permettendo il rilassamento del 
superavvolgimento. La ligasi successivamente chiude il DNA. Nei procarioti sono presenti solo le topoisomarasi II, negli eucarioti anche le I. 
\subsection{Sintesi di primer a RNA o RNA-DNA dalla DNA primasi}
Dopo l'apertura dell'\emph{Ori} la DNA primasi \emph{DnaG} in E. coli \`e reclutata dalla DNA elicasi. Agisce esclusivamente alla forcella di replicazione producendo una corta sequenza
RNA o RNA-DNA. Non richiede un esistente $3'-OH$ per la sintesi a differenza della polimerasi. La primasi batterica possiede due subunit\`a e crea un primer di $10$-$30$ basi, mentre 
quella eucariotica ne possiede $3$ e crea un primer misto DNA-RNA. Due subunit\`a funzionano come primasi mentre una come DNA polimerasi $\alpha$ aggiungendo un primer di DNA 
all'RNA. L'attivit\`a della primasi \`e unita a quella dell'elicasi e insieme formano il complesso primosoma. Dopo che la subunit\`a DNA polimerasi $\alpha$ della primasi eucariotica
crea una corta lunghezza di DNA la DNA polimerasi replicativa III per i batteri e $\delta$ o $\epsilon$ negli eucarioti la sostituisce e sintetizza il resto del DNA nel processo di 
polimerase switching. Questo avviene ogni volta che un frammento di Okazaki viene creato tra i primer. La polimerasi replicativa viene reclutata dalla pinza scorrevole e determina 
l'inizio dell'allungamento. 
\section{Allungamento}
\subsection{Pinza scorrevole}
L'allungamento inizia con il reclutamento della pinza scorrevole che permette l'alta processivit\`a della DNA polimerasi mantenendola stabilmente legata al DNA. A una sequenza di ssDNA
stampo viene caricata la pinza da una proteina di caricamento. Sia la pinza che il suo caricatore sono conservati in batteri, archea ed eucarioti: 
\begin{itemize}
	\item Pinza scorrevole: pinza $\beta$ nei procarioti e \emph{PCNA} (proliferating cell nuclear antigen) negli eucarioti. 
	\item Proteina caricatrice della pinza: fattore di replicazione $C$: \emph{RFC}.
\end{itemize}
La pinza scorrevole \`e un anello con un buco da \num{35}\si{\angstrom} che racchiude il primer ssDNA. \`E molto stabile e rimane associata con il DNA una volta caricata. Affinch\`e
possa associarsi al DNA il caricatore della pinza, una struttura ad anello a $5$ subunit\`a deve aprirla. Specifiche subunit\`a del caricatore sono \emph{$AAA^+$ ATPasi} e quando lega
\emph{ATP} causano cambi conformazionali che guidano il legame con la pinza, la sua apertura e il reclutamento al DNA. La pinza scorrevole, una volta legata al primer recluta
la DNA polimerasi attraverso un motivo a $8$ amminoacidi. Il caricatore della pinza ha bassa affinit\`a per la pinza scorrevole fino a che non \`e legato all'\emph{ATP}. Quando lo lega
il caricatore lega la pinza. Il complesso ha un alta affinit\`a per il primer a ssDNA. 
\subsubsection{Legame con il primer}
Il legame con il primer stimola l'attivit\`a ATPasica del caricatore che chiude la pinza e si rilascia. L'idrolisi dell'\emph{ATP} riduce l'affinit\`a di \emph{RFC} per il DNA. La
pinza rimane associata con il DNA e recluta l'oloenzima DNA polimerasi permettendo l'inizio dell'allungamento. Il caricatore della pinza rilasciato pu\`o essere ricaricato con
\emph{ATP} per ripeter il processo a un altro primer. Si dice oloenzima un complesso multiproteico in cui un enzima centrale \`e associato con componenti addizionali che ne aumentano 
la funzione. 
\subsection{Sintesi del DNA}
\subsubsection{Materiali richiesti}
La sintesi del DNA richiede il complesso stampo a ssDNA e primer. In vitro il primer \`e a DNA, mentre in vivo \`e a RNA o RNA-DNA. Lo stampo \`e il filamento di ssDNA che \`e letto dalla
DNA polimerasi. Oltre al complesso sono richiesti deossiribonucleotidi: i monomeri \emph{dNTP} come \emph{dATP}, \emph{dCTP}, \emph{dCTP}, \emph{dGTP} e \emph{dTTP}. La sintesi avviene
sempre nella direzione $5'$-$3'$, mentre la lettura nella direzione inversa. 
\subsubsection{Polimerizzazione}
La polimerizzazione del DNA consiste nella formazione dei legami fosfodiestere. \emph{\textsuperscript{$\alpha$}P} in \emph{dNTP} si lega alla terminazione $3'OH$ del primer e viene 
rilasciato pirofosfato \emph{\textsuperscript{$\gamma$}P-\textsuperscript{$\beta$}P}. L'energia che spinge la reazione in avanti deriva dall'idrolisi del pirofosfato da parte della
pirofosfatasi che causa l'irreversibilit\`a della reazione.
\subsection{DNA polimerasi}
Le principali DNA polimerasi processive coinvolte nella replicazione del DNA sono la DNA polimerasi III nei batteri e le DNA polimerasi $\gamma$ e $\epsilon$ negli eucarioti. Le DNA
polimerasi rimangono attaccate al DNA grazie alla pinza scorrevole per lunghe sequenze prima di dissociarsi rendendo la polimerasi processiva. Le polimerasi processive sono altamente 
conservate e contengono multipli domini e regioni con diverse funzioni tra cui la ricerca di errori. I tre domini della DNA polimerasi sono detti pollice, dita e palmo che insieme 
assomigliano a una mano destra. Il DNA in crescita a doppio filamento ``il braccio" si trova nel palmo mentre il ssDNA passa attraverso le dita. I domini delle dita aiutano a posizionare
del nucleotide in arrivo, il pollice mantien il dsDNA allungato ma non contribuisce alla reazione di polimerizzazione. 
\subsubsection{Funzione}
L'addizione corretta di un nucleotide al filamento in sintesi attiva la polimerasi attraverso un cambio conformazionale: la mano rilascia il DNA dopo aver aggiunto il nucleotide e si 
muove di una base per leggere lo stampo da $3'$ a $5'$. Il sito attivo della DNA polimerasi possiede gruppi carbossilati di due residui di aspartato con due ioni $2mg^{2+}$. I siti
attivi catalizzano un tasferimento di fosforile unendo il $5'P$ del nucleotide in arrivo al $3'OH$ del DNA in crescita per formare un legame fosfodiestere. La reazione consiste 
dell'attacco nucleofilo dal $3'OH$ al $\alpha$-fosfato del \emph{dNTP} in arrivo, rilasciando i fosfati $\beta$ e $\gamma$ come pirofosfato. Gli ioni magnesio sono critici: uno attiva 
il priming di $3;OH$ abbassando il suo $pKa$ e favorendo l'attacco nucleofilo, l'altro interagisce con l'ossigeno negativo dei gruppi fosfato $\beta\gamma$ e posiziona $\alpha$-fosfato
vicino al priming $3'OH$. 
\subsection{Fedelt\`a della polimerizzazione del DNA}
La DNA polimerasi processiva fa un errore ogni \num{100000} nucleotidi. L'identit\`a dei monomeri \`e controllata durante e dopo l'addizione di un \emph{dNTP} al filamento in crescita. 
La fedelt\`a \`e mantenuta da ``proofreading". Durante l'addizione la polimerasi riconosce il nucleotide corretto grazie alla forma precisa nel palmo quando accoppiato con lo stampo. 
Nucleotidi scorretti hanno forme diverse e non entrano nel sito attivo con la stessa precisione. Non \`e richiesta alcuna energia per questo processo. 
\subsubsection{Riparazione dell'errore}
Dopo l'aggiunta del nucleotide la DNA polimerasi si muove al nucleotide successivo nel filamento di stampo ma rallenta quando viene aggiunto un nucleotide scorretto. L'operazione di 
proofreading serve a riconoscere questi errori. Successivamente la terminazione scorretta \`e rotta nei legami a idrogeno e viene ``flipped out" nel sito dell'esonucleasi che rimuove 
le basi mal-accoppiate attraverso attivit\`a di delezione $3'$-$5'$ di esonucleasi: le funzioni di polimerasi ed esonucleasi sono spazialmente separate. La terminazione $3'OH$ nel nuovo
filametno \`e successivamente riimmessa nel sito attivo e reinizia la sintesi. Per questo passaggio \`e richiesta energia. 
\subsubsection{Cause di errori}
\paragraph{Tautomeria dei nucleotide}
L'inserzione di un nucleotid errato pu\`o essere dovuta a un riposizionamento transiente dei doppi legami delle quattro basi all'azoto che cambia la posizione dell'idrogeno legato 
al gruppo ammnino o forme tautomeriche. Una volta accoppiate le basi tautomeriche possono riconvertirsi in posizione normale ma la coppia rimane mal-accoppiata. 
\begin{itemize}
	\item La timina si lega con la forma enolica della guanina.
	\item L'adenina si lega con la forma imminica della citosina.
	\item La guanina si lega con la forma enolica della timina.
	\item La citosina si lega con la forma imminica dell'adenina. 
\end{itemize}
Un altro caso \`e la depurinazione delle purine, ovvero rimozione o perdita del nucleotide che pu\`o accadere a bassi pH e pu\`o causare transizioni e trasnversioni. 
\begin{itemize}
	\item Transizioni: una purina \`e sostituita da un'altra purina.
	\item Transversioni: una purina \`e sostituita da una pirimidina.
\end{itemize}
I mal-accoppiamenti tautomerici sono riconosciuti dalla DNA polimerasi e corretti dall'attivit\`a dell'esonucleasi o da sistemi di riparazione post-replicazionali. Quando non
corretti causano delle mutazioni.
\begin{itemize}
	\item Transizioni (di entrambi i tipi) una purina \`e sostituita da un'altra purina ($A\leftrightarrow G$).
	\item Transversioni (del secondo tipo) una purina \`e sostituita da una pirimidina: ($A\leftrightarrow C\lor T$, $G\leftrightarrow C\lor T$).
\end{itemize}
\paragraph{Inclusione di un \emph{NTP}}
Le concentrazioni intracellulari di \emph{NTP} sono molto pi\`u alte rispetto a quelle di \emph{dNTP}. Questi vengono discriminati in base alla posizione $2'OH$. La DNA polimerasi 
contiene anche una tirosina nel dominio catalitico che si scontra con il $2'OH$ del ribosio prevenendo il loro posizionamento nel sito catalitico e la formazione del legame fosfodiestere.
La capacit\`a di riconoscimento di questi errori \`e diversa per ogni DNA polimerasi. Viene comunque inserita la base corretta per l'accoppiamento e viene corretta attraverso un processo
di riparazione post-replicatorio attraverso \emph{RNasi H1} e \emph{RNasi H2} che rimuovono il ribonucleotide dal filamento e una DNA polimerasi non processiva e una DNA ligasi 
chiudono il buco. 
\subsection{Sintesi del DNA discontinua}
La DNA polimerasi \`e in grado di sintetizzare in direzione $5'$-$3'$ ma entrambe le forcelle si muovono bidirezionalmente. Pertanto per forcella un filamento pu\`o essere sintetizzato
$5'$-$3'$ usando un primer a RNA nel mezzo della bolla e la polimerasi segue l'elicasi. Questo viene detto filamento continuo o guida. Il secondo filamento deve essere sintetizzato in 
maniera discontinua da $5-$ a $3'$ e viene detto filamento ritardato. Nessun primer a monte \`e disponibile sul DNA svolto mentre l'elicasi si muove. Viene detto ``lagging" in 
quanto la sintesi \`e pi\`u lenta rispetto all'altro filamento in quanto dipende dalla produzione di molti primer. Corti primer a RNA o RNA-DNA sono posti alla forcella di replicazione
da una DNA primasi legata alla DNA elicasi nel complesso primosoma. Nei procarioti i primer a RNA sono posti ogni \num{1000}-\num{2000} basi con superavvolgimenti a monte. Negli 
eucarioti invece gli intervalli sono di \num{150}-\num{200} basi in quanto lo svolgimento \`e pi\`u difficile della cromatina con i nucleosomi. La terminazione $3'OH$ del primer
\`e il sito di inizio per la DNA polimerasi e la sintesi di un pezzo di DNA da un primer al successivo viene detto frammento di Okazaki. 
\subsubsection{Maturazione dei frammenti di Okazaki in E. coli}
Il DNA viene sintetizzato come frammenti corti e discontinui e i frammenti sono uniti quando i primer a RNA sono rimossi. In E. coli la DNA polimerasi III sintetizza il DNA che si 
estende dal primer e si dissocia dal DNA quando raggiunge il primer successivo, mentre la pinza scorrevole rimane attaccata. La DNA polimerasi I rimuove il primer attraverso 
``nick trasnlation" e riempie il buco lasciato dal primer con DNA, mentre il nick viene unito da una DNA ligasi. 
\paragraph{DNA polimerasi I}
La DNA polimerasi I di E. coli \`e una polimerasi ad alta fedelt\`a e non processiva hce viene usata per rimuovere i primer a RNA e per riparare a danni del DNA. Ha una struttura diversa 
rispetto a quella processiva. \`E una proteina con capacit\`a di polimerasi $5'$-$3'$ ed esonucleasi in entrambe le direzioni. L'enzima pu\`o essere rotto con la proteasi subtilisina, 
e l'esonucleasi e genera il frammento di Klenow con attivit\`a esonucleasica che viene usata per rimouvere overhang in $OH3'$ o riempirlo in $5'$ per rendere blunt le terminazioni di 
DNA trattato con enzimi di restrizione. 
\paragraph{DNA ligasi}
La DNA ligasi \`e un enzima che catalizza la formazione del legame fosfodiestere tra due molecole di DNA. Lo fa rimuovendo $\gamma\beta$-fosfato in $5'$ e favorendo l'attacco nucleofilo
grazie allo ione magnesio. 
\subsubsection{Maturazione dei frammenti di Okazaki negli eucarioti}
La rimozione del primer e il riempimento dello spazio nel nuovo DNA \`e diversa negli eucarioti. La DNA polimerasi $\delta$ sintetizza il DNA partendo da un primer fino a che 
raggiunge il successivo. Successivamente sposta il primer a valle insieme al frammento di Okazaki creando un ``flap". La Flap endonucleasi \emph{Fen1} lo taglia (con omologo presente in
archea). La DNA polimerasi $\delta$ successivamente si stacca dal DNA e la DNA ligasi unisce il nick. La DNA polimerasi $\delta$ viene reclutata da una pinza scorrevole alla terminazione
$3'OH$ al primer successivo. 
\subsection{Attivit\`a del replisoma alla forcella di replicazione}
La DNA polimerasi replicativa \`e parte di un grande complesso: il replosoma. Questo \`e composto da $3$ DNA polimerasi, $3$ pinze scorrevoli, $3$ proteine Tau, un caricatore della
pinza, una DNA elicasi, una DNA primasi, \emph{SSB/RFC}, e solo nei procarioti da una DNA polimerasi I. La proteina Tau $\tau$ lega la DNA polimerasi al caricatore della pinza. Si 
trova un replisoma per forcella di replicazione e sintetizza contemporaneamente sia il filamento principale che quello ritardato. 
\begin{itemize}
	\item Negli eucarioti la DNA polimerasi $\epsilon$ sintetizza il filamento guida, mentre le DNA polimerasi $\alpha$ e $\delta$ sintetizzano quello ritardato.
	\item Nei procarioti una DNA polimerasi III sintetizza il filamento guida, mentre due DNA polimerasi III sintetizzano quello ritardato.
\end{itemize}
\subsubsection{Il modello a trombone}
Le DNA polimerasi si muovono in direzioni diverse sui due filamenti, ma si muovono seguendo la forcella di replicazione e la stessa elicasi. Questo viene permesso dalla creazione di 
un loop da parte del filamento ritardato. Il loop deve essere rilasciato dalla polimerasi ogni paio di secondi. Si nota come le polimerasi sul filamento guida e ritardato si trovano
insieme spazialmente e sono regolate insieme. 
\section{Terminazione}
Quando la replicazione inizia il processo non si ferma fino a quando il replicone \`e replicato. Nei batteri le forcelle a destra e a sinistra si fondono su un plasmide o sul genoma
batterico circolare, mentre negli eucarioti una bolla si fonde con un'altra in arrivo da destra e un'altra in arrivo da sinistra sul cromosoma lineare. 
\subsection{Terminazione nei batteri}
Nei batteri una sequenza di terminazione \emph{Ter} determina dove entrambe le forcelle si fondono. \`E composta da $10$ ripetizioni di $23bp$ ed \`e localizzata diametricamente opposta
a \emph{OriC}. Le ripetizioni di \emph{Ter} sono $5$ in senso orario e $5$ antiorario. La proteina terminatrice \emph{Tus} si lega a ognuna delle $10$ sequenze \emph{Ter} fermando 
la forcella di replicazione, causando il disassemblaggio del complesso di replicazione e il corto pezzo di DNA non replicato \`e riempito dalla DNA polimerasi I e chiuso dalla DNA
ligasi. \emph{Tus} possiede due facce: una non permissiva che blocca la replicazione e una permissiva che permette la sua continuazione. Dopo la terminazione e ligazione i genomi 
sono incastrati e vengono liberati da topoisomerasi durante la decatenazione. 
\subsection{Terminazione negli eucarioti}
In alcune regioni dei cromosomi eucariotici la replicazione \`e bloccata in una direzione per impedire scontri con il macchinario di trascrizione che arriva da un altro lato. Nel
lievito tale sito \`e detto barriera della forcella di replicazione e si trova nel rDNA. La proteina \emph{FOB1} si lega al sito e previene la progressione della forcella di replicazione.
\subsubsection{Replicazione alla fine di un cromosoma lineare}
Il meccanismo per la sintesi del filamento ritardato non pu\`o replicare la terminazione di un cromosoma lineare in quanto la rimozione dell'ultimo primer da parte di \emph{RNAasi 
H} lascia un vuoto che non pu\`o essere riempito, pertanto una porzione del cromosoma (telomero) potrebbe essere accorciata durante diversi cicli di replicazione. 
\section{Replicazione dei telomeri}
Il problema della replicazione delle terminazioni dei cromosomi \`e risolto dai virus \emph{T4} unendo diverse copie del cromosoma lineare che poi si dissociano. Gli eucarioti invece
possiedono una telomerasi per prevenire il problema della replicazione delle terminazioni. I telomeri variano tra i $100bp$ e i \num{20000}\si{bp} in base alla specie. Si 
accorciano ad ogni ciclo di replicazione e possono essere allungati attraverso la telomerasi che aggiunge nuove sequenze telomeriche alla terminazione. Nella Drosophila i \emph{non-LTR}
retrotrasposoni \emph{HeT-A} e \emph{TART} si traspongono ripetutamente nelle terminazioni cromosomiche per produrre una regione simile ai telomeri. Il mantenimento della 
lunghezza del telomero attraverso eventi di trasposizione addizionali. 
\subsection{Telomerasi}
La telomerasi \`e una speciale DNA polimerasi, una ribonucleoproteina \emph{RNP} formata da un'enzima e una molecola di RNA. L'enzima o \emph{TERT}, telomerasi trascrittasi inversa
\`e conservata negli eucarioti. L'RNA fornisce lo stampo per la sintesi delle ripetizioni della telomerasi. La telomerasi si lega al overhang di DNA nel telomero a singolo filamento. 
L'overhang alla terminazione $3'$ si lega con l'RNA che viene usato come stampo per sintetizzare DNA. La sintesi viene ripetuta pi\`u volte. 
\subsection{Mantenimento della lunghezza}
Il mantenimento della lunghezza dei telomeri viene raggiunto in due passi:
\begin{enumerate}
	\item Allungamento dell'overhang $3'$: la telomerasi si lega all'overhang $3'$, sintetizza del DNA, lo allunga, si trasloca e riallunga il filamento.
	\item Replicazione del filamento complementare: viene sintetizzato un primare da una primasi, il gap viene riempito da una polimerasi, il primer viene rimosso e una ligasi 
		lega i filamenti.
\end{enumerate}
Si noti come rimane comunque un overhang che \`e coinvolto nella formazione del D-loop. 
\subsection{Problemi nel mantenimento della lunghezza del telomero}
Nelle cellule somatiche adulte il gene che codifica \emph{hTERT} non \`e ben espresso a causa di repressione epigenetica, ma altamente espresso in cellule fetali e staminali. I telomeri
si accorciano nelle cellule causando la risposta del danno al DNA e un arresto \emph{p53} dipendente e morte cellulare. Inoltre la repressione epigenetica pu\`o impedire il 
legame della telomerasi con la cromatina del telomero (approfondire). 
\subsection{Il limite di Haflick}
La lunghezza dei telomeri alla nascita \`e di \num{15000}\si{bp} e per divisione cellulare il telomero si accorcia di $300$-\num{1000} coppie di basi. Ad un certo punto tra le $50$ e le
$60$ divisione si raggiunge il limite di Hayflick in cui una cellula smette di dividersi per evitare ulteriore erosione del cromosoma e raggiunge la senescenza. Molte cellule 
cancerogene possiedono una telomerasi sovraespressa e continuano a dividersi e proliferare o uno stato epigenetico alterato ai telomeri che porta alle stesse conseguenze. 
\section{Correzione degli errori post-replicativa}
Il DNA mismatch repair \emph{MMR} non \`e parte dell'attivit\`a della DNA polimerasi e riduce il tasso di errore di $100$ volte. Un nucleotide mismatch causa una distorsione 
locale nella doppia elica riconosciuta dal dimero \emph{MutS} che si lega ad essa e recluta il dimero \emph{MutL} che stabilizza il legame con il DNA distorto. La riparazione avviene
sul nucleotide di nuova sintesi, riconosciuto grazie alla metilazione sullo stampo di $GATC$. Il compleso \emph{MutS-MutL} recluta l'endonucleasi \emph{MutH} che si lega alla 
sequenza metilata sul filamento stampo pi\`u vicino al mismatch, taglia il filamento di nuova sintesi vicino alla sequenza, aiutata occasionalmente dall'elicasi \emph{UvD}. L'esonucleasi
$3'$-$5'$ rimuove il DNA fino al sito di errore e la DNA polimerasi III lo risintetizza. Quando si incontra il DNA dopo il taglio i filamenti sono riattaccati da una ligasi. 
L'efficienza di riparazione si abbassa allontanandosi da $GA^{me}TC$
\section{Mantenimento delle modifiche istoniche}
Si nota come i nucleosomi sono in parte rimossi dal replisoma in movimento. Le proteine istoniche devono pertanto essere reclutate ai posti corretti nel DNA appena reclutato e mantenere
le modifiche epigenetiche (eredit\`a epigenetica). I nucleosomi sono distrutti dalla forcella di replicazione e devono rilegarsi dopo che questa li ha superata e le modifiche epigenetiche
sono aggiunte su essi. Questo avviene in quanto il nucleosoma parentale non \`e rimosso: $50\%$ sono distribuiti equamente tra i filamenti singoli: avviene una segregazione degli istoni
parentali. Di questi solo quelli marcati epigeneticamente \emph{2(H3-H4)} sono mantenuti, mentre gli altri sono rimossi anche se rimangano vicino. Quelli non marcati sono inclusi 
nei nuovi nucleosomi attraverso il chaperone \emph{CAf1} insieme ai \emph{2(H2A-H2B)} parentali o nuovi attraverso il chaperone \emph{NAP1}. Pertnato si trovano $4$ tipi di nucleosomi
nei filamenti sintetizzati. Infine il nucleosoma parentale originale serve come stampo per produrre lo stato epigenetico su tutti i nucleosomi che sono rimasti vicini. 
\section{DNA polimerasi specializzate}
Tutte le DNA poliemrasi catalizzano la stessa sintesi del DNA ma differiscono nella velocit\`a e tasso di errore. 
\begin{itemize}
	\item DNA polimerasi replicative: sono processive e catalizzano la replicazione dei genomi durante la fase $S$, hanno un'alta fedelt\`a di polimerizzazione e un'attivit\`a
		esonucleasica $3'$-$5'$ altamente efficiente. 
	\item DNA polimerasi non-processive o distributive: rappresentano la maggior parte di DNA polimerasi, polimerizzano corte lunghezze per mantenerlo in uno stato sano (riparazione),
		hanno bassa fedelt\`a, sintetizzano il DNA sul filamento danneggiato o sintesi di translezione \emph{TLS}.
\end{itemize}
Certe DNA polimerasi hanno altre attivit\`a e tutte sono reclutate dalla pinza scorrevole sul DNA che permette loro di stare attaccate al DNA, altrimenti dopo ogni addizione di 
nucleotide potrebbero separarsi. 
\subsection{DNA polimerasi batteriche}
Altamente conservate in tutti i procarioti E. coli possiede $5$ DNA polimerasi:
\begin{itemize}
	\item DNA polimerasi I: attiva nella riparazione del DNA ed estende e matura i frammenti di Okazaki.
	\item DNA polimerasi II: attiva nella riparazione del DNA.
	\item DNA polimerasi III: polimerasi replicatica che catalizza la polimerizzazione del DNA a grande lunghezza.
	\item DNA polimerasi IV e V: riparano danni al DNA che bloccano il replisoma. 
\end{itemize}
\subsection{DNA polimerasi specializzate}
\subsubsection{DNA polimerasi \emph{TLS}}
Esistono in tutti gli organismi e sono le DNA polimerasi per la sintesi di traslesione. Il DNA in una cellula subisce danno costante e deve essere riparato, in particolare il dimero di 
timidina nello stampo del filamento singolo causa un arresto della forcella, rottura edl doppio filamento e morte cellulare. Le DNA polimerasi \emph{TLS} promuovono la replicazione, ma
hanno alto tasso di errore a causa dell'assenza dell'attivit\`a nucleasica. Quando il dimero di timidina causa uno stacco della DNA polimerasi III ad esso si lega la DNA polimerasi 
\emph{TLS} che avendo un sito di legame pi\`u aperto riesce a superare il blocco e continuare per un breve tratto la sintesi con minore fedelt\`a prima di essere di nuovo sostituita
dalla DNA polimerasi III. 
\subsubsection{Trascrittasi inversa}
La trascrittasi inversa \`e una DNA polimerasi dipendente da RNA: copia l'RNA in ssDNA o cDNA. Usa un singolo filamento di RNA come stampo e necessita di un primer. Sono molto simili
alle DNA polimerasi e sono codificate da virus e retrotrasposoni, elementi di DNA mobile negli eucarioti. La telomerasi \`e una trascrittasi inversa specializzati. I virus a RNA
o retrovirus come \emph{HIV-1} e influenza A hanno un genoma a RNA che codifica per una trascrittasi inversa nella cellula ospiteL il gema \`e trascritto in dsDNA che si integra 
nel genoma ospite. 

