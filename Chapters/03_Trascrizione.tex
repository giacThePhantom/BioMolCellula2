\chapter{Trascrizione}
\section{Panoramica della trascrizione}
L'informazione conservata nel DNA \`e utilizzata per creare proteine o molecole di RNA funzionale. Si intende per trascrizione il processo di copia di un filamento di DNA in una molecola
di RNA detta trascritto. Il processo di trascrizione viene svolto da una RNA polimerasi. Un cofattore esterno RNA polimerasi elicasi separa i filamenti di DNA e permette i 
ribonucleosidi trifosfato di accoppiarsi con il filamento stampo. Per produrre una proteina da una molecola di RAN la sequenza di RNA \`e letta dal ribosoma nella traduzione. L'RNA
in questo processo viene detto RNA messaggero \emph{mRNA}. Si nota come la RNA polimerasi non possegga attivit\`a esonucleasica e pertanto non possa correggere errori di 
mal-accoppiamento. Il tasso di errore \`e di $10^{-4}$ e il tasso di da $40$ a $80$ nucleotidi al secondo. Si nota come rispetto alla DNA polimerasi \`e pi\`u lenta, inefficiente e 
meno accurata. 
\subsection{Il processo di trascrizione}
La trascrizione pu\`o essere divisa in iniziazione, allungamento e terminazione. Inizia quando la RNA polimerasi si lega a una sequenza di DNA che precede il gene: il promotore. Il 
sito di inizio di trascrizione \emph{TSS} \`e la prima base ad essere trascritta ed \`e notata con $+1$. L'RNA viene trascritto nella direzione $5'$-$3'$ con il filamento letto in 
direzione opposta come nella sintesi del DNA. Pertanto si indica come basi a monte quelle $5'$ e a valle quelle $3'$. 
\subsubsection{Iniziazione}
Durante l'iniziazione la RNA polimerasi separa i filamenti di DNA per creare una bolla di trascrizione tra le $12$ e le $14bp$ e inseriscce i primi ribonucleoside trifosfati \emph{NTP} 
mentre si trova al promotore. 	Quando il RNA \`e di lunghezza sufficiente la RNA polimerasi lascia il promotore ``promoter clearance" e cambia conformazione per essere pi\`u 
stabilmente associata con il DNA permettendo l'allungamento del RNA. 
\subsubsection{Allungamento}
L'allungamento inizia dopo la clearance del promotore e la RNA polimerasi si muove lungo il DNA aggiungendo ribonucleotidi e allungando il trascritto di RNA. La RNA polimerasi 
svolge il DNA a valle e close quello a monte mantenendo la dimensione della bolla di trascrizione di dimensione costante per impedire la fomrazione di R-loop. Nella bolla una
regione del trascritto di $8$-$10bp$ \`e accoppiata con il DNA mentre il resto \`e estruso dalla polimerasi.
\subsubsection{Terminazione}
L'allungamento continua fino a che la polimerasi incontra una sequenza di DNA detta terminatore che segnala la fine della sintesi di RNA. Il RNA \`e rilasciato e la RNA polimerasi si 
dissocia dal DNA. 
\subsection{Nomenclatura dei geni}
In un gene viene indicato con $+1$ il sito di inizio della trascrizione \emph{TSS}, con numeri negativa la zona del promotore e con \emph{TTS} il sito di terminazione della trascrizione.
Si dice con prossimale la zona pi\`u vicina al \emph{TSS}, distale quella che si trova allontanandosi verso il \emph{TTS}. Si intende per sequenza codificante \emph{CDS} la sequenza del 
gene senza introni, mentre con open reading frame \emph{ORF} la sequenza con gli introni. 
\subsection{Regolazione della trascrizione}
La trascrizione \`e regolata per produrre il RNA richiesto al tempo corretto. La cromatina negli eucarioti presenta una sfida per la trascrizione in quanto i nucleosomi prevengono
il legame e il movimento del macchinario di trascrizione attraverso la cromatina. Sono pertanto richiesti:
\begin{itemize}
	\item Rimodellamento dei nucleosomi: per riposizionare gli istoni lontano dal DAN che deve essere trascritto. 
	\item Chaperone degli istoni per riassemblare e disassemblare i dimeri nucleosoma-istone.
	\item Enzimi che modificano le proteine istoniche epigeneticamente per permettere o prevenire il legame di proteine che regolano la trascrizione.
\end{itemize}
\section{L'enzima centrale della RNA polimerasi}
Se i procarioti possiedono $1$ RNA polimerasi gli eucarioti ne possiedono $3$ principali:
\begin{itemize}
	\item RNA polimerasi I \emph{Pol I}: trascrive i geni di grandi RNA ribosomali \emph{rDNA}. Agisce nei nucleoli $5$-$6$ negli umani, $1$ nel lievito.
	\item RNA polimerasi II \emph{Pol II}: trascrive i geni di RNA messaggero \emph{mRNA}, RNA non codificanti corti e lunghi, \emph{miRNA} con ruolo nella regolazione 
		dell'espressione genica attraverso interferenza e \emph{snRNA} RNA piccoli nucleari con un ruolo nel processamento di \emph{mRNA}. Agisce nel nucleoplasma.
	\item RNA polimerasi III \emph{Pol III}: trascrive una variet\`a di RNA come tutti gli RNA transfer \emph{tRNA}, il piccolo $5S$ RNA ribosomale, \emph{snRNA U6} componente 
		dello spliceosoma, \emph{7SL RNA} o long non-coding RNA parte della particella di riconoscimento del segnale che regola la traduzione. Agisce sia nei nucleoli che nel
		citoplasma.
\end{itemize}
Le piante possiedono una quarta RNA polimerasi con ruolo di trascrizione degli RNA non codificanti con ruolo nell'espressione genica. Gli eucarioti possiedono anche RNA polimerasi 
mitocondriali. Si intende per nucleoli raggruppamenti non circondati da membrana di $6$ ripetizioni di rDNA trascritti da \emph{Pol I} e \emph{Pol III}. 
\subsection{Struttura}
Le RNA polimerasi sono formate da diverse subunit\`a che formano il nucleo polimerasico. 
\subsubsection{RNA polimerasi batteriche}
Le RNA polimerasi batteriche sono le pi\`u piccole con $5$ subunit\`a che si assemblano in un complesso nucleare con lobi simili a mascelle che formano una pinza. Le mascelle sono formate
dalle subunit\`a $\beta$ e $\beta'$ con le due $\alpha$ e $\omega$ alla base. La base della fessura \`e il sito attivo dell'enzima. Ogni subunit\`a $\alpha$ possiede un dominio $N$ 
terminale \emph{$\alpha$NTD} e un dominio $C$ terminale \emph{$\alpha$CTD} unite da un collegatore flessibile. 
\paragraph{Subunit\`a e loro ruolo}
\begin{itemize}
	\item $\alpha$ ($2$): responsabile per l'assemblaggio del complesso.
	\item Domini $N$ terminali: interagiscono con le subunit\`a $\beta$ e $\beta'$. 
	\item Domini $C$ terminali: si legano al DNA promotore: elemento a monte.
	\item $\beta$ ($1$): contengono il sito catalitico, formano i legami fosfodiestere legando \emph{\ce{Mg2+}} e svolgono la correzione degli errori.
	\item $\beta'$ ($1$): mantiene l'enzima legato al filamento stampo, svolge e riavvolge il dsDNA. 
	\item $\omega$ ($1$): \`e un chaperon: promuove la stabilit\`a strutturale della RNA polimerasi.
\end{itemize}
\subsubsection{RNA polimerasi di eucarioti ed archea}
Tutte le RNA polimerasi possiedono le $5$ subunit\`a centrali viste prima altamente conservate, specialmente al sito attivo. Si trovano ulteriori subunit\`a in RNA polimerasi di archea
ed eucarioti ordinate intorno alle $5$ centrali. L'attivit\`a catalitica pertanto rimane conservata in tutti e tre gli alberi della vita. 
\paragraph{Ruoli della RNA polimerasi II} 
Oltre a trascrivere il DNA la RNA polimerasi II accoppia la trascrizione con il processamento del trascritto di RNA: il dominio $C$ terminale \emph{CTD} di \emph{Pol II} \`e cruciale per 
questa funzione: \`e la terminazione della subunit\`a \emph{Rpb1} e esiste come ripetizioni di una sequenza di $7$ amminoacidi: \emph{Tyr-Ser-Pro-Thr-Ser-Pro-Ser} con $26$ ripetizioni
nel lievito e $52$ negli umani. 	
\section{Riconoscimento dei promotori}
Il nucleo della RNA polimerasi pu\`o sintetizzare il RNA ma non pi\`o riconoscere e legarsi alla sequenza promotrice di un gene. Si rendono pertanto necessarie ulteriori subunit\`a
che si legano direttamente al promotore. Si forma pertanto un oloenzima dalla loro unione con l'enzima nucleare.
\subsection{Batteri}
Nei batteri le subunit\`a che si legano al promotore sono dette fattori $\sigma$. Ne esistono diverse che riconoscono promotori specifici per promuovere la trascrizione di geni specifici 
in base a particolari condizioni di crescita. I fattori $\sigma$ si legano a sequenze che definiscono i promotori batterici composti tipicamente da due elementi: uno a $-35$ e uno a
$-10$ detto box di Pribnow. Ogni fattore sigma ha una sequenza di legame specifico e una particolare spaziazione tra i due elementi. Questo in modo da regolare la trascrizione: pi\`u
vicine le sequenze e la spaziazione a quella particolare del fattoreo $\sigma$  pi\`u forte il legame e pi\`u alti tassi di trascrizione. 
\subsubsection{Fattori sigma}
Nell'oloenzima della RNA polimerasi il fattore $\sigma70$ \`e costitutivo e $3$ dei suoi $4$ domini riconoscono specifici elementi dei promotori:
\begin{itemize}
	\item Il dominio $2$ si lega alla regione $-10$ e aiuta a separare il dsDNA ``promoter melting".
	\item Il dominio $3$ riconosce le due basi della regione estesa $-10$.
	\item Il dominio $4$ riconosce l'elemento $-35$, \`e attaccato a una parte flessibile del nucleo dell'enzima che permette diversa spaziazione tra $-35$ 3 $-10$. 
\end{itemize}
Alcuni fattori sigma sono regolati in risposta a condizioni ambientali o di sviluppo. Possono essere regolati sia a livello trascrizionale o traduzionale o alterando la stabilit\`a 
del proprioi mRNA. Sono inoltre regolati da:
\begin{itemize}
	\item Fattori pro-sigma: proteine con domini inibitori che devono essere rotte prima che il fattore $\sigma$ possa associarsi con l'enzima RNA polimerasi. 
	\item Fattori anti-sigma: proteine che legano i fattori $\sigma$ inibendo la loro funzione. 
\end{itemize}
\paragraph{Fattori anti-sigma e regolazione dell'assemblaggio del flagello in Salmonella typhimurium}
Mentre le proteine che formano la base del flagello sono sintetizzate il fattore anti-sigma \emph{FlgM} si lega a $\sigma^F$ impedendo il suo legame con la RNA polimerasi-$\sigma^{70}$. 
$\sigma^F$ promuove la trascrizione di geni necessari per il completamento dell'assemblaggio del flagello. Negli ultimi passi della sintesi delle proteine del flagello \emph{FlgM} 
\`e esportata dalla cellula in modo che $\sigma^F$ possa legarsi alla RNA polimerasi sostituendo $\sigma^{70}$ e promuovere la trascrizione dei geni dell'assemblaggio permettendo
la creazione del flagello. 
\subsection{Eucarioti}
Anche i promotori eucariotici e di archea necessitano di ulteriori proteine per direzionare la RNA polimerasi ai promotori. Questi sono detti fattori di trascrizione generali
\emph{TF} seguito dal numero della \emph{Pol}. Negli eucarioti \emph{TFII} si assemblano al promotore e con l'enzima del nucleo formano il complesso di pre-iniziazione. \emph{Pol I} e 
\emph{Pol III} richiedono diverse proteine per formare il complesso di pre-iniziazione. I promotori per la RNA \emph{Pol II} possiedono spesso una TATA box con sequenza di consenso
$TATAA$ tra le $25$ e le $30bp$ a monte dell'inizio della trascrizione. Tutte le polimerasi eucariotiche necessitano della $TATA$ binding protein \emph{TBP} per iniziare la trascrizione
che viene riconosciuta da \emph{TBP} associated factors \emph{TAF} formando il complesso \emph{TFIID}. Alcuni altri elementi includono l'elemento di riconoscimento \emph{TFIIB} 
\emph{BRE}, l'elemento iniziatore \emph{INR} e l'elemento promotore a valle \emph{DPE}. Molti promotori non hanno questi elementi e la loro variabilit\`a rende difficile la loro 
predizione. Il complesso di pre-iniziazione contiene $32$ proteine fattori di trascrizione generali e la RNA \emph{Pol II} con $12$ subunit\`a. I promotori per \emph{Pol II} possono
dividersi in:
\begin{itemize}
	\item Prossimali: distanti meno di $200bp$ da \emph{TSS} nei lieviti unicellulari.
	\item Distali: distanti fino a $10kb$ dal \emph{RSS} come negli eucarioti multicellulari. 
\end{itemize}
\subsection{Formazione del complesso di pre-inizio}







\section{Iniziazione della trascrizione e transizione a un complesso di allungamento}

\section{Allungamento della trascrizione}

\section{Terminazione della trascrizione}

\section{Principi della regolazione della trascrizione}

\section{Domini leganti il DNA in proteine che regolano la trascrizione}

\section{Meccanismi per regolare l'iniziazione della trascrizione nei batteri}

\section{L'operone \emph{lac} in E. coli}

\section{L'operone triptofano \emph{trp} in E. coli}

\section{Regolazione della trascrizione da parte di riboswitches trascritti}

\section{Regolazione dell'espressione genica del batteriofago $\lambda$ in E. coli}

\section{Regolazione della trascrizione da sistemi di trasduzione del segnale a due componenti}

\section{Regolazoine dell'iniziazione della trascrizione ed allungamento negli eucarioti}

\section{Il ruolo delle cascate di segnalazione nella regolazione della trascrizione}

\section{Silenziamento genico attraverso imprinting genomico}
