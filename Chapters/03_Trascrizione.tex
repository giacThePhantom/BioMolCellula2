\chapter{Trascrizione}
\section{Panoramica della trascrizione}
L'informazione conservata nel DNA \`e utilizzata per creare proteine o molecole di RNA funzionale. Si intende per trascrizione il processo di copia di un filamento di DNA in una molecola
di RNA detta trascritto. Il processo di trascrizione viene svolto da una RNA polimerasi. Un cofattore esterno RNA polimerasi elicasi separa i filamenti di DNA e permette i 
ribonucleosidi trifosfato di accoppiarsi con il filamento stampo. Per produrre una proteina da una molecola di RAN la sequenza di RNA \`e letta dal ribosoma nella traduzione. L'RNA
in questo processo viene detto RNA messaggero \emph{mRNA}. Si nota come la RNA polimerasi non possegga attivit\`a esonucleasica e pertanto non possa correggere errori di 
mal-accoppiamento. Il tasso di errore \`e di $10^{-4}$ e il tasso di da $40$ a $80$ nucleotidi al secondo. Si nota come rispetto alla DNA polimerasi \`e pi\`u lenta, inefficiente e 
meno accurata. 
\subsection{Il processo di trascrizione}
La trascrizione pu\`o essere divisa in iniziazione, allungamento e terminazione. Inizia quando la RNA polimerasi si lega a una sequenza di DNA che precede il gene: il promotore. Il 
sito di inizio di trascrizione \emph{TSS} \`e la prima base ad essere trascritta ed \`e notata con $+1$. L'RNA viene trascritto nella direzione $5'$-$3'$ con il filamento letto in 
direzione opposta come nella sintesi del DNA. Pertanto si indica come basi a monte quelle $5'$ e a valle quelle $3'$. 
\subsubsection{Iniziazione}
Durante l'iniziazione la RNA polimerasi separa i filamenti di DNA per creare una bolla di trascrizione tra le $12$ e le $14bp$ e inseriscce i primi ribonucleoside trifosfati \emph{NTP} 
mentre si trova al promotore. 	Quando il RNA \`e di lunghezza sufficiente la RNA polimerasi lascia il promotore ``promoter clearance" e cambia conformazione per essere pi\`u 
stabilmente associata con il DNA permettendo l'allungamento del RNA. 
\subsubsection{Allungamento}
L'allungamento inizia dopo la clearance del promotore e la RNA polimerasi si muove lungo il DNA aggiungendo ribonucleotidi e allungando il trascritto di RNA. La RNA polimerasi 
svolge il DNA a valle e close quello a monte mantenendo la dimensione della bolla di trascrizione di dimensione costante per impedire la fomrazione di R-loop. Nella bolla una
regione del trascritto di $8$-$10bp$ \`e accoppiata con il DNA mentre il resto \`e estruso dalla polimerasi.
\subsubsection{Terminazione}
L'allungamento continua fino a che la polimerasi incontra una sequenza di DNA detta terminatore che segnala la fine della sintesi di RNA. Il RNA \`e rilasciato e la RNA polimerasi si 
dissocia dal DNA. 
\subsection{Nomenclatura dei geni}
In un gene viene indicato con $+1$ il sito di inizio della trascrizione \emph{TSS}, con numeri negativa la zona del promotore e con \emph{TTS} il sito di terminazione della trascrizione.
Si dice con prossimale la zona pi\`u vicina al \emph{TSS}, distale quella che si trova allontanandosi verso il \emph{TTS}. Si intende per sequenza codificante \emph{CDS} la sequenza del 
gene senza introni, mentre con open reading frame \emph{ORF} la sequenza con gli introni. 
\subsection{Regolazione della trascrizione}
La trascrizione \`e regolata per produrre il RNA richiesto al tempo corretto. La cromatina negli eucarioti presenta una sfida per la trascrizione in quanto i nucleosomi prevengono
il legame e il movimento del macchinario di trascrizione attraverso la cromatina. Sono pertanto richiesti:
\begin{itemize}
	\item Rimodellamento dei nucleosomi: per riposizionare gli istoni lontano dal DAN che deve essere trascritto. 
	\item Chaperone degli istoni per riassemblare e disassemblare i dimeri nucleosoma-istone.
	\item Enzimi che modificano le proteine istoniche epigeneticamente per permettere o prevenire il legame di proteine che regolano la trascrizione.
\end{itemize}
\section{L'enzima centrale della RNA polimerasi}

\section{Riconoscimento dei promotori in batteri ed eucarioti}

\section{Iniziazione della trascrizione e transizione a un complesso di allungamento}

\section{Allungamento della trascrizione}

\section{Terminazione della trascrizione}

\section{Principi della regolazione della trascrizione}

\section{Domini leganti il DNA in proteine che regolano la trascrizione}

\section{Meccanismi per regolare l'iniziazione della trascrizione nei batteri}

\section{L'operone \emph{lac} in E. coli}

\section{L'operone triptofano \emph{trp} in E. coli}

\section{Regolazione della trascrizione da parte di riboswitches trascritti}

\section{Regolazione dell'espressione genica del batteriofago $\lambda$ in E. coli}

\section{Regolazione della trascrizione da sistemi di trasduzione del segnale a due componenti}

\section{Regolazoine dell'iniziazione della trascrizione ed allungamento negli eucarioti}

\section{Il ruolo delle cascate di segnalazione nella regolazione della trascrizione}

\section{Silenziamento genico attraverso imprinting genomico}
