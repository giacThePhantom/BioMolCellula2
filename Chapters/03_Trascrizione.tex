\chapter{Trascrizione}
\section{Panoramica della trascrizione}
L'informazione conservata nel DNA \`e utilizzata per creare proteine o molecole di RNA funzionale. Si intende per trascrizione il processo di copia di un filamento di DNA in una molecola
di RNA detta trascritto. Il processo di trascrizione viene svolto da una RNA polimerasi. Un cofattore esterno RNA polimerasi elicasi separa i filamenti di DNA e permette i 
ribonucleosidi trifosfato di accoppiarsi con il filamento stampo. Per produrre una proteina da una molecola di RAN la sequenza di RNA \`e letta dal ribosoma nella traduzione. L'RNA
in questo processo viene detto RNA messaggero \emph{mRNA}. Si nota come la RNA polimerasi non possegga attivit\`a esonucleasica e pertanto non possa correggere errori di 
mal-accoppiamento. Il tasso di errore \`e di $10^{-4}$ e il tasso di da $40$ a $80$ nucleotidi al secondo. Si nota come rispetto alla DNA polimerasi \`e pi\`u lenta, inefficiente e 
meno accurata. 
\subsection{Il processo di trascrizione}
La trascrizione pu\`o essere divisa in iniziazione, allungamento e terminazione. Inizia quando la RNA polimerasi si lega a una sequenza di DNA che precede il gene: il promotore. Il 
sito di inizio di trascrizione \emph{TSS} \`e la prima base ad essere trascritta ed \`e notata con $+1$. L'RNA viene trascritto nella direzione $5'$-$3'$ con il filamento letto in 
direzione opposta come nella sintesi del DNA. Pertanto si indica come basi a monte quelle $5'$ e a valle quelle $3'$. 
\subsubsection{Iniziazione}
Durante l'iniziazione la RNA polimerasi separa i filamenti di DNA per creare una bolla di trascrizione tra le $12$ e le $14bp$ e inseriscce i primi ribonucleoside trifosfati \emph{NTP} 
mentre si trova al promotore. 	Quando il RNA \`e di lunghezza sufficiente la RNA polimerasi lascia il promotore ``promoter clearance" e cambia conformazione per essere pi\`u 
stabilmente associata con il DNA permettendo l'allungamento del RNA. 
\subsubsection{Allungamento}
L'allungamento inizia dopo la clearance del promotore e la RNA polimerasi si muove lungo il DNA aggiungendo ribonucleotidi e allungando il trascritto di RNA. La RNA polimerasi 
svolge il DNA a valle e close quello a monte mantenendo la dimensione della bolla di trascrizione di dimensione costante per impedire la fomrazione di R-loop. Nella bolla una
regione del trascritto di $8$-$10bp$ \`e accoppiata con il DNA mentre il resto \`e estruso dalla polimerasi.
\subsubsection{Terminazione}
L'allungamento continua fino a che la polimerasi incontra una sequenza di DNA detta terminatore che segnala la fine della sintesi di RNA. Il RNA \`e rilasciato e la RNA polimerasi si 
dissocia dal DNA. 
\subsection{Nomenclatura dei geni}
In un gene viene indicato con $+1$ il sito di inizio della trascrizione \emph{TSS}, con numeri negativa la zona del promotore e con \emph{TTS} il sito di terminazione della trascrizione.
Si dice con prossimale la zona pi\`u vicina al \emph{TSS}, distale quella che si trova allontanandosi verso il \emph{TTS}. Si intende per sequenza codificante \emph{CDS} la sequenza del 
gene senza introni, mentre con open reading frame \emph{ORF} la sequenza con gli introni. 
\subsection{Regolazione della trascrizione}
La trascrizione \`e regolata per produrre il RNA richiesto al tempo corretto. La cromatina negli eucarioti presenta una sfida per la trascrizione in quanto i nucleosomi prevengono
il legame e il movimento del macchinario di trascrizione attraverso la cromatina. Sono pertanto richiesti:
\begin{itemize}
	\item Rimodellamento dei nucleosomi: per riposizionare gli istoni lontano dal DAN che deve essere trascritto. 
	\item Chaperone degli istoni per riassemblare e disassemblare i dimeri nucleosoma-istone.
	\item Enzimi che modificano le proteine istoniche epigeneticamente per permettere o prevenire il legame di proteine che regolano la trascrizione.
\end{itemize}
\section{L'enzima centrale della RNA polimerasi}
Se i procarioti possiedono $1$ RNA polimerasi gli eucarioti ne possiedono $3$ principali:
\begin{itemize}
	\item RNA polimerasi I \emph{Pol I}: trascrive i geni di grandi RNA ribosomali \emph{rDNA}. Agisce nei nucleoli $5$-$6$ negli umani, $1$ nel lievito.
	\item RNA polimerasi II \emph{Pol II}: trascrive i geni di RNA messaggero \emph{mRNA}, RNA non codificanti corti e lunghi, \emph{miRNA} con ruolo nella regolazione 
		dell'espressione genica attraverso interferenza e \emph{snRNA} RNA piccoli nucleari con un ruolo nel processamento di \emph{mRNA}. Agisce nel nucleoplasma.
	\item RNA polimerasi III \emph{Pol III}: trascrive una variet\`a di RNA come tutti gli RNA transfer \emph{tRNA}, il piccolo $5S$ RNA ribosomale, \emph{snRNA U6} componente 
		dello spliceosoma, \emph{7SL RNA} o long non-coding RNA parte della particella di riconoscimento del segnale che regola la traduzione. Agisce sia nei nucleoli che nel
		citoplasma.
\end{itemize}
Le piante possiedono una quarta RNA polimerasi con ruolo di trascrizione degli RNA non codificanti con ruolo nell'espressione genica. Gli eucarioti possiedono anche RNA polimerasi 
mitocondriali. Si intende per nucleoli raggruppamenti non circondati da membrana di $6$ ripetizioni di rDNA trascritti da \emph{Pol I} e \emph{Pol III}. 
\subsection{Struttura}
Le RNA polimerasi sono formate da diverse subunit\`a che formano il nucleo polimerasico. 
\subsubsection{RNA polimerasi batteriche}
Le RNA polimerasi batteriche sono le pi\`u piccole con $5$ subunit\`a che si assemblano in un complesso nucleare con lobi simili a mascelle che formano una pinza. Le mascelle sono formate
dalle subunit\`a $\beta$ e $\beta'$ con le due $\alpha$ e $\omega$ alla base. La base della fessura \`e il sito attivo dell'enzima. Ogni subunit\`a $\alpha$ possiede un dominio $N$ 
terminale \emph{$\alpha$NTD} e un dominio $C$ terminale \emph{$\alpha$CTD} unite da un collegatore flessibile. 
\paragraph{Subunit\`a e loro ruolo}
\begin{itemize}
	\item $\alpha$ ($2$): responsabile per l'assemblaggio del complesso.
	\item Domini $N$ terminali: interagiscono con le subunit\`a $\beta$ e $\beta'$. 
	\item Domini $C$ terminali: si legano al DNA promotore: elemento a monte.
	\item $\beta$ ($1$): contengono il sito catalitico, formano i legami fosfodiestere legando \emph{\ce{Mg2+}} e svolgono la correzione degli errori.
	\item $\beta'$ ($1$): mantiene l'enzima legato al filamento stampo, svolge e riavvolge il dsDNA. 
	\item $\omega$ ($1$): \`e un chaperon: promuove la stabilit\`a strutturale della RNA polimerasi.
\end{itemize}
\subsubsection{RNA polimerasi di eucarioti ed archea}
Tutte le RNA polimerasi possiedono le $5$ subunit\`a centrali viste prima altamente conservate, specialmente al sito attivo. Si trovano ulteriori subunit\`a in RNA polimerasi di archea
ed eucarioti ordinate intorno alle $5$ centrali. L'attivit\`a catalitica pertanto rimane conservata in tutti e tre gli alberi della vita. 
\paragraph{Ruoli della RNA polimerasi II} 
Oltre a trascrivere il DNA la RNA polimerasi II accoppia la trascrizione con il processamento del trascritto di RNA: il dominio $C$ terminale \emph{CTD} di \emph{Pol II} \`e cruciale per 
questa funzione: \`e la terminazione della subunit\`a \emph{Rpb1} e esiste come ripetizioni di una sequenza di $7$ amminoacidi: \emph{Tyr-Ser-Pro-Thr-Ser-Pro-Ser} con $26$ ripetizioni
nel lievito e $52$ negli umani. 	
\section{Riconoscimento dei promotori}
Il nucleo della RNA polimerasi pu\`o sintetizzare il RNA ma non pi\`o riconoscere e legarsi alla sequenza promotrice di un gene. Si rendono pertanto necessarie ulteriori subunit\`a
che si legano direttamente al promotore. Si forma pertanto un oloenzima dalla loro unione con l'enzima nucleare.
\subsection{Batteri}
Nei batteri le subunit\`a che si legano al promotore sono dette fattori $\sigma$. Ne esistono diverse che riconoscono promotori specifici per promuovere la trascrizione di geni specifici 
in base a particolari condizioni di crescita. I fattori $\sigma$ si legano a sequenze che definiscono i promotori batterici composti tipicamente da due elementi: uno a $-35$ e uno a
$-10$ detto box di Pribnow. Ogni fattore sigma ha una sequenza di legame specifico e una particolare spaziazione tra i due elementi. Questo in modo da regolare la trascrizione: pi\`u
vicine le sequenze e la spaziazione a quella particolare del fattoreo $\sigma$  pi\`u forte il legame e pi\`u alti tassi di trascrizione. 
\subsubsection{Fattori sigma}
Nell'oloenzima della RNA polimerasi il fattore $\sigma70$ \`e costitutivo e $3$ dei suoi $4$ domini riconoscono specifici elementi dei promotori:
\begin{itemize}
	\item Il dominio $2$ si lega alla regione $-10$ e aiuta a separare il dsDNA ``promoter melting".
	\item Il dominio $3$ riconosce le due basi della regione estesa $-10$.
	\item Il dominio $4$ riconosce l'elemento $-35$, \`e attaccato a una parte flessibile del nucleo dell'enzima che permette diversa spaziazione tra $-35$ 3 $-10$. 
\end{itemize}
Alcuni fattori sigma sono regolati in risposta a condizioni ambientali o di sviluppo. Possono essere regolati sia a livello trascrizionale o traduzionale o alterando la stabilit\`a 
del proprioi mRNA. Sono inoltre regolati da:
\begin{itemize}
	\item Fattori pro-sigma: proteine con domini inibitori che devono essere rotte prima che il fattore $\sigma$ possa associarsi con l'enzima RNA polimerasi. 
	\item Fattori anti-sigma: proteine che legano i fattori $\sigma$ inibendo la loro funzione. 
\end{itemize}
\paragraph{Fattori anti-sigma e regolazione dell'assemblaggio del flagello in Salmonella typhimurium}
Mentre le proteine che formano la base del flagello sono sintetizzate il fattore anti-sigma \emph{FlgM} si lega a $\sigma^F$ impedendo il suo legame con la RNA polimerasi-$\sigma^{70}$. 
$\sigma^F$ promuove la trascrizione di geni necessari per il completamento dell'assemblaggio del flagello. Negli ultimi passi della sintesi delle proteine del flagello \emph{FlgM} 
\`e esportata dalla cellula in modo che $\sigma^F$ possa legarsi alla RNA polimerasi sostituendo $\sigma^{70}$ e promuovere la trascrizione dei geni dell'assemblaggio permettendo
la creazione del flagello. 
\subsection{Eucarioti}
Anche i promotori eucariotici e di archea necessitano di ulteriori proteine per direzionare la RNA polimerasi ai promotori. Questi sono detti fattori di trascrizione generali
\emph{TF} seguito dal numero della \emph{Pol}. Negli eucarioti \emph{TFII} si assemblano al promotore e con l'enzima del nucleo formano il complesso di pre-iniziazione. \emph{Pol I} e 
\emph{Pol III} richiedono diverse proteine per formare il complesso di pre-iniziazione. I promotori per la RNA \emph{Pol II} possiedono spesso una TATA box con sequenza di consenso
$TATAA$ tra le $25$ e le $30bp$ a monte dell'inizio della trascrizione. Tutte le polimerasi eucariotiche necessitano della $TATA$ binding protein \emph{TBP} per iniziare la trascrizione
che viene riconosciuta da \emph{TBP} associated factors \emph{TAF} formando il complesso \emph{TFIID}. Alcuni altri elementi includono l'elemento di riconoscimento \emph{TFIIB} 
\emph{BRE}, l'elemento iniziatore \emph{INR} e l'elemento promotore a valle \emph{DPE}. Molti promotori non hanno questi elementi e la loro variabilit\`a rende difficile la loro 
predizione. Il complesso di pre-iniziazione contiene $32$ proteine fattori di trascrizione generali e la RNA \emph{Pol II} con $12$ subunit\`a. I promotori per \emph{Pol II} possono
dividersi in:
\begin{itemize}
	\item Prossimali: distanti meno di $200bp$ da \emph{TSS} nei lieviti unicellulari.
	\item Distali: distanti fino a $10kb$ dal \emph{RSS} come negli eucarioti multicellulari. 
\end{itemize}
\subsection{Formazione del complesso di pre-iniziazione}
Il primo passo nell'assemblaggio del complesso di pre-iniziazione \`e il legame di \emph{TFIID} alla $TATA$ box attraverso \emph{TBP} che si lega alla fessura minore del DNA causando
una piegatura locale. Altre componenti di \emph{TFIID} i \emph{TBP}-associated factors \emph{TAF} mediano il riconoscimento di altri elementi del promotore com \emph{INR} e \emph{DPE} per
\emph{TFIID}. Dopo l'associazione di \emph{TFIID} con il DNA viene reclutata \emph{TFIIB} che riconosce l'elemento \emph{BRE} e ci si lega asimmetricamente aiutando a determinare la
direzione di trascrizione. \emph{TFIIB} \`e simile al fattore $\sigma$ batterico. Dopo il legame di \emph{TFIID} e \emph{TFIIB} viene reclutata \emph{TFIIA} che lega e stabilizza 
l'interazoine \emph{TBP-DNA}. Successivamente si legano \emph{TFIIF}, \emph{TFeIIE} e \emph{TFIIH}. L'ultima di queste catalizza attivamente lo svolgimento del DNA. In vivo questo 
complesso necessita di un complesso \emph{Mediatore} per attivare molti geni trascritti da  \emph{Pol II} (non agisce con le altre RNA polimerasi). Altre proteine necessarie contengono
enzimi modificatori degli istoni e rimodellatori dei nucleosomi. 
\subsection{Differenze tra le RNA polimerasi eucariotiche}
Il nucleo delle RNA polimerasi \`e simile in \emph{Pol I}, \emph{Pol II} e \emph{Pol III} e tutte usano \emph{TBP} per iniziare la trascrizione. \emph{TBP} \`e un monomero con 
due domini simmetrici attraverso i quali lega il DNA e recluta diversi complessi iniziatori della trascrizione per ogni polimerasi. Le tre RNA polimerasi infatti usano proteine diverse
per iniziare la trascrizione che riconoscono promotori diversi e on diversi fattori di trascrizione generali. 
\subsubsection{RNA polimerasi I}
La RNA polimerasi I trascrive il rRNA e si lega a un promotore con un elemento di nucleo riconosciuto da \emph{TP} e da un elemento di controllo a monte \emph{UCE}. 
\subsubsection{RNA polimerasi III}
La RNA polimerasi III trascrive $5S$ RNA e tutti i tRNA e riconosce elementi promotori chiave: box $A$ e $C/B$ nella sequenza codificante insieme ad altri elementi a monte.
\section{Iniziazione della trascrizione e transizione a un complesso di allungamento}
Una volta che la RNA polimerasi \`e in posizione l'olocomplesso di RNA polimerasi \`e il promotore a doppio filamento chiusto vengono detti il complesso chiuso. Le RNA polimerasi 
bateriche aprono da $12$ a $14bp$ di dsDNA formando la bolla di trascrizione. Invece in \emph{Pol II} la bolla di trascrizione \`e aperta dalle subunit\`a elicasiche di \emph{RFIIH} e 
richiede \emph{ATP}. A questo punto la RNA polimerasi unita a un promotore aperto viene detta complesso aperto. La RNA polimerasi non crea un RNA di lunghezza completa al primo 
tentativo: inizialmente produce un RNA di $9$-$12$ nucleotidi detto trascritto abortivo di iniziazione. 
\subsection{Modello di inizio abortivo}
\begin{itemize}
	\item Inizio a passaggio transiente: la RNA polimerasi produce un RNA abortivo, si ferma e trona indietro.
	\item Inizio a bruco: la RNA polimerasi produce il RNA abortivo allungandosi flessibilmente prima di tornare indietro.
	\item Inizio ad accartocciamento: la RNA polimerasi accartoccia il DNA al suo interno per produrre il RNA abortivo. 
\end{itemize}
Sia il fattore $\sigma$ batterico che l'eucariote \emph{TFIIB} sono coinvolti nell'iniziazione abortiva: possiedono infatti un loop che si estende nel sito attivo delle RNA polimerasi in
modo che blocchi il trascritto in allungamento dopo $9$-$12$ nucleotidi. Lo spostamento ``chiusura" del loop aiuta la polimerasi a separarsi da promotore nella promoter clearance e
andare nella modalit\`a di allungamento. 
\subsection{Sintesi del RNA}
Una volta che si forma la bolla di trascrizione un timone nella RNA polimerasi mantiene il filamento di stampo e l'altro separati. Il filamento di stampo si lega alla RNA polimerasi nel 
sito attivo e il RNA formato esce attraverso un canale di uscita tra il muro e il coperchio. I ribonucleotidi \emph{NTP} entrano nel sito attivo attraverso un poro e si accoppiano con le 
basi del filamento stampo. I nucleotidi successivi sono attaccati attraverso attacco nucleofilo alla terminazione $3'$ della moelcola di RNA in crescita, formando un legame fosfodiestere
e rilasciando pirofosfato. Due ioni di magnesio sono presenti nel sito attivo e catalizzano l'addizione di nucleotidi attivando il $3'OH$ dell'ultimo nucleotide nel filamento di RNA e 
stabilizzando una carica negativa sull'ossigeno che sta lasciando sul $\gamma\beta$-pirofosfato. 
\subsection{Promoter clearance}
Durante la promoter clearance la RNA polimerasi subisce un cambio conformazionale nel coperchio e nel timone nella subunit\`a $\beta'$ che associa l'enzima molto stabilmente con il DNA
e riduce il legame con i fattori di iniziaione dissociandosi da essi. Le polimerasi eucariotiche sono anche fosforilate mentre si convertono al complesso di allungamento. L'enzima del
nucleo della RNA polimerasi svolge la sintesi del RNA attraverso il sito attivo nella subunit\`a $\beta$.
\section{Allungamento della trascrizione}
La trascrizione \`e processiva una volta che la fase di allungamento inizia. La RNA polimerasi rimane associata con il DNA e aggiunge centinaia o migliaia di basi sul RNA crescente 
con una velocit\`a tra i $20$ e i $50$ nucleotidi al secondo. La bolla di trascrizione si muove a $12$-$14$ basi costanti. La RNA polimerasi infatti separa il filamento di DNA 
a valle senza elicasi ulteriori e le coppie di basi si riformano quando questa si muove nella base successiva. I ribonucleotidi di addizione pi\`u recente si trovano nella bolla di 
trascrizione. 
\subsection{Pause e arresti nella trascrizione}
Le RNA polimerasi possono fermarsi a causa di ostruzioni fisiche nel ``transcriptional pausing" che pu\`o avvenire quando il filamento stampo forma una forcina a causa di corte sequenze
complementari o attraverso la formazione di deboli ibridi DNA-RNA nella bolla. Il ``transcriptional arrest" avviene quando l'enzima non pu\`o riprendere la sintesi. Il transcriptional
pausing pu\`o essere aumentato o diminuito da fattori di allungamento.
\subsection{Processamento del mRNA}
La RNA polimerasi II crea mRNA ma non si trova nella forma attiva \emph{pre-mRNA}. Questo deve essere processato in modo da diventare attivo. Negli eucarioti l'allungamento trascrizionale
\`e associato con il processamento del mRNA. La regione \emph{CTD} fosforilata della subunit\`a \emph{Rpb1} viene coinvolta: la quinta serina nella ripetizione di $7$ amminoacidi \`e 
fosforilata \emph{S5-P} mentre la RNA polimerasi II esce dalla regione del promotore e inizia il processo di allungamento. Questa recluta enzimi di processamento del RNA come
quelli che aggiungono un cap di guanosina metilata alla terminazione $5'$ del mRNA. L'allungamento \`e brevemente messo in pausa per permettere questo evento. Successivamente il capping
porta alla fosforilazione della seconda serina sulla \emph{CTD} $S2-P$ che permette la ripresa dell'allungamento da parte della polimerasi. 
\subsection{Backtrack}
Quando c'\`e una pausa nella sintesi del RNA il complesso di allungamento pi\`u fare del backtrack. Questo processo pu\`o avere un ruolo nella correzione di errori: se sono incorporate
basi mal-accoppiate il doppio DNA-RNA pu\`o essere distorto e causare pausa fisica della RNA polimerasi. La regione mal-accoppiata pu\`o essere poi rotta permettendo alla RNA polimerasi
di avere un'altra occasione per essere incorporata nel nucleotide corretto. La RNA polimerasi inverte la sua direzione e il RNA appena sintetizzato protrude dal complesso.La pausa
della trascrizione e fattori di rottura \emph{TFIIS} per gli eucarioti e \emph{GreB} in E. coli rompono il RNA protrundente $3'$ aumentando l'attivit\`a endonucleasic della RNA 
polimerasi e poi la trascrizione pu\`o riprendere. Questi fattori si legano nella regione a imbuto della RNA polimerasi attraverso cui protrude il RNA in caso di backtrack. I fattori 
posizionano uno ione magnesio al sito attivo che attiva una molecola di acqua per l'idrolisi del legame fosfodiestere. Il taglio \`e fatto $3$ nucleotidi prima della fine. 
\subsection{Problemi dell'allungamento}
\subsubsection{I nucleosomi impediscono il movimento della RNA polimerasi}
Gli eucarioti usano chaperones degli istoni per disassemblare i nuclesomi delicatamente a valle della RNA polimerasi e li riassemblano dietro essa. Questi chaperones sono detti 
\emph{FACT} (facilitano la trascrizione della cromatina) e un esempio \`e \emph{Spt6}. I \emph{FACT} disassemblano parzialmente il nucleosoma rimuovendo un singolo dimero
\emph{H2A-H2B} che allenta sufficientemente il DNA intorno al nucleosoma in modo che la RNA polimerasi II possa trascriverlo. \emph{FACT} assiste inoltre con il reposizionamento 
del dimero associato mentre \emph{Spt6} aiuta a garantire l'assemblaggio corretto dietro la RNA polimerasi II. La ristorazione dell'organizzazione cromatinica corretta potrebbe 
prevenire l'occupazione di siti TATA criptici nelle regioni codificanti da \emph{TBP} e il reclutamento di RNA polimerasi II. 
\subsubsection{Superavvolgimento a valle del DNA trascritto}
La bolla di trascrizione si muove lungo il DNA svolgendo la doppia elica continuamente. Questo porta a cambi nel superavvolgimento: un aumento di positivo a valle e di negativo a monte.
Tali cambi potrebbero causare pause della RNA polimerasi e la tensione deve essere sollevata da topoisomerasi I o II. 
\section{Terminazione della trascrizione}
La RNA polimerasi deve fermarsi a trascrivere il gene nel posto corretto, rilasciare il trascritto e dissociarsi dal DNA. Sequenze di terminazioni nel DNA segnalano la RNA polimerasi 
di fermare la trascrizione. 
\subsection{Batteri}
Le classi di sequenze terminatrici batteriche sono intrinseche Rho-indipendenti e Rho-dipendenti. 
\subsubsection{Intrinseche o semplici}
I terminatori intrinsechi terminano la trascrizione senza altri fattori. Quelli batterici presentano una sequenza di ripetizioni invertite che quando trascritte formano uno stem-loop 
nel RNA. Sono spesse ricche di $G-C$ per legami forti. Presentano inoltre stringhe di $8$-$10$ residui che si accoppia con il trascritto \emph{poli-U} nella bolla di trascrizione. La 
formazione di forcine di RNA forzata tira fuori il RNA dal sito attivo grazie all'accoppiamento con il DNA debole nella bolla di trascrizione finale. 
\subsubsection{Rho-dipendenti}
Dove la sequenza terminatrice non \`e sufficiente intervengono fattori proteici: in E. coli questi geni necessitano dell'elicasi Rho per terminare la trascrizione: 
\emph{terminatori $\rho$-dipendenti}. Questi geni contengono ripetizioni invertite. 
\paragraph{Rho} Questo \`e un anello formato da \emph{ATPasi} esameriche che legano a un'area terminale del RNA ricca in $C$: la sequenza \emph{RUT} (rho utilization site) lunga fino
a $40$ nucleotidi. L'anello ha una struttura aperta che si lega al RNA. Una volta legato l'anello si chiude e l'indrolisi del \emph{ATP} spinge il RNA attraverso l'anello. La formazione
di forcine risulta in una pausa della RNA polimerasi. Rho successivamente raggiunge la RNA polimerasi e la fa separare dal DNA. Rho si svolge e rilascia il RNA dallo stampo ssDNA
a cui \`e legato. 
\subsection{Eucarioti}
\subsubsection{Terminazione della RNA polimerasi I}
Sono presenti sequenze \emph{Poli-A} ma il grande gene rRNA contiene una sequenza ripetuta riconosciuta da una proteina terminatrice \emph{TTF-1} (transcription termination factor for
RNA polymerase I). \emph{TTF-1} legato blocca fisicamente la trascrizione causando la disassociazione di \emph{RNA polimerasi I} e il rilascio del RNA di nuova sintesi. 
\subsubsection{Terminazione della RNA polimerasi II}
La terminazione dei geni trascritti dalla RNA polimerasi II \`e accoppiata con un processamento della terminazione $3'$ del mRNA. La RNA polimerasi II continua a trascrivere il DNA
dopo il segnale di rottura e poliadenilazione e il mRNA \`e rotto e viene aggiunta una coda \emph{poli-A} alla terminazione $C$. Questa coda non \`e codificata dal DNA. 
\paragraph{Modello allosterico}
La RNA polimerasi II trascrive attraverso il segnale di rottura e poliadenilazione, proteine di procesamento del RNA si associano con i segnali di processamento e il \emph{CTD}. La 
rottura o riconoscimento delle proteine di processamento causa cambi conformazionali che portano alla dissociazione di \emph{Pol II} dal DNA. 
\paragraph{Modello torpedo}
Dopo la rottura come nel modello allosterico il RNA a valle \`e digerito dalla endonucleasi \emph{Rat1}, il torpedo che continua a degradare il RNA fino a che incontra la RNA 
polimerasi II che causa la sua dissociazione dal DNA. 
\paragraph{Conclusioni}
\`E possibile che una combinazione dei due modelli sia responsabile per la terminazione. Inoltre fosforilazione di \emph{CTD} a $S2$ e $S7$ potrebbe ridurre addizionalmente 
l'affinit\`a della RNA polimerasi II per il DNA dopo la rottura e polieadenilazione. 
\subsubsection{Terminazione della RNA polimerasi III}
La RNA polimerasi III riconosce siti di terminazione intrinsechi ricchi di $A$ che destabilizzano l'ibrido DNA-RNA come quelli batterici. 
\section{Principi della regolazione della trascrizione}
La regolazione della trascrizione \`e la chiave per fornire a una cellula la corretta quantit\`a di prodotto genico al momento corretto. I regolatori di trascrizione come i fattori di 
trascrizione possono avere effetti drammatici sull'espressione genica. Inoltre tale regolazione sottost\`a alla differenziazione cellulare e allo sviluppo in tutti gli eucarioti 
grazie all'utilizzo di una grande variet\`a di meccanismi. 
\subsection{Meccanismi di regolazione}
L'oloenzima RNA poliemrasi pu\`o trascrivere qualunque gene con un promotore funzionale, pertanto un livello di regolazione \`e la forza del promotore (per esempio la fedelt\`a delle 
sequenze $-35$ e $-10$). La maggior parte della regolazione avviene da regolazione su un gene obiettivo durante l'iniziazione (pi\`u efficiente e utilizzata), all'allungamento o 
terminazione o regolando il RNA trascritto. Le proteine che aumentano la trascrizione sono dette attivatori e stimolano il legame della RNA polimerasi al promotore attraverso contatto 
fisico. Le proteine che diminuiscono la trascrizione sono dette repressori e impediscono il legame della RNA polimerasi con il DNA. 
\subsubsection{Sequenze regolatorie}
Molti geni sono regolati da proteine che si legano vicino al gene, principalmente a monte del promotore.Le sequenze regolatrici sono regioni specifiche del DNA a cui la proteina 
regolatrice si lega. Si possono trovare vicino al promotore o a molte kilobasi di distanza.
\paragraph{Batteri} 
Nei procarioti le sequenze riconosciute dai regolatori sono dette siti operatori e sono vicine al promotore o si sovrappongono ad esso. Se gli operatori sono distali al gene il DNA
deve formare un loop in modo che la proteina regolatrice possa interagire con al polimerasi aiutata da proteine piegatrici del DNA.
\paragraph{Eucarioti}
Negli eucarioti i geni sono controllati da sequenze regolatorie distanti fino a migliaia di basi. Sono classificate come enhancers (amplificatori), silencers (repressori) o insulators. 
Le sequenze regolatorie legano diverse proteine aumentando la capacit\`a e l'intensit\`a della regolazione. Le sequenze possono trovarsi a monte, valle o nel gene. Come nei batteri 
sequenze distanti regolano il gene attraverso loop.
\subsection{Enhancer}
Si dice enhancer un effettore positivo a lunga distanza che potrebbe trovarsi fino a \num{10000}\si{bp} dal \emph{TSS}. \`E una sequenza di DNA promotrice agente in \emph{cis}. I 
regolatori che si legano a un enhancer si dicono agenti in \emph{trans}. Questi comprendono fattori di trascrizione base e attivatori trascrizionali. \`E lungo $500bp$ e contiene 
tra i $10$ e i $12$ siti di legami per fattori di trascrizione. Il legame dei fattori \`e cooperativo o esclusivo ma tutti hanno un effetto additivo sulla trascrizione. Le sequenze 
possono agire in entrambi gli orientamenti rispetto alla direzione della trascrizione. Il complesso formato da sequenza enhancer e dagli attivatori legati ad esso viene detto 
\emph{enhanceosome}. Questo complesso forma un anello e connette con il \emph{PIC} attraverso il complesso mediatore. Il livello di trascrizione \`e direttamente proporzionale con 
il numero di attivatori legati. 
\subsection{Silencer}
Il silencer \`e assolutamente analogo al enhancer: lega repressori trascrizionali formando il \emph{silenceosoma} e compete con il enhanceosoma per il legame con il complesso 
mediatore che attiva la trascrizione.
\subsection{Insulator}
Lo insulator \`e una sequenza di DNA a cui proteine repressori si legano che hanno effetto negativamente tra il enhancer e il promotore, di fatto isolandolo. Si pu\`o trovare
a monte, a valle o in una sequenza codificante di un gene. Spesso separano zone eterocromatiche da quelle eucromatiche, sono pertanto gli elementi di barriera. 
\subsection{Struttura delle proteine regolatrici}
Le proteine regolatrici devono essere in grado di riconoscere specificatamente la corretta sequenza regolatoria: ogni regolatore ha un dominio legante il DNA che riconosce una sequenza
specifica. Sono spesso modulari, con domini addizionali che aiutano l'oligomerizzzione, coattivano o coreprimono la trascrizione o interagiscono con altri regolatori e RNA polimerasi. 
I regolatori eucariotici controllano la trascrizione reclutando coattivatori o coreppressori che non sono in grado di legare direttamente il DNA> 
\subsection{Eventi con effetto sulla trascrizione}
Programmi di sviluppo o cambiamenti nelle condizioni di crescita possono influenzare la trascrizione di molti geni. Inoltre piccole molecole come effettori allosterici (estrogeno) 
possono legarsi direttamente a proteine regolatrici cambiando la loro conformazione. La fosforilazione o altre modifiche covalenti possono modificare come proteine regolatorie
interagiscono con il DNA e altre proteine. Queste modifiche con altri fattori come abbondanza e localizzazione del regolatore servono per aggiustare i livelli di trascrizione.
\subsubsection{La struttura cromatinica}
La struttura cromatinica ha un ruolo cruciale nella trascrizione eucariote: cromatina iper-acetilata tende ad essere attivamente trascritta, mentre i livelli di trascrizione sono pi\`u
bassi in cromatina ipo-acetilata. L'attivazione dei gene \`e accoppiata con il reclutamento di acetil-trasferasi istoniche, mentre la deacetilasi istonica pu\`o reprimerla. Altre
modifiche istoniche come metilazione, fosforilazione, ubiquitinazione e SUMOilazione possono avere un effetto sulla trascrizione. I codici istonici sono proposti in certe combinazioni
di modifiche istoniche che portano a eventi di trascrizione. 
\subsection{Confronto con replicazione}
Si nota come la trascrizione \`e pi\`u lenta della replicazione in quanto:
\begin{itemize}
	\item La DNA polimerasi processiva non cade dal ssDNA.
	\item La RNA polimerasi si ferma quando si formano deboli ibridi RNA-DNA, forcine e altre strutture che possono portare allo stop e alla rimozione della RNA polimerasi.
	\item Il repliosoma che incontra una RNA polimerasi sul DNA causa una sua rimozione.
	\item R-loop possono formarsi nella bolla di trascrizione e bloccare la sua progressione fino alla risoluzione da parte di RNAasi.
	\item ssDNA pu\`o formare strutture secondarie. 
	\item Ci sono difficolt\`a nel riavvolgimento del dsDNA dietro la RNA polimerasi.
	\item La correzione degli errori della RNA polimerasi \`e debole e richiede altre proteine, rallentando il processo.
	\item L'inizio della trascrizione \`e pi\`u complicato rispetto alla replicazione.
	\item Negli eucarioti sono importanti modifiche epigenetiche a valle che richiedono enzimi modificatori degli istoni attaccati alla RNA poliemrasi.
	\item Il processamento del RNA avviene in contemporanea con la sua trascrizione. 
\end{itemize}
\section{Domini leganti il DNA in proteine che regolano la trascrizione}
\subsection{Helix-turn-helix}
Un dominio comune delle proteine regolatrici \`e il helix-turn-helix. La prima $\alpha$-elica e la $\alpha$-elica di riconoscimento $R$-elica che entrano nella fessura principale. 
Molti agiscono come dimeri con le eliche di riconoscimento spaziate di $3.4nm$ (un giro di elica) per potersi trovare nella fessura principale vicina. La struttura a dimero permette di
aumentare la selettivit\`a e l'affinit\`a con la sequenza di $3$-$7bp$ riconosciuta. La $\alpha$-elica di riconoscimento legge la sequenza attraverso interazioni con le coppie di basi.
Vicino ai siti di legame e della loro connessione si trova un segnale di localizzazione nucleare \emph{NLS} attraverso cui viene localizzata nel nucleo.
\subsubsection{Homeodomain}
Il homeodomain \`e un comune dominio eucariote legante il DNA. \`E monomerico. La elica $3$ si trova nella fessura principale e catene laterali interagiscono con le basi del DNA. Il
braccio N terminale fa contatto con la fessura minore aumentando la specificit\`a e stabilit\`a del legame. 
\subsection{Zinc finger}
Gli zinc finger \emph{ZnF} sono domini che comprendono $30$ amminoacidi con $1$ $\alpha$-elica e due $\beta$-filamenti intorno a uno ione di zinco centrale che interagisce con due 
cisteine e due istidine. $12$ amminoacidi separano \emph{$Cys_2His_2$}, $3$ amminoacidi separano $2Cys$ da $2His$ affiancati da $4$ amminoacidi. Le proteine spesso hanno diversi 
zinc finger, ognuno dei quali inserisce la sua $\alpha$-elica nella fessura maggiore. Sono il dominio legane il DNA pi\`u comune nel genoma umano. I motivi \emph{ZnF} sono spesso 
presenti in ripetizioni nei fattori di trascrizione. Ognuno dei quali aiuta affinit\`a e specifit\`a del legame. \emph{TFII2} contiene $9$ ripetizioni. 
\subsection{Leucine zipper}
I domini leucine zipper \emph{bZIP} comprende due lunghe $\alpha$-eliche di $60$ amminoacidi. Queste sono formate come coiled-coils, ovvero si arrotolano tra di loro. Le due eliche
possiedono leucine idrofobiche sulla superficie interna che si uniscono insieme. Alla terminazione N le eliche si separano e si trovano nella fessura maggiore. Si legano a sequenze di 
DNA di $4bp$ invertite separate da $1$ nucleotide. I monomeri dei leucine zipper possono combinarsi in diverse combinazioni di eterodimeri per attivare la trascrizione. Il legame pu\`o 
avvenire al di fuori del DNA o sul DNA quando avviene la dimerizzazione tra monomeri legati al DNA. La proteina \emph{C/EBP} \emph{AP-1} \`e un eterodimero formato dal proto-oncogene
\emph{c-Jun} e \emph{c-Fos}. Regola l'espressione genica in risposta allo stress o infezioni virali o batteriche. 
\subsection{Helix-loop-helix}
Le proteine helix-loop-helix \emph{bHLH} hanno coiled-coils e un'elica con un residuo basico che si trova nella fessura principale. \`E tipicamente dimerico con ogni monomero
contenente $2$ $\alpha$-eliche unite da un loop. L'elica $N$-terminale che lega il DNA \`e pi\`u grande rispetto a quella $C$-terminale. Potrebbe legarsi come monomero o formare 
eterodimeri con domini \emph{HLH} o di altre proteine per funzione specifiche. La flessibilit\`a del loop permette di impacchettamento con altre eliche e dimerizzazione attraverso 
interazioni idrofobiche. Si lega a sequenze di DNA palindromiche dette \emph{E-box}. Attiva l'espressione genica \emph{Myo-D} per lo sviluppo muscolare e \emph{c-Myc} per il ciclo
cellulare, la biogenesi dei ribosomi o il metabolismo. 
\subsection{Ribbon-helix-helix}
Il dominio Ribbon-helix-helix \emph{RHH} contiene due $\beta$-foglietti sul ribbon, ognuno che arriva da un monomero proteico coinvolto nella formazione del dimero e nelle interazioni
specifiche con le basi del DNA nella fessura principale. 
\subsection{Interazioni con gli enhancer}
Gli elementi di sequenze enhancer e le loro sequenze nella regione promotrice si legano ad attivatori per stimolare l'espressione genica. La loro variet\`a riflette la presenza di
diverse proteine attivatrice. La loro combinazione pu\`o avere un forte effetto sulla trascrizione genica. Un elemento enhancer pu\`o contenere fino a $12$ siti di legame per 
diversi fattori di trascrizioni permettendo un controllo combinatorio della trascrizione genica. Specifici enhanceosomi si legano a \emph{TAF}, al mediatore e alla RNA polimerasi II per
regolare la trascrizione. Mutazioni in uno di questi attivatori o nella sequenza sottostante possono ridurre o aumentare la trascrizione. 
\subsubsection{Il sistema del lievito a due ibridi basato sui regolatori trascrizionali}
Il fattore di trascrizione attivatore contiene un dominio legante il DNA e un sito attivante della RNA poliemrasi. UN esempio sono i \emph{Cal4} dei zinc finger. Attiva l'espressione
di geni per la digestione del galattosio in S. cerevisiae. Il repressore \emph{LexA} lega il DNA nella sequenza clonata di fronte a \emph{lacZ}. Reprime l'espressione di \emph{lacZ} 
quando il dominio  legante il DNA di \emph{LexA} \emph{DBD} lega. L'attivazione di \emph{lacZ} avviene quando il dominio di attivazione \emph{Gal4} \`e legato a \emph{LexA's DBD}, 
l'ibrido \emph{Gal4-LexA}. Si nota come i domini di \emph{Gal4} possono essere separati porta alla tecnica di screen del lievito a due-ibrido. 

\section{Meccanismi per regolare l'iniziazione della trascrizione nei batteri}
I geni nei procarioti sono organizzati in operoni, gruppi di geni (fino a $12$) le cui proteine codificate agiscono in un pathway. I geni sono organizzati in maniera back-to-back e 
trascritti da un promotore. Viene prodotto un lungo trascritto di mRNA detto mRNA poli-cistronico che non viene spliced: i ribosomi lo traducono in proteine separate. Anche i 
virus lo utilizzano. Negli eucarioti invece i geni sono singole unit\`a trascrizionali tranne che nei nematodi. In C. elegans i trascritti poli-cistronici vengono spliced. 
\subsection{Operone}
Nell'operone a monte dei geni si trovano un promotore e un regolatore. L'operatore \`e una sequenza regolatrice che agisce in cis a cui si possono legare proteine codificate da altri
geni regolatori o sequenze regolatrici agenti in trans. Un modo per regolare la trascrizione \`e di prevenire il legame della RNA polimerasi al promotore o inibire il repressore che 
compie tale operazione. 
\subsection{Regolazione della trascrizione}
\subsubsection{Regolazione positiva}
Si dice regolazione positiva quando un attivatore determina il destino dell'operone. Si dice trascrizione inducibile positiva quando un attivatore legato all'operone deve essere attivato
per permettere la trascrizione. Si dice trascrizione reprimibile positiva quando tale attivatore viene disattivato e si separa dall'operone,
\subsubsection{Regolazione negativa}
Si dice regolazione negativa quando un repressore determina il destino dell'operone. Si dice trascrizione inducibile negativa quando il repressore attivo viene spento e si separa dal
DNA e trascrizione negativa reprimibile quando il repressore inattivo viene attivato. 
\section{L'operone \emph{lac} in E. coli}
L'operone \emph{lac} di E. coli \`e un esempio di un operone negativo inducibile. Pertanto sull'operone agisce un repressore che deve essere inibito per permettere l'inizio della 
trascrizione. In assenza di lattosio il gene \emph{lacl} viene trascritto creando il repressore che si lega all'operatore $O$, pertanto i geni seguenti \emph{lacZYA} sono molto poco 
espressi. In presenza di lattosio invece il repressore lega un induttore \emph{allolattosio} che impedisce il legame con il sito $O$. La RNA polimerasi \`e pertanto libera di legarsi 
all'operone permettendo la trascrizione dei geni \emph{lacZ}, \emph{lacY} e \emph{lacA}. Questi poi produrranno le proteine \emph{$\beta$-galattosidasi}, \emph{permeasi} e 
\emph{transacetilasi}. La fine della trascrizione \`e $\rho$-indipendente. Il promotore appena a monte del sito $O$ o sito $P$ viene bloccato fisicamente da \emph{Lacl} che si lega alla
sequenza palindromica dell'operatore $O$. 
\subsection{Genetica e analisi funzionale}
Isolando mutanti di E. coli si separano quelli che producono \emph{$\beta$-galattosidasi} in maniera costitutiva e si nota come ne esistono di due gruppi:
\begin{itemize}
	\item Mutazioni localizzate a un gene lontano da \emph{lacZ}: \emph{$lacl^c$}. 
	\item Mutazioni localizzate a una sequenza di DNA non codificante a monte di \emph{lacZ}: operatore $O^c$.
\end{itemize}
I mutanti sono divisi in mutazioni che agiscono in cis e in trans. Quelle in cis avvengono in una sequenza non codificante sull'operone e non possono essere complementate esprimendo la
sequenza wild type da un plasmide. Le altre in trans sono mutazioni sulla sequenza codificante una proteina e possono essere complementate esprimendo il gene wild type da un plasmide. 
Inoltre possono essere recessive o dominanti. Successivamente avviene l'analisi delle complementazioni delle mutazioni introducendo un operone \emph{lac} wild-type sul plasmide 
$F'$ detto operone esogeno. Il plasmide viene costruito in modo che faccia complementazione $1:1$: contiene sia il gene \emph{lacl} che l'intero operone. 
\subsubsection{Analisi dei mutanti complementari: crescita senza lattosio}
\paragraph{Mutazioni a \emph{lacl}}
Il mutante costitutivo del repressore che agisce in trans causa una produzione costitutiva del mRNA poli-cistronico da P. Con il plasmide la produzione viene repressa in quanto 
$F'$  \`e un operone $1:1$ complementare. La trascrizione endogena pertanto si blocca a causa di \emph{Lacl}.
\paragraph{Mutazioni al sito \emph{O}}
Il mutante costitutivo del sito $O$ che agisce in cis causa anch'esso una produzione cotitutiva del mRNA poli-cistronico da P. L'aggiunta del plasmide non \`e in grado comunque di 
disattivare la sua produzione in quanto la proteina \emph{Lacl} non \`e in grado di attaccarsi al sito $O^c$. Si dice pertanto che questa mutazione \`e dominante. 
\subsection{Il repressore \emph{Lacl}}
Il repressore \emph{Lacl} contiene un dominio helix-turn-helix per legarsi al DNA con un'elica di riconoscimento. Forma un tetramero quando si lega al sito $O$ e contiene inoltre dei 
domini per il legame con l'allolattosio. L'inibizione allosterica di \emph{Lacl} causata dall'allolattosio \`e dovuta a cambi conformazionali che riduscono l'affinit\`a per il sito 
$O$.
\subsubsection{Mutazioni di \emph{Lacl}}
Si riconoscono $3$ categorie di mutanti di \emph{Lacl}.
\paragraph{$\mathbf{lacl^d}$}
Questo mutante pu\`o di e tetramerizzare ma non pu\`o legarsi all'operatore. Con complementazione di $F'$ \emph{lacZYA} si nota come non c'\`e quasi possibilit\`a di avere un 
di/tetramero completamente funzionale.
\paragraph{$\mathbf{lacl^-}$}
Questo mutante non pu\`o di-tetramerizzare e non pu\`o legarsi all'operatore. Con complementazione non pu\`o legare a $O$ e reprimere la trascrizione, mentre quello espresso dal plasmide
pu\`o. 
\paragraph{$\mathbf{lacl^s}$}
Questo mutante non pu\`o legare l'allolattosio ma si lega fermamente all'operatore. Pertanto la complementazione non pu\`o competere per il legame e i geni saranno costitutivamente 
disattivati. 
\subsection{L'operone \emph{lac} e induzione positiva}
L'operone \emph{lac} \`e anche un esempio di un operone inducibile positivamente: un attivatore agisce e deve essere attivato affinch\`e avvenga la trascrizione. L'operone subisce
regolazione positiva solo in presenza di lattosio e assenza di glucosio. 
\subsubsection{\emph{CAP} e \emph{cAMP}}
Si nota come E. coli preferisca il glucosio come fonte di carbonio ed energia, pertanto in presenza di glucosio la trascrizione di \emph{lacZYA} rimane repressa da \emph{Lacl} senza 
che il glucosio lo inibisca allostericamente. In assenza di glucosio E. coli cerca un altro carboidrato e se il lattosio \`e presente induce la trascrizione di \emph{lacZYA}. Si nota
come nonostante l'induzione dall'allolattosio l'intensit\`a di trascrizione \`e bassa, pertanto E. coli deve aumentarla per consumare il lattosio efficientemente e velocemente in modo
da sopravvivere. Viene pertanto aumentata l'affinit\`a per la RNA polimerasi al promotore attraverso un attivatore trascrizionale \emph{CAP} (catabolic gene activating protein) la cui
attivit\`a \`e indotta da \emph{cAMP}. Infatti quando i livelli di glucosio sono bassi e il lattosio \`e presente \emph{ATP} viene convertito in \emph{cAMP}. A questo punto si forma 
il complesso \emph{CAP+cAMP} che si lega a monte del promotore nel proprio operatore. Questo aumenta l'affinit\`a con RNA polimerasi legandosi al dominio $C$ terminale \emph{CTD} 
della subunit\`a $\alpha$ dell'enzima. 
\section{L'operone triptofano \emph{trp} in E. coli}
L'operone \emph{trp} \`e un operone negativo reprimibile: un repressore agisce e deve essere attivato per reprimere la trascrizione. L'operone \emph{trpE-A} codifica l'enzima che
converte l'acido corismico in triptofano. In E. coli la regolazione trascrizionale della sintesi del triptofano avviene a due fasi: all'iniziazione della trascrizione (regolazione
reprimibile negativa) quando si hanno alti livelli di triptofano intracellulari. Nella seconda fase alla terminazione della trascrizione e della traduzione attraverso attenuazione: 
quando si trovano bassi o medi (la trascrizione ha iniziato ma deve essere fermata) livelli intracellulari di triptofano. Entrambi i meccanismi sono utilizzati per gli altri amminoacidi.
\subsection{Regolazione all'iniziazione della trascrizione}
In questa fase si trovano livelli di triptofano medio alti, pertanto non si rende necessaria sintesi attiva di triptofano: quello intracellulare induce l'attivit\`a del repressore
inattivo \emph{TrpR} per bloccare la propria sintesi. Funge pertanto da co-repressore. Un modo per regolare la trascrizione negativamente \`e impedire che la RNA polimerasi possa
accedere al promotore come \emph{TrpR}, una proteina helix-turn-helix. Il legame del repressore al DNA dipende dal livello di triptofano nella cellula. Quando i livelli sono bassi 
l'operone \emph{trpE-A} viene trascritto per sintetizzarlo, mentre quando sono alti il legame di \emph{TrpR} all'operatore blocca il legame della RNA polimerasi. \emph{TrpR} pu\`o 
legarsi al DNA quando \`e legata a triptofano (indicatori che i livelli sono alti). 
\subsection{Regolazione della trascrizione attraverso attenuazione}
Questa regolazione avviene quando i livelli di triptofano sono bassi. L'attenuazione \`e il controllo della trascrizione quando \`e gi\`a cominciata. Quando i livelli sono bassi non
avviene la repressione da parte di \emph{TrpR}, pertanto la trascrizione inizia. Quando il trascritto \`e lungo $161nt$ la trascrizione pu\`o continuare (livelli bassi di triptofano) o 
fermarsi (livelli medi). L'attenuazione controlla la sintesi dei geni batterici necessari per la sintesi di triptofano e degli altri amminoacidi sfruttando il piegamento di una sequenza
di RNA leader in strutture secondarie come forcine. 
\subsubsection{Attenuazione mediata da \emph{trpL}}
Il mRNA \emph{trp} posisede una sequenza leader \emph{TrpL} vicino all'inizio con un terminatore intrinseco \emph{poli-U}. Questa sequenza codifica parzialmente per un peptide di $14$
amminoacidi che include due codoni per il triptofano. La sequenza leader contiene $4$ blocchi di sequenza che possono formare accoppiamenti alternativi: $1:2$ un debole stem-loop per
il ribosoma, $3:4$ un terminatore $\rho$-indipendente molto forte per la RNA-polimerasi, $2:3$ un debole stem-loop per il ribosoma. 
\paragraph{Livelli bassi di triptofano nella cellula}
In presenza di questi livelli bassi il ribosoma si blocca quando traduce il peptide leader: la regione $1$ \`e occupata dal ribosoma, e la $2$ e la $3$ formano uno stem-loop impedendo
il legame di $3$-$4$ e la formazione del terminatore permettendo la continuazione della trascrizione, pertanto i geni nel mRNA \emph{trpE-A} verranno trascritti e singolarmente tradotti.
\paragraph{Medi livelli di triptofano nella cellula}
In presenza di livelli medi il ribosoma procede occupando $1$ e $2$ permettendo la formazione del terminatore $3:4$: la RNA polimerasi lascia il RNA leader e non trascrive oltre 
l'operone. 
\section{Regolazione della trascrizione da parte di riboswitches del trascritto}
I riboswitches sono porzioni di un trascritto che legano piccole molecole che controllano la struttura secondaria del RNA regolando la trascrizione. Questi hanno due regioni: 
\begin{itemize}
	\item L'aptamero che lega un metabolita.
	\item La piattaforma di espressione che controlla la trascrizione. 
\end{itemize}
Si nota come i riboswitches sono in grado di controllare sia la trascrizione che la traduzione. 
\subsection{Riboswitch di adenina di B. subtilis}
Il riboswitch di B. subtilis regola la sintesi e il trasporto di adenina: l'espressione genica dipende se si forma un terminatore o un anti-terminatore.
\subsubsection{Bassi livelli di adenina}
Con bassi livelli di adenina le regioni $2$ e $3$ del RNA formano deboli loop e la trascrizione procede.
\subsubsection{Alti livelli di adenina}
Con alti livelli di adenina le regioni $3$ e $4$ formano un forte terminatore $\rho$-indipendente. Il RNA \`e rimosso dalla RNA polimerasi che si separa dal filamento di DNA stampo. 
\section{Regolazione dell'espressione genica del batteriofago $\lambda$ in E. coli}

\section{Regolazione della trascrizione da sistemi di trasduzione del segnale a due componenti}

\section{Regolazoine dell'iniziazione della trascrizione ed allungamento negli eucarioti}

\section{Il ruolo delle cascate di segnalazione nella regolazione della trascrizione}

\section{Silenziamento genico attraverso imprinting genomico}

