\chapter{Modifica e targeting delle proteine}
\section{Piegamento delle proteine assistito da chaperones}
Le proteine devono essere piegate correttamente in modo da assumere la struttura tridimensionale corretta per funzionare correttamente. Tipicamente questo processo richiede degli aiutanti o chaperones. Il mal-piegamento delle
proteine e la loro aggregazione pu\`o avere conseguenze patologiche come Alzheimer e Parkinson. La maggior parte dei peptidi emerge dal canale di uscita del ribosoma in forma estrusa, con struttura lineare. Il piegamento pu\`o 
cominciare appena la terminazione N esce dal ribosoma dove il peptide incontra dei chaperoni. Nei batteri i peptidi interagiscono con un chaperone associato al ribosoma detto ``trigger factor'' e successivamente possono
piegarsi non assistite, legarsi a chaperones \emph{Hsp70} come \emph{DnaK} o legarsi ai chaperon \emph{Hsp60}. Gli chaperones lavorano interagendo con regione idrofobiche sul peptide che emerge impedendo che si associno 
scorrettamente con altre regioni idrofobiche. Il nome heat-shock proteins deriva dal fatto che le proteine formate da ribosomi ad alte temperature si mal-piegano causando la morte della cellula. 
\subsection{Ponti disolfuro}
I legami disolfuro tra i residui di cisteina sono importanti per la stabilit\`a e funzione delle proteine. Le cisteine infatti possono esistere in forma:
\begin{itemize}
	\item Ridotta con il tiolo \emph{-SH}, esiste pi\`u frequentemente in ambiente riducende come il citosol. Subisce minimo stress ossidativo grazie all'attivit\`a della glutationina reduttasi.
	\item Ossidata con il legame disolfuro \emph{-S-S-}, esiste pi\`u frequentemente in ambiente ossidativo come ER e mitocondri.
\end{itemize}
Essendo i citosol di un batterio ridotto le proteine batteriche non possiedono legami disolfuro, mentre l'attivit\`a della sulfidril ossidasi permette alle proteine di essere disecrete con legami disolfuro. fi
\section{Targeting di cellule attraverso la cellula}
Le proteine sono spesso trasportate a regioni cellulari distinte. Per direzionare la localizzazione sono presenti motivi specifici nella proteina che possono essere presenti alla terminazione N o C, con le prime spesso rimosse
dopo che il loro lavoro si \`e svolto. I motivi di ordinamento possono essere presenti in diversi numeri per una localizzazione precisa. I motivi non sono rigidi ma si trova una certa flessibilit\`a a livello di composizione. 
\subsection{Ordinamento delle proteine}
L'ordinamento in compartimenti cellulari richiede il riconoscimento dei sorting motif, trasporto e traslocazione in o attraverso una membrana. I trasportatori, recettori o motori relocano le proteine e le passano a traslocatori. 
I trasportatori includono un peptide di ricnoscimento del segnale \emph{SRP} che riconosce e si lega al segnale sulla porteina che emerge dal ribosoma. Successivamente la proteina si muove attraverso un traslocatore nella
membrana venendo o spinta o tirata utilizzando l'energia messa a disposizione dai chaperones. 
\subsection{Entrata ed uscita delle proteine dal nucleo}
Il trasporto da e verso il nucleo avviene attraverso il complesso dei pori nucleari \emph{NPC}, che sono traslocatori. Le proteine ribosomali sono create nel citoplasma e trasportate nel nucleo dove si assemblano con rRNA. 
La subunit\`a maggiore e minore sono poi esportate nel citoplasma. \emph{NPC} sono ricchi di amminoacidi carichi positivamente e sono riconosciuti dai trasportatori importine ed esportine che si legano a parti del \emph{NPC}. 
La proteina legata pu\`o attraversare ora il poro. 
\subsubsection{Direzionalit\`a}
La \emph{GTPasi} \emph{Ran} governa la direzionalit\`a del trasporto. La \emph{RanGTP} si trova principalmente nel nucleo mentre la \emph{RanGDP} si trova nel citoplasma. Il gradiente di \emph{GDP/GTP Ran} promuove il legame e 
il rilascio delle proteine dentro e al di fuori del nucleo. Il cargo potrebbe includere \emph{RNP}. Nel nucleo si trova \emph{RCC1} o guanine-nucleotide exchange factor che cambia \emph{RanGDP} in \emph{RanGTP} causando la sua 
esportazione dal nucleo. Nel citoplasma invece si trova una \emph{RanGAP} o \emph{RanGTPasi} activating protein che stimola l'idrolisi di \emph{RanGTP}. 
\section{Rottura post-traduzionale della catena polipeptidica}
Molte proteine devono essere rotte da proteasi prima che possano diventare funzionali. Questo in quanto si pu\`o dover rimuovere la prima metionina da parte di metionina aminopeptidasi o rottura proteolitica di una sequenza 
leader del polipeptide. La rottura proteolitica pu\`o garantire una sua attivit\`a temporizzata. La chimotripsinogina \`e un precursore inattivo della chimotripsina, un enzima digestivo dell'intestino e sintetizzata nel 
pancreas. Il precursore \`e inattivo quando sintetizzato per impedire la digestione di cellule pancreatiche. Successivamente si muove nell'intestino dove viene tagliato e diventa attiva. 
\subsection{Insulina}
L'insulina viene prodotta dal pancreas e subisce un processamento da parte di endopeptidasi. Il precursore preproinsuina creato dal ribosoma nel ER ruvido possiede una sequenza segnale che la direziona al ER. Il segnale
viene poi rotto nel ER formando ponti disolfuro (ER \`e un ambiente ossidativo) tra le catene $A$ e $B$. Ora la proinsulina si muove nel Golgi. Ulteriori rotture della proinsulina nel GOlgi rimuove la catena $C$ creando
cos\`i l'insulina attiva. I livelli della catena $C$ o antigene sono misurati per il diabete: quando i livelli di glucosio nel sangue sono alti attraverso \emph{ELISA}. La catena $C$ infatti possiede un'emivita molto pi\`u
alta ($20$-$30min$) rispetto all'insulina ($3$-$5min$), in quanto la seconda viene degradata da enzimi di fegato e reni. L'insulina \`e un ormone che aiuta le cellule nell'uptake del glucosio. Esistono due tipi di diabete:
di tipo $1$ in cui il pancreas non produce abbastanza insulina o di tipo $2$ in cui l'insulina non stimola l'uptake di glucosio nella cellula. 
\subsection{Rottura post traduzionale delle rotture e Alzheimer}
La porteina \emph{APP} amyloid precursor proteine, errore di rottura, placche amiloidi (aggiungere poi) 
i neuroni poi non sono pi\`u in grado di funzionare causando problemi al cervello. 
\subsection{Splicingdi proteine: rimozione di inteine}
Alcune porteine nei batteri, archea e eucarioti unicellulari possiedono inteine, lunghe tra i $138$ e $844$ sono parti di una porteine rimosse attraveso un processo di splicing delle proteine. Le inteine catalizzano la 
propria rimozione e legano le exeine che le affiancano. Il risultato sono due proteine: le eseine unite e l'inteina libera. L'inteina deve essere escissa in quanto altrimenti la proteina che la contiene rimane inattiva. 
L'inteina agisce tipicamente come un elemento genetico, muovendo la propria sequenza di DNA nel genoma. 
\subsubsection{Splicing di inteine}
\begin{enumerate}
	\item Un gruppo \emph{HX} un un residuo conservato nella inteina: cisteina, serina o treonina attacca il gruppo carbossile del legame peptide nell'esteina N terminale. 
	\item Lo stesso carbossile viene attaccato da un residuo simile nella terminazione C dell'esteina.
	\item Il gruppo ammino dell'ultimo residuo dell'inteina attacca il proprio gruppo carbossile causando il rilascio dell'inteina. 
	\item Le due eseine unite subiscono ulteriore attacco di un gruppo ammino sul carbossile per creare un legame peptidico.
\end{enumerate}
Si nota come non viene coinvolto \emph{\ce{Mg^{2+}}}. Lo sono le istidine cariche positivamente nella proteina. 
\subsubsection{Gene di endonucleasi homing}
Le inteine possono includere un homing endonuclease gene \emph{HEG} dominio oltre ai domini di splicing. Questo dominio \`e responsabile per la diffusione dell'inteina rompendo il DNA sull'allele libero dell'inteina
sul cromosoma omologo, attivando il sistema di riparazione del danno che copia il DNA codificante l'inteina in un sito prima privo di essa. Sono simili agli introni mobili di gruppo I. Il dominio \emph{HEEG} non \`e necessario
per lo splicing della proteina. 
\section{Regolazione e modifica post-traduzionale di proteine}
L'attivit\`a e stabilit\`a delle proteine \`e regolata da modifiche post-traduzionali, la via pi\`u veloce per regolare l'attivit\`a della cellula ed \`e regolata da enzimi. Pu\`o essere reversibile o irreversibile, 
fisiologica o patologica. Tutte le varie modifiche a livello di trascrizione, traduzione e modifiche post-traduzionale porta a un grado di complessit\`a estremamente elevato. L'alto livello di complessit\`a delle informazioni
\`e dovuto a modifiche post-traduzionali delle proteine, che permettono per cambi veloci in risposte e adattamenti cellulari: il $5\%$ del proteoma umano \`e composto da enzimi che catalizzano pi\`u di $200$ diversi tipi 
di modifiche post-traduzionali di proteine. 
\subsection{Ubiquitinaizone delle proteine}
Grandi modificazioni post-traduzionali come l'ubiquitinazione sono l'addizione di una proteina ad un'altra. L'ubiquitina \emph{Ubi} \`e un peptide eucariote di $76aa$ che marca la proteina per degradazione o per processi non
degradativi e regolatori come la risposta al danno del DNA. Viene attaccata covalentemente principalmente al gruppo ammino della lisina ma anche a metionina, cisteina, serina o treonina non canoniche. Per molecola di 
ubiquitina sono aggiunti alla proteina $8.5kDa$. L'ubiquitina si lega alla terminazione C sulla glicina. Inoltre le ubiquitine possono concatenarsi nella modifica per ottenere una poli-ubiquitinazione attraverso la lisina
$8$. Proteine poli-ubiquitinate sono direzionate verso la degradazione. L'ubiquitina pu\`o attacare una o pi\`u isine in una proteina o a una molecola di ubiquitina creando una catena di poli-ubiquitina lineare o 
ramificata, omogenea o eterogenea e pu\`o portare a una variet\`a di conformazioni secondo al legame. Multiple ubiquitinazioni sono classificate dalla lisina K nella prima ubiquitina a cui la prossima si attacca. 
\section{\emph{SUMOilazione} delle proteine}
