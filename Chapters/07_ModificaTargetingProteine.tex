\chapter{Modifica e targeting delle proteine}
\section{Piegamento delle proteine assistito da chaperones}
Le proteine devono essere piegate correttamente in modo da assumere la struttura tridimensionale corretta per funzionare correttamente. Tipicamente questo processo richiede degli aiutanti o chaperones. Il mal-piegamento delle
proteine e la loro aggregazione pu\`o avere conseguenze patologiche come Alzheimer e Parkinson. 
\subsection{Processo di piegamento}
La maggior parte dei peptidi emerge dal canale di uscita del ribosoma in forma estrusa, con struttura lineare. Il piegamento pu\`o 
cominciare appena la terminazione N esce dal ribosoma dove il peptide incontra dei chaperoni. Nei batteri i peptidi interagiscono con un chaperone associato al ribosoma detto ``trigger factor'' e successivamente possono
piegarsi non assistite, legarsi a chaperones \emph{Hsp70} come \emph{DnaK} con co-chaperones \emph{DnaJ} e \emph{GrpE} o legarsi ai chaperon \emph{Hsp60}. Un complesso di piegatura \`e formato da \emph{GroES} e \emph{GroEL}, che
formano un cilindro. I chaperones lavorano interagendo con regione idrofobiche sul peptide che emerge impedendo che si associno scorrettamente con altre regioni idrofobiche. Il nome heat-shock proteins deriva dal fatto che le 
proteine formate da ribosomi ad alte temperature si mal-piegano causando la morte della cellula. 
\subsection{Ponti disolfuro}
I legami disolfuro tra i residui di cisteina sono importanti per la stabilit\`a e funzione delle proteine. Le cisteine infatti possono esistere in forma:
\begin{itemize}
	\item Ridotta con il tiolo \emph{-SH}, esiste pi\`u frequentemente in ambiente riducende come il citosol. Subisce minimo stress ossidativo grazie all'attivit\`a della glutationina reduttasi.
	\item Ossidata con il legame disolfuro \emph{-S-S-}, esiste pi\`u frequentemente in ambiente ossidativo come ER e mitocondri.
\end{itemize}
Essendo i citosol di un batterio ridotto le proteine batteriche non possiedono legami disolfuro, mentre l'attivit\`a della sulfidril ossidasi permette alle proteine di essere disecrete con legami disolfuro.
\section{Targeting di cellule attraverso la cellula}
Le proteine sono spesso trasportate a regioni cellulari distinte. Per direzionare la localizzazione sono presenti motivi specifici nella proteina che possono essere presenti alla terminazione N o C, con le prime spesso rimosse
dopo che il loro lavoro si \`e svolto. I motivi di ordinamento possono essere presenti in diversi numeri per una localizzazione precisa. I motivi non sono rigidi ma si trova una certa flessibilit\`a a livello di composizione. 
\subsection{Ordinamento delle proteine}
L'ordinamento in compartimenti cellulari richiede il riconoscimento dei sorting motif, trasporto e traslocazione in o attraverso una membrana. I trasportatori, recettori o motori relocano le proteine e le passano a traslocatori.
I trasportatori includono un peptide di riconoscimento del segnale \emph{SRP} che riconosce e si lega al segnale sulla proteina che emerge dal ribosoma. Successivamente la proteina si muove attraverso un traslocatore nella
membrana venendo o spinta o tirata utilizzando l'energia messa a disposizione dai chaperones. 
\subsection{Entrata ed uscita delle proteine dal nucleo}
Il trasporto da e verso il nucleo avviene attraverso il complesso dei pori nucleari \emph{NPC}, che sono traslocatori. Le proteine ribosomali sono create nel citoplasma e trasportate nel nucleo dove si assemblano con rRNA\@. 
La subunit\`a maggiore e minore sono poi esportate nel citoplasma. \emph{NPC} sono ricchi di amminoacidi carichi positivamente e sono riconosciuti dai trasportatori importine ed esportine che si legano a parti del \emph{NPC}. 
La proteina legata pu\`o attraversare ora il poro. 
\subsubsection{Direzionalit\`a}
La \emph{GTPasi} \emph{Ran} governa la direzionalit\`a del trasporto. La \emph{RanGTP} si trova principalmente nel nucleo mentre la \emph{RanGDP} si trova nel citoplasma. Il gradiente di \emph{GDP/GTP Ran} promuove il legame e 
il rilascio delle proteine dentro e al di fuori del nucleo. Il cargo potrebbe includere \emph{RNP}. Nel nucleo si trova \emph{RCC1} o guanine-nucleotide exchange factor che cambia \emph{RanGDP} in \emph{RanGTP} causando la sua 
esportazione dal nucleo. Nel citoplasma invece si trova una \emph{RanGAP} o \emph{RanGTPasi} activating protein che stimola l'idrolisi di \emph{RanGTP}. 
\section{Rottura post-traduzionale della catena polipeptidica}
Molte proteine devono essere rotte da proteasi prima che possano diventare funzionali. Questo in quanto si pu\`o dover rimuovere la prima metionina da parte di metionina aminopeptidasi o rottura proteolitica di una sequenza 
leader del polipeptide. La rottura proteolitica pu\`o garantire una sua attivit\`a temporizzata. La chimotripsinogina \`e un precursore inattivo della chimotripsina, un enzima digestivo dell'intestino e sintetizzata nel 
pancreas. Il precursore \`e inattivo quando sintetizzato per impedire la digestione di cellule pancreatiche. Successivamente si muove nell'intestino dove viene tagliato e diventa attiva. 
\subsection{Insulina}
L'insulina viene prodotta dal pancreas e subisce un processamento da parte di endopeptidasi. Il precursore preproinsuina creato dal ribosoma nel ER ruvido possiede una sequenza segnale che la direziona al ER\@. Il segnale
viene poi rotto nel ER formando ponti disolfuro (ER \`e un ambiente ossidativo) tra le catene $A$ e $B$. Ora la proinsulina si muove nel Golgi. Ulteriori rotture della proinsulina nel GOlgi rimuove la catena $C$ creando
cos\`i l'insulina attiva. I livelli della catena $C$ o antigene sono misurati per il diabete: quando i livelli di glucosio nel sangue sono alti attraverso \emph{ELISA}. La catena $C$ infatti possiede un'emivita molto pi\`u
alta ($20$-$30min$) rispetto all'insulina ($3$-$5min$), in quanto la seconda viene degradata da enzimi di fegato e reni. L'insulina \`e un ormone che aiuta le cellule nell'uptake del glucosio. Esistono due tipi di diabete:
di tipo $1$ in cui il pancreas non produce abbastanza insulina o di tipo $2$ in cui l'insulina non stimola l'uptake di glucosio nella cellula. 
\subsection{Rottura post traduzionale delle proteine e Alzheimer}
La porteina \emph{APP} amyloid precursor proteine, errore di rottura, placche amiloidi (aggiungere poi) 
i neuroni poi non sono pi\`u in grado di funzionare causando problemi al cervello. 
\subsection{Splicingdi proteine: rimozione di inteine}
Alcune porteine nei batteri, archea e eucarioti unicellulari possiedono inteine, lunghe tra i $138$ e $844$ sono parti di una porteine rimosse attraveso un processo di splicing delle proteine. Le inteine catalizzano la 
propria rimozione e legano le exeine che le affiancano. Il risultato sono due proteine: le eseine unite e l'inteina libera. L'inteina deve essere escissa in quanto altrimenti la proteina che la contiene rimane inattiva. 
L'inteina agisce tipicamente come un elemento genetico, muovendo la propria sequenza di DNA nel genoma. 
\subsubsection{Splicing di inteine}
\begin{enumerate}
	\item Un gruppo \emph{HX} un un residuo conservato nella inteina: cisteina, serina o treonina attacca il gruppo carbossile del legame peptide nell'esteina N terminale. 
	\item Lo stesso carbossile viene attaccato da un residuo simile nella terminazione C dell'esteina.
	\item Il gruppo ammino dell'ultimo residuo dell'inteina attacca il proprio gruppo carbossile causando il rilascio dell'inteina. 
	\item Le due eseine unite subiscono ulteriore attacco di un gruppo ammino sul carbossile per creare un legame peptidico.
\end{enumerate}
Si nota come non viene coinvolto \emph{\ce{Mg^{2+}}}. Lo sono le istidine cariche positivamente nella proteina. 
\subsubsection{Gene di endonucleasi homing}
Le inteine possono includere un homing endonuclease gene \emph{HEG} dominio oltre ai domini di splicing. Questo dominio \`e responsabile per la diffusione dell'inteina rompendo il DNA sull'allele libero dell'inteina
sul cromosoma omologo, attivando il sistema di riparazione del danno che copia il DNA codificante l'inteina in un sito prima privo di essa. Sono simili agli introni mobili di gruppo I. Il dominio \emph{HEEG} non \`e necessario
per lo splicing della proteina. 
\section{Regolazione e modifica post-traduzionale di proteine}
L'attivit\`a e stabilit\`a delle proteine \`e regolata da modifiche post-traduzionali, la via pi\`u veloce per regolare l'attivit\`a della cellula ed \`e regolata da enzimi. Pu\`o essere reversibile o irreversibile, 
fisiologica o patologica. Tutte le varie modifiche a livello di trascrizione, traduzione e modifiche post-traduzionale porta a un grado di complessit\`a estremamente elevato. L'alto livello di complessit\`a delle informazioni
\`e dovuto a modifiche post-traduzionali delle proteine, che permettono per cambi veloci in risposte e adattamenti cellulari: il $5\%$ del proteoma umano \`e composto da enzimi che catalizzano pi\`u di $200$ diversi tipi 
di modifiche post-traduzionali di proteine. Che possono esistere di vari tipi in parallelo sulla proteina. Le modifiche sono enzimatiche, pertanto molto dinamiche e reversibili.
\subsection{Ubiquitinaizone delle proteine}
L'ubiquitinazione \`e una grande modificazione post-traduzionale e consiste nell'addizione di una proteina in un altra. L'ubiquitina \`e un peptide eucariote di $76$ amminoacidi che marca la proteina per
degradazione o ha un ruolo nel processo regolatorio come la risposta al danno del DNA\@. Viene covalentemente attaccata al gruppo amminoacido della lisina o anche non canonicamente a metionina, cisteina,
serina o treonina. La molecola di ubiquitina aggiunge $8.5\si{k\dalton}$. Oltre al proteosoma pu\`o portare proteine anche a proteasi nei lisozimi. Il legame tra Ubiquitina e lisina avviene tra 
gruppo carbossilico e amminico. Un processo enzimatico complesso che coinvolge tre enzimi \emph{E1-3} con \emph{ATP}. Il gruppo di ubiquitina \`e legato da lisina con gruppo carbossilico dell'ultimo 
amminoacido della coda: glicina $76$.
\subsubsection{Poli-ubiquitinazione}
Nella proteina si trovano anche lisine distribuite su tutta la lunghezza e che possono essere substrati per l'ubiquitina: pu\`o pertanto avvenire poli-ubiquitinazione, importante in quanto fa parte di
pathway di segnalazione. In base al punto di poli-ubiquitinazione si attiva un pathway diverso. Sulla lisina della proteina viene ubiquitinato attraverso il legame tra glicina $76$ e lisina. 
Nell'ubiquitina si osserva una lisina $48$, substrato per il prossimo gruppo ubiquitina. La lisina $48$ si lega pertanto alla glicina di un'altra ubiquitina. Diverse ubiquitine inoltre si possono
legare a pi\`u lisine diverse di una proteina oltre che a un'ubiquitina gi\`a aggiunta. Si formano pertanto catene di poli-ubiquitina lineari, ramificate, eterogenee od omogenee che danno vita a 
diverse conformazioni strutturali in base al linkage. Ubiquitinazioni multiple sono classificate dalla lisina nella prima ubiquitina a cui si attacca la successiva. Si possono trovare nella cellula
catene di ubiquitina libere: questo avviene in quanto essa si stacca dalla proteina in modo da poter essere riciclata. Esistono inoltre anche altre modifiche simili  all'ubiquitinazione che possono
essere incluse nelle catene.
\paragraph{Mono-ubiquitinazione}
La mono-ubiquitinazione causa alla proteina del trafficking, endocitosi o espressione-silenziamento genico. 
\paragraph{Ubiquitinazione \emph{K11}}
L'ubiquitinazione \emph{K11} porta a degradazione della proteina.
\paragraph{Ubiquitinazione \emph{K48}}
L'ubiquitinazione \emph{K48} porta a degradazione della proteina nel proteosoma.
\paragraph{Ubiquitinazione \emph{K63}}
L'ubiquitinazione \emph{K63} porat a tolleranza del danno al DNA, attivazione delle chinasi, trafficking e traduzione non-proteolitica. 
\paragraph{Ubiquitinazione \emph{K6}}
L'ubiquitinazione \emph{K6} porta a una risposta infiammatoria.
\subsubsection{Pathway di ubiquitinazione}
Il processo di ubiquitinazione coinvolge tre gruppi di enzimi \emph{E1-3} e dipende dalla presenza di \emph{ATP}. La grande variet\`a di enzimi diversi garantisce grande specificit\`a per 
l'ubiquitinazione.
\paragraph{\emph{E1} - attivazione dell'ubiquitina}
L'ubiquitina viene ricevuta da un ubiquitin-activating enzyme creando un legame tra il gruppo carbossilico sulla glicina $76$ e il gruppo sulfidrilico sulla cisteina di \emph{E1}. Il processo
richiede \emph{ATP}, che funziona da sistema di controllo del processo. \emph{E1} in presenza di \emph{ATP} si lega all'ubiquitina e rilascia \emph{AMP-$PP_i$} creando un legame tioestere. 
\paragraph{\emph{E2} - coniugazione dell'ubiquitina}
Avviene una trans-tioesterificazione: \emph{E1} rilascia ubiquitina e si forma un nuovo legame tioestere con \emph{E2}.
\paragraph{\emph{E3} - trasferimento dell'ubiquitina}
L'enzima ubiquitina ligasi \emph{E3} quando riceve l'ubiquitina da \emph{E2} si trova gi\`a legato al substrato e catalizza il trasferimento di ubiquitina da \emph{E2} alla proteina. 
\subparagraph{\emph{E3 RING} ubiquitina ligasi}
Il trasferimento avviene da \emph{E2} al substrato direttamente. Un dominio RING finger coordina \emph{$Zn^{2+}$} attraverso cisteine e residui di istidina a intervalli regolari. Il RING unisce 
\emph{E2} e il substrato insieme e media il trasferimento dell'ubiquitina. 
\subparagraph{\emph{E3 HECT} ubiquitina ligasi}
IL trasferimento avviene da \emph{E2} a un legame estere intermedio con \emph{E3} sul gruppo suflidrilico della cisteina prima che avvenga il trasferimento sul substrato. Il dominio HECT consiste di un
lobo N N terminale che interagisce con \emph{E2} e un lobo C C terminale che contiene la cisteina del sito attivo. Il movimento di un hinge loop flessibile porta vicini il loco C e \emph{E2}. 
\subsubsection{Degradazione della proteina nel proteosoma}
La degradazione di proteine poli-ubiquitinate $K11$ e $K40$ avviene attraverso il proteosoma. Il proteosoma \`e un complesso proteico $26S$ che viene diviso in una parte centrale o particella nucleare
$20S$ che contiene $30$ proteine e due complessi regolatori $19S$ che contengono $19$ proteine e sono formati da cap e lid. 
\paragraph{Funzioni delle componenti del proteosoma}
\subparagraph{Proteine del lid}
Queste proteine si legano alle proteine del substrato e de-ubiquitinasi.
\subparagraph{Proteine del cap}
Queste proteine sono \emph{AAA-proteasi}, denaturano e spiegano la struttura terziaria della proteina substrato usando \emph{ATP} e le spingono nella parte centrale.
\subparagraph{Proteine del nuclep}
Queste proteine tagliano la proteina in frammenti lunghi tra i $3$ e i $15$ amminoacidi che possono essere poi degradati ulteriormente da proteasi. Il nucleo si divide in quattro subunit\`a: 
due $\alpha$ e due $\beta$ simmetriche rispetto al centro. Per ogni subunit\`a $\beta$ si trovano tre peptidasi. 
\paragraph{Localizzazione del proteosoma}
Il proteosoma localizza sia nel citosol, dove si associa con i centromeri, la rete del citoscheletro e le membrane esterne del ER o nel nucleoplasma, dove si associa con i corpi \emph{PML} ma non con i nucleoli.
\paragraph{Processo di degradazione}
La coda di poli-ubiquitina si lega al lid e l'anello di \emph{ATPasi} la trasporta nel nucleo, dove le peptidasi la tagliano. A questo punto i peptidi escono attraverso l'anello di \emph{ATPasi} 
sull'altro canale. La direzionalit\`a del processo \`e data da?
\subsection{\emph{SUMOilazione} delle proteine}
Altre piccole proteine simili all'ubiquitina (ubiquitin-like proteins) possono essere usate per modificare le proteine. In particolare \emph{SUMO} si attacca alle catene laterali della lisina 
durante la sumoilazione. 
Il processo viene catalizzato da enzimi simili a quelli dell'ubiquitina ed \`e importante per l'espressione genica e il targeting delle proteine, oltre a controllare la 
degradazione competendo con l'ubiquitina per i siti di legame. 
\subsubsection{Caratteristiche}
\emph{SUMO} o small ubiquitin-like modifier contiene $100$ amminoacidi. Inizia con una metionina e termina con una glicina e pesa $12kDa$. 
Negli eucarioti \`e conservata: se ne trovano tre negli umani e una nel lievito.
Si coniuga a proteine nel substrato a lisine, con consenso \emph{yKxE}, dove $y$ \`e un amminoacido idrofobico, $x$ un amminoacido e $E$ acido glutammico. 
Si attacca attraverso l'ultimo amminoacido che viene esposto solo dopo il processamento.
Usa lo stesso pathway di ubiquitina e viene tipicamente utilizzata per targettare proteine nel nucleo. 
Il processo di attacco di \emph{SUMO} \`e reversibile attraverso \emph{de-SUMOilasi}, che sono delle isopeptidasi. 
Pu\`o formare catene eterogenee con altre proteine \emph{SUMO}, con ubiquitina e \emph{NEDD8} (altra proteina simile a ubiquitina).
\subsubsection{Substrati e funzioni}
\paragraph{Proteine dei pori nucleari}
\emph{SUMO} permette il traffico nucleare.
\paragraph{Fattori di trascrizione}
\emph{SUMO} influenza l'espressione genica.
\paragraph{Proteine della riparazione e replicazione del DNA}
\emph{SUMO} influenza la stabilit\`a genomica.
\paragraph{Proteine dei cinetocori e dei centromeri}
\emph{SUMO} influenza l'integrit\`a cromosomica.
\paragraph{\emph{SUMOilazione} bilanciata}
\emph{SUMOilazione} bilanciata porta alla progressione ed arresto del ciclo cellulare, sopravvivenza cellulare o apoptosi, divisione e proliferazione cellulare, differenziamento ed invecchiamento.
\paragraph{\emph{SUMOilazione} deregolata}
\emph{SUMOilazione} deregolata porta a tumori, malattie neurodegenerative, diabete, infezioni virali e difetti nello sviluppo. 
\subsection{Regolazione dell'attivit\`a di \emph{PCNA} attraverso ubiquitinazione e \emph{SUMOilazione}}
Questo esempio viene preso dal lievito e \emph{PCNA} \`e l'antigene per le cellule nucleari proliferanti. 
Equivale alla pinza scorrevole nei batteri: recluta la DNA polimerasi al DNA ed \`e coinvolta nella replicazione e riparazione del danno al DNA.
\subsubsection{Replicazione del DNA}
\emph{Mono-SU} di \emph{$PCNA^{K_{127},K_{164}}$} con \emph{E2 Ubc9} causa una promozione della replicazione da parte di \emph{PCNA}.
\subsubsection{Danno al DNA}
\emph{Mono-Ub} di \emph{$PCNA^{K_{164}}$} attraverso \emph{E2 Rad6} ed \emph{E3 Rad18}. 
\emph{Rad5} si lega a \emph{$PCNA^{K_{164}}$} e recluta \emph{UBC13}, un \emph{E2} e \emph{MMS2}, un \emph{E3} causando una catena \emph{Poly-Ub} \emph{$Ub^{K_{63}}$} in modo che \emph{PCNA} promuova la riparazione del danno al DNA.
\subsection{Sindrome di Alzheimer}
La sindrome di Alzheimer viene causata una rottura errata di \emph{APP} (amiloid precursor protein), che contiene una regione che pu\`o essere tagliata da tre enzimi diversi a posizioni diverse: $\alpha\beta$. 
In caso di rottura corretta interviene la \emph{$\alpha$-secretasi} che forma \emph{sAPP, APPs$\alpha$}, una proteina solubile che riduce la concentrazione di \emph{$Ca^{2+}$} aumentando la neuroprotezione e la neuroplasticit\`a. 
In caso di rottura sbagliata da parte di \emph{$\beta$-secretasi} e \emph{$\gamma$-secretasi} si forma la proteina amiloide $\beta$ che forma degli aggregati essendo insolubile portando a un aumento di \emph{$Ca^{2+}$}, neurotossicit\`a e una dimensione anormale dei neuriti, portando infine alla sindrome di Alzheimer. 
La \emph{SUMOilazione} della \emph{$\beta$-secretasi} le impedisce di creare rotture non volute.
