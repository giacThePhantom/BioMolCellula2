\chapter{Traduzione}
\section{Panoramica della traduzione}
Si intende per traduzione del RNA la produzione della proteina dall'informazione contenuta in un mRNA, mentre per sintesi proteica la polimerizzazione degli amminoacidi a formare una proteina. Questi 
processi sono un processo complesso ed altamente conservato che coinvolge diverse componenti: mRNA, amminoacidi, tRNA, amminoacil-tRNA sintetasi, ribosomi e rRNA. Questo processo \`e impegnativo per la
cellula in quanto richiede molta energia e risorse nutrizionali: in una cellula di lievito che si divide ogni $90min$ le proteine devono essere sintetizzate per la crescita e divisione, pertanto vengono
prodotti $33$ ribosomi al secondo.
\subsection{Ribosomi}
Il ribosoma \`e formato da:
\begin{itemize}
	\item $4$ rRNA sintetizzati negli eucarioti da RNA polimerasi I e III.
	\item $80$ proteine strutturali i cui geni sono parti del \emph{RP regulon} (ribosomal protein).
	\item $50\%$ dell'attivit\`a della RNA polimerasi II \`e dedicata alla produzione di proteine ribosomali.
	\item $236$ proteine sono necessarie per costruire un ribosoma attive e questi geni fanno parte del \emph{Ribi regulon}.
\end{itemize}
La sintesi delle proteine varia negli eucarioti da $2$ (crescita lenta) a $6$ (crescita veloce) amminoacidi al secondo, mentre nei procarioti tra i $10$ e i $20$. La velocit\`a pu\`o essere aumentata 
dall'attivit\`a di multipli ribosomi (da $2$ a $100$) su un mRNA singolo che formano il polisoma. Si definisce pertanto ribosoma il macchinario che opera la traduzione, il monosoma $1$ ribosoma 
attaccato a $1$ mRNA traduzionalmente attivo e polisoma un numero di ribosomi legati a $1$ mRNA traduzionalmente attivi. 
\subsection{Produzione e traduzione di mRNA}
Si nota come si la produzione che la traduzione di mRNA avvengono in direzione $5'$-$3'$. Inoltre avvenendo la traduzione nel citoplasma si rende necessario negli eucarioti un esporto nucleare degli mRNA.
\subsection{Confronto tra trascrizione e traduzione}
\subsubsection{Procarioti}
La trascrizione totale del RNA in una cellula di E. coli viene svolta da \num{1000} fino a \num{10000} RNA polimerasi e procede a velocit\`a massimale di $40$-$80\frac{nt}{sec}$. La traduzione 
di E. coli ha velocit\`a di $20\frac{aa}{sec}$ si nota come i due tassi sono quasi uguali. Infatti se la traduzione fosse pi\`u veloce della trascrizione i ribosomi colliderebbero con la RNA polimerasi 
nei procarioti, dove i processi possono avvenire simultaneamente. Inoltre i ribosomi sono importanti per trascrizione veloce: sembrano impedire alla RNA polimerasi di fare backtrack e pause, creando
pertanto un accoppiamento inverso tra traduzione e trascrizione. 
\subsubsection{Eucarioti}
La trascrizione in cellule mammifere ha tassi di allungamento simili a quelli misurati in E. coli tra i $50$ e i $100\frac{nt}{sec}$. Viene suggerito che queste lunghezze di trascrizione rapida siano
intervallate con lunghe pause portando a un tasso medio un ordine di grandezza inferiore rispetto a E. coli. L'allungamento della traduzione in cellule mammifere \`e anch'esso pi\`u lento: in condizioni
ottime $6\frac{nt}{sec}$. 
\section{Ribosoma}
\subsection{Scoperta}
Nel $1941$ viene scoperta la correlazione tra RNA e livelli di proteine. Nel $1943$ attraverso centrifugazione ad alta velocit\`a viene identificata una frazione che contiene la maggior parte del RNA
nel citoplasma e poi identificata come il ER ruvido. Nel $1955$, attraverso microscopia elettronica vengono osservate granuli densi liberi nel citoplasma o legati al ER che contengono il RNA. Nel
$1958$ viene scoperto il ribosoma. Si nota come i ribosomi legati alla membrana ER sintetizzano proteine per l'esporto cellulare o inserzione in membrane, mentre quelli liberi nel citoplasma le 
sintetizzano per tutte le attivit\`a citosoliche e nucleari. 
\subsection{Composizione}
I ribosomi sono grandi da $2.5$ fino a $4mDa$ e sono formati per $\frac{2}{3}$ da rRNA e per $\frac{1}{3}$ da proteine. Si dividono nella subunit\`a minore che decifra il mRNA e media l'interazione
tra mRNA e tRNA e la subunit\`a maggiore che catalizzala formazione di legami peptide tra gli amminoacidi e possiede un tunnel attraverso cui esce il peptide di crescita. L'interfaccia tra le 
subunit\`a \`e importante per i movimenti di tRNA e mRNA nel ribosoma. Un errore avviene una volta ogni \num{1000}-\num{10000} monomeri. I fattori di traduzione, spesso \emph{GTPasi} si associano 
con i ribosomi per aiutare la traduzione, composta da quattro fasi principali: iniziazione, allungamento, terminazione, separazione delle subunit\`a ribosomiali e riciclo. 
\subsubsection{Confronto tra ribosomi eucarioti e procarioti}
I ribosomi eucarioti e batterici sono funzionalmente e strutturalmente conservati ma differiscono nella composizione proteica.
\begin{multicols}{2}
	Ribosoma batterico ($70S$):
	\begin{itemize}
		\item Subunit\`a maggiore ($50S$):
			\begin{itemize}
				\item rRNA $5S$ lungo $120nt$.
				\item rRNA $23S$ lungo \num{2900}$nt$.
				\item Circa $34$ proteine \emph{L1-L34}.
			\end{itemize}
		\item Subunit\`a minore ($30S$):
			\begin{itemize}
				\item rRNA $16S$ lungo \num{1540}$nt$.
				\item Circa $21$ proteine \emph{S1-S21}.
			\end{itemize}
	\end{itemize}
	\columnbreak
	Ribosoma eucariote ($80S$):
	\begin{itemize}
		\item Subunit\`a maggiore ($60S$):
			\begin{itemize}
				\item rRNA $5S$ lungo $120nt$.
				\item rRNA $5.8S$ lungo $160nt$.
				\item rRNA $28S$ lungo \num{4700}$nt$.
				\item Circa $49$ proteine.
			\end{itemize}
		\item Subunit\`a minore ($40S$):
			\begin{itemize}
				\item rRNA $18S$ lungo \num{1900}$nt$.
				\item Circa $33$ proteine.
			\end{itemize}
	\end{itemize}
\end{multicols}
\subsubsection{Ruolo del rRNA}
La struttura del ribosoma \`e determinata da quella del proprio rRNA. In una subunit\`a le proteine estendono braccia nelle regioni di rRNA, solitamente altamente basiche e si pensa aiutino con 
l'impacchettamento dei backbone fosfati degli rRNA negativamente carichi. A causa dell'alto bisogno dei ribosomi i geni degli rRNA sono i pi\`u intensamente trascritti: il $90\%$ di tutto il RNA \`e
rRNA e viene organizzato in vettori. Gli rRNA processati da pre-rRNA vengono divisi in domini: il $16S$ ($18S$ negli eucarioti) possiede tre domini principali e uno minore, mentre il $23S$ ($28S$ negli
eucarioti) possiede $6$ domini. Questi domini sono distinti nella piccola subunit\`a, mentre nella grande sono intrecciati. La subunit\`a maggiore possiede inoltre sia in batteri che eucarioti il 
$5S$ rRNA e solo negli eucarioti il $5.8S$ rRNA. Gli RNA sono cruciali per l'attivit\`a dei ribosomi in quanto la regione decodificante del rRNA $16S$ media l'interazione tra il tRNA e mRNA e tra
mRNA e il ribosoma, mentre il rRNA $23S$ nel centro della peptidil-trasferasi interagisce con il tRNA. 
\subsection{Il ribosoma come un ribozima}
Gli rRNA catalizzano la sintesi proteica. Si nota come le componenti dei ribosomi a RNA erano presenti prima di quelle proteiche che sono state reclutate al ribosoma pi\`u tardi nell'evoluzione. Si
pu\`o pertanto considerare il ribosoma come un ribozima coperto da proteine. Questo si deduce sia funzionalmente: il rRNA $23S$ estratto dalla subunit\`a maggiore $50S$ dei procarioti senza
proteine attaccate ad esso pu\`o catalizzare il legame tra i peptidi con bassa efficienza che strutturalmente: i siti  rRNA $16S$ e peptidil-trasferasi (rRNA $23S$) sono formati da rRNA e non si 
trova vicino alcuna proteina ribosomiale. 
\subsubsection{Modifiche ai nucleosidi degli rRNA}
Gli rRNA eucarioti e procarioti contengono nucleotidi alternativi a causa dei modifiche post-trascrizionali enzimatiche dei quattro nucleotidi canonici.
\subsubsection{Proteine ribosomiali}
Le proteine nel ribosoma supportano il piegamento del rRNA garantendo ad esso e al ribosoma la struttura corretta per l'attivit\`a, aumentano l'efficienza del processo di traduzione. Per questi motivi ogni
proteina ribosomiale \`e essenziale per la vitalit\`a della cellula. Sono pertanto altamente conservate. Le proteine sono grandi, globulari e basiche. Si localizzano esternamente ma con code che
protrudono nella struttura del rRNA. I ribosomi eucariotici possiedono $82$ proteine, $27$ in pi\`u rispetto a quelli procarioti. Sono identificate come \emph{rpS``numero''} per quelle presenti 
nella subunit\`a minore e \emph{rpL``numero''} per quelle nella maggiore. 
\subsection{Assemblaggio dei ribosomi negli eucarioti}
Negli eucarioti il primo passo per l'assemblaggio dei ribosomi inizia nel nucleolo, dove rRNA $35S$, rRNA $5S$, proteine ribosomiali, fattori proteici non ribosomiali e snoRNA vanno a formare il pre-$90S$.
Questo complesso esce dal nucleolo e si divide in pre-$60S$ con gli rRNA $27S$ e $5S$ e nel pre-$40S$ con il rRNA $20S$. Questi due sono i pre-ribosomi. I pre-ribosomi escono dal nucleo attraverso due 
complessi del poro nucleare ed entrano nel citoplasma, dove maturano nel $60S$ con rRNA $25S$, $5.8S$ e $5S$ e nel $40S$ con rRNA $18S$.
\subsubsection{Nucleoli}
I nucleoli sono ripetizioni di loci di rDNA eucariote clustered trascritto da RNA polimerasi I e III.
\subsection{Rappresentazione funzionale del ribosoma}
Il ribosoma possiede nella subunit\`a minore un canale per il passaggio del mRNA. Dal tunnel poi si formano due fori: il sito A e il sito P. Nel sito A viene decodificato il codice genetico e caricato 
l'amminoacido corretto. Il sito $A$ finisce nella subunit\`a minore dove comunica con il tunnel di uscita della catena peptidica nascente. Dopo che \`e stato caricato nel sito A il tRNA si sposta 
nel sito P dove avviene il legame peptide con la catena peptidica in crescita. Il tRNA perde l'amminoacido e passa al sito E dove viene rilasciato nel citoplasma. 
\section{tRNA e codice genetico}
Negli anni $50$ Crick ipotizza l'esistenza di una molecola adattartice, un link fisico tra un codone nel mRNA e l'amminoacido a lui corrispondente. Gli adattatori sono piccoli RNA lunghi tra i $74$ e i 
$94nt$ o tRNA. Questi decifrano il mRNA e trasportano un amminoacido. Si trova almeno $1$ tRNA per ogni amminoacido, per $20$ se ne trovano pi\`u di $40$. 
\subsection{Struttura}
La struttura del tRNA possiede quattro regioni di RNA a doppio filamento e tre stem-loop $T$, $A$ e $D$. Il tRNA possiede nucleotidi modificati il loop $D$ $DHU$ prende il nome dalla di-idro-uridina, 
mentre il loop $T$ $TpsiC$ possiede ribo-timidina e pseudo-uridina $\phi$. Le terminazioni $5'$ e $3'$ si accoppiano e formano lo stem accettore, con una coda $3'$ conservata $CCA$ che lega l'amminoacido. 
Il loop dell'anticodone possiede $3$ nucleotidi che si accoppiano con il codone nel mRNA. I tRNA hanno sequenze abbastanza conservate, mentre la struttura lo \`e latamente. Possiedono molti nucleotidi 
modificati a causa di modifiche post-trascrizionali enzimatiche che gli danno un identificativo univoco. In particolare l'inosina e le sue varianti modificate e le puirine $A$ o $G$ in posizione $37$
che affiancano l'anticodone sono ipermodificate per impedire che la base $37$ si accoppi con il codone nel mRNA e per produrre un loop stabile $AC$. 
\subsection{Sintesi}
Nel nucleo si forma il trascritto iniziale dei tRNA che negli eucarioti possiede un introne. Successivamente vengono processate le terminazioni e aggiunta la coda $3'$ $CCA$. Viene trasportato nel 
citoplasma dove si trova nella forma matura. 
\subsection{Codoni}
Un codone tripletta che specifica per un singolo amminoacido si dice sense codon ($61$), mentre uno che non lo fa \`e detto stop codone o nonsense codone ($3$). Si noti come il codice a triplette \`e il
pi\`u semplice che pu\`o specificare tutti e $20$ gli amminoacidi. Viene utilizzato ogni codone possibile, pertanto alcuni amminoacidi sono codificati da pi\`u di un codone: in molti casi i primi due
nucleotidi sono lo stesso e differisce lo stesso, ma non sempre. Solo i tRNA per metionina e triptofano riconoscono un singolo codone. I diversi tRNA che trasportano lo stesso amminoacido sono
detti iso-accettori. 
\subsubsection{Interazione codone anti-codone}
Il codone nel mRNA interagisce con l'anti-codone nel $U$-turn del tRNA. La posizione $1$ e $2$ sul mRNA (da $5'$ a $3'$) sono letti da accoppiamenti rigidi con le posizioni $2$ e $3$ sull'anticodone. 
Per la posizione $3$ del mRNA sono permesse deviazioni o wobble pairing che permette interazioni non canoniche come $G$-$U$. In alcuni casi con inosina. Pertanto ogni codone non necessita del proprio tRNA:
ce ne sono circa $40$ per i $61$ codoni di senso. I pi\`u comuni sono: $U-G$, $I-U$, $I-C$, $I-A$. 
\subsubsection{Codoni rari}
Alcuni codoni sono utilizzati meno frequentemente di altri e sono detti codoni rari, codificati da tRNA rari. Il codice genetico \`e pi\`u o meno lo stesso in tutti gli organismi e l'evoluzione ha 
conservato i codoni in modo che le mutazioni che cambiano l'amminoacido codificato risultano in un amminoacido simile che lo modifica. Ci sono eccezzioni come nei mitocondri e in mycoplasma. 
\subsection{Legame con il ribosoma}
Il tRNA si lega in sequenza a tre siti nel ribosoma: il sito amminoacile, il sito peptide e il sito di uscita. Il centro di decodifica \`e l'area tra il basso di A e il canale del mRNA e il
centro di trasferimento peptide \`e all'alto del sito P.
\section{Amminoacil-tRNA sintetasi}
L'amminoacil tRNA sintetasi attacca un amminoacido sul tRNA a $3'AAC$, ne esistono di $24$ tipi per i $20$ amminocidi. Una \emph{aaRS} per un amminoacido, come la glicina \`e definita come \emph{GlyRS}.
Il tRNA su cui un certo amminoacido deve essere taricato \`e identificato da certi marcatori a modifiche nucleosidiche. Si nota come ogni amminoacido possieda una amminoacil-tRNA sintetasi specifica e
quello che viene caricato su un tRNA specifico determina la nomenclatura del tRNA com e\emph{$tRNA^{Met}$}. Si dice affine (``cognate'') l'amminoacido corretto. I tRNA vengono riconosciuti grazie a 
caratteristiche di struttura e di sequenza o elementi di identit\`a.
\subsection{Struttura}
La maggior parte delle proteine \emph{aaRS} contengono un sito di amminocilazione e uno di editing. Esistono due classi dell'enzima con siti attivi differenti strutturalmente e che si legano a tRNA
diversi strutturalmente. 
\subsubsection{Classe I}
Le \emph{aaRS} di classe $I$ legano l'amminoacido al \emph{$2'$-OH} del ribosio dell'adenosina, agiscono come monomeri o dimeri e si legano alla fessura minore dello stem accettore.
\subsubsection{Classe II}
Le \emph{aaRS} di classe $II$ legano l'amminoacido al \emph{$3'$-OH} del ribosio dell'adenosina, agiscono come dimeri o tetrameri e si legano alla fessura maggiore dello stem accettore. 
\subsubsection{Sito catalitico o di attivazione}
In questo dominio avviene il caricamento dell'amminoacido a due passaggi e dipendente dall'ATP.
\subsubsection{Sito di legame per l'anticodone}
Questo dominio interagisce con l'anticodone del tRNA e garantisce il legame del tRNA corretto all'amminoacido.
\subsubsection{Sito di editing}
Questo dominio rimuove un amminoacido aggiunto scorrettamente. 
\subsection{Caricamento}
Il caricamento dell'amminoacido \`e accurato, con meno di un errore ogni $10^4$ eventi di amminoacilazione grazie al controllo qualit\`a del modello a doppio setaccio. L'amminoacido corretto 
viene scelto in un processo a due passi.
\subsubsection{Attivazione amminoacilica}
Quando deve essere caricato l'amminoacido viene attivato attaccando \emph{AMP} che rilascia pirofosfato e fornisce l'energia. L'amminoacil-adenilato attivato rimane attaccato all'enzima \emph{aaRS}. 
\subsubsection{Trasferimento dell'amminoacido}
Dopo l'attivazione l'enzima trasferisce l'amminoacido al $2'$ o al $3'$ del ribosio dell'adenosina sulla coda \emph{CCA} del tRNA.
\subsection{Correzione degli errori}
Alcuni amminoacidi sono molto simili strutturalmente come i tRNA ed essendo il ribosoma incapace di distinguere tra i tRNA caricati correttamente o scorrettamente avviene un processo di correzione
degli errori da parte della \emph{aaRS}: $10$ su $24$ compiono editing nei loro siti.
\subsubsection{Selezione degli amminoacidi - il modello a doppio setaccio}
\paragraph{Primo setaccio}
Il primo setaccio si trova al sito di attivazione e svolge un esclusione per dimensione, \`e grossolano ed esclude amminoacidi troppo grandi. 
\paragraph{Secondo setaccio}
Il secondo setaccio si trova al sito di editing o di proofreading o idrolitico, \`e fine e idrolizza gli \emph{aa-AMP} troppo piccoli nel pre-transfer editing. 
\paragraph{Post-transfer editing}
Se un \emph{aa-AMP} viene incluso nonostante il doppio setaccio avviene post-transfer editing in trans. In questo processo il tRNA mal-etichettato lascia il \emph{aaRS} e fattori idrolitici indipendenti
come \emph{YbaK} o \emph{AlaXP} rimuovono l'amminoacido. 
\subsubsection{\emph{aaRS} senza capacit\`a di editing}
Solo $10$ delle $24$ \emph{aaRS} possono editare e quelle di classe II sono le pi\`u efficienti. La presenza di siti di editing garantisce un vantaggio della sopravvivenza con migliori condizioni con
una pi\`u alta proposizione di proteine correttamente sintetizzate. Gli amminoacidi pi\`u usati correlano con le \emph{aaRS} con un sito di editing. 
\paragraph{Mal-traduzioni}
Le \emph{aaRS} senza capacit\`a di editing creano un rischio di mal-traduzione che possono causare la risposta delle proteine non piegate che causa la loro degradazione. Altre cellule tollerano 
la mal-traduzione con un effetto sulla vitalit\`a visibile solo in condizioni di stress. La mal-traduzione pu\`o avere un beneficio adattivo con sopravvivenza in condizioni negative.
\section{mRNA}
Il rRNA \`e molto stabile e non permette alla cellula di rispondere a condizioni variabili abbastanza velocemente. Si necessita pertanto di un RNA come portatore delle informazioni genetiche con una
vita corta e che pu\`o essere creato e degradato velocemente: il RNA messaggero. Ha un'abbondanza percentuale bassa rispetto a rRNA $80\%$, tRNA $15\%$ del $5\%$ e viene rapidamente turned over, con
un tempo vitale di minuti nei batteri e dai minuti alle ore negli eucarioti.
\subsection{Struttura nei procarioti}
Nei procarioti il mRNA si presenta principalmente policistronico con una corta \emph{$5'$-UTR} e varie regioni codificanti separate da regioni intercistroniche. Alla terminazione presenta una 
corta sequenza \emph{$3'$-UTR}. 
\subsection{Struttura negli eucarioti}
Negli eucarioti il mRNA si presenta principalmente monocistronico con un cap $5'$, una lunga \emph{$5'$-UTR}, una regione codificante, una lunga \emph{$3'$-UTR} e una coda poli-A. La \emph{$3'$-UTR} 
contiene elementi stabilizzatori e di regolazione. 
\section{Ciclo di traduzione}
La traduzione coinvolge quattro passaggi. 
\subsection{Iniziazione}
Il $AUG$ all'inizio del reading frame aperto viene identificato da fattori di iniziazione, dal ribosoma e da un tRNA iniziatore metionina speciale. Iniziazione precoce coinvolge il legame della
subunit\`a minore ribosomiale al mRNA e la subunit\`a maggiore alla minore. Questo risulta in un ribosoma legato al mRNA con un tRNA caricato a metionina legato al sito P. Il ribosoma \`e ora pronto
per muoversi lungo il mRNA. 
\subsection{Allungamento}
Il fattore di allungamento \emph{Tu} nei batteri, \emph{eEF1A} negli eucarioti carica il successivo tRNA carico nel sito A in base al codone nel mRNA. La formazione del legame peptidico \`e catalizzata
tra l'amminoacido nel sito P e quello nel sito A. La reazione trasferisce il polipeptide in crescita al tRNA nel sito $A$. Successivamente \emph{EFG} nei batteri e \emph{eEF2} negli eucarioti promuove
il movimento del ribosoma sul codone successivo attraverso traslocazione. Questo muove il peptidil-tRNA che era nel sito $A$ nel sito $P$ e porta un nuovo codone nel sito $A$. Il tRNA che si trova
ora nel sito $E$ lascia il ribosoma.
\subsection{Terminazione e riciclo dei ribosomi}
La terminaizone avviene quando il ribosoma raggiunge un codone di stop $UAG$, $UAA$ o $UGA$ che vengono riconosciuti da fattori di rilascio di classe I. Il fattore di rilascio batterico \emph{RF1} 
riconosce $UAA$ e $UAG$, \emph{RF2} $UAA$ e $UGA$. Negli eucarioti \emph{eRF1} riconosce tutti i codoni di stop. L'interazione tra codone di stop e \emph{RF1/2} promuove il rilascio del polipeptide. 
Infine la subunit\`a maggiore e minore si dissociano e rilasciano il rimanente tRNA e mRNA. Il fattore di riciclo \emph{RRF} e \emph{EFG} aiutano i ribosomi a dissociarsi nei batteri.
\section{Iniziazione della traduzione - caratteristiche comuni a batteri ed eucarioti}
Ci sono tre passaggi per l'iniziazione della traduzione raggiunti diversamente in eucarioti e batteri:
\begin{itemize}
	\item La subunit\`a ribosomiale minore identifica il codone di inizio nel mRNA.
	\item Un \emph{metionil-$tRNA^{Met}$} \`e caricato nel sito $P$ del ribosoma e si accoppia con le basi del codone di inizio.
	\item La subunit\`a ribosomiale maggiore unisce il complesso.
\end{itemize}
Il codone di inizio \`e tipicamente $AUG$ e viene decodificato dall'iniziatore $tRNA^{Met}$ che differisce in eucarioti $tRNA_I^{Met}$ e batteri $tRNA_f^{Met}$, dove $f$ denota un gruppo formile. 
Inoltre diversi fattori di iniziazione $IF$ (\emph{GTPasi}) sono ciunvolti nel legame del \emph{metionil-$tRNA^{Met}$} o di \emph{fmetionil-$tRNA^{Met}$} al sito $P$ di eucarioti e batteri. 
L'iniziatore batterico ha un mis-match $C$-$A$ nello stem accettore, mentre l'iniziatore \emph{tRNA} eucariotico ha un match $A$-$U$. Entrambi possiedono tre coppie $G$-$C$ nello stem anticodone, 
importanti per legare il fattore di allungamento \emph{EF-Tu} (batteri) o \emph{eIF1A} (eucarioti).
\subsection{Il tRNA iniziale nei procarioti}
Sia che in procarioti che in eucarioti la sintesi della proteina inizia con l'inclusione di una metionina $AUG$, $GUG$, $UUG$, ma solo nei procarioti la \emph{Met} iniziale \`e una \emph{formil-Met}. 
Dopo il legame al proprio $tRNA_I$ la metionina viene modificata enzimaticamente con un gruppo formile alla terminazione N formando \emph{fMet-$tRNA_f^{Met}$}. L'enzima coinvolto \`e la
metionil-tRNA trasformilasi e \emph{fMet} non pu\`o essere utilizzata per l'allungamento delle proteine in quanto il gruppo \emph{\ce{NH2}} \`e bloccato. 
\section{Iniziazione della traduzione batterica}
Gli mRNA batterici sono tipicamente policistronici: contengono diversi reading frame aperti, ognuno con il proprio codone di start e stop. I codoni di iniziazione possiedono tipicamente una 
sequenza di Shine-Dalgarno, un tratto a poli-purine che si trova tra i $7$ e i $13nt$ a monte del codone di inizio $AUG$. Shine-Dalgarno si accoppia con una regione a poli-pirimidine nella terminazione
$3'$ del rRNA $16S$ batterico, la sequenza anzi-Shine-Dalgarno. L'interazione tra le basi guida l'iniziatore $AUG$ nel sito P del ribosoma. 
\subsection{Riconoscimento della sequenza Shine-Dalgarno}
Il complesso di pre-iniziazione formato dalla subunit\`a $30S$, mRNA, \emph{fMet-$tRNA^{Met}$} e IF deve assemblarsi sul codone di inizio $AUG$. Si deve pertanto distinguere tra i diversi codoni $AUG$
quello di inizio. Mettendo in vitro i ribosomi, mRNA, tRNA e GTP questi formano un complesso che non pu\`o allungarsi. Successivamente si tratta con RNA endonucleasi ed esonucleasi e si isola mRNA 
protetto dai ribosomi. Si sequenzia il frammento trovato e si nota come si trova una sequenza conservata di $7nt$ $AGGAGGU$ dai $7$ ai $13nt$ a monte di $AUG$ che \`e il sito di legame del ribosoma
o sequenza di Shine-Dalgarno \emph{SD}. Si nota come il consenso della sequenza anti-Shine-Dalgarno con essa controlla la forza della traduzione. 
\subsection{Assemblaggio del complesso di inizio}
La subunit\`a minore del ribosoma $30S$ si posiziona sul mRNA grazie alla terminazione $3'$ del rRNA $16S$ che riconosce la sequenza di \emph{SD}. Successivamente viene reclutata la subunit\`a maggiore
$50S$. Si nota come essendo il mRNA procariote poli-cistronico sequenze di Shine-Dalgarno e $AUG$ sono presenti all'inizio di ogni cistrone. 
\subsubsection{Fattori di inizio}
Nei procarioti tre fattori di inizio guidano \emph{fMet-$tRNA_f^{Met}$} al sito $P$: \emph{IF1}, \emph{IF2} e \emph{IF3}.
\paragraph{\emph{IF1} e \emph{IF3}} Questi due fattori legano i siti $A$ ed $E$ nella subunit\`a minore in assenza di mRNA o di \emph{fMet-$tRNA_f^{Met}$}, direzionando il tRNA iniziatore al sito $P$ e 
impedendo un legame inappropriato con la subunit\`a maggiore. 
\paragraph{\emph{IF2}} Questo fattore di inizio \`e una \emph{GTPasi} che idrolizza \emph{GTP} per fornire l'energia necessaria all'unione delle subunit\`a del ribosoma. L'idrolisi del \emph{GTP} causa
inoltre il suo rilascio.
\paragraph{\emph{IF1} e \emph{IF3}} Questi due poi vengono rilasciati quando le subunit\`a si combinano completando l'iniziazione.
\section{Iniziazione della traduzione eucariotica}
Gli mRNA eucarioti codificano tipicamente per una proteina (sono monocistronici). Non si trovano equivalenti eucariotici per la sequenza di Shine-Dalgarno e la subunit\`a minore non si lega direttamente
al mRNA e si necessitano di pi\`u fattori di inizio. L'iniziazione avviene tipicamente al primo $AUG$ del mRNA. Questo pu\`o essere inefficiente e causare l'iniziazione al secondo o terzo $AUG$. Il
riconoscimento di $AUG$ \`e sensibile al contesto di sequenza: sequenze di Kozak con consenso $A/GXXAUGG$. Si nota come il primo metile non \`e formilato e la subunit\`a minore si lega al cap $5'$ del
mRNA che diventa il sito di legame per il ribosoma. Successivamente si muove lungo il mRNA al primo $AUG$ facendo scanning in una sequenza di Kozak. La sequenza di Kozako lo scansionato ha un consenso
debole e nei vertebrati \`e $GCC\ CCAUGG$ e a differenza dei procarioti non \`e il sito di legame del ribosoma \emph{RBS}. Il sito di legame del ribosoma \`e il cap $5'$ o \emph{IRES} (internal ribosome
entry site). 
\subsection{Processo di iniziazione}
\subsubsection{Complesso del loop chiuso}
Il cap $5'$ e la coda poli-A del mRNA sono coinvolti nell'iniziazione in quanto preparano il mRNA affinch\`e venga scansionato dal ribosoma. Il cap $5'$ viene legato da \emph{eIF4E} e la coda
$3'$ da \emph{PABP} (polyA binding protein) che interagiscono tra di loro  attraverso un complesso di altri fattori di inizio (\emph{4G, 4A, 4B}) formando un complesso a loop chiuso. Questo complesso 
protegge il mRNA dall'attivit\`a esonucleasica e potrebbe funzionare anche come un sistema di controllo di qualit\`a per eliminare mRNA non processati o danneggiati impedendo che vengano tradotti in 
proteine. 
\subsubsection{Reclutamento del complesso di iniziazione}
\emph{eEF1A} e \emph{eEF1} si legano nei siti $A$ ed $E$ del ribosoma e \emph{met-$tRNA_I^{Met}$} al sito $P$ della subunit\`a minore del ribosoma $40S$. Il complesso viene poi reclutato 
al complesso del loop chiuso attraverso interazioni tra \emph{eIF3} e \emph{eIF4G}, rispettivamente sul complesso di iniziazione e del loop chiuso. La subunit\`a ribosomiale minore, legata a un
numero di fattori di inizio forma il complesso di pre-iniziazione $43S$ e pu\`o legarsi al mRNA formando il fattore di pre-iniziazione $48S$. \emph{eIF4A}, un'elicasi \emph{eIF4B}, suo attivatore 
svolgono la struttura secondaria del cap $5'$ del mRNA e permettono al \emph{Met-$tRNA_I^{Met}$} nel complesso di pre-iniziazione $48S$ di fare una scansione per la sequenza di Kozak. \emph{eIF5B} 
catalizza l'unione della subunit\`a maggiore $60S$ e tutti i fattori di iniziazione si dissociano. Il complesso di iniziazione $80S$ formatosi \`e ora competente per l'allungamento. 
\subsection{Metodi di scansione}
\subsubsection{Scansione dipendente dal cap $5'$ e da \emph{eIF4F}}
Il complesso \emph{eIF4} formato da \emph{eIF4E + G + A + B} si lega al cap $5'$ attraverso \emph{eIF4E}. Questo recluta il complesso di pre-inizio $43S$ formando il $48S$ che scansiona il mRNA 
per $AUG$. La $5'$ \emph{UTR} inizia a formare un anello uscendo durante la scansione e quando la subunit\`a $40S$ raggiunge la sequenza di Kozak e il $AUG$ nella sequenza vengono rilasciati i fattori
di inizio ad esclusione di \emph{eIF4E-G} che rimangono legati al cap. Viene rilasciato il loop formato dal cap e viene reclutata la subunit\`a maggiore $60S$ sulla cima di $AUG$. 
\subsubsection{Iniziazione indipendente dal cap $\mathbf{5'}$ e dipendente da \emph{IRES}}
L'iniziazione indipendente dal cap $5'$ viene identificata negli RNA dei virus eucariotici e per certi mRNA. Necessita di una sequenza \emph{IRES} nel \emph{$5'$-UTR} del mRNA. Si trova spesso negli 
mRNA eucarioti policistronici.
\subsection{Presenza della metionina all'inizio della proteina prodotta}
\subsubsection{Procarioti}
Nei procarioti il gruppo formile viene sempre rimosso dalla prima metionina e spesso \`e l'intera metionina ad essere rimossa.
\subsubsection{Eucarioti}
Negli eucarioti la metionina iniziale viene rimossa da aminopeptidasi \emph{MAP1}, \emph{MAP2} essenziali per la vitalit\`a, in $\frac{2}{3}$ delle proteine. Viene rimossa quando il secondo amminoacido
\`e di piccola o media dimensione come alanina, cisteina, glicina, prolina, serina, treonina e valina. Non viene rimossa quando il secondo amminoacido \`e grande come arginina, aspargina, acido 
aspartico, glutammina, acido glutammico, isoleucina, leucina, lisina o metionina.
\subsubsection{Rimozione della metionina}
La rimozione della metionina permette il suo riciclo in quanto \`e l'amminoacido pi\`u laborioso da sintetizzare. Potrebbe inoltre ridurre l'emivita di una proteina secondo la regola della terminazione
$N$ o permettere modifiche $N$ terminali del secondo amminoacido. 
\section{Allungamento della traduzione}
L'allungamento della traduzione \`e un processo conservato. Si nota come alla fine dell'iniziazione il ribosoma \`e posizionato al primo $AUG$. \emph{Met-$tRNA_I^{Met}$} o\emph{fMet-$tRNA_f^{Met}$} 
localizza al sito $P$ e si lega al $AUG$. Ogni amminoacido si attacca alla catena polipeptidica in crescita attraverso un ciclo di:
\begin{itemize}
	\item Decodifica: entrata di un amminoacido \emph{aa-$tRNA^{aa}$} nel sito $A$.
	\item Formazione del legame peptidico tra amminoacidi nel sito $P$ grazie al rRNA $23S$ peptidiltrasferasi.
	\item Traslicazione: del mRNA:peptidil-tRNA da $A$ a $G$ lasciando ancora il sito $A$ aperto.
\end{itemize}
\subsection{Decodifica, legame di un amminoacil-$\mathbf{tRNA^{aa}}$ al sito $\mathbf{A}$}
Questo processo richiede per i procarioti i fattori di allungamento \emph{EF}: \emph{EF-Tu} (thermo unstable), \emph{Ef-Ts} (thermo stable) e \emph{EF-G} (translocation factor) e \emph{GTP}. Al sito 
$P$ si trova \emph{fMet-$tRNA_I^{Met}$} che si lega a $AUG$ portata da \emph{IF2-GTP}. Nel sito $A$ l'entrata di \emph{aa-$tRNA^{aa}$} \`e meditata dalla \emph{GTPasi} \emph{EF-Tu} legata a 
\emph{GTP} formando il complesso \emph{aa-$tRNA^{aa}$:EF-TU-GTP}. Il legame tra l'anticodone del complesso e il codone del mRNA avviene nel centro di decodifica del sito $A$, la tasca di rRNA a $16S$. 
Il corretto legame tra codone e anticodone avviene grazie a legami a idrogeno e il segno di un'associazione stabile nel sito $A$ \`e un cambio strutturale nella tasca del rRNA $16S$. Successivamente
l'idrolisi del \emph{GTP} da parte di \emph{EF-TU} causa il rilascio di \emph{ET-TU-GDP} e se il fit no \`e buono viene rimosso anche \emph{aa-$tRNA^{aa}$}. \emph{Ef-Ts}, un fattore di scambio di 
\emph{GTP} ricarica \emph{EF-Tu} ripristinandolo nello stato carico con \emph{GTP} rendendolo pronto per caricare il prossimo \emph{aa-$tRNA^{aa}$}. 
\subsubsection{Il complesso \emph{aa-$tRNA^{aa}$:EF-Tu-GTP}}
\emph{EF-Tu} \`e una delle proteine pi\`u abbondanti nei batteri, e costituisce il $5\%$ di tutte le proteine. Protegge il legame labile tra il tRNA e il proprio amminoacido quando
\emph{aa-$tRNA^{aa}$} lascia il proprio \emph{aaRS}. Inoltre \emph{EF-Tu-GTP} carica \emph{aa-$tRNA^{aa}$} rapidamente e con alta fedelt\`a nel sito $A$ del ribosoma. Infine 
aiuta a decidere se l'anticodone del \emph{aa-$tRNA^{aa}$} \`e un fit abbastanza buono quando legato al codone. 
\subsubsection{Fit corretto tra codone ed anti-codone}
L'interazione tra codone e anti-codone attraverso legami a idrogeno pu\`o essere cognate quando \`e completamente accurata, near-cognate quando ha un mal-appaiamento su una singola 
base o non-cognate quando ci sono $2$ o $3$ mal-appaiamenti. Il riconoscimento del cognate stimola un cambio conformazionale nel ribosoma determinando l'accettazione del 
\emph{aa-$tRNA^{aa}$}. L'alta fedelt\`a dell passaggio di decodifica \`e ottenuta attraverso un processo a due passi di accettazione o rifiuto. 
\paragraph{Selezione iniziale basata sul tempo}
\emph{EF-Tu} \`e una \emph{GTPasi} che usa l'energia fornita dall'idrolisi del \emph{GTP} per valutare il fit tra codone e anticodone. Prende forma un equilibrio: il ribosoma 
decide quanto bene \emph{aa-$tRNA^{aa}$} portato da \emph{EF-Tu-GTP} interagisce con il codone nel sito $A$. 
\subparagraph{Buon legame} 
Il \emph{aa-$tRNA^{aa}$} si lega pi\`u a lungo ed \`e pi\`u probabile che avvenga l'idrolisi del \emph{GTP}.
\subparagraph{Cattivo legame}
\emph{aa-$tRNA^{aa}$+EF-Tu-GFP} si separa dal ribosoma prima che possa avvenire l'idrolisi del \emph{GTP}.
\paragraph{Proofreading basato sulla struttura}
La valutazione dell'accoppiamento delle basi tra codone e anti-codone avviene grazie al $16S$ rRNA attraverso movimenti interni delle due adenosine e della guanina conservate
presenti nel centro di decodifica. Quando il $16S$ rRNA riconoscei il legame come corretto avviene un cambio confomrazionale nel ribosoma che stimola l'idrolisi del \emph{GTP} da 
parte di \emph{EF-Tu}. In questo modo \emph{EF-Tu-GDP} lascia il ribosoma e l'amminoacido viene incluso nella coda polipeptidica. Se il legame \`e scorretto \emph{aa-$tRNA^{aa}$} lascia
il sito $A$. 
\subsection{Formazione del legame peptide tra amminoacidi nel centro di trasferimento del peptidil}
Tra l'ultimo amminoacido alla terminazione $3'$ del peptidil-tRNA nel sito $P$ e il nuovo amminoacido alla terminazione $3'$ del \emph{aa-$tRNA^{aa}$} nel sito $A$ avviene un 
trasferimento del poli-peptide dal tRNA al sito $P$ sull'amminoacido di \emph{aa-$tRNA^{aa}$} nel sito $A$. Questo avviene nel sito attivo del ribosoma grazie all'attivit\`a di
peptidiltrasferasi, un ribozima rRNA $23S$. Ora i tRNA si trovano in stati intermedi $E/P$ e $P/A$. 
\subsubsection{Catalisi mediata da rRNA}
Il $23S$ rRNA o $28S$ rRNA::$5.8S$ rRNA negli eucarioti porta il \emph{peptidil-tRNA} e \emph{aa-$tRNA^{aa}$} insieme grazie loop conservati $A$ e $P$ presenti nei corrispettivi siti e
ai nucleotidi conservati in entrambi. La reazione coinvolge un attacco nucleofilo \emph{:N} su un  centro deficiente di elettroni \emph{$C^{\delta+}$}. Vengono utilizzati ioni di 
magnesio come co-fattori. 
\subsection{Traslocazione del peptidil-tRNA dal sito $\mathbf{A}$ al sito $\mathbf{P}$}
S dice traslocazione ilmovimento di \emph{mRNA:peptidil-tRNA} dal sito $A$ al sito $P$, che causa il movimento del ribosoma di $3$ nucleotidi lungo il mRNA. Il \emph{$tRNA^{aa}$}
lascia il sito $P$ per entrare nel sito $E$ da cui esce dal ribosoma. Il prossimo codone viene esposto nel sito $A$ vuoto per il prossimo \emph{aa-$tRNA^{aa}$}. La traslocazione
richiede il fattore di traslocazione \emph{GTPasi EF-G+GTP} che si lega al sito $A$ vuoto nello stato legato al \emph{GTP}. Il movimento del \emph{mRNA:peptidil-tRNA} dal sito $A$
al sito $P$ richiede l'energia fornita dall'idrolisi del \emph{GTP} legato a \emph{EF-G}. A questo punto \emph{EF-G-GDP} lascia il sito $A$ e viene ricaricato con \emph{GTP}
dal fattore di scambio del nucleotide guanina \emph{GEF}. Il \emph{aa-$tRNA^{aa}$} \`e reclutato e l'allungamento pu\`o continuare. \emph{EF-G-GTP} sposta il peptidil-tRNA dal
sito $A$ al sito $P$ e si nota come il passo di traslocazione \`e irreversibile dal momento in cui \emph{GTP} viene idrolizzata da \emph{EF-G}. \emph{EF-G} \`e simile strutturalmente
a \emph{EF-Tu:aa-$tRNA^{aa}$} e causa il movimento in avanti spingendo il tRNA. Durante la traslocazione la testa della subunit\`a $30S$ ruota di $17\si{\degree}$ attorno il 
$16S$ rRNA causando il movimento in avanti del ribosoma lungo il mRNA. Oltre a far passare i tRNA dagli stati ibridi $E/P$ e $P/A$ in $E/E$ e $P/$ l'idrolisi del \emph{GTP}
rompe il contatto tra tRNA ed mRNA permettendo l'uscita dal sito $E$. 
\subsection{Allungamento negli eucarioti}
Il processo di allungamento negli eucarioti \`e simile a quello dei procarioti. Cambiano i fattori di allungamento, che sono detti \emph{eEF} (eukaryotic Elongation Factor):
\begin{itemize}
	\item \emph{eEF1A} corrisponde a \emph{EF-Tu}.
	\item \emph{eEF1B} corrisponde a \emph{EF-Ts}.
	\item \emph{eEF2} corrisponde a \emph{EF-G}.
\end{itemize}
\section{Terminazione e reinizio della traduzione}
La traduzione continua fino a che il ribosoma incontra un codone di stop nel mRNA: $UAA$, $UAG$, $UGA$. 
\subsection{Fattori di rilascio}
\subsubsection{Fattori di rilascio di classe I}
I fattori di rilascio di classe I \emph{RF} non sono tRNA e riconosono i codoni di stop. Per i batteri sono \emph{RF1} per $UAA$ e $UAG$, \emph{RF2} per $UAA$ e $UGA$. Quello eucariotico 
\emph{eRF1} riconosce tutti e tre i codoni di stop. I \emph{RF} entrano ed occupano il sito $A$ e promuovono il rilascio idrolitico del peptide finito portando una molecola d'acqua che 
attacca il legame estere liberandolo. Si nota come \emph{RF} e tRNA sono molto simili in struttura. Gli \emph{RF} batterici e quelli eucariotici non sono imparentati ma hanno lo stesso 
motivo $GCQ$ necessario per la catalisi della reazione del rilascio del polipeptide. Nell'idrolisi del legame estere del peptide avviene la deprotonazione del \emph{$2'$-OH} dove 
l'acqua  agisce come donatrice di protoni ed elettroni.
\subsubsection{Fattori di rilascio di classe II}
I fattori di rilascio di classe II sono \emph{GTPasi}. Nei batteri \emph{RF3-GTP} rimuove \emph{RF1} o \emph{RF2} dal sito $A$ dio il rilascio del peptide \emph{RF1/RF2}: 
l'energia ottenuta dall'idrolisi del \emph{GTP} \`e utilizzata per la loro rimozione. L'ultimo tRNA nel sito $P$ prende lo stato ibrido $E/P$ vicino al sito $A$ vuoto. Negli eucarioti
il fattore di rliascio di classe II \emph{eRF3-GTP} si lega al fattore di rilascio di classe I \emph{eRF1}. Entrambi si legano al sito $A$ e questo si accoppia con l'idrolisi del
\emph{GTP} di \emph{eRF3-GTP} con il rilascio del peptide. Successivamente la \emph{ATPasi} \emph{ABCE1} rimuove \emph{eRF3-GDP} dal sito $A$ legandosi a \emph{eRF1}. 
\subsection{Riconoscimento del codone di stop}
Il riconoscimento del codone di stop \`e molto accurato e il rilascio prematuro del peptide \`e molto raro. Nonostante questo gli errori di decodifica sono pi\`u probabili durante
condizioni di stress. Un mismatch tra un codone::anti-codone viene mosso nel sito $P$ dopo il movimento di un codone da parte del ribosoma che causa un cambio conformazionale 
che diminiusce la fedelt\`a nel sio $A$ che diventa un sito $A$ a bassa fedelt\`a e \emph{RF3} viene reclutata a tale sito indipendentemente dal codone di stop. Il peptide difettoso
viene riconosciuto e degradato da peptidasi e proteasi cellulari. Questo agisce come un addizionale controllo di qualit\`a. 
\subsection{Riciclo del ribosoma}
Dopo che la traduzione \`e terminata il ribosoma deve essere rilasciato in modo che sia riciclato. Nei batteri il fattore di riciclo del ribosoma \emph{RRF} e la \emph{GTPasi} 
di traslocazione \emph{EF-G-GTP} localizza al sito $A$ vuoto promuovendo il disassemblaggio del ribosoma con l'energia fornita dall'idrolisi del \emph{GTP} di \emph{EF-G}. tRNA
e mRNA si dissociano dalla subunit\`a minore stabilizzata dal legame di \emph{IF3} al sito $E$ che previene l'evento di ri-associazione tra le subunit\`a del ribosoma. Negli eucarioti
in assenza di un ortologo del batterico \emph{RRF} \emph{eRF1} rimane legato al peptide dopo il rilascio con \emph{ABCE1} che divide il ribosoma idrolizzando \emph{ATP}. I fattori 
di iniziazione principali \emph{eIF1} \emph{eIF1a} e \emph{eIF3} successivamente si legano alla subunit\`a ribosomiale minore per impedire che si riunisca con la maggiore.
\section{Energia richiesta per la traduzione}
Circa il $60\%$ dell'energia di una cellula viene utilizzata per la sintesi delle proteine. 
\subsection{Iniziazione}
\subsubsection{Procarioti}
\begin{itemize}
	\item $1$ \emph{ATP} per caricare $aa$ sul \emph{$tRNA^{aa}$} da parte di \emph{aaRS}. 
	\item $1$ \emph{GTP} per l'iniziazione.
\end{itemize}
\subsubsection{Eucarioti}
\begin{itemize}
	\item $3$ \emph{ATP}: $1$ per \emph{aa-$tRNA^{aa}$}, $1$ per la RNA elicasi e $1$ per la scansione di $AUG$.
	\item $2$ \emph{GTP}: $1$ consumato da \emph{eIF2} per il rilascio del fattore di iniziazione e $1$ consumato da \emph{eLF5B} per unire le subunit\`a $40S$ e $60S$. 
\end{itemize}
\subsection{Allungamento}
\begin{itemize}
	\item $1$ \emph{ATP} da parte di \emph{aaRS} per creare l'amminoacido-tRNA legame acile ad alta energia. L'energia generata dalla rottura di questo legame viene utilizzata per
		catalizzare $1$ legame peptide dalla peptidil trasferasi.
	\item $1$ \emph{GTP} da parte di \emph{EF-Tu} per portare \emph{aa-$tRNA^{aa}$} al sito $A$.
	\item $1$ \emph{GTP} per la traduzione di \emph{EF-G}.
\end{itemize}
\subsection{Terminazione}
\begin{itemize}
	\item $1$ \emph{GTP} da parte di \emph{RF3/eRF3} per rilasciare \emph{RF1/2/eRF1} dal sito $A$. 
\end{itemize}
\section{Velocit\`a ed accuratezza della traduzione}
La velocit\`a della sintesi delle proteine \`e di $15\frac{aa}{sec}$ nei procarioti e $4\frac{aa}{sec}$ negli eucarioti. La maggior lentezza \`e dovuta al fatto che avviene in
un ambiente diverso rispetto alla trascrizione e non sono coordinate come nei procarioti, coinvolge pi\`u proteine e alcuni passaggi sono pi\`u intricati: processing del mRNA e 
scansione per $AUG$. Si nota per\`o come la traduzione negli eucarioti sia $3$ volte pi\`u accurata rispetto ai procarioti: in E. coli il tasso di errore \`e $10^{-4}$-$10^{-5}$. 
Tipicamente gli errori di sostituzione di amminoacidi sono dovuti a: amminoacilazioni sbagliate del \emph{$tRNA^{aa}$} da parte di \emph{aaRS} o di selezione sbagliata 
di \emph{aa-$tRNA^{aa}$} da parte di \emph{EF-Tu-GTP} o di \emph{eEF1A-GTP}. Altri errori possono essere selezione sbagliata del sito di inizio, frameshift. 
Si consideri ora: 
\begin{itemize}
	\item Sia $r$ la frequenza di errore della traduzione.
	\item Sia $N$ il numero di amminoacidi contenuti in una proteina.
	\item La probabilit\`a che un amminoacido sia sbagliato $P_w=r$.
	\item La probabilit\`a che un amminoacido sia corretto $P_c=1-r$.
	\item La probabilit\`a che una proteina contenente $N$ amminoacidi sia sintetizzata correttamente $P_{Pc} = (1-r)^N$
\end{itemize}
\begin{center}
	\begin{tabular}{|c|c|c|}
		\hline
		$N=300$  & $r=10^{-4}$ & $r=10^{-2}$\\
		\hline
		$P_w$	 & $0.01\%$ & $1\%$\\
		\hline
		$P_c$    & $99.9\%$ & $99\%$\\
		\hline
		$P_{Pc}$ & $97\%$ & $5\%$\\
		\hline
	\end{tabular}
\end{center}
Si nota come la sintesi delle proteine sia pi\`u prona ad errori rispetto a trascrizione e replicazione in quanto un errore durante l'allungamento di una proteina potrebbe essere molto
meno drammatico rispetto a una mutazione del DNA o nella produzione del RNA, che porterebbe alla produzione di molte pi\`u proteine errate da esso. 
\section{Salvataggio dei ribosomi e controllo qualit\`a degli mRNA}
I ribosomi possono stallare o arrestarsi e devono essere salvati: il mRNA associato viene degradato, la proteina sintetizzata viene targettata per la proteolisi e il ribosoma e i 
tRNA sono riciclati. I ribosomi possono leggere il mRNA e riconoscere errori in esso: il controllo qualit\`a mediato dai ribosomi garantisce che mRNA difettivi o tradotti non 
completamente sono eliminati. 
\subsection{mRNA senza stop codone}
\subsubsection{La risposta tmRNA nei procarioti}
In quanto trascrizione e traduzione non sono separati spazialmente nei batteri, la seconda pu\`o avvenire su mRNA troncati. Il ribosoma stalla quando raggiunge la terminazione di un 
mRNA senza uno stop codone. La situazione \`e risolta dal transfer mRNA o tmRNA, lungo circa $300nt$ simile sia a tRNA che a mRNA: si adatta strutturalmente come un tRNA: le terminazioni
$3'$ e $5'$ si uniscono per formare un dominio simile a tRNA. Viene tipicamente caricato con un alanina: \emph{Ala-tmRNA} (alcuni \emph{Gly-$tmRNA^{Gly}$}). Si trova inoltre nella
sequenza un \emph{ORF} e un codone di stop. \emph{Ala-tmRNA} viene caricato da \emph{EF-Tu-GTP} nel sito $A$ dove il legame viene stabilizzato da \emph{smpB} (small protein B). 
\emph{smpB} inserisce la sua coda nel canale a mRNA vuoto sotto il sito $A$ confermando che il mRNA \`e troncato. \emph{Ala-tmRNA} agisce come un \emph{Ala-$tRNA^{Ala}$} e l'alanina \`e 
aggiunta al poli-peptide in stallo. Si nota come il legame avviene grazie alla similitudine strutturale tra un tRNA e tmRNA. Dopo l'aggiunta dell'alanina il tmRNA crea una strettura 
simile a un codone disponibile nel sito E. Si trovano ora $27$ codoni che codificano per $9$ amminoacidi, in questo modo il ribosoma continua a leggere e la catena catena a crescere fino
a quando il ribosoma incontra il codone di stop del tmRNA. Il tag di amminoacidi codificato dal tmRNA attaccato al peptide segnala per la usa degradazione. Il ribosoma si disassembla, 
il no-stop RNA \`e rilasciato e degradato dalla esonucleasi \emph{RNAasi R} da $3'$ a $5'$. 
\paragraph{Evoluzione del tmRNA} 
Si nota come il tmRNA \`e simile a un introne che si trova negli eucarioti. 
\subsubsection{La risposta del non-stop decay negli eucarioti}
Differisce dai batteri in quanto i ribosomi possono continuare a leggere il trascritto fino alla fine della coda poli-A. Il ribosoma deve pertanto rimuovere la \emph{PABP} legato ad
essa e allunga la catena peptidica crescente con lisine \emph{AAA}. Il ribosoma stalla all'ultimo $AAA$. Il ribosoma recluta il complesso \emph{Ski7/Ski} che rectuta l'esosoma. 
\emph{Ski7} \`e simile a \emph{RF3} e rilascia la proteina, mentre l'esosoma degrada il mRNA mutante da $3'$ a $5'$. La proteina taggata con poli-lisina anormale viene ubiquitinata
e degradata dal proteasoma. 
\subsection{Il ribosoma stalla a un ostacolo}
\subsubsection{Risposta del no-go decay negli eucarioti}
Il mRNA che circonda il ribosoma \`e tagliato da endonucleasi e degradato dall'esonucleasi $5'$-$3'$ \emph{Xrn1} e da $3'$ a $5'$ dall'esosoma. \emph{Dom34} \`e
simile strutturalmente a \emph{eRF1} e a \emph{Hbs1} simile a \emph{eRF3}. Queste si legano al sito $A$ e causano la dissociazione del ribosoma dal mRNA ed entrambe le subunit\`a 
sono riciclate. Il frammento di mRNA coperto dal ribosoma, lungo circa $18nt$ viene degradato e la proteina immatura viene rilasciata e ubiquitinata causando la sua degradazione dal
proteasoma. 
\subsubsection{Risposta dei procarioti}
I ribosomi batterici contengono attivit\`a di RNA elicasica nelle proteine della piccola subunit\`a: \emph{rpS1}, \emph{rpS3} e \emph{rpS4}. \emph{rpS1} viene utilizzata per l'iniziazione
della traduzione regione \emph{SD} in un hairpin che antagonizza un sRNA legato a una terminazione $5'$. \emph{rpS3} e \emph{rpS4} vengono utilizzati per l'allungamento, \`e presente
nell'entrata al mRNA e svolge gli hairpins. La differenza nella risposta, essendo i ribosomi conservati.
\subsection{Stop codone prematuro}
\subsubsection{Risposta nonsense-mediated decay negli eucarioti}
Gli mRNA con un codone di terminazione prematuro a causa di mutazioni del DNA, splicing errato o trascrizione errata produrrebbero una proteina troncata. Negli eucarioti gli 
stop codoni si trovano nell'esone finale, pertanto codoni di stop trovati da altre parti sono marcati come prematuri. Il nonsense-mediated decay viene pertanto causato dal ribosoma
e dal exon junction complex. Lo splicing lascia un complesso proteico, l'exon junction complex che localmente marca l'unione di due esoni. Codoni di stop che si trovano a monte di un 
\emph{EJC} indicano il loro nonsenso. Veri codoni di stop si trovano dopo l'ultimo \emph{EJC} e non sono seguiti da un altro. Mentre traduce il ribosoma elimina il \emph{EJC} quando non 
trova un codone di stop di fronte ad esso. In combinazione di codone di stop e \emph{EJC} il ribosoma stalla. Successivamente vengono reclutati \emph{UPF} fattori di sorveglianza 
\emph{eRF1-GTP} e \emph{eRF3} al sito $A$. Il ribosoma si disassembla e il mRNA con il codone di stop prematuro viene degradato. 
\section{Ricodifica - codone di stop programmato read-through e frame-shifting}
Nel recoding un codone viene interpretato diversamente in un mRNA specifico. Avviene molto raramente e pu\`o produrre diverse proteine da un singolo gene:
\begin{itemize}
	\item Soppressione non senso: un codone di stop \`e mal letto e non avviene terminazione.
	\item Frame-shifting: il mRNA fa shift in modo che la sintesi della proteina proceda in un diverso reading frame. 
\end{itemize}
\subsection{Frame-shifting}
I ribosomi si muovono lungo il mRNA e leggono i codoni di $3$ nucleotidi in modo sequenziale. Dopo la lettura del codone di inizio se il ribosoma si muove di un numero diverso di 
nucleotidi avviene un frame-shift del reading frame. In questo modo sono letti diversi codoni dal ribosoma producendo cos\`i una proteina diversa. Il frame-shifting \`e programmato
e pu\`o essere coinvolto nella regolazione genica. Tipicamente avviene un nucleotide in avanti o indietro dopo che il codone $AUG$ di inizio \`e letto correttamente e il codone di 
stop pu\`o cos\`i venire ignorato. 
\subsubsection{Fattore di terminazione batterico \emph{RF2}}
\emph{RF2} possiede $2$ open reading frame: un corto \emph{ORF1} seguito da uno pi\`u lungo seguito in un frame $+1$. Questo frameshift $+1$ accade alla sequenze $CUU UGA C$. Una
sequenza \emph{SD} in \emph{ORF1} stimola lo shift. In abbondanza di \emph{RF2} il ribosoma riconosce il codone di stop $CUUUGAC$, mentre in deficienza il legame di esso a $UGA$ 
\`e pi\`u lento e avviene un frameshift $+1$ in quanto \emph{Leu-$tRNA^{Leu}$} riconosce il near-cognate $UUU$ nel frame $+1$ e pertanto il prossimo codone \`e $GAC$. 
\subsection{Soppressione non senso}
La soppressione non senso \`e rara ma pi\`u comune quando associata a specifici elementi di mRNA. UN motivo a esanucleotide $CARYYA$ dopo lo stop codone in certi mRNA virali causa
un aumento di read-through. Il precursire \emph{GalPol} nel virus della leucemia murina \`e creato da una soppressione non senso: nella sintesi di \emph{GalPol} si forma un 
pseudo-knot prossimale a $3'$. Questo incoraggia la mal-lettura di $UAG$ da \emph{Glu-$tRNA^{Glu}$} che deve competere coni fattori di terminazione. 
\subsubsection{Retrovirus}
Il frame-shifting programmato \`e anche necessario in molti retrovirus e retrotrasposoni \emph{LTR} per produrre \emph{Pol}, una proteina con attivit\`a di trascrittasi inversa, \emph{RNAasi} e integrasi.
La maggior parte dei ribosomi terminano al gene \emph{gag} ma alcuni subiscono uno shift $-1$ risultando in una proteina di fusione \emph{Gag-Pol}. Questo shift avviene a causa dello scivolamento 
del mRNA nel ribosoma facilitato da uno pseudoknot a valle. La sequenza \`e scivolosa in quanto il tRNA nel sito $P$ pu\`o legarsi nel frame $-1$. 
\subsubsection{Inclusione di amminoacidi non standard}
I codoni di stop possono permettere l'incorporazione di amminoacidi non standard.
\paragraph{Selenocisteina}
La selenocisteina \emph{Sec} \`e simile a serina e cistena ma possiede selenio invece di un atomo di ossigeno o zolfo. Questa viene incorporato in diversi enzimi in siti cataliticii 
dove pu\`o agire come un forte agente riduttore. Il Selenocisteine-tRNA \`e generato caricando \emph{$tRNA^{Sec}$} con serina e convertendo la serina in selenocisteina. 
\emph{$tRNA^{Sec}$} ha un anticodone che fa match con il codone di stop $UGA$. In E. coli la sequenza di insterzione della selenocisteina \emph{SECIS} si trova a $3'$ del $UGA$ che viene 
ricodificato. Per determinare il $UGA$ che viene riconosciuto la proteina simile a \emph{EF-Tu} \emph{SelB} porta il tRNA carico al codone e agisce come guida per interagire con lo stem 
loop \emph{SECIS}. 
\paragraph{Pirrolisina}
La pirrolisina \`e una lisina modificata prodotta da due lisine che si trova nei siti catalitici delle metiltasferasi. Si trova in archea metagenomici e batteri. La pirrolisina viene
incorporata a un codone di stop $UGA$ in modo simile alla selenocisteina. Viene utilizzato un \emph{$tRNA^{Pyl}$} specializzato ma non si conosce se si trovano elementi di mRNA simili
alla sequenza \emph{SEICS}. \emph{EF-Tu-GTP} viene coinvolto nel caricamento di \emph{Pyl-$tRNA^{Pyl}$}
\section{Antibiotici che hanno come obiettivo l'attivit\`a del ribosoma}
Gli antibiotici sono piccole molecole chimiche che uccidono o distruggono la crescita di organismi. Gli antibiotici pi\`u efficaci per uso terapeutico sono quelli che hanno
come obiettivo un processo batterico o fungineo senza alterare quello dell'organismo ospite. Piccole differenze nella traduzione tra batteri ed eucarioti permette agli antibiotici di targettare 
selettivamente il ribosoma batterico e le proteine di traduzione, pertanto gli antibiotici tendono a legarsi a regioni critiche del ribosoma o a fattori di traduzione. 
\subsection{Esempi}
\subsubsection{Clorafenicolo}
Si lega al $23S$ rRNA e inibisce la formazione del legame peptidico. 
\subsubsection{Eritromicina}
Si lega alla porzione $50S$ e impedisce il movimento di traslocazione del ribosoma lungo il mRNA.
\subsubsection{Tetraciclina}
Interferisce con l'attacco di tRNA al complesso di mRNA-ribosoma.
\subsubsection{Streptomicina}
Cambia la forma della porzione $30S$ causando una lettura scorretta del codice sul mRNA. Sono incorporati gli amminoacidi sbagliati che producono proteine non funzionali causando la morte
del microorganismo. Interagisce con la proteina ribosomale $S12$ causando cambi conformazionali nel centro di decodifica. Avvengono pertanto delle sostituzioni che rendono la proteina non funzionale.
\subsubsection{Puromicina}
Antibiotico che agisce tra procarioti ed eucarioti. Utilizzato unicamente nei laboratori per selezionare le cellule umane che esprimono il gene per la resistenza alla puromicina. \`E
simile strutturalmente alla parte esterna del braccio accettore del tirosil-tRNA. Si lega al sito $A$ della subunit\`a maggiore e la peptidil trasferasi lo aggiunge alla catena 
poli-peptidica in crescita causando una terminazione prematura della catena. 
\subsubsection{Acido fusidico}
Anti-batterico, blocca la traslocazione del ribosoma in quanto si lega a \emph{EF-G-GDP} impedendo il suo rilascio in modo che il ribosoma non possa pi\`u caricare il nuovo
\emph{aa-$tRNA^{aa}$} bloccando l'allungamento della proteina. 
\subsection{Resistenza batterica agli antibiotici}
La resistenza dei batteri agli antibiotici pu\`o essere dovuta a un influsso impedito, mutazioni dell'obiettivo, sue modifiche, sovra-produzione di un simile dell'obiettivo, protezione
associata a fattori, modifica dell'antibiotico o sua degradazione. 
\section{Regolazione globale dell'iniziazione della traduzione in batteri ed eucarioti}
La traduzione \`e regolata globalmente in risposta a cambi di sviluppo ed ambientali. Avviene tipicamente all'iniziazione e si pu\`o discriminare tra regolazione a livello globale o 
a livello locale. In condizioni sfavorevoli come pochi amminoacidi si deve ridurre la sintesi delle proteine per salvare risorse ed energia per sopravvivere portando
a una riduzione globale di attivit\`a metaboliche e sintetiche. I tRNA sono etichettati con l'amminoacido cognate e i tRNA non carichi nella cellula sono segno di bassa presenza di 
amminoacidi. Pertanto in caso di legame nel sito $A$ di un tRNA scarico il batterio attiva la risposta stringente. 
\subsection{Stringent response}
Quando arriva nel sito $A$ un tRNA libero non avviene allungamento e il ribosoma si blocca. La \emph{RelA (p)ppGppp sintetasi} o fattore stringent e \emph{rpL11} sentono il 
tRNA scarico nel sito $A$. \emph{RelA} produce \emph{(p)ppGpp}, un pentafosfato guanina nucleotide da \emph{GTP} utilizzando \emph{ATP} e \emph{AMP}. \emph{ppGpp} o l'effettore di 
risposta stringent causando la risposta stringent. Questa molecola poi si lega a RNA Polimerasi abbassando l'affinit\`a dei batteri per $\sigma^{70}$. Vengono inoltre attivati i 
geni in risposta allo stress favorendo l'interazione della RNA polimerasi con i fattori di stress \emph{$\sigma S$}, \emph{$\sigma E$} e \emph{$\sigma N$} e con il fattore di trascrizione \emph{DksA}. 
In queste condizioni rimane invariata la produzione di rRNA e tRNA alter all'attivit\`a e produzione dei ribosomi. Diminuisce del $10\%$ la sintesi delle proteine, mentre aumenta la sintesi di amminoacidi
e di operoni legati allo stress.
\subsubsection{Miglioramento delle condizioni}
Quando le condizioni migliorano \emph{SpoT} idrolizza \emph{ppGpp} in \emph{GDP$+$PPi} e la RNA polimerasi trascrive i target normali usando \emph{$\sigma70$}. 
\subsection{Eucarioti}
\subsubsection{Inibizione}
Negli eucarioti a base concentrazioni di amminoacidi i tRNA scarichi si legano alla chinasi \emph{Gcn2} e il complesso fosforila il fattore di inizio della traduzione \emph{eIF2-GDP} che porterebbe
il tRNA iniziale al sito $P$, ma quando fosforilata si lega fortemente a \emph{eIF2B} in modo che \emph{eIF2-GDP} non possa essere ricaricato con \emph{GTP} causando un arresto globale della traduzione.
Le cellule successivamente operano una risposta allo stress in modo da indurre cambi trascrizionali e traduzionali per ripristinare i livelli di amminoacido. 
\subsubsection{Promozione}
L'iniziazione eucariotica \`e stimolata dalla formazione di un mRNA circolarizzato, processo che coinvolge \emph{eIF4E} al cap $5'$ e \emph{PABP} alla coda poli-A che si associano attraverso interazioni
con \emph{eIF4G}. \emph{eIF4E} viene regolato a livello trascrizionale e da fosforilazione e pu\`o essere sequestrato da un numero di \emph{4E-BP} (proteine leganti \emph{eIF4E}) in risposta a diverse
condizioni di crescita diminuendo la circolarizzazione e attivit\`a di traduzione globali. La fosforilazione di uno di questi due attori inibisce la loro interazione permettendo la circolarizzazione del
mRNA e il conseguente aumento della traduzione.
\section{Regolazione dell'iniziazione attraverso sequenze agenti in \emph{cis} nella \emph{$\mathbf{5'}$-UTR} in batteri ed eucarioti}
La regolazione della traduzione pu\`o anche avvenire al livello di un singolo trascritto oltre che a livello cellulare. La risposta pi\`u rapida \`e attraverso la regolazione dell'iniziazione della
traduzione. 
\subsection{Cambi di conformazione degli mRNA}
Gli mRNA assumono strutture diverse in condizioni diverse con effetti sui livelli di traduzione. Nei batteri la sequenza di Shine-Dalgarno viene spesso oscurata impedendo l'iniziazione. 
\subsubsection{Traduzione di \emph{PrfA}}
La traduzione del mRNA di \emph{PrfA} che codifica un attivatore di trascrizione patogenico necessario per l'infezione di Listeria monocytogene \`e regolata da cambi di temperatura: a basse temperatura
la sequenza \emph{SD} \`e inaccessibile, mentre a $37\si{\degree}$ diventa accessibile.
\subsection{Riboswitches}
I riboswitches sono piccole molecole regolatorie che controllano la propria sintesi. 
\subsubsection{Auto-repressione traduzionale della sintesi delle proteine ribosomiali in E. coli}
Il ribosoma di E. coli contiene $55$ proteine e $3$ rRNA. Le \emph{RP} sono codificate in $19$ operoni poli-cistronici. Se il numero di \emph{RP} \`e corretto per i propri rRNA viene prodotto il 
ribosoma, mentre se il numero di \emph{RP} \`e maggiore dei corrispettivi rRNA la produzione del riboosma \`e bloccata: $1$ \emph{RP} si lega alla \emph{$5'$-UTR} del mRNA policistronico per impedire 
il legame al ribosoma reprimendo la propria sintesi e quella delle altre \emph{RP} da esso prodotte. 
\subsubsection{Regolazione del ferro}
La regolazione \emph{$5'$-UTR} \`e meno comune negli eucarioti, ma viene utilizzata per regolare il ferro: ferro libero nella cellula \`e tossico, pertanto viene legato strettamente dalla ferritina. 
Maggiore la concentrazione di ferro, maggiore ferritina necessaria. La \emph{$5'$-UTR} del mRNA della ferritina contiene un elemento responsivo del ferro \emph{IRE} che si lega alla proteina regolatrice
del ferro \emph{IRP}. 
\paragraph{Ferro scarso}
\emph{IRP} si lega a \emph{IRE} impedendo l'accesso di $AUG$ da parte del ribosoma.
\paragraph{Ferro abbondante}
Il ferro si lega a \emph{IRP} che non possono legarsi a \emph{IRE}, pertanto la traduzione continua a produrre ferritina. 
\subsubsection{Controllo globale della sintesi degli amminoacidi nel lievito regolando la traduzione del mRNA \emph{GCN4} alla \emph{$\mathbf{5'}$-UTR}}
\emph{Gcn4} \`e un attivatore trascrizionale di pi\`u di $500$ geni coinvolti nella biosintesi degli amminoacidi il cui livello \`e regolato traduzionalmente in base ai livelli di amminoacidi. Il suo
mRNA nella \emph{$5'$-UTR} contiene $4$ \emph{ORF} a valle \emph{uORF1-4} falsi prima della sequenza codificante. Gli \emph{uORF} sono formati da un codone di start, un paio di codoni e un codone di 
stop.
\paragraph{Alti livelli di amminoacidi}
In questa situazione non si necessita di produrre \emph{Gcn4}: dal cap $5'$ del mRNA il complesso di pre-iniziazione $48S$ scansiona il mRNA fino a trovare la sequenza di Kozak di \emph{uORF1}: viene
reclutata la subunit\`a $60S$, tradotti due amminoacidi, si incontra il codone di stop, si termina la traduzione e il ribosoma si disassembla. Nel $50\%$ dei casi quando la subunit\`a $60S$ lascia il
complesso $48S$ di pre-inizio rimane legato al mRNA al codone di stop e a causa di alti livelli di \emph{eIF2-GTP} avviene un reinizio fino a \emph{uORF2}. Il processo si ripete fino a \emph{uORF4}, 
dove avviene la scarica completa del $40S$ e del $60S$: il ribosoma non arriva al vero $AUG$ e non avviene la traduzione del mRNA. 
\paragraph{Bassi livelli di amminoacidi}
In questa situazione si necessita di produrre \emph{Gcn4}: pochi tRNA sono etichettati con amminoacid, pertanto non avviene un reinizio veloce e il complesso di pre-inizio $48S$ continua a scansionare con
una bassa frequenza saltando degli \emph{uORF} fino a trovare il $AUG$ di \emph{GCN4}, pertanto con bassa frequenza viene reclutata la subunit\`a $60S$ in modo da tradurre \emph{GCN4} e produrne bassi
livelli. 
\subsubsection{Auto-repressione traduzionale della proteina legante poli-A citoplasmatica}
\emph{PABP} si lega alla coda poli-A di mRNA e permette insieme a \emph{eIF4G} la sua circolarizzazione aiutando a stabilire il complesso di inizio $43S$. \emph{PABP} viene prodotta in eccesso e si lega
nella propria coda poli-A impedendo il legame del complesso di pre-iniziazione nella \emph{$5'$-UTR}. Non avviene scansione n\`e reclutamento della subunit\`a maggiore.
\subsection{Regolazione dell'attivit\`a di \emph{$\mathbf{5'}$-UTR} da parte di sRNA nei procarioti}
I piccoli RNA non codificanti aiutano i batteri ad adattarsi a condizioni di stress ossidativo variabili: l'ossidante \emph{\ce{H2O2}} prodotto in un batterio in crescita aerobica in quanto pu\`o essere
convertito in radicali superossidanti che possono danneggiare il DNA. Gli sRNA sono lunghi dai $50$ ai $500nt$, altamente strutturati con diversi stem-loop. 
\subsubsection{Repressione di \emph{flhA} mRNA da \emph{oxyS} sRNA}
\emph{oxiS} copre la sequenza \emph{SD} e parte della sequenza codificante impedendo il legame del ribosoma. 
\subsubsection{Attivazione di \emph{rpoS} mRNA da parte di \emph{DsrA} sRNA}
Il fattore sigma di stress in E. coli o \emph{Rpos=$\sigma^S=\sigma^{38}$}.
\paragraph{Condizioni di crescita normali}
La sequenza \emph{SD} e $AUG$ sono seppellite nella struttura secondaria del \emph{$5'$-UTR} del mRNA di \emph{rpoS} che non viene prodotto in condizioni normali con il proprio mRNA degradato.
\paragraph{Condizioni di stress}
\emph{DsrA} a $87nt$ con un chaperone \emph{Hfq} che si lega alla sua terminazione $5'$ si accoppiano con le basi della terminazione $5'$ del mRNA \emph{rpoS} smascherando \emph{SD} e $AUG$. La regione 
a valle \`e rimossa, \emph{rpoS} viene tradotto e viene prodotto $\sigma^S$. 
\section{Regolazione della traduzione attraverso sequenze agenti in \emph{cis} nella \emph{$\mathbf{3'}$-UTR} negli eucarioti}
Questo tipo di regolazione avviene negli mRNA degli eucarioti e la regola inibendo la circolarizzazione del mRNA o inibendo la traduzione dopo la sua circolarizzazione.
\subsection{Xenopus laevis}
Negli oociti di Xenopus laevis gli mRNA non sono tradotti inizialmente: si trovano in stato dormente. La loro traduzione dipende dalla lunghezza della coda di poli-A, inizialmente corta. L'elemento
\emph{$3'$-UTR} di poli-adenilazione \emph{CPE} citoplasmatico viene legato da \emph{CPEB} che sequestra \emph{eIF4E} attraverso Maskin bloccando la formazione del loop chiuso richiesto per l'iniziazione
della traduzione. La chinasi \emph{eg2} fosforila \emph{CPEB} e \emph{CPEB-P} reclita \emph{CPSF} a una sequenza $AAUAAA$ nel mRNA. Successivamente \emph{CPSD} recluta la poli-A polimerasi \emph{PAP}
che estende la coda poli-A. Ora \emph{PABP} pu\`o legarsi alla coda permettendo il legame di \emph{eIF4G} e l'allontanamento di Maskin permettendo alla traduzione di procedere.
\section{Trasporto e localizzazione degli mRNA}
Negli eucarioti gli mRNA sono trasportati dal nucleo al citoplasma come mRNP. Gli mRNA possono muoversi a un sito specifico nel citoplasma per essere ancorati e tradotti solo l\`a. Si nota come i gradienti
di mRNA sono asimmetrici nella cellula e la localizzazione avviene grazie a proteine motrici legate agli RNA nei complessi mRNP. La traduzione degli mRNA viene inibita durante il trasporto. Pertanto
gli mRNA contengono ``zip codes'' lunghi da $10$ a \num{1000} basi nella regione \emph{$3'$-UTR} o metazoans o nella sequenza codificante.
\subsection{Localizzazione del mRNA di \emph{ASH1} nel lievito}
\emph{Ash1} \`e un fattore di traduzione che agisce unicamente nella cellula figlia. Il suo mRNA viene espresso dalla cellula madre e trasportato nel citoplasma e poi nella gemma. Contiene uno ``zip 
code'' a stem-loop. Viene trasportato dal loop da proteine trasportatrici \emph{She} lungo fili di actina. Il mRNA si accumula nella cellula figlia dove viene prodotta la proteina \emph{Ash1}. 
\section{Granuli citoplasmatici di RNA e P-bodies}
La conservazione degli mRNA non utilizzati nelle cellule mammifere avviene nei processing bodies o P-bodies o nei granuli di stress.
\subsection{P-bodies}
I P-bodies sono sempre presenti e contengono la maggior parte degli enzimi degradanti il RNA: enzimi di decapping $5'$, $3'$ deadenilasi e $5'$-$3'$ esonucleasi. Questi enzimi degradano gli RNA con 
codoni non senso, mRNA con \emph{ARE} e agiscono nel silenziamento degli mRNA. Sono il vero sito di degradazione degli mRNA. 
\subsubsection{Componenti}
\begin{multicols}{2}
\begin{itemize}
	\item mRNA.
	\item $5'$-$3'$ esonucleasi \emph{XRN1}.
	\item Fattori di deadenilazione \emph{CCR4, CAF1, NOT1-4}.
	\columnbreak
	\item Fattori di rimozione del cap $5'$ \emph{LSM1-7, DCP192, PAT1, DHH1, CPEB}.
	\item Fattori coinvolti nella degradazione non-senso \emph{SMG5.7. UPF1}.
	\item Fattori di pathway del miRNA \emph{miRNA, AGO1-4, GW182}.
\end{itemize}
\end{multicols}
\subsection{Granuli di stress}
I granuli di stress si formano unicamente durante condizioni di stress e sono zone di conservazione degli mRNA durante lo stress. Contengono mRNA, miRNA, subunit\`a riboosmali, fattori di iniziazione
della traduzione, enzimi nucleolitici, elicasi proteine leganti il RNA e altro. Una cellula sotto stress riprogramma il proprio metabolismo di mRNA: che sono presenti come complessi di pre-iniziazione
in stallo rilasciati dai granuli e tradotti quando necessario. Questi sono legati all'\emph{ER}. Quando la condizione di stress termina il proteasoma $26S$ lo elimina. 
\subsubsection{Componenti}
\begin{multicols}{2}
\begin{itemize}
	\item mRNA con code poli-A.
	\item  Fattori di iniziazione della traduzione \emph{40S, eIF4E, eIF4G, eIF3, eIF2}.
	\item Fattori di controllo della traduzione \emph{CPEB, PABP, DHH1}.
	\columnbreak
	\item Fattori di degradazione del mRNA \emph{DHH1, Staufen}.
	\item Fattori di assemblaggio \emph{snRNP}.
	\item Fattori di processamento del RNA.
\end{itemize}
\end{multicols}
