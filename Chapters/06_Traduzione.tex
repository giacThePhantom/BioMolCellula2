\chapter{Traduzione}
\section{Panoramica della traduzione}
Si intende per traduzione del RNA la produzione della proteina dall'informazione contenuta in un mRNA, mentre per sintesi proteica la polimerizzazione degli amminoacidi a formare una proteina. Questi 
processi sono un processo complesso ed altamente conservato che coinvolge diverse componenti: mRNA, amminoacidi, tRNA, amminoacil-tRNA sintetasi, ribosomi e rRNA. Questo processo \`e impegnativo per la
cellula in quanto richiede molta energia e risorse nutrizionali: in una cellula di lievito che si divide ogni $90min$ le proteine devono essere sintetizzate per la crescita e divisione, pertanto vengono
prodotti $33$ ribosomi al secondo.
\subsection{Ribosomi}
Il ribosoma \`e formato da:
\begin{itemize}
	\item $4$ rRNA sintetizzati negli eucarioti da RNA polimerasi I e III.
	\item $80$ proteine strutturali i cui geni sono parti del \emph{RP regulon} (ribosomal protein).
	\item $50\%$ dell'attivit\`a della RNA polimerasi II \`e dedicata alla produzione di proteine ribosomali.
	\item $236$ proteine sono necessarie per costruire un ribosoma attive e questi geni fanno parte del \emph{Ribi regulon}.
\end{itemize}
La sintesi delle proteine varia negli eucarioti da $2$ (crescita lenta) a $6$ (crescita veloce) amminoacidi al secondo, mentre nei procarioti tra i $10$ e i $20$. La velocit\`a pu\`o essere aumentata 
dall'attivit\`a di multipli ribosomi (da $2$ a $100$) su un mRNA singolo che formano il polisoma. Si definisce pertanto ribosoma il macchinario che opera la traduzione, il monosoma $1$ ribosoma 
attaccato a $1$ mRNA traduzionalmente attivo e polisoma un numero di ribosomi legati a $1$ mRNA traduzionalmente attivi. 
\subsection{Produzione e traduzione di mRNA}
Si nota come si la produzione che la traduzione di mRNA avvengono in direzione $5'$-$3'$. Inoltre avvenendo la traduzione nel citoplasma si rende necessario negli eucarioti un esporto nucleare degli mRNA.
\subsection{Confronto tra trascrizione e traduzione}
\subsubsection{Procarioti}
La trascrizione totale del RNA in una cellula di E. coli viene svolta da \num{1000} fino a \num{10000} RNA polimerasi e procede a velocit\`a massimale di $40$-$80\frac{nt}{sec}$. La traduzione 
di E. coli ha velocit\`a di $20\frac{aa}{sec}$ si nota come i due tassi sono quasi uguali. Infatti se la traduzione fosse pi\`u veloce della trascrizione i ribosomi colliderebbero con la RNA polimerasi 
nei procarioti, dove i processi possono avvenire simultaneamente. Inoltre i ribosomi sono importanti per trascrizione veloce: sembrano impedire alla RNA polimerasi di fare backtrack e pause, creando
pertanto un accoppiamento inverso tra traduzione e trascrizione. 
\subsubsection{Eucarioti}
La trascrizione in cellule mammifere ha tassi di allungamento simili a quelli misurati in E. coli tra i $50$ e i $100\frac{nt}{sec}$. Viene suggerito che queste lunghezze di trascrizione rapida siano
intervallate con lunghe pause portando a un tasso medio un ordine di grandezza inferiore rispetto a E. coli. L'allungamento della traduzione in cellule mammifere \`e anch'esso pi\`u lento: in condizioni
ottime $6\frac{nt}{sec}$. 
\section{Ribosoma}
\subsection{Scoperta}
Nel $1941$ viene scoperta la correlazione tra RNA e livelli di proteine. Nel $1943$ attraverso centrifugazione ad alta velocit\`a viene identificata una frazione che contiene la maggior parte del RNA
nel citoplasma e poi identificata come il ER ruvido. Nel $1955$, attraverso microscopia elettronica vengono osservate granuli densi liberi nel citoplasma o legati al ER che contengono il RNA. Nel
$1958$ viene scoperto il ribosoma. Si nota come i ribosomi legati alla membrana ER sintetizzano proteine per l'esporto cellulare o inserzione in membrane, mentre quelli liberi nel citoplasma le 
sintetizzano per tutte le attivit\`a citosoliche e nucleari. 
\subsection{Composizione}
I ribosomi sono grandi da $2.5$ fino a $4mDa$ e sono formati per $\frac{2}{3}$ da rRNA e per $\frac{1}{3}$ da proteine. Si dividono nella subunit\`a minore che decifra il mRNA e media l'interazione
tra mRNA e tRNA e la subunit\`a maggiore che catalizzala formazione di legami peptide tra gli amminoacidi e possiede un tunnel attraverso cui esce il peptide di crescita. L'interfaccia tra le 
subunit\`a \`e importante per i movimenti di tRNA e mRNA nel ribosoma. Un errore avviene una volta ogni \num{1000}-\num{10000} monomeri. I fattori di traduzione, spesso \emph{GTPasi} si associano 
con i ribosomi per aiutare la traduzione, composta da quattro fasi principali: iniziazione, allungamento, terminazione, separazione delle subunit\`a ribosomiali e riciclo. 
\subsubsection{Confronto tra ribosomi eucarioti e procarioti}
I ribosomi eucarioti e batterici sono funzionalmente e strutturalmente conservati ma differiscono nella composizione proteica.
\begin{multicols}{2}
	Ribosoma batterico ($70S$):
	\begin{itemize}
		\item Subunit\`a maggiore ($50S$):
			\begin{itemize}
				\item rRNA $5S$ lungo $120nt$.
				\item rRNA $23S$ lungo \num{2900}$nt$.
				\item Circa $34$ proteine \emph{L1-L34}.
			\end{itemize}
		\item Subunit\`a minore ($30S$):
			\begin{itemize}
				\item rRNA $16S$ lungo \num{1540}$nt$.
				\item Circa $21$ proteine \emph{S1-S21}.
			\end{itemize}
	\end{itemize}
	\columnbreak
	Ribosoma eucariote ($80S$):
	\begin{itemize}
		\item Subunit\`a maggiore ($60S$):
			\begin{itemize}
				\item rRNA $5S$ lungo $120nt$.
				\item rRNA $5.8S$ lungo $160nt$.
				\item rRNA $28S$ lungo \num{4700}$nt$.
				\item Circa $49$ proteine.
			\end{itemize}
		\item Subunit\`a minore ($40S$):
			\begin{itemize}
				\item rRNA $18S$ lungo \num{1900}$nt$.
				\item Circa $33$ proteine.
			\end{itemize}
	\end{itemize}
\end{multicols}
\subsubsection{Ruolo del rRNA}
La struttura del ribosoma \`e determinata da quella del proprio rRNA. In una subunit\`a le proteine estendono braccia nelle regioni di rRNA, solitamente altamente basiche e si pensa aiutino con 
l'impacchettamento dei backbone fosfati degli rRNA negativamente carichi. A causa dell'alto bisogno dei ribosomi i geni degli rRNA sono i pi\`u intensamente trascritti: il $90\%$ di tutto il RNA \`e
rRNA e viene organizzato in vettori. Gli rRNA processati da pre-rRNA vengono divisi in domini: il $16S$ ($18S$ negli eucarioti) possiede tre domini principali e uno minore, mentre il $23S$ ($28S$ negli
eucarioti) possiede $6$ domini. Questi domini sono distinti nella piccola subunit\`a, mentre nella grande sono intrecciati. La subunit\`a maggiore possiede inoltre sia in batteri che eucarioti il 
$5S$ rRNA e solo negli eucarioti il $5.8S$ rRNA. Gli RNA sono cruciali per l'attivit\`a dei ribosomi in quanto la regione decodificante del rRNA $16S$ media l'interazione tra il tRNA e mRNA e tra
mRNA e il ribosoma, mentre il rRNA $23S$ nel centro della peptidil-trasferasi interagisce con il tRNA. 
\subsection{Il ribosoma come un ribozima}
Gli rRNA catalizzano la sintesi proteica. Si nota come le componenti dei ribosomi a RNA erano presenti prima di quelle proteiche che sono state reclutate al ribosoma pi\`u tardi nell'evoluzione. Si
pu\`o pertanto considerare il ribosoma come un ribozima coperto da proteine. Questo si deduce sia funzionalmente: il rRNA $23S$ estratto dalla subunit\`a maggiore $50S$ dei procarioti senza
proteine attaccate ad esso pu\`o catalizzare il legame tra i peptidi con bassa efficienza che strutturalmente: i siti  rRNA $16S$ e peptidil-trasferasi (rRNA $23S$) sono formati da rRNA e non si 
trova vicino alcuna proteina ribosomiale. 
\subsubsection{Modifiche ai nucleosidi degli rRNA}
Gli rRNA eucarioti e procarioti contengono nucleotidi alternativi a causa dei modifiche post-trascrizionali enzimatiche dei quattro nucleotidi canonici.
\subsubsection{Proteine ribosomiali}
Le proteine nel ribosoma supportano il piegamento del rRNA garantendo ad esso e al ribosoma la struttura corretta per l'attivit\`a, aumentano l'efficienza del processo di traduzione. Per questi motivi ogni
proteina ribosomiale \`e essenziale per la vitalit\`a della cellula. Sono pertanto altamente conservate. Le proteine sono grandi, globulari e basiche. Si localizzano esternamente ma con code che
protrudono nella struttura del rRNA. I ribosomi eucariotici possiedono $82$ proteine, $27$ in pi\`u rispetto a quelli procarioti. Sono identificate come \emph{rpS``numero''} per quelle presenti 
nella subunit\`a minore e \emph{rpL``numero''} per quelle nella maggiore. 
\subsection{Assemblaggio dei ribosomi negli eucarioti}
Negli eucarioti il primo passo per l'assemblaggio dei ribosomi inizia nel nucleolo, dove rRNA $35S$, rRNA $5S$, proteine ribosomiali, fattori proteici non ribosomiali e snoRNA vanno a formare il pre-$90S$.
Questo complesso esce dal nucleolo e si divide in pre-$60S$ con gli rRNA $27S$ e $5S$ e nel pre-$40S$ con il rRNA $20S$. Questi due sono i pre-ribosomi. I pre-ribosomi escono dal nucleo attraverso due 
complessi del poro nucleare ed entrano nel citoplasma, dove maturano nel $60S$ con rRNA $25S$, $5.8S$ e $5S$ e nel $40S$ con rRNA $18S$. DA APPROFONDIRE.
\subsubsection{Nucleoli}
I nucleoli sono ripetizioni di loci di rDNA eucariote clustered trascritto da RNA polimerasi I e III.
\subsection{Rappresentazione funzionale del ribosoma}
Il ribosoma possiede nella subunit\`a minore un canale per il passaggio del mRNA. Dal tunnel poi si formano due fori: il sito A e il sito P. Nel sito A viene decodificato il codice genetico e caricato 
l'amminoacido corretto. Il sito $A$ finisce nella subunit\`a minore dove comunica con il tunnel di uscita della catena peptidica nascente. Dopo che \`e stato caricato nel sito A il tRNA si sposta 
nel sito P dove avviene il legame peptide con la catena peptidica in crescita. Il tRNA perde l'amminoacido e passa al sito E dove viene rilasciato nel citoplasma. 
\section{tRNA e codice genetico}

\section{Amminoacil-tRNA sintetasi}

\section{mRNA}

\section{Ciclo di traduzione}

\section{Iniziazione della traduzione - caratteristiche comuni a batteri ed eucarioti}

\section{Iniziazione della traduzione batterica}

\section{Iniziazione della traduzione eucariotica}

\section{Allungamento della traduzione}

\section{Terminazione e reinizio della traduzione}

\section{Energia richiesta per la traduzione}

\section{Velocit\`a ed accuratezza della traduzione}

