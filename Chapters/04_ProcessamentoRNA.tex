\chapter{Processamento dell'RNA}
\section{Panoramica del processamento del RNA}
Gli RNA prodotti sono spesso non funzionali. Questi RNA precursori \emph{pre-RNA} devono essere modificati per diventare funzionali. Questo avviene durante il processamento del RNA o 
maturazione. Il processamento avviene nel nucleo per impedire che i pre-RNA siano tradotti nel citoplasma. Dopo il processamento sono trasportati nel citoplasma per traduzione 
seguente da parte dei ribosomi. Il processamento avviene per tre ragioni:
\begin{itemize}
	\item Regolazione dell'attivit\`a genica.
	\item Diversit\`a: molti RNA diversi possono essere prodotti da un gene attraverso splicing alternativo rimuovendo diverse combinazioni di introni.
	\item Controllo della qualit\`a: mRNA difettivi sono individuati e degradati.
\end{itemize}
\subsection{Modifiche al RNA}
Le modifiche al RNA coinvolgono un grande numero di complessi molecolari e molti di essi contengono sia proteine che RNA e sono ribonucleoproteine \emph{RNP}. Il RNA negli \emph{RNP}
pu\`o avere ruolo strutturale ma anche attivit\`a catalitiche come i ribozimi e i ribosomi. Alcuni \emph{RNP} contengono guide a RNA che si accoppiano con le basi dei pre-RNA e li 
guidano alla sequenza corretta per il processamento del pre-RNA obiettivo. 
\section{Processamento di rRNA e di tRNA}
\subsection{Procarioti}
\subsubsection{rRNA}
Il rRNA nei procarioti \`e prodotto come lunghi pre-rRNA $30S$. Questi sono rotti in un numero di rRNA da endonucleasi. Gli rRNA rotti sono successivamente raffinati da esonucleasi alle
estremit\`a per produrre gli rRNA finali. Questi non sono tradotti ma diventano il backbone strutturale delle subunit\`a grande e piccola dei ribosomi. 
\paragraph{E. coli}
In E. coli il pre-rRNA $30S$ forma degli stem loop in corrispondenza del rRNA $16S$, del $23S$ e una struttura a forcina con due stem-loop in corrispondenza del $5S$. Le RNAasi 
\emph{RNAasi III}, \emph{RNAasi M16} e da \emph{RNAasi M23} rilasciano il $16S$ e il $23S$, mentre la \emph{RNAasi E} rilascia il $5S$. La sequenza che viene processata contiene
anche dei tRNA interni che vengono elaborati diversamente e specificatamente. 
\paragraph{Ribonucleasi}
Le ribonucleasi rompono o raffinano gli RNA in pezzi pi\`u piccoli.
\subparagraph{Esonucleasi}
Le esonucleasi rimuovono nucleotidi dalle terminazioni di un trascritto, non sono specifiche alla sequenza e la maggior parte agiscono in direzione $3'$-$5'$. La maggior parte sono
processive. La \emph{PNPasi} e l'esosoma sono esonucleasi $3'$-$5'$ di E. coli. \emph{Xrn1} e \emph{Exol}, sempre di E. coli sono esonucleasi $5'$-$3'$ e la seconda \`e processiva.
\subparagraph{Endonucleasi}
Le endonucleasi romponon il RNA nel filamento. Alcune rompono dsRNA come \emph{RNAasi III}, mentre altre ssRNA come \emph{RNAasi P} o \emph{tRNAasi Z}. Possono rompere a $3'$ o a $5'$. 
La \emph{RNAasi P} possiede componenti a RNA e proteine, in quella batterica solo la parte a RNA pu\`o rompere il RNA, mentre la parte proteica aumenta l'attivit\`a e l'intervallo di 
substrati. In quella eucariotica, di archea e mitocondriale invece il componente a RNA da solo non pu\`o tagliare il RNA ma \`e essenziale per la funzione.
\subsubsection{tRNA}
I tRNA hanno una struttura variabile composta da un sito accettore \emph{ACC}, un braccio \emph{T$\Psi$C} con pseudo-iridina, un braccio variabile, un braccio dell'\emph{AC} e un 
braccio \emph{D} contenente di-idro-uridina. Si nota la presenza di molte basi modificate. In E. coli vengono prodotti come un pre-tRNA poli-cistronico $30S$ che viene processato. A $5'$
interviene il ribozima \emph{RNAasi P} che taglia il pre-tRNA, mentre una endonucleasi $3'$ lo separa e esonucleasi di tRNA invece affinano la terminazione $3'$ fino alla sequenza di 
stop prima di \emph{CCA}. Il caricamento degli amminoacidi avviene sul \emph{CCA-$3'OH$} grazie a un aminoacil-tRNA sintetasi che idrolizza \emph{ATP} per creare un legame estere
tra la terminazione e l'amminoacido. 
\paragraph{Degenerazione del codice genetico}
Si nota come in E. coli per $20$ amminoacidi si trovano $64$ codoni e $43$ tRNA. Pertanto esistono pi\`u codoni per $1$ amminoacido che pu\`o usare multipli tRNA per la sua inclusione
in una catena peptidica in base al suo codone. 
\subparagraph{La degenerazione diminuisce gli effetti deleteri delle mutazioni}
Una mutazione pu\`o rimanere silente o non causare uno stop nella sintesi della proteina.	
\subparagraph{Gerarchia dei codoni}
Si trova una gerarchia di importanza tra i diversi codoni per la codifica di uno stesso amminoacido: certi codoni sono usati in proteine a bassa priorit\`a che non sono sintetizzate in
caso l'amminoacido sia poco presente. Altri sono utilizzati in proteine ad alta priorit\`a fatte senza tener conto della disponibilit\`a di amminoacido. Pertanto diverse sequenze di 
codoni non hanno la stessa importanza creando il codon bias. In questo modo la cellula conosce quali proteine devono essere fatte e quali possono essere ignorate quando la loro 
disponiblilit\`a \`e bassa. Si nota come l'utilizzo dei codoni influenza il tasso di traduzione del RNA e la produzione di proteine: pi\`u sono frequentemente usati pi\`u veloce 
\`e la traduzione. 
\subsection{Eucartioti}
\subsubsection{rRNA}
Negli umani nel nucleolo si trova il rDNA array formato da un non-transcribed spacer \emph{NTS} e il gene per il rRNA. La RNA polimerasi I sintetizza il $47S$ pre-rRNA che contiene
tre rRNA con a $5'$ un external trasnscribed spacer \emph{ETS}, seguito dal rRNA $18S$, un internal transcribed spacer \emph{ITS1}, poi un rRNA $5.8S$, un \emph{ITS2} e il $28S$. 
Codificare gli rRNA diversi in un solo precursore assicura che le quantit\`a di ogni RNA siano bilanciate. Il pre-rRNA viene elaborato: prima viene eliminato il \emph{ETS}, poi 
rotto il \emph{ITS1} e infine viene affinato il $18S$ e vengono separati il $5.8S$ e il $28S$. Il $5.8S$ andr\`a poi a legarsi al $28S$. Inoltre piccoli RNA nucleaolari \emph{snoRNA}
con proteine associate formano i \emph{snoRNP}, processosomi delle piccole e grandi subunit\`a ribosomali che catalizzano il processo di maturazione del rRNA. 
\subsubsection{tRNA}
GLi eucarioti producono trascritti di tRNA mono-cistronici sintetizzati dalla RNA polimerasi III. Un gene per un tRNA contiene una met\`a $5'$ separata dalla met\`a $3'$ da un
introne. Dopo la trascrizione il macchinario di splicing dei tRNA rimuove l'introne, la \emph{RNAasi P} rimuove la terminazione $5'$ mentre la \emph{tRNAasi Z} la $3'$. Successivamente
dopo questo processamento viene aggiunto il \emph{CAA} dalla \emph{tRNA nucleotidil trasferasi} catalizzata dal sito di legame del nucletoide. La tasca possiede tre conformazioni: libera
legata a \emph{CTP} e legata a \emph{ATP} che controllano se viene aggiunta una \emph{C} o una \emph{A} alla terminazione $3'$ del tRNA. 
\section{Modifiche dei nucleotidi di rRNA e di tRNA}
I nucleotidi negli rRNA e nei tRNA vengono modificati chimicamente dopo la trascrizione: vari enzimi modificano le basi: 
\begin{itemize}
	\item Al gruppo ammino: adnenina e guasina \emph{\ce{NH2}}.
	\item A un atomo di azoto: guanina \emph{N7} e citosina \emph{N3}.
	\item A un atomo di carbonio: citosina e uracile \emph{C3}.
\end{itemize}
Al ribosio a \emph{$2'$-OH}. Le modifiche sono enzimatiche e reversibili. Possono essere piccole come metilazioni o grandi come l'addizione di un amminoacido come la treonina. Il sito, 
la quantit\`a e la distribuzione delle modifiche variano tra molecole di RNA, organismi e compartimenti intracellulari. Molte modifiche sono essenziali per la crescita e la 
sopravvivenza. Le modifiche pi\`u comuni al rRNA sono la metilazione del ribosio \emph{$2'$-OH} e la pseudo-uridilazione della base. Molte modifiche degli rRNA si trovano in basi 
importanti per la funzione dei ribosomi. Sono state identificate pi\`u di $100$ modifiche negli RNA attraverso epitranscriptome sequencing e il $75\%$ di esse si trova nei rRNA. 
Queste aumentano il numero di dimensioni e struttura e la stabilit\`a della molecola di tRNA. ANche gli mRNA sono modificati con il capping $m^7G$ aggiungendo una guanosina metilata
a \emph{n7}. Gli RNA spliceosomali, RNA piccoli nucleari \emph{snRNA}, piccoli RNA nucleolari \emph{snoRNA}, \emph{miRNA} e \emph{siRNA} sono modificati. Molte modifiche sono collegate
al piegamento, attivit\`a e stabilita degli RNA, differenziazione della cellula, determinazione del sesso e risposta allo stress. Mutazioni negli enzimi modificanti il RNA causano 
malattie negli esseri umani. 
\subsection{Modifiche chimiche dei tRNA procarioti ed eucarioti}
I tRNA subiscono delle modfiiche a posizioni specifiche. Nella regione dell'anticodone aumentano l'efficienza della traduzione, mentre nel corpo centrale correggono la struttura 
secondaria e terziaria e aumentano la stabilit\`a del tRNA in quanto l'assenza di certe di esse causa degradazione. Le modifiche combinate definiscono l'identi\`a di ogni tRNA e 
tRNA che portano lo stesso amminoacido possono essere diversamente modificati. 
\subsection{Modifiche chimiche degli rRNA}
Dopo la sintesi i pre-rRNA subiscono il processamento e due tipi di modifiche catalizzate dagli \emph{snoRNP}:
\begin{itemize}
	\item \emph{$2'$-OH} metilazione del ribosio in certi nucleotidi mediata dal \emph{snoRNA C/D}.
	\item \emph{Pseudo-uridilazione} dell'uracile creando uridina o pseudo-uridina $\Psi$ mediata da \emph{snoRNA H/ACA}.
\end{itemize}
Queste potrebbero contribuire alla stabilit\`a e al piegamento del rRNA. L'interazione con le proteine ribosomali  modula la biogenesi dei ribosomi e ne aumenta l'attivitù\`a. 
\subsubsection{Enzimi coinvolti}
Gli enzimi che catalizzano la metilazione del ribosio e la conversione dell'uridina in pseudo-uridina nel pre-rRNA son conservati in tutti gli organismi, ma varia il meccanismo di 
riconoscimento. Negli eucarioti ed archea si utilizzano piccoli RNA guida nucleolari \emph{snoRNA} che portano l'enzima modificatore al sito corretto. Questi si associano con un numero
di proteine per formare un \emph{RNP} attivo detto \emph{snoRNP}. \emph{snoRNP C/D} e \emph{H/ACA} contengono $4$ diverse proteine ognuna e i loro nomi derivano da box conservate 
presenti nella loro sequenza a RNA. Gli \emph{snoRNA} sono lunghi tra i $60$ e i $300$ nucletoidi e l a maggior parte sono fatti dagli introni dei pre-mRNA. Gli \emph{snoRNA} guida si 
accoppiano con le basi di regioni specifiche dei pre-rRNA obiettivo e guidano gli enzimi come metil tasferasi firillarina \emph{Nop1}, e la pseudo-uridina sintestasi discherina a quelle
posizioni. Gli \emph{snoRNA} che guidano la metilazione $2'$-\emph{OH} sono \emph{C/D}, mentre quelli che guidano la pseudo-uridilazione delle basi sono \emph{H/ACA}. 
\section{Capping e poliadenilazione di mRNA}
Entrambe le terminazioni degli RNA eucarioti sono modificati durante la trascrizione. Queste modifiche proteggono gli mRNA dalla degradazione da parte delle esonucleasi e aiutano 
con le interazioni proteiche come i ribosomi. La terminazione $5'$ \`e incappucciata con la guanosina-P attraverso legame $5'$-$5'$ trifosfato. Questa guanina \`e poi metilata a
\emph{N7}. Il cap $5'$ \`e necessario per allungamento efficiente e terminazione del trascritto, per il processamento del RNA, esporto dal nucleo e direzionamento della traduzione. In
eucartioti complessi il $2'$-\emph{OH} della prima, seconda e qualche volta terza base sono metilate. 
\subsection{Capping dei pre-mRNA eucarioti al $\mathbf{5'}$}
Il cap $5'$ viene aggiunto in tre passi poco dopo che il mRNA emerge dalla RNA polimerasi II ed \`e lungo tra i $20$ e i $30$ nucleotidi e prima che avvenga una qualsiasi altra 
attivit\`a di processamento: 
\begin{enumerate}
	\item La RNA $5'$ trifosfatasi rimuove il $\gamma$-fosfato dalla terminazione $5'$.
	\item La guanil trasferasi attacca una guanosina monofosfato \emph{GMP} alla terminazione $\beta\alpha$-difosfato in un legame $5'$-$5'$ trifosfato. 
	\item La guanina-$7$-metil trasferasi metila la guanina in posizione \emph{N7}.
	\item La $2'$-\emph{OH} metiltrasferasi trasferisce un gruppo metile dalla \emph{S-adenosilmetionina} al $2'$-\emph{OH} al ribosio dei primi due o tre nucleotidi alla 
		terminazione $5'$. 
\end{enumerate}
Nel lietivo il processo viene svolto da tre enzimi diversi, mentre in C. elegans e nei mammiferi avvengono grazie a un singolo complesso enzimatico di capping che contiene le tre 
attivit\`a enzimatiche e ulteriori metilazioni di $2'$-\emph{OH}. Inoltre pu\`o avvenire un'iper-metilazione del cap $5'$ anche in piccoli RNA nucleari o nucleolari come 
\emph{snRNA U1/U5/U3}.
\subsection{Poli-adenilazione $3'$ dei pre-mRNA eucarioti}
La terminazione $3'$ della maggior parte degli mRNA eucarioti contiene $200$ adenosine aggiunte dette coda poli-A. Gli mRNA possiedono un sito di poli-adenilazione interno dove questi 
sono rotti e aggiunta la coda. Un mRNA pu\`o contenere multipli siti di poli-adenilazione e le sequenze in mezzo possono partecipare alla loro regolazione: la poli-adenilzione
a un sito distale trattiene multiple regioni regolatorie, mentre ad un sito prossimale le rimuove. La coda $3'$ poli-A protegge il RNA dalle esonucleasi $3'$ nel nucleo e citoplasma, 
aumenta l'efficienza per il trasporto nel citoplasma e promuove la traduzione del trascritto. 	
\subsubsection{Riconoscimento del sito di poli-adenilazione}
Quando la RNA polimerasi II arriva alla terminazione $3'$ di un gene, trascrive il motivo \emph{AATAAA} crea nel trascritto il segnale di rottura e poli-adenilazione \emph{AAUAAA}, 
un motivo \emph{CA} sito di rottura e poli-A e una regione ricca di \emph{U} o $G$ o $C$. Quest'ultima \`e importante in quanto molti introni sono ricchi in $A$ e $T$ e pertanto 
fornisce la specificit\`a necessaria per impedire che vengano rotti questi erroneamente. I siti \emph{AAUAAA} sono riconosiuti da \emph{CPSF}: cleavage and polyadenylation specific 
factor. Il pre-mRNA \`e tagliato a \emph{CA} da RNA endonucleasi complessi fattori di rottura \emph{CFI} e \emph{CFII} la cui attivit\`a \`e stimolata da \emph{CStF} (fattore di 
stimolazione della rottura). Dopo la rottura sono aggiunte $200$ adenosine dalla polimerasi poli-A. Viene utilizzata la $A$ in quanto \emph{ATP} \`e il nucleotide pi\`u abbondante. 
L'apparato di poli-adenilazione pu\`o riconoscere diversi siti \emph{CA} rotti nel trascritto di pre-mRNA creando diversi mRNA da esso. 
\subsubsection{Il processo di poli-adenilazione}
La rottura e poli-adenilazione avvengono durante la trascrizione: \`e un segnale per la RNA polimerasi II che deve terminare la trascrizione e dissociarsi dal DNA. 
\paragraph{Iniziazione}
L'iniziazione consiste nell'aggiunta delle prime $10$ adenosine: dipende dal segnale di rottura e poli-adenilazione e dalle sequenze segnale \emph{GU} e \emph{CA}. Richiede 
\emph{CPSF}, \emph{CStF}, \emph{CFI} e \emph{CFII} oltre alla polimerasi poli-A \emph{PAP}. 
\paragraph{Allungamento}
L'allungamento consiste dell'aggiunta di tutte le adenosine successive: richiede \emph{PAP} e il \emph{PABPN1} nucleare: la proteina legante poli-A nucleare 1.
\subsubsection{Effetti della lunghezza della coda poli-A}
La lunghezza della coda poli-A determina quanto a lungo il mRNA sopravvive: questa infatti aumenta l'efficienza dell'inizio della traduzione nel citoplasma. Il legame di \emph{PABPC1}:
proteina legante poli-A citoplasmatica 1 alla coda e i fattori di inizio della traduzione \emph{eIF4E} e \emph{eIF4G} circolarizzano mRNA con la terminazione $5'$ e permettono il legame
del ribosoma. 
\subsubsection{Poli-adenilazioni alternative}
\paragraph{Poli-adenilazione nei procarioti e negli organelli}
Nei procarioti, nei mitocondri e nei cloroplasti la coda poli-A \`e un segnale per la degradazione del mRNA: attraverso \emph{PNPasi} polinucleotide fosforilasi $3'$-$5'$ RNAsi, dalla
\emph{ss RNAasi II} che compiono il ciclo di degradazione.
\paragraph{Istoni}
Gli mRNA che producono le proteine istoniche hanno il cap $5'$ ma non la coda poli-A: presentano uno stem-loop nella regione $3'$ UTR che marca la fine del trascritto dopo rottura
$3'$ endonucleolitica. 
\subparagraph{Maturazione degli mRNA istonici}
Il processamento di questi mRNA inizia dal legame con la proteina legante lo stem-loop \emph{SLBP} allo stem-loop. IL legame del \emph{sRNP U7} all'elemento istonico \emph{HDE} a valle
crea contatti attraverso accoppiamenti di base con la terminazione $5'$ del \emph{U7} \emph{snRNA}. Il legame al pre-mRNA \`e stabilizzato da interazioni con uno zinc finger 
\emph{U7} \emph{snRNP} e \emph{SLBP}. Il pre-mRNA \`e rotto dal complesso endonucleasi \emph{CPSF73/100/symplekin}. Il mRNA poi si chiude attraverso la circolarizzazione della 
terminazione $3'$ e del cappuccio $5'$ e lega \emph{eiF4G/4e} e \emph{SLBP} per reclutare un ribosoma per la traduzione. 
\subsection{Legame con altre attivit\`a di polimerizzazione del RNA}
Il capping $5'$ e la poli-adenilazione $3'$ sono legate con altre attivit\`a di polimerizzazione del RNA: il capping \`e necessario per l'allungamento della trascrizione e la
poli-adenilazione $3'$ per terminazione efficiente. Il dominio C-terminale \emph{CTD=$[YS^2PTS^5PS]_{26-52}$} della subunit\`a \emph{Rbp1} della RNA polimerasi II media tutti i 
processamenti del RNA come una piattaforma di evento. La \emph{CTD} recluta sequenzialmente diversi complessi di processamento:
\begin{itemize}
	\item La \emph{CTD} \`e fosforilata all'inizio della trascrizione e recluta il complesso o gli enzimi per il capping $5'$. 
	\item L'allungamento porta ad ulteriore fosforilazione di \emph{CTD} che recluta il macchinario di splicing.
	\item Questo viene seguito dal reclutamento del complesso di rottura e poli-adenilazione.
\end{itemize}
\section{RNA splicing}
Il RNA finale \`e costituito da sequenze di esoni discreti, originariamente separati da introni rimossi dal pre-RNA. Ci sono quattro diverse classi di introni e tutte devono essere
rimosse dal pre-RNA. La maggior parte degli introni non contengono geni e sono escissi e degradati. Eccezioni sono costituite da \emph{snoRNA} e \emph{miRNA}. Alcuni entroni sono
rimossi da enzimi come \emph{tRNA}, altri da \emph{RNP} come lo spliceoosma, mentre altri ancora si separano da soli. Gli introni sono prevalenti negli eucarioti, anche se alcuni virus
ne posseggono alcuni che fanno self-splicing. Una rimozione alternativa degli introni permette la creazione di diversi trascritte e proteine isoforme dallo stesso gene, che tra gli
umani variano tra i \num{100000} e i \emph{1000000}. Gli introni permettono inoltre un mescolamento degli introni a livello del DNA, dove gli esoni sono scambiati e riordinati attraverso
ricombinazione, permettendo la produzione di diversi gene e proteine che codificano. Lo splicing avviene nel nucleo e negli organelli contenenti DNA genomico come mitocondri e 
cloroplasti.
\subsection{Scoperta degli introni}
Nel $1977$ si nota come la sequenza di un RNA funzionale \`e diversa dalla sequenza dei geni: la seconda \`e pi\`u lunga e presenta sezioni assenti nel gruppo finale: si chiamano 
tali zone introni in quanto non sono incluse nel trascritto maturo. I geni pertanto sono discontinui. Questi vengono scoperti attraverso la tecnica del \emph{R-loop mapping}. Dopo 
un incubazione ad alte temperature avviene l'ibridazione tra pre-mRNA e la sequenza genica e mRNA e la sequenza. Il prodotto viene macchiato con uranil-acetato reso scuro con
platino/palladio e visualizzato attraverso microscopia elettronica. Gli introni sono stati successivamente confermati sequenziando il DNA del gene, il cDNA derivato dal pre-mRNA e dal
mRNA maturo. Originariamente si fecero due ipotesi: nella prima si supponeva che la RNA polimerasi fosse discontinua e saltasse delle sequenze, mentre nella seconda il trascritto 
\`e completo ed \`e successivamente elaborato. Si nota come la corretta \`e la seconda attraverso northern hybridization blots: isolamento degli RNA, ibridazione con una sequenza
ss radioattivamente etichettata: analizzando il campione nel tempo si nota la scomparsa dei pre-mRNA di lunghezza intera e l'apparizione di mRNA intermedio e maturo. 
\subsubsection{Caratteristiche degli introni}
Il numero di introni in un gene varia tra le specie. In S. cerevisiae il $5\%$ dei geni possiede introni e tipicamente $1$ per gene, mentre per gli umani li possiedono il $85\%$ con 
$11$ introni per proteina in media. La connettina ne possiede $363$. Gli introni sono presenti nei geni codificanti mRNA, rRNA e tRNA. Negli umani il $95\%$ della sequenza di un pre-mRNA
\`e formata da introni con sequenze molto pi\`u lunghe rispetto agli esoni. 
\subsection{Tipi di splicing del RNA}
Ci sono quattro tipi di RNA splicing:
\begin{itemize}
	\item Splicing nucleare fatto dallo spliceosoma, un grande complesso ribonucleoproteico, e il tipo pi\`u comune di splicing. Avviene per i prodotti della RNA polimerasi II.
	\item Gruppo di introni I self-splicing.
	\item Gruppo di introni II self-splicing.
	\item tRNA splicing.
	\item Trans-splicing da parte dello spliceosoma.
\end{itemize}
\subsubsection{Splicing nucleare}
Le sequenze degli introni non sono conservate tranne che per corte sequenze di segnale che corrispondono ai segnali di riconoscimento e rimozione. Queste sono il sito donatore
$5'$ introne, il sito $A$ di branching, la lunghezza riccadi \emph{CT} poli-pirimidina. \emph{AG} finale $3'$ o sito accettore. Lo spliceosoma riconosce specifici siti intronici nel 
pre-mRNA in quanto i siti $5'$ e $3'$ sono molto meno conservati. Altre sequenze aiutano a definire i confini tra gli introni e gli esoni e aiutano le cellule a produrre diversi mRNA 
maturi dallo stesso gene. 
\paragraph{Meccanismo di splicing}
I due esoni di ogni parte sono uniti da un processo a due passi: entrambi coinvolgono una reazione di transesterificazione in cui un legame fosfo-estere \`e rotto ma un altro 
\`e formato. L'energia simile tra i due legami indica che la reazione non richiede \emph{ATP}. L'introne \emph{lariat} viene degradato e pertanto la reazione non \`e reversibile. Si nota
come \`e necessario \emph{ATP} per l'assemblaggio dello spliceosoma. 
\paragraph{Lo spliceosoma}
Lo spliceoosma \`e un macchinario formato da $60$ proteine e $4$ molecole di RNA nel lievito e $5$ negli umani. Le dimensioni sono simili a quelle della piccola subunit\`a ribosomiale
e la composizione proteica differisce leggermente tra le specie. Lo splicing \`e mediato dal RNA in quanto lo spliceosoma \`e un ribozima. Non \`e coinvolta l'attivit\`a di RNAasi. 
Contiene $5$ corti RNA o short nuclear RNA \emph{snRNA} ricchi in uracile \emph{U1}, \emph{U2}, \emph{U4}, \emph{U5} e \emph{U6}. Il numero indica l'ordine di attivit\`a e \emph{U3}
manca per motivi storici in quanto necessario nella maturazione del rRNA. I $5$ \emph{snRNA} formano accoppiamenti specifici con le basi di sequenze introniche conservate nel pre-mRNA. 
Ognuno di essi si lega a un insieme specifico di proteine formando $5$ \emph{snRNP}: small nuclear ribonucleoprotein. Ognuno di essi lega sempre a proteine \emph{7 Sm} che formano una
ciambella che si lega a una sequenza conservata di $9nt$  nel \emph{snRNA}: \emph{$5'$-AUUUGUG-$3'$}. L'accoppiamento di basi avviene tra \emph{U1} e la sequenza donatrice $5'$
dell'introne e $3'$ dell'esone e tra \emph{U2} e il branch point $A$ e le basi intorno. Insieme decidono dove avviene lo splicing.
\subparagraph{Formazione dello spliceosoma}
\begin{enumerate}
	\item \emph{U1} riconosce il sito donatore $5'$ dell'introne.
	\item \emph{BBP} branching point binding protein riconosce il branch point, recluta \emph{U2}.
	\item \emph{U2AF} \emph{U2} associated factor si lega al sito di splice $3'$ includendo la sequenza accettrice \emph{AG}.
	\item \emph{U2} riconosce il sito accettore $3'$ dell'introne. 
	\item \emph{U4} recluta \emph{U5} e \emph{U6} al sito di splice $5'$ includendo la sequenza \emph{GU}. Avviene la prima reazione di trans-esterificazione da parte di \emph{U6}/
	\item \emph{U1} e \emph{U4} lasciano il RNA. 
	\item \emph{U2}, \emph{U5} e \emph{U6} mediano il riordinamento unendo i due esoni nella seconda reazione di transesterificazione da parte di \emph{U2} e l'introne \`e rimosso.
\end{enumerate}
\subparagraph{\emph{snRNA}}\mbox{}\\
\begin{center}
\begin{tabular}{|c|c|c|}
	\hline
	\emph{\textbf{snRNA}} & \textbf{Lunghezza (nt)} & \emph{Funzione}\\
	\hline
	\emph{U1} & $164$ & \makecell{Riconosce il sito di splicing al $5'$ mediante \\l'appaiamento di una regione complementare.} \\
	\hline
	\emph{U2} & $187$ & \makecell{Riconosce il sito di ramificazione mediante \\l'appaiamento con una regione complementare.}\\
	\hline
	\emph{U4} & $144$ & Forma un duplex con \emph{U6}.\\
	\hline
	\emph{U5} & $116$ & Funzione sconosciuta, si lega a esone 1 e 2.\\
	\hline
	\emph{U6} & $106$ & \makecell{Forma un duplex con \emph{U4}, scalza \emph{U1} \\nell'appaiamento con il sito di splicing al $5'$.}\\
	\hline
\end{tabular}
\end{center}
\paragraph{Exon junction complex}
Il exon junction complex \emph{EJC} \`e lasciato alla giunzione di splice dopo lo splicing in modo da marcare il trascritto come localmente processato. 
\paragraph{Splicing errato}
Errori nello splicing portano a malattia come la distrofia muscolare di Duchenne e derivano da siti donatori, accettori o di branch mutati o mutazioni dei regolatori, proteine o 
\emph{snRNA} spliceosomali. 
\subsubsection{Gruppi di introni I e II self-splicing}
Nel $1982$ viene scoperto un introne nel pre-rRNA $26S$ nel Tetrahymena thermophila ha fatto splicing di s\`e stesso dal proprio trascritto senza intervento di altre proteine o enzimi, 
co-fattori o \emph{ATP}. Si nota come clonando il gene di rDNA in un plasmide, purificando la RNA polimerasi I batterica si produce un trascritto di rRNA che aggiungendo 
\emph{\ce{Mg^{2+}}} fa splicing. Il RNA viene pertanto considerato come un'entit\`a simile a un enzima, un ribozima. Gli introni autocatalitici si dividono in quelli di gruppo I e di 
gruppo II con due meccanismi di self-splicing attraverso trans-esterificazione.
\begin{itemize}
	\item Nel gruppo I avviene attraverso il \emph{$3'$-OH} di una guanosina al di fuori della sequenza di mRNA.
	\item Nel gruppo II avviene attraverso il \emph{$2'$-OH} di un'adenosina interna branchpoint simile all'attivit\`a dello spliceosoma. 
\end{itemize}
\paragraph{Introni di gruppo I self-splicing}
Questi introni si escindono da soli al trascritto primario. Hanno una lunghezza che varia tra i $250$ e i $500nt$ e si trovano in microorganismi eucarioti, piante, batteri e virus
eucarioti nel DNA nucleare, mitocondriale e cloroplasto. Contengono una struttura secondaria conservata che contiene $9$ regioni a stem-loop \emph{P1-P9}. I siti di splice sono 
definiti dalla struttura tridimensionale dell'introne e dal riconoscimento di una $G$ conservata in \emph{P1} che forma una coppia wobble con $U$ ultimo nucleotide dell'esone $1$. Il
loop \emph{P7} lega \emph{GTP}, \emph{GDP}, \emph{GMP} e guanina come il nucleofilo per la reazione. Gli introni codificano una maturasi, proteina senza attivit\`a catalitica che
svolge il ruolo di RNA chaperone aiutando la reazione di splicing stabilizzando la struttura del RNA e l'attivit\`a di auto-splicing dell'introne. Gli introni potrebbero o no
codificare una DNA endonucleasi che se presente rende l'introni mobile, ovvero pu\`o retro-trasporsi nel proprio allele nel meccanismo di homing. 
\subparagraph{Introni di gruppo I immobili}
Negli introni di gruppo I immobili il gruppo \emph{$3'$-OH} della guanosina libera \emph{GTP} si localizza nella tasca \emph{P7} e attacca il  \emph{$5'$-P} al primo nucleotide 
dell'introne. Avviene la trans-esterificazione 1: si stacca la terminazione $5'$ dall'esone 1. Successivamente la terminazione rilasciata dell'esone 1 \emph{U $3'$-OH} attacca la
giunzione introne-esone 2: \emph{$3'$-P} dell'ultimo nucleotide dell'introne e avviene la seconda reazione di trans-esterificazione. L'introne lineare non forma un lariat. 
\subparagraph{Introni di gruppo I mobili}
Negli introni di gruppo I mobili la trascrizione e lo splicing avviene come in quelli immobili, ma l'introne lineare viene esportato nel citoplasma dove viene tradotto. Si produce
un'endonucleasi che ritorna nel nucleo. L'allele omologo subisce una rottura a doppio filamento e l'allele originale agisce come donatore per la riparazione. Avviene una riparazione
\emph{DSB} o \emph{SDSA} (synthesis-dependent strand annealing). Il processo di homing avviene attraverso intermedi del DNA.
\subparagraph{Ruoli evolutivi secondari delle endonucleasi homing}
Membri di varie famiglie di endonucleasi homing presentano omologia strutturale e relazioni funzionali con una grande variet\`a di proteine da vari organismi. Le funzioni 
biologiche includono enzimi di degradazione del DNA non specifici, enodnucleasi di restrizione, enzimi di riparazione del DNA, resolvasi e fattori di trascrizione anche auto-repressivi.
Queste relazioni suggeriscono che queste endonucleasi homing condividono antenati comuni con proteine coinvolte nella fedelt\`a dei genomi, loro mantenimento ed espressione genica. 
Pertanto quando sono espresse le endonucleasi homing possono contenere attivit\`a enzimatiche addizionali che agiscono nella cellula. Il homing amplifica il loro numero genico e il
suo effetto nella biologia della cellula. 
\paragraph{Introni di gruppo II self splicing}
Questi introni sono lunghi tra i $400$ e i \num{1000}$nt$. Si trovano in eucarioti, piante, archea e batteri. Possiedono una struttura secondaria conservata con $6$ domini a stem-loop
\emph{D1-D6}. Il dominio \emph{D4} potrebbero codificare per maturasi o maturasi, endonucleasi e trascrittasi inversa rendendo l'introne mobile che fa splicing invertito. La struttura
terziaria degli esoni e il branching pooint $A$ in \emph{D6} si trovano vicini. Il meccanismo di splicing \`e simile a quello dello spliceosoma e guidato dagli ioni magnesio. Non 
\`e necessario il co-fattore nucleotide: avviene l'attacco nucleofilo dal \emph{$2'$-PH} da $A$ al \emph{$5'$-P} dell'esone 1. Successivamente avviene l'attacco nucleofilo dal 
\emph{$3'$-OH} dall'esone 1 al \emph{$3'$-P} dell'introne. L'introne rimosso forma un lariat. Si nota come questi introni sono antenati dello spliceosoma eucariote. 
\subparagraph{Introni mobili di gruppo II: retrohoming}
Se \emph{D6} codifica per un endonucleasi questa taglia il filamento basso di DNA e il ss lariat liberato si re-integra nel filamento superiore invertendo i passi di 
trans-esterificazione della reazione in avanti. Dopo che la maturasi promuove lo splicing dell'introne questo individua grazie a regioni l'allele omologo e crea una rottura doppio
filamento. Il lariat si inserisce e una trascrittasi inversa sintetizza il ssDNA in cui il taglio basso agisce come primer. Successivamente una DNA polimerasi sintetizza il filamento
complementare. 	
\subsubsection{Splicing del tRNA}
I trascritti di tRNA eucarioti sono mono-cistronici e sono creati dalla RNA polimerasi III. Sono tipicamente formati da una met\`a $5'$ seguita da un introne e infine una met\`a $3'$. 
Dopo la trascrizioen avviene l'escissione dell'introne dal macchinario di splicing del tRNA. Successivamente una RNasi P e una tRNAasi Z tagliano le estremit\`a e una tRNA nucleotidil
trasferasi aggiunge \emph{CCA} formando il tRNA maturo. La maturazione dei pre-tRNA in archea ed eucarioti non coinvolge reazioni di trans-esterificazione e necessita pertanto di 
un numero di enzimi: un'endonucleasi rimuove l'introne. Successivamente nella met\`a $5'$ si forma $2'$-$3'$ fosfato ciclico. Nella met\`a $3'$ una chinasi dipendente da \emph{ATP} 
fosforialzione la terminazione \emph{$5'$-OH} e una ligasi tRNA-specifica e \emph{ATP} aggiunge alla \emph{$5'$-P} un \emph{AMP}. A questo punto nella met\`a $5'$ una fosfodiesterasi
\emph{PDE} apre l'anello di fosfato. Si forma il legame tra \emph{$3'$-OH} nella met\`a $5'$ e il \emph{$5'$-OH} dell'altra con il rilascio di \emph{AMP}. Una fosfatasi rimuove il gruppo
fosfato in \emph{$2'$-OH} della prima met\`a e si forma il legame fosfodiestere. Si nota come lo splicing avviene nel citoplasma: il tRNA subisce delle modifiche, viene esportato nel
citoplasma, reimportato nel nucleo e infine re-esportato. 
\subsubsection{Trans-splicing}
Si dice cis-splicing quando gli esoni nello stesso pre-RNA sono uniti insieme. Si nota come \`e possibile come esoni di due diversi pre-RNA si possano unire nel trans-splicing. Questo
processo avviene in archea, eucarioti unicellulari, piante e nematodi ma non negli umani. Si riconosce un pre-mRNA accettore con un $A$ branch point e un corto \emph{SL RNA} donatore 
(spliced leader). Il secondo \`e pi\`u corto di $150nt$, possiede un cap $5'$ e una sequenza non codificante leader di $20nt$ e un introne ed \`e contenuto in un \emph{Sm snRNP}. 
Il \emph{SL RNA} sostituisce il \emph{U1 snRNP} e interagisce on altri \emph{snRNP} al sito di splicing $3'$. Il processo coinvolge lo spliceosoma. Si nota come
i siti \emph{GU} donatore e \emph{AG} accettore risiedono in due diverse molecole di RNA. Avvengono due reazioni di trans-esterificazione. \emph{SL RNA} viene trascritto dalla RNA 
polimerasi II. 
\section{Definizione degli esoni e splicing alternativo}
Lo splicing pu\`o portare a pi\`u di un mRNA maturo: la maggior parte dei geni un eucarioti complessi subisce splicing alternativo, dove sono usate diverse combinazioni di esoni. La
maggior parte di essi sono costitutivi e sempre inclusi, mentre alcuni sono regolati e possono essere esclusi. Possono essere anche usati siti di splicing alternativi alla terminazione
$5'$ o $3'$. Si possono usare inizi di trascrizione alternativi e diversi di terminazione. Lo splicing alternativo \`e importante per la diversit\`a genetica come 
nel gene \emph{dscam} in Drosophila che pu\`o creare \num{38000} diversi trascritti maturi. Questo gene infatti contiene $24$ geni e $4$ che sono cluster che a loro volta contengono
$98$ esoni alternativi. Questo codifica per un axal guidance reporter per lo sviluppo neurale. In diversi tessuti vengono espressi diversi isoformi della proteina e lo splicing 
\`e regolato in tempo e spazio durante lo sviluppo. 
\subsection{Splicing alternativo: diversit\`a e complessit\`a delle specie}
Negli eucarioti si nota un numero di geni simile ma una grade diversit\`a nella complessit\`a della specie. Questo avviene grazie allo splicing alterantivo: da $1$ pre-mRNA nel 
$90\%$ dei geni umani si possono produrre diverse proteine. Si nota come gli introni sono sempre eliminati, mentre gli esoni potrebbero esserlo. 
\subsubsection{Riconoscimento dei veri siti di splicing}
Le sequenze che definiscono le giunzioni tra introni ed esoni sono semplici, corte e degenerative: possono trovarsi da qualche altra parte e portare a splicing non voluto o i siti di 
splicing criptici. Essendo che lo splicing avviene con alta fedelt\`a lo spliceosoma deve poter riconoscere i veri siti di splicing. Ci sono due modelli che propongono il meccanismo di 
riconoscimento. 
\paragraph{Definizione degli esoni}
Nei mammiferi le terminazioni $5'$ e $3'$ di un esone sono portate insieme ta interazioni tra i complessi \emph{U1} e \emph{U2}: gli introni che li affiancano subiscono splicing se 
\`e presente a monte \emph{U1} o \emph{U2} a valle. Mutazioni nel sito di splice risultano in un esclusione di un esone a causa della mancanza di legame di \emph{RNP U1}. 
\paragraph{Definizione degli introni}
In invertebrati, funghi e piante gli introni sono definiti da interazioni tra \emph{U1} e \emph{U2} legati ai confini $5'$ e $3'$ dell'introne. Mutazioni al sito di splice $5'$ 
risulterebbero in assenza di splicing e inclusione dell'introne nel trascritto maturo. 
\paragraph{Conclusione}
Si nota come in entrambi i modelli i siti di splice $3'$ e $5'$ sono marcati mentre sono trascritti: gli esoni da proteine \emph{SR} e gli introni da \emph{hnRNP}. 
\subsection{Elementi di sequenze di RNA addizionali}
Elementi di sequenze di RNA addizionali hanno un effetto sulla funzione dello spliceosoma. L'attivit\`a di questi elementi dipende dal contesto locale: lo stesso motivo pu\`o agire 
come repressore o attivatore se si trova in un esone o in un introne. 
\subsubsection{Slicing enhancers}
Si dicono splicing enhancers \emph{SE} sequenze che promuovono lo splicing. Possono essere intronici \emph{ISE} o esonici \emph{ESE}. Si legano a un \emph{SR} proteine (ricche di 
serina/arginina) e promuovono l'assemblaggio dello spliceosoma. \subsubsection{Splicing silencers}
Si dicono splicing silencers \emph{SS} sequenze che inibiscono lo splicing. Possono essere intronici \emph{ISS} o esonici \emph{ESS}. Sono legati da proteine \emph{hnRNP} 
(ribonucleoproteine nucleari eterogenee) che mascherano siti di splicing criptici in un introne e inibiscono l'interazione tra \emph{RNP U1} e \emph{RNP U2} dello spliceosoma. 
\subsection{Exon shuffling}A
Si definisce exon shuffling il processo di mescolamento degli esoni che avviene durante la meiosi. Il mescolamento di tali domini necessari per la funzione delle proteine permette
la creazione di nuove combinazioni di proteine funzionali. 
\section{\emph{miRNA} e \emph{siRNA}}

\section{Ribozimi auto-catalitici}

\section{RNA editing}

\section{Traslocazione nucleo-citoplasmatica di RNA processati}

\section{Degradazione di RNA endogeni}

\section{Degradazione di RNA esogeni \emph{siRNA} \emph{CRISPR}}

