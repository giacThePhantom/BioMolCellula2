\chapter{Processamento dell'RNA}
\section{Panoramica del processamento del RNA}
Gli RNA prodotti sono spesso non funzionali. Questi RNA precursori \emph{pre-RNA} devono essere modificati per diventare funzionali. Questo avviene durante il processamento del RNA o 
maturazione. Il processamento avviene nel nucleo per impedire che i pre-RNA siano tradotti nel citoplasma. Dopo il processamento sono trasportati nel citoplasma per traduzione 
seguente da parte dei ribosomi. Il processamento avviene per tre ragioni:
\begin{itemize}
	\item Regolazione dell'attivit\`a genica.
	\item Diversit\`a: molti RNA diversi possono essere prodotti da un gene attraverso splicing alternativo rimuovendo diverse combinazioni di introni.
	\item Controllo della qualit\`a: mRNA difettivi sono individuati e degradati.
\end{itemize}
\subsection{Modifiche al RNA}
Le modifiche al RNA coinvolgono un grande numero di complessi molecolari e molti di essi contengono sia proteine che RNA e sono ribonucleoproteine \emph{RNP}. Il RNA negli \emph{RNP}
pu\`o avere ruolo strutturale ma anche attivit\`a catalitiche come i ribozimi e i ribosomi. Alcuni \emph{RNP} contengono guide a RNA che si accoppiano con le basi dei pre-RNA e li 
guidano alla sequenza corretta per il processamento del pre-RNA obiettivo. 
\section{Processamento di rRNA e di tRNA}
\subsection{Procarioti}
\subsubsection{rRNA}
Il rRNA nei procarioti \`e prodotto come lunghi pre-rRNA $30S$. Questi sono rotti in un numero di rRNA da endonucleasi. Gli rRNA rotti sono successivamente raffinati da esonucleasi alle
estremit\`a per produrre gli rRNA finali. Questi non sono tradotti ma diventano il backbone strutturale delle subunit\`a grande e piccola dei ribosomi. 
\paragraph{E. coli}
In E. coli il pre-rRNA $30S$ forma degli stem loop in corrispondenza del rRNA $16S$, del $23S$ e una struttura a forcina con due stem-loop in corrispondenza del $5S$. Le RNAasi 
\emph{RNAasi III}, \emph{RNAasi M16} e da \emph{RNAasi M23} rilasciano il $16S$ e il $23S$, mentre la \emph{RNAasi E} rilascia il $5S$. La sequenza che viene processata contiene
anche dei tRNA interni che vengono elaborati diversamente e specificatamente. 
\paragraph{Ribonucleasi}
Le ribonucleasi rompono o raffinano gli RNA in pezzi pi\`u piccoli.
\subparagraph{Esonucleasi}
Le esonucleasi rimuovono nucleotidi dalle terminazioni di un trascritto, non sono specifiche alla sequenza e la maggior parte agiscono in direzione $3'$-$5'$. La maggior parte sono
processive. La \emph{PNPasi} e l'esosoma sono esonucleasi $3'$-$5'$ di E. coli. \emph{Xrn1} e \emph{Exol}, sempre di E. coli sono esonucleasi $5'$-$3'$ e la seconda \`e processiva.
\subparagraph{Endonucleasi}
Le endonucleasi romponon il RNA nel filamento. Alcune rompono dsRNA come \emph{RNAasi III}, mentre altre ssRNA come \emph{RNAasi P} o \emph{tRNAasi Z}. Possono rompere a $3'$ o a $5'$. 
La \emph{RNAasi P} possiede componenti a RNA e proteine, in quella batterica solo la parte a RNA pu\`o rompere il RNA, mentre la parte proteica aumenta l'attivit\`a e l'intervallo di 
substrati. In quella eucariotica, di archea e mitocondriale invece il componente a RNA da solo non pu\`o tagliare il RNA ma \`e essenziale per la funzione.
\subsubsection{tRNA}
I tRNA hanno una struttura variabile composta da un sito accettore \emph{ACC}, un braccio \emph{T$\Psi$C} con pseudo-iridina, un braccio variabile, un braccio dell'\emph{AC} e un 
braccio \emph{D} contenente di-idro-uridina. Si nota la presenza di molte basi modificate. In E. coli vengono prodotti come un pre-tRNA poli-cistronico $30S$ che viene processato. A $5'$
interviene il ribozima \emph{RNAasi P} che taglia il pre-tRNA, mentre una endonucleasi $3'$ lo separa e esonucleasi di tRNA invece affinano la terminazione $3'$ fino alla sequenza di 
stop prima di \emph{CCA}. Il caricamento degli amminoacidi avviene sul \emph{CCA-$3'OH$} grazie a un aminoacil-tRNA sintetasi che idrolizza \emph{ATP} per creare un legame estere
tra la terminazione e l'amminoacido. 
\paragraph{Degenerazione del codice genetico}
Si nota come in E. coli per $20$ amminoacidi si trovano $64$ codoni e $43$ tRNA. Pertanto esistono pi\`u codoni per $1$ amminoacido che pu\`o usare multipli tRNA per la sua inclusione
in una catena peptidica in base al suo codone. 
\subparagraph{La degenerazione diminuisce gli effetti deleteri delle mutazioni}
Una mutazione pu\`o rimanere silente o non causare uno stop nella sintesi della proteina.	
\subparagraph{Gerarchia dei codoni}
Si trova una gerarchia di importanza tra i diversi codoni per la codifica di uno stesso amminoacido: certi codoni sono usati in proteine a bassa priorit\`a che non sono sintetizzate in
caso l'amminoacido sia poco presente. Altri sono utilizzati in proteine ad alta priorit\`a fatte senza tener conto della disponibilit\`a di amminoacido. Pertanto diverse sequenze di 
codoni non hanno la stessa importanza creando il codon bias. In questo modo la cellula conosce quali proteine devono essere fatte e quali possono essere ignorate quando la loro 
disponiblilit\`a \`e bassa. Si nota come l'utilizzo dei codoni influenza il tasso di traduzione del RNA e la produzione di proteine: pi\`u sono frequentemente usati pi\`u veloce 
\`e la traduzione. 
\subsection{Eucartioti}
\subsubsection{rRNA}
Negli umani nel nucleolo si trova il rDNA array formato da un non-transcribed spacer \emph{NTS} e il gene per il rRNA. La RNA polimerasi I sintetizza il $47S$ pre-rRNA che contiene
tre rRNA con a $5'$ un external trasnscribed spacer \emph{ETS}, seguito dal rRNA $18S$, un internal transcribed spacer \emph{ITS1}, poi un rRNA $5.8S$, un \emph{ITS2} e il $28S$. 
Codificare gli rRNA diversi in un solo precursore assicura che le quantit\`a di ogni RNA siano bilanciate. Il pre-rRNA viene elaborato: prima viene eliminato il \emph{ETS}, poi 
rotto il \emph{ITS1} e infine viene affinato il $18S$ e vengono separati il $5.8S$ e il $28S$. Il $5.8S$ andr\`a poi a legarsi al $28S$. Inoltre piccoli RNA nucleaolari \emph{snoRNA}
con proteine associate formano i \emph{snoRNP}, processosomi delle piccole e grandi subunit\`a ribosomali che catalizzano il processo di maturazione del rRNA. 
\subsubsection{tRNA}
GLi eucarioti producono trascritti di tRNA mono-cistronici sintetizzati dalla RNA polimerasi III. Un gene per un tRNA contiene una met\`a $5'$ separata dalla met\`a $3'$ da un
introne. Dopo la trascrizione il macchinario di splicing dei tRNA rimuove l'introne, la \emph{RNAasi P} rimuove la terminazione $5'$ mentre la \emph{tRNAasi Z} la $3'$. Successivamente
dopo questo processamento viene aggiunto il \emph{CAA} dalla \emph{tRNA nucleotidil trasferasi} catalizzata dal sito di legame del nucletoide. La tasca possiede tre conformazioni: libera
legata a \emph{CTP} e legata a \emph{ATP} che controllano se viene aggiunta una \emph{C} o una \emph{A} alla terminazione $3'$ del tRNA. 
\section{Modifiche dei nucleotidi di rRNA e di tRNA}
I nucleotidi negli rRNA e nei tRNA vengono modificati chimicamente dopo la trascrizione: vari enzimi modificano le basi: 
\begin{itemize}
	\item Al gruppo ammino: adnenina e guasina \emph{\ce{NH2}}.
	\item A un atomo di azoto: guanina \emph{N7} e citosina \emph{N3}.
	\item A un atomo di carbonio: citosina e uracile \emph{C3}.
\end{itemize}
Al ribosio a \emph{$2'$-OH}. Le modifiche sono enzimatiche e reversibili. Possono essere piccole come metilazioni o grandi come l'addizione di un amminoacido come la treonina. Il sito, 
la quantit\`a e la distribuzione delle modifiche variano tra molecole di RNA, organismi e compartimenti intracellulari. Molte modifiche sono essenziali per la crescita e la 
sopravvivenza. Le modifiche pi\`u comuni al rRNA sono la metilazione del ribosio \emph{$2'$-OH} e la pseudo-uridilazione della base. Molte modifiche degli rRNA si trovano in basi 
importanti per la funzione dei ribosomi. Sono state identificate pi\`u di $100$ modifiche negli RNA attraverso epitranscriptome sequencing e il $75\%$ di esse si trova nei rRNA. 
Queste aumentano il numero di dimensioni e struttura e la stabilit\`a della molecola di tRNA. ANche gli mRNA sono modificati con il capping $m^7G$ aggiungendo una guanosina metilata
a \emph{n7}. Gli RNA spliceosomali, RNA piccoli nucleari \emph{snRNA}, piccoli RNA nucleolari \emph{snoRNA}, \emph{miRNA} e \emph{siRNA} sono modificati. Molte modifiche sono collegate
al piegamento, attivit\`a e stabilita degli RNA, differenziazione della cellula, determinazione del sesso e risposta allo stress. Mutazioni negli enzimi modificanti il RNA causano 
malattie negli esseri umani. 
\subsection{Modifiche chimiche dei tRNA procarioti ed eucarioti}
I tRNA subiscono delle modfiiche a posizioni specifiche. Nella regione dell'anticodone aumentano l'efficienza della traduzione, mentre nel corpo centrale correggono la struttura 
secondaria e terziaria e aumentano la stabilit\`a del tRNA in quanto l'assenza di certe di esse causa degradazione. Le modifiche combinate definiscono l'identi\`a di ogni tRNA e 
tRNA che portano lo stesso amminoacido possono essere diversamente modificati. 
\subsection{Modifiche chimiche degli rRNA}
Dopo la sintesi i pre-rRNA subiscono il processamento e due tipi di modifiche catalizzate dagli \emph{snoRNP}:
\begin{itemize}
	\item \emph{$2'$-OH} metilazione del ribosio in certi nucleotidi mediata dal \emph{snoRNA C/D}.
	\item \emph{Pseudo-uridilazione} dell'uracile creando uridina o pseudo-uridina $\Psi$ mediata da \emph{snoRNA H/ACA}.
\end{itemize}
Queste potrebbero contribuire alla stabilit\`a e al piegamento del rRNA. L'interazione con le proteine ribosomali  modula la biogenesi dei ribosomi e ne aumenta l'attivitù\`a. 
\subsubsection{Enzimi coinvolti}
Gli enzimi che catalizzano la metilazione del ribosio e la conversione dell'uridina in pseudo-uridina nel pre-rRNA son conservati in tutti gli organismi, ma varia il meccanismo di 
riconoscimento. Negli eucarioti ed archea si utilizzano piccoli RNA guida nucleolari \emph{snoRNA} che portano l'enzima modificatore al sito corretto. Questi si associano con un numero
di proteine per formare un \emph{RNP} attivo detto \emph{snoRNP}. \emph{snoRNP C/D} e \emph{H/ACA} contengono $4$ diverse proteine ognuna e i loro nomi derivano da box conservate 
presenti nella loro sequenza a RNA. Gli \emph{snoRNA} sono lunghi tra i $60$ e i $300$ nucletoidi e l a maggior parte sono fatti dagli introni dei pre-mRNA. Gli \emph{snoRNA} guida si 
accoppiano con le basi di regioni specifiche dei pre-rRNA obiettivo e guidano gli enzimi come metil tasferasi firillarina \emph{Nop1}, e la pseudo-uridina sintestasi discherina a quelle
posizioni. 
\section{Capping e poliadenilazione di mRNA}

\section{RNA splicing}

\section{Definizione degli esoni e splicing alternativo}

