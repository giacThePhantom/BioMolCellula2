\chapter{RNA regolatori}
\section{Panoramica degli RNA regolatori}
Gli RNA regolatori controllano molti processi biologici in tutti i regni della vita.
\subsection{Principi fondamentali}
Per tutti gli RNA regolatori:
\begin{itemize}
	\item Il trascritto primario \`e processato per formare la molecola funzionante.
	\item Utilizzano accoppiamento di basi con i target di RNA o DNA.
	\item Interagiscono spesso con altre componenti con altre proteine per permettere la loro funzione. Lo fanno aumentando la loro funzionalit\`a o trasportandole.
\end{itemize}
\subsection{Iterazioni tra le basi}
Le interazioni tra le basi con altre molecole di RNA o DNA con gli RNA regolatori possono avere diversi effetti:
\begin{itemize}
	\item Possono distruggere il legame della molecola legata con una proteina modificandone la struttura tridimensionale.
	\item Possono cambiare la struttura del RNA.
	\item Possono promuovere il legame con proteine.
\end{itemize}
\subsection{Codifica}
Gli RNA regolatori possono essere codificati in diversi modi rispetto al RNA target.
\begin{itemize}
	\item Nella maggior parte dei casi si trovano codificati su una regione di DNA separata: \emph{trans-encoded}.
	\item Si possono trovare sullo stesso filamento di DNA ma in anti-senso rispetto al target.
	\item Si possono trovare come parte della stessa sequenza target \emph{cis-encoded}.
\end{itemize}
\section{Piccoli RNA batterici}
Gli RNA regolatori batterici sono sintetizzati come trascritti lunghi tra i $100$ e i $300nt$ e si dicono small RNA \emph{sRNA}. Alcuni sono ulteriormente processati in frammenti pi\`u 
piccoli, ma la maggior parte agiscono come molecole intatte a differenza di quanto succede negli eucarioti in cui sono molto pi\`u processati e piccoli. La maggior parte degli sRNA
che fa accoppiamento di basi sono codificati \emph{in trans} e sono prodotti in risposta all'ambiente. 
\subsection{Funzioni}
Molti si legano al mRNA vicino al motivo di Shine-Dalgarno, il luogo di legame dei ribosomi e inizio della traduzione: operano un suo controllo negativo. Alcune volte la aumentano 
promuovendo il legame del ribosoma distruggendo strutture sul mRNA che lo impedivano. Alcuni sRNA influenzano la degradazione del RNA target reclutando ribonucleasi. Si nota come
la flessibilit\`a strutturale degli RNA regolatori permette il legame di altre molecole rispetto a RNA e DNA come proteine \emph{RNP} e metaboliti per la formazione di riboswitches, 
molecole che legano il RNA per impedire la traduzione. 
\subsection{Traduzione del trascritto regolata dal ferro}
Il sRNA \emph{RyhB} \`e lungo $90nt$ in E. coli e reprime la traduzione di enzimi conservatori e utilizzatori del ferro quando i livelli di ferro cellulare sono bassi. 
\subsubsection{Bassi livelli di ferro}
Questo RNA interagisce con il trascritto in diversi modi: impedisce il legame del ribosoma con certi mRNA e promuove il legame con quelli codificanti prodotti necessari in condizioni di 
basso ferro. Recluta inoltre la ss endonucleasi \emph{RNAasi E} degradando il mRNA obiettivo. 
\subsubsection{Alti livelli di ferro}
Con alti livelli di ferro i trascritti degli enzimi conservatori e utilizzatori del ferro sono tradotti o stabili e la traduzione di prodotti necessari in condizione di basso livello 
di ferro \`e impedita dalla struttura del mRNA secondaria. 
\subsection{Livelli di complementariet\`a}
Il sRNA pu\`o interagire con multipli RNA target a diversi livelli in base al livello di complementariet\`a. Si nota come la complementariet\`a di sRNA \emph{trans-encoded} \`e limitata
a $10$-$20bp$ e tipicamente una corta regione di coppie \`e critica e questo filamento \`e conservato per un certo livello. 
\subsubsection{Proteina chaperon del RNA}
La proteina chaperone del RNA \emph{Hfq} aiuta il sRNA a trovare il proprio target all'interno dei migliaia di trascritti della cellula. \`E un anello proteico esamerico e il legame
con \emph{Hfq} al motivo \emph{$5'$-AAYAAYAA-$3"$} porta insieme le molecole di sRNA e mRNA per promuovere l'accoppiamento tra le basi. Le due molecole spesso si legano a parti diverse 
di \emph{Hfq}. Quando i livelli di \emph{Hfq} sono pi\`u bassi del affinch\`e avvengano tutte le interazioni sRNA compete per \emph{Hfq} e diventa parte del sistema regolatorio. 
\subsubsection{mRNA esca}
mRNA decoy o esca pu\`o antagonizzare il sRNA in quanto si lega a una molecola di sRNA impedendo la sua azione al mRNA target.
\paragraph{Chitobiosi}
Si nota come alti livelli di chitobiosi, un dimero di glucosammine legate $\beta-1,4$ induce la trascrizione di un RNA esca \emph{chbB} che lega al sRNA \emph{ChiX} che libera il
mRNA obiettivo e causa la sua degradazione permettendo la traduzione dell'operone policistronico del chitobiosi. 
\section{\emph{sRNA} eucarioti: \emph{miRNA}, \emph{siRNA} e \emph{piRNA}}
Gli RNA piccoli negli eucarioti sono lunghi circa $22nt$ e sono derivati da trascritti pi\`u lunghi. Si associano con la famiglia \emph{Argonauta} di elicasi-RNAasi che facilitano
l'interazione con l'obiettivo.
\subsection{Classi}
\subsubsection{MicroRNA}
I \emph{miRNA} sono derivati da filamenti primari endogeni creati da RNA polimerasi II e sotto-regolano gli RNA citoplasmatici attraverso degradazione degli mRNA e repressione 
traduzionale. Sono pi\`u di \num{1000} quelli codificati dal genoma umano e regolano $\frac{1}{3}$ dei geni che codificano proteine. Reclutano una proteina argonauta che in caso
di perfect hit porta alla degradazione dell'obiettivo, altrimenti blocca la traduzione e lo trasloca nei P-bodies dove viene degradato. Successivamente avviene un rientro nucleare che
porta alla formazione di eterocromatina nel DNA dove \`e trascritto il RNA bersaglio. 
\subsubsection{RNA piccoli interferenti}
I \emph{siRNA} sono derivati da RNA a doppio filamento esogeni come RNA virali e hanno come target RNA per la loro degradazione come meccanismo di difesa cellulare.
\subsubsection{RNA interagenti con Piwi}
I \emph{piRNA} sono derivati da regioni ripetitive del genoma o trasposoni codificanti troncati e sotto-regolano la trascrizione degli elementi trasponibili. 
\section{La famiglia di proteine \emph{Argonauta}}
Tutti gli sRNA eucarioti svolgono la loro funzione associati a proteine Argonauta che si lega ad essi per facilitare la loro interazione con gli mRNA obiettivo.
\subsection{Classi}
\subsubsection{Proteine Argonauta}
Le proteine argonauta sono coinvolte nei meccanismi di miRNA e siRNA in animali, piante e funghi. 
\subsubsection{Proteine \emph{Piwi}}
Le proteine \emph{Piwi} sono coinvolte nei meccanismi di piRNA.
\subsubsection{Proteine \emph{WAGO}}
Le proteine \emph{WAGO} si trovano in C. elegans.
\section{Processamento di sRNA eucarioti}
\subsection{Il pathway di microRNA}
\subsubsection{Processamento}
Il trascritto \emph{pri-miRNA} a singolo filamento si piega in una struttura a stem-loop. Questi a ds sono rimossi dal pri-miRNA dal ``complesso microprocessore" nucleare 
\emph{Drosha+DGCR8} che produce un pre-miRNA lungo tra i $60$ e i $70nt$ a hairpin con un \emph{$3'$-OH} e un $5'$ monofosfato. Il hairpin \`e esportato dal nucleo attraverso esportine
ed \`e caricato nel complesso \emph{RISC} contenente \emph{Dicer-Argonauta} citoplasmatico. Il Dicer taglia il pre-miRNA per produrre un $22bp$ \emph{$miR:miR^*$}. 
\subsubsection{Effetti}
Si nota come un complesso \emph{miRNA-RISC} possa avere come target centinaia di copie di mRNA. Inoltre se il contatto tra \emph{miRNA-RISC} \`e debole e il target mRNA non \`e tagliato 
pu\`o subire un blocco della traduzione anche quando iniziata. Il ribosoma e il mRNA viene incluso nei corpi di processamento citoplasmatici \emph{P-bodies} dove il ribosoma si dissocia 
e il mRNA \`e ultimamente degradato. Il miRNA si pu\`o trovare in introni, esoni e sequenze codificanti o non-codificanti. Da un singolo pri-miRNA possono essere prodotti diversi miRNA: 
in C. elegans ne vengono prodotti $7$. 
\subsubsection{Caricamento su Argonauta}
Una volta che \`e generato il miRNA viene caricato sull'Argonauta del complesso \emph{RISC} che include Dicer e una proteina legante dsRNA \emph{TRBP} (TAR RNA binding protein). Un
filamento guida o attivo \emph{miR} \`e mantenuto, mentre l'altro, il filamento passeggero viene separato e tagliato dall'attivit\`a elicasica ed RNAasica dell'Argonauta  nel processo
di sorting. Il filamento guida \`e coinvolto nel silenziamento o degradazione del RNA target. 
\subsubsection{il cammino dei piccoli RNA interferenti}
\subsubsection{Processamento}
I \emph{siRNA} sono derivati da dsRNA da fonti esogene e sono coinvolti nella difesa cellulare attraverso RNA interference. Non si trova un passaggio di rottura nel nucleo e Drosha non
\`e coinvolta: Dicer taglia il dsRNA attraverso tagli sequenziali ogni $22bp$. Sono prodotte le molecole di \emph{$siR:siR^*$}. 
\subsubsection{Caricamento su Argonauta}
Una volta generato la molecola di siRNA viene caricata sulla proteina Argonauta: un filamento \`e mantenuto mentre l'altro rimosso dall'attivit\`a elicasica e RNAasica di Argonauta. Il
filamento mantenuto \`e coinvolto nel silenziamento del RNA target. 
\subsubsection{Confronto tra \emph{$miR:miR^*$} e \emph{$siR:siR^*$}}
Si nota come entrambe siano simili: entrambe possiedono \emph{$3'$-OH} e $5'$ monofosfati e possono subire modifiche post-trascrizionali come metilazioni che aumentano la loro 
stabilit\`a intracellulare. \emph{$siR:siR^*$} sono tipicamente completamente complementari e formano dimeri perfetti per regolare un mRNA target specifico, mentre le altre non lo
sono e possono regolare target diversi. 
\subsection{Il pathway di piwi interacting RNA}
Gli elementi trasponibili \emph{TE} sono componenti strutturali principali dei genomi eucarioti, ma la loro mobilizzazione ha tipicamente effetti negativi sul genoma ospite. Per
contrastarli le cellule hanno sviluppato meccanismi genetici ed epigenetici per tenerli silenziati. Uno di questi coinvolge il complesso \emph{Piwi-piRNA} che reprime i \emph{TE}
rompendo il loro trascritto nel citoplasma e direzionando specifici rimodellatori della cromatina ai loci \emph{TE} nel nucleo. La maggior parte degli RNA interagenti con Piei
sono derivati da cluster di piRNA genomici. 
\subsubsection{Processamento}
I piRNA sono trascritti dai cluster e processati fino a raggiungere la lunghezza di $24$-$30nt$ attraverso un meccanismo a ping pong o amplificazione. Ogni trasposone inserito in
orientamento inverso nel cluster a piRNA pu\`o dare origine a piRNA anti-senso. I piRNA anti-senso sono incorporati in una proteina Argonauta Piwi e direzionano la sua attivit\`a
RNAasica attaverso il trascritto trasposone di sensoL il prodotto della rottura \`e legato da un'altra proteina Piwi e accorciato a dimensione di piRNA. Il piRNA senso \`e utilizzato
per rompere cluster di piRNA trascritti per generare pi\`u piRNA anti-senso. Alla fine il complesso di piRNA anti-senso e Piei si muovono nel nucleo per reprimere i trasposoni
attraverso metilazione del DNA e modifica degli isoni delle sequenze promotrici o codificanti. 
\subsection{Gli enzimi Drosha e Dicer RNAasi III}
Gli enzimi della famiglia delle ds RNAasi III sono coinvolti nel processamento di miRNA e siRNA in quanto tagliano il dsRNA. Si dividono in tre classi:
\begin{itemize}
	\item Enzimi di classe I come RNAasi III con un dominio catalitico e agiscono come dimeri.
	\item Enzimi di classe II come Drosha con due domini catalitici e agiscono come monomeri.
	\item Enzimi di classe III come Dicer con due domini catalitici e agiscono come monomeri.
\end{itemize}
\subsubsection{Drosha}
Drosha determina la lunghezza dei frammenti tagliati attraverso la proteina accessorio \emph{DGCR8} che si lega a $11bp$ dalla base e forma con Drosha il complesso microprocessore.
\subsubsection{Dicer}
Dicer determina la lunghezza dei frammenti tagliati assicurando la terminazione $3'$ del RNA con il dominio \emph{PAZ} e tagliando a una certa distanza da esso. 
\paragraph{Dominio \emph{PAZ}} 
Questo dominio determina la lunghezza del dsRNA tagliato a $22nt$. 
\subsubsection{Argonauta}
Le proteine Argonauta possiedono $4$ domini: 
\begin{itemize}
	\item \emph{PAZ} si lega alla terminazione $3'$ del dsRNA legato.
	\item \emph{Mid} interagisce con la terminazione $5'$ del dsRNA.
	\item \emph{Piwi} lega $2$ \emph{\ce{Mg^{2+}}} e taglia il mRNA target ss.
	\item \emph{N}.
\end{itemize}
I primi tre insieme orientano il RNA guida legato per facilitare la scansione di mRNA target. 
\section{Silenziamento genico da parte di RNA eucarioti}
I siti di legame per miRNA si trovano tipicamente in $3'$ \emph{UTR} del mRNA target e possono essere l'obiettivo di diversi \emph{miRISC}. Questi tipicamente si accoppiano inizialmente
attraverso una sequenza $5'$ di $2$-$8$ nucleotidi ``seed" mentre la terminazione $3'$ del miRNA rimane strettamente legata al dominio \emph{PAZ}. L'accoppiamento tra miRNA e il target
\`e tipicamente imperfetto con mismatch a posizioni $10$ e $11$, il dominio \emph{PIWI} rimane inattivo e no li taglia: questo legame causa repressione traduzionale. Quando 
l'accoppiamento \`e completo avviene una rottura del mRNA, enzimi di deadenilazione $3'$ e decapping $5'$ vengono reclutati e comincia la degradazione del mRNA. \emph{siRISC} si pu\`o
legare lungo la sua intera lunghezza, rilascia il domino \emph{PAZ} inducendo un cambio conformazionale che attiva l'attivit\`a RNAasica del dominio \emph{PIWI} che rompe e degrada il
mRNA. 
\section{Ruolo della difesa virale di \emph{sRNA} batterici, eucarioti e di archea}
I virus hanno sviluppato meccanismi per contrastare il silenziamento del RNA: il virus carnation ringspot possiede una proteina $p19$ che lega a una molecola di $21$ nucleotidi di 
siRNA e impedisce che venga incorporata nel complesso \emph{RISC}. 
\section{Regolazione mediata da RNA in \emph{cis}}
Nella cellula sono attivi RNA regolatori pi\`u grandi e complessi degli sRNA> Per esempio gli RNA possono essere reponsivi dei livelli di amminoacidi e reprimere o attivare la traduzione
in accordo a tale informazione. 
\subsection{Sintesi della amminoacil tRNA sintetasi}
i tRNA controllano la sintesi della propria amminoacil tRNA sintetasi \emph{aaRS}: quando abbastanza di essa \`e presente i tRNA carichi si legano in cis al mRNA del proprio enzima 
\emph{aaRS} e promuovono la terminazione della sua trascrizione. Quando non \`e abbastanza presente i tRNA scarichi si legano al mRNA dei propri \emph{aaRS} cambiando la sua 
conformazione e permettendo alla RNA polimerasi di leggerlo. 
\subsection{Riboswitches}
I riboswitches si trovano principalmente nei batteri e possono controllare trascrizione, traduzione e splicing. Possiedono due domini principi: l'aptamero e l'effettore. Il metabolita
si lega all'aptamero e induce un cambio conformazionale all'effettore che ha un effetto sull'espressione genica. L'effetto varia per promuovere o impedire la trascrizione o 
traduzione. I metaboliti sono vitamine, ioni, zuccheri, purine e altri, oltre a diversi stati come la fosforilazione. 
\section{RNA regolatori leganti proteine}
Gli RNA leganti proteine possono avere un gran numero di effetti biologici attraverso molti meccanismi, uno dei quali \`e la titolazione di una proteina lontano dal RNA target. Per
esempio il \emph{CsrB} RNA non codificante batterico (carbon storage regulator B) possiede multipli motivi leganti \emph{GGA} che possono legarsi e sequestrare il repressore 
traduzionale \emph{CsrA} che quando libero si lega al mRNA e blocca la traduzione durante il metabolismo del glicogeno. Un altro esempio \`e il RNA $6S$ che mimica il DNA in una
bolla di trascrizione aperta: quando i livelli di \emph{NTP} sono bassi il $6S$RNA compete con il promotore del DNA per la RNA polimerasi e il suo fattore $\sigma$ sequestrandola. 
Quando i livelli sono alti il $6S$ viene trascritto causando il rilascio della RNA polimerasi sequestrata.
\section{RNA non codificanti lunghi intergenici}
Pu\`o avvenire anche la trascrizione di regioni non codificanti proteine, creando trascritti molto lunghi o lunghi RNA non codificanti intergenici \emph{lincRNA}. Un esempio \`e il RNA
\emph{Xist} coinvolto nella disattivazione del cromosoma $X$: il trascritto processato copre il cromosoma $X$ che lo ha prodotto e recluta il complesso polycomb che silenzia il cromosoma
$X$ attraverso metilazione di \emph{H3K9} e di \emph{H3K27}. 
